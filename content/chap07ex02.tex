\section{译者补充:初等数论基础}\label{sec:译者补充:初等数论基础}

\begin{remark}
    本节内容不是原书内容,而是译者根据\citet{ElementaryNumberTheory}
    以及\citet{wiki:ExtendedEuclideanAlgorithm}补充的,请酌情参考和斧正。
\end{remark}

\begin{notation}
    本节我们重申以下记号:
    \begin{itemize}
        \item 用$\mathbb{N}$表示全体正整数构成的集合;$\mathbb{Z}$表示全体整数构成的集合。
        \item 若命题$p$能推出命题$q$,则记为$p\Rightarrow q$;若$p$与$q$等价,则记为$p\Leftrightarrow q$.
    \end{itemize}
\end{notation}


\begin{theorem}[\protect\keyindex{最小自然数原理}{least number principle}{}]\label{theorem:7.ex02.1}
    设$T$是$\mathbb{N}$的一非空子集,则必有$t_0\in T$,
    使对任意的$t\in T$有$t_0\le t$,即$t_0$是$T$中最小的自然数。
\end{theorem}

% \begin{theorem}[最大自然数原理]
%     设$M$是$\mathbb{N}$的一非空子集,若$M$有上界(即存在$a\in \mathbb{N}$使
%     对任意的$m\in M$有$m\le a$),则必有$m_0\in M$,使对任意的$m\in M$有$m\le m_0$,
%     即$m_0$是$M$中最大的自然数。
% \end{theorem}

% \begin{theorem}[\protect\keyindex{归纳原理}{principle of induction}{}]
%     设$S\subseteq \mathbb{N}$,且满足
%     \begin{enumerate}
%         \item 有$1\in S$;
%         \item 对任意$n\in S$都有$n+1\in S$;
%     \end{enumerate}
%     则$S=\mathbb{N}$.
% \end{theorem}

% \begin{theorem}[\protect\keyindex{数学归纳法}{mathematical induction}{}]
%     设$P(n)$是关于自然数$n$的命题,若
%     \begin{enumerate}
%         \item 当$n=1$时,$P(1)$成立;
%         \item $P(n)$成立时必能推出$P(n+1)$成立;
%     \end{enumerate}
%     则$P(n)$对所有自然数$n$均成立。
% \end{theorem}

% \begin{theorem}[\protect 第二种数学归纳法]
%     设$P(n)$是关于自然数$n$的命题,若
%     \begin{enumerate}
%         \item 当$n=1$时,$P(1)$成立;
%         \item 设$n>1$,对所有自然数$m<n$都有$P(m)$成立时必能推出$P(n)$成立;
%     \end{enumerate}
%     则$P(n)$对所有自然数$n$均成立。
% \end{theorem}

\begin{theorem}[\protect\keyindex{鸽巢原理}{pigeonhole principle}{}]\label{theorem:7.ex02.2}
    对于某$n\in\mathbb{N}$,现有$n$个笼子和$n+1$只鸽子,
    所有的鸽子都被关在鸽笼里,那么至少有一个笼子有至少2只鸽子。
    也称\keyindex{狄利克雷抽屉原理}{Dirichlet's drawer principle}{}。
\end{theorem}

\subsection{整除与带余除法}\label{sub:整除与带余除法}
\begin{definition}
    设$a,b\in\mathbb{Z}$且$a\neq0$,若存在$q\in\mathbb{Z}$使得$b=aq$,
    则称$a$\keyindex{整除}{divide evenly}{}$b$,或说$b$能被$a$整除,记作$a|b$,
    并称$a$是$b$的\keyindex{因数}{divisor}{},也称{\sffamily 约数}、{\sffamily 除数},
    $b$是$a$的\keyindex{倍数}{multiple}{}。$a$不能整除$b$时记作$a\nmid b$.
\end{definition}

\begin{example}
    6能整除18,记作$6|18$,6是18的因数,18是6的倍数。
\end{example}

\begin{theorem}\label{theorem:7.ex02.3}
    整除具有以下性质:
    \begin{enumerate}
        \item $a|b\Leftrightarrow -a|b \Leftrightarrow a|-b \Leftrightarrow |a|||b|$;
        \item $a|b$且$b|c \Rightarrow a|c$;
        \item $a|b$且$a|c \Leftrightarrow$对任意的$x,y\in\mathbb{Z}$有$a|bx+cy$;
        \item 设$m\neq0$,则$a|b\Leftrightarrow ma|mb$;
        \item $a|b$且$b|a\Rightarrow b=\pm a$;
        \item 设$b\neq0$,则$a|b\Rightarrow |a|\le|b|$.
    \end{enumerate}
\end{theorem}
% \begin{corollary}
%     非零整数的因数只有有限个。
% \end{corollary}
% \begin{theorem}
%     设整数$b\neq0$,而$d_1,d_2,\ldots,d_k$是$b$的全体因数,
%     则$\displaystyle\frac{b}{d_1},\frac{b}{d_2},\ldots,\frac{b}{d_k}$也是
%     $b$的全体因数。此外,若$b>0$,则当$d$遍历$b$的全体正因数时,
%     $\displaystyle\frac{b}{d}$也遍历$b$的全体正因数。
% \end{theorem}
\begin{definition}
    设整数$p\neq0,\pm1$,若$p$除了$\pm1,\pm p$外没有其他因数,
    则称$p$为\keyindex{质数}{prime number}{},也称{\sffamily 素数}、{\sffamily 不可约数}。
    若$a\neq0,\pm1$且$a$不是质数,则称$a$是\keyindex{合数}{composite number}{}。
\end{definition}
\begin{example}
    3、5、11是质数,4、6、12是合数。0和1既不是质数也不是合数。
\end{example}
\begin{notation}
    下文若无特别说明,所指的质数总是正的。
\end{notation}
% \begin{theorem}
%     \begin{enumerate}
%         \item $a>1$是合数$\Leftrightarrow$$a=de,1<d<a,1<e<a$;
%         \item 若$d>1$,$p$是质数且$d|p$,则$d=p$.
%     \end{enumerate}
% \end{theorem}
% \begin{theorem}
%     若$a$是合数,则必存在质数$p|a$.
% \end{theorem}
% \begin{definition}
%     若一个整数的因数是质数时,称该因数为\keyindex{质因数}{prime factor}{}。
% \end{definition}
% \begin{theorem}
%     设整数$a\ge2$,则$a$一定可表示为质数的乘积(包括$a$本身是质数),即
%     \begin{align}\label{eq:7.ex02.primefactor}
%         a=p_1p_2\cdots p_s\, ,
%     \end{align}
%     其中$p_j(1\le j\le s)$是质数。
% \end{theorem}
% \begin{example}
%     1260共有6个质因数(包括相同的),其中不相同的有4个,即
%     $1260=2\times2\times3\times3\times5\times7=2^2\times3^2\times5\times7$.
% \end{example}
% \begin{corollary}
%     设整数$a\ge2$,
%     \begin{enumerate}
%         \item 若$a$是合数,则必有质数$p|a$且$p\le\sqrt{a}$;
%         \item 若$a$有表示\refeq{7.ex02.primefactor},则必有质数$p|a$且$p\le a^{\frac{1}{s}}$.
%     \end{enumerate}
% \end{corollary}
% \begin{theorem}
%     质数有无穷多个。
% \end{theorem}
% \begin{theorem}
%     设全体质数按大小排序成
%     \begin{align}
%         p_1=2,\quad p2=3,\quad p_3=5,\ldots\, .
%     \end{align}
%     则有
%     \begin{align}
%         p_n\le2^{2^{n-1}},\quad n=1,2,\ldots\, ,
%     \end{align}
%     及
%     \begin{align}
%         \pi(x)>\log_2{\log_2{x}},\quad x\ge2\, ,
%     \end{align}
%     其中$\pi(x)$表示不超过$x$的质数个数。
% \end{theorem}

初等数论的证明中最重要、最基本、最直接的工具是下面的
\keyindex{带余除法}{division with remainder}{},
也称\keyindex{欧几里德除法}{Euclidean division}{}。
\begin{theorem}\label{theorem:7.ex02.4}
    % \label{theorem:7.ex02.EuclideanDivision}
    对于给定的$a,b\in\mathbb{Z}$且$a\neq0$,必存在唯一一对$q,r\in\mathbb{Z}$,满足
    \begin{align}\label{eq:7.ex02.EuclideanDivision}
        b=qa+r,\quad 0\le r<|a|\, .
    \end{align}
    此外,$a|b \Leftrightarrow r=0$.
\end{theorem}
\begin{prove}
    {\sffamily 唯一性}\quad 若还有整数$q'$与$r'$满足
    \begin{align}\label{eq:7.ex02.prove-theorem4-01}
        b=q'a+r',\quad 0\le r'<|a|\, ,
    \end{align}
    不妨设$r'\ge r$.则由\refeq{7.ex02.EuclideanDivision}和\refeq{7.ex02.prove-theorem4-01}得$0\le r'-r<|a|$,及
    \begin{align}
        r'-r=(q-q')a\, .
    \end{align}
    若$r'-r>0$,则由上式及定理\ref{theorem:7.ex02.3}(6)
    推出$|a|\le r'-r$.这和$r'-r<|a|$矛盾。所以必有$r'=r$,进而得$q'=q$.

        {\sffamily 存在性}\quad 当$a|b$时,可取$q=\displaystyle\frac{b}{a}$,$r=0$.
    当$a\nmid b$时。考虑集合
    \begin{align}
        T=\{b-ka:k=0,\pm1,\pm2,\ldots\}\, .
    \end{align}
    容易看出,集合$T$中必有正整数,所以由定理\ref{theorem:7.ex02.1}知,
    $T$中必有一个最小正整数,设为
    \begin{align}
        t_0=b-k_0a>0\, .
    \end{align}
    现在来证明必有$t_0<|a|$.因$a\nmid b$,所以$t_0\neq |a|$.
    若$t_0>|a|$,则$t_1=t_0-|a|>0$,显然$t_1\in T$,$t_1<t_0$.
    这和$t_0$的最小性矛盾。取$q=k_0$,$r=t_0$就满足要求。

    最后,显然当$b=qa+r$时,$a|b \Leftrightarrow a|r$.
    当满足$0\le r<|a|$时,由定理\ref{theorem:7.ex02.3}(6)就
    推出$a|r \Leftrightarrow r=0$.这就证明了定理的最后一部分。
\end{prove}

上述定理还有更灵活的形式。
\begin{theorem}\label{theorem:7.ex02.5}
    对于给定的$a,b,d\in\mathbb{Z}$且$a\neq0$,必存在唯一一对$q_1,r_1\in\mathbb{Z}$,满足
    \begin{align}\label{eq:7.7.ex02.remainder}
        b=q_1a+r_1,\quad d\le r_1<|a|+d\, .
    \end{align}
    此外,$a|b \Leftrightarrow a|r_1$.
\end{theorem}

只要对$a$和$b-d$用定理\ref{theorem:7.ex02.4}即可推出定理\ref{theorem:7.ex02.5}。
适当选取$d$可令\refeq{7.7.ex02.remainder}变形为下面的形式:
\begin{align}
    b & =q_1a+r_1, & -\frac{|a|}{2}< r_1\le\frac{|a|}{2}\, ,\label{eq:7.ex02.remainder02} \\
    b & =q_1a+r_1, & -\frac{|a|}{2}\le r_1<\frac{|a|}{2}\, ,\label{eq:7.ex02.remainder03} \\
    b & =q_1a+r_1, & 1\le r_1\le |a|\, .\label{eq:7.ex02.remainder04}
\end{align}
通常称\refeq{7.ex02.EuclideanDivision}中的$r$为$b$被$a$除后的\keyindex{最小非负余数}{least non-negative remainder}{remainder\ 余数},
\refeq{7.ex02.remainder02}和\refeq{7.ex02.remainder03}中的$r_1$都称为\keyindex{绝对最小余数}{least absolute remainder}{remainder\ 余数},
\refeq{7.ex02.remainder04}中的$r_1$称为\keyindex{最小正余数}{least positive remainder}{remainder\ 余数},
\refeq{7.7.ex02.remainder}中的$r_1$统称为\keyindex{余数}{remainder}{}。

% \begin{corollary}
%     设$a>0$,任意整数被$a$除后所得的最小非负余数是且仅是$0,1,\ldots,a-1$这$a$个数中的一个。
% \end{corollary}
\begin{corollary}
    给定正整数$a\ge2$,则任一正整数$n$必可唯一表示为
    \begin{align}
        n=r_ka^k+r_{k-1}a^{k-1}+\cdots+r_1a+r_0\, ,
    \end{align}
    其中整数$k\ge0,0\le r_j\le a-1(0\le j\le k),r_k\neq0$.
    这即正整数的$a$进制表示。
\end{corollary}
\begin{prove}
    对正整数$n$必有唯一的$k\ge 0$,使得$a^k\le n<a^{k+1}$.
    由带余除法知,必有唯一的$q_0,r_0$满足
    \begin{align}
        n=q_0a+r_0,\quad 0\le r_0<a\, .
    \end{align}
    若$k=0$,则必有$q_0=0$,$1\le r_0<a$,所以结论成立。
    设结论对$k=m\ge0$成立,则当$k=m+1$时,上式中的$q_0$必满足
    \begin{align}
        a^m\le q_0<a^{m+1}\, .
    \end{align}
    由假设知
    \begin{align}
        q_0=s_ma^m+\cdots+s_0\, ,
    \end{align}
    其中$0\le s_j\le a-1(0\le j\le m-1)$,$1\le s_m\le a-1$.因而有
    \begin{align}
        n=s_ma^{m+1}+\cdots+s_0a+r_0\, ,
    \end{align}
    即结论对$m+1$也成立。由数学归纳法,推论得证。
\end{prove}

\subsection{最大公因数与最小公倍数}\label{sub:最大公因数与最小公倍数}
\begin{definition}
    设$a_1,a_2\in\mathbb{Z}$,若$d|a_1$且$d|a_2$,则称$d$是
    $a_1$与$a_2$的\keyindex{公因数}{common divisor}{divisor\ 因数}。
    一般地,设$a_1,\ldots,a_k$是$k$个整数,若$d|a_1,\cdots,d|a_k$,
    则称$d$是$a_1,\ldots,a_k$的公因数。
\end{definition}
\begin{example}
    12和18的公因数是$\pm1,\pm2,\pm3,\pm6$.$n$和$n+1$的公因数是$\pm1$.
    当$a_1,\ldots,a_k$中有一个不为零时,它们的公因数个数有限。
\end{example}
\begin{definition}
    设$a_1,a_2\in\mathbb{Z}$不全为零,称$a_1$和$a_2$的公因数中
    最大的为$a_1$和$a_2$的\keyindex{最大公因数}{greatest common divisor}{divisor\ 因数}(GCD),
    记作$(a_1,a_2)$.一般地,设$a_1,\ldots,a_k$是$k$个不全为零的整数,
    称$a_1,\ldots,a_k$的公因数中最大的为$a_1,\ldots,a_k$的最大公因数,
    记作$(a_1,\ldots,a_k)$.用$\mathcal{D}(a_1,\ldots,a_k)$表示$a_1,\ldots,a_k$的
    所有公因数组成的集合。于是
    \begin{align}
        (a_1,a_2)        & =\max\limits_{d\in\mathcal{D}(a_1,a_2)}{d}\, ,        \\
        (a_1,\ldots,a_k) & =\max\limits_{d\in\mathcal{D}(a_1,\ldots,a_k)}{d}\, .
    \end{align}
\end{definition}
\begin{example}
    $\mathcal{D}(12,16)=\{\pm1,\pm2,\pm3,\pm6\}$,$(12,18)=6$;
    $\mathcal{D}(6,10,-15)=\{\pm1\}$,$(6,10,-15)=1$;
    $(n,n+1)=1$.
\end{example}
\begin{theorem}\label{theorem:7.ex02.6}
    最大公因数满足以下性质:
    \begin{enumerate}
        \item $(a_1,a_2)=(a_2,a_1)=(-a_1,a_2)$;一般地,\\
              $(a_1,a_2,\ldots,a_i,\ldots,a_k)=(a_i,a_2,\ldots,a_1,\ldots,a_k)=(-a_1,a_2,\ldots,a_i,\ldots,a_k)$;
        \item $a_1|a_j(j=2,\ldots,k)\Rightarrow (a_1,a_2)=(a_1,a_2,\ldots,a_k)=|a_1|$;
        \item 对任意整数$x$,$(a_1,a_2)=(a_1,a_2,a_1x)$;$(a_1,\ldots,a_k)=(a_1,\ldots,a_k,a_1x)$;
        \item 对任意整数$x$,$(a_1,a_2)=(a_1,a_2+a_1x)$;\\
              $(a_1,a_2,a_3,\ldots,a_k)=(a_1,a_2+a_1x,a_3,\ldots,a_k)$;
        \item 若$p$是质数,则
              \begin{align}
                  (p,a_1)=\left\{
                  \begin{array}{ll}
                      p, & \text{若}p|a_1\, ,      \\
                      1, & \text{若}p\nmid a_1\, ;
                  \end{array}
                  \right.
              \end{align}
              一般地
              \begin{align}
                  (p,a_1,\ldots,a_k)=\left\{
                  \begin{array}{ll}
                      p, & \text{若}p|a_j,\quad j=1,2,\ldots,k, \\
                      1, & \text{其他。}
                  \end{array}
                  \right.
              \end{align}
    \end{enumerate}
\end{theorem}
\begin{definition}
    若$(a_1,a_2)=1$,则称$a_1$和$a_2$是\keyindex{互质}{coprime}{}
    (或relatively prime、mutually prime)的,也称{\sffamily 互素}、{\sffamily 既约}。
    一般地,若$(a_1,\ldots,a_k)=1$,则称$a_1,\ldots,a_k$是互质的。
\end{definition}
\begin{theorem}\label{theorem:7.ex02.7}
    若存在整数$x_1,\ldots,x_k$使得$a_1x_1+\cdots+a_kx_k=1$,则$a_1,\ldots,a_k$是互质的。
\end{theorem}
\begin{prove}
    因为$a_1,\ldots,a_k$的任意公因数$d$一定要整除1,所以必有$d=\pm1$.定理得证。
\end{prove}
\begin{theorem}\label{theorem:7.ex02.8}
    设正整数$m|(a_1,\ldots,a_k)$,则
    \begin{align}\label{eq:7.ex02.prove-theorem8-01}
        m\left(\frac{a_1}{m},\cdots,\frac{a_k}{m}\right)=(a_1,\ldots,a_k)\, .
    \end{align}
    特别地有
    \begin{align}\label{eq:7.ex02.prove-theorem8-02}
        \left(\frac{a_1}{(a_1,\cdots,a_k)},\ldots,\frac{a_k}{(a_1,\cdots,a_k)}\right)=1\, .
    \end{align}
\end{theorem}
\begin{prove}
    记$D=(a_1,\ldots,a_k)$.由$m|D$,$D|a_j(1\le j \le k)$知
    $m|a_j(1\le j \le k)$,故
    \begin{align}
        \frac{D}{m}\bigg|\frac{a_j}{m},\quad j=1,\ldots,k\, ,
    \end{align}
    即$\displaystyle\frac{D}{m}$是$\displaystyle\frac{a_1}{m},\ldots,\frac{a_k}{m}$的公因数
    且为正,所以由定义知
    \begin{align}\label{eq:7.ex02.prove-theorem8-03}
        \frac{D}{m}\le\left(\frac{a_1}{m},\ldots,\frac{a_k}{m}\right)\, .
    \end{align}
    另一方面,若$\displaystyle d\bigg|\frac{a_j}{m}(1\le j\le k)$,
    则$md|a_j(j=1,\ldots,k)$,由定义知
    \begin{align}
        md\le D,\quad \text{即}d\le\frac{D}{m}\, .
    \end{align}
    取$d=\displaystyle\left(\frac{a_1}{m},\ldots,\frac{a_k}{m}\right)$,
    由此及\refeq{7.ex02.prove-theorem8-03}即得\refeq{7.ex02.prove-theorem8-01}。
    在\refeq{7.ex02.prove-theorem8-01}中取$m=(a_1,\ldots,a_k)$即得\refeq{7.ex02.prove-theorem8-02}。
\end{prove}
\begin{definition}
    设$a_1,a_2\in\mathbb{Z}$均不为零,若$a_1|l$且$a_2|l$,
    则称$l$是$a_1$和$a_2$的\keyindex{公倍数}{common multiple}{multiple\ 倍数}。
    一般地,设$a_1,\ldots,a_k$是$k$个均不为零的整数,
    若$a_1|l,\ldots,a_k|l$,则称$l$是$a_1,\ldots,a_k$的公倍数。
    此外,以$\mathcal{L}(a_1,\ldots,a_k)$表示$a_1,\ldots,a_k$的所有公倍数构成的集合。
\end{definition}
\begin{example}
    $\mathcal{L}(2,3)=\{0,\pm6,\pm12,\ldots,\pm6k,\ldots\}$.
\end{example}
\begin{definition}
    设$a_1,a_2\in\mathbb{Z}$均不为零,我们把$a_1$和$a_2$公倍数中的最小正数
    称为$a_1$和$a_2$的\keyindex{最小公倍数}{least common multiple}{multiple\ 倍数},记作$[a_1,a_2]$,即
    \begin{align}
        [a_1,a_2]=\min\limits_{l\in\mathcal{L}(a_1,a_2),l>0}{l}\, .
    \end{align}
    一般地,设$a_1,\ldots,a_k\in\mathbb{Z}$均不为零,我们把
    $a_1,\ldots,a_k$公倍数中的最小正数称为$a_1,\ldots,a_k$的最小公倍数,
    记作$[a_1,\ldots,a_k]$,即
    \begin{align}
        [a_1,\ldots,a_k]=\min\limits_{l\in\mathcal{L}(a_1,\ldots,a_k),l>0}{l}\, .
    \end{align}
\end{definition}
\begin{example}
    $[2,3]=6$;$[2,3,4]=12$.
\end{example}
\begin{theorem}\label{theorem:7.ex02.9}
    最小公倍数满足以下性质:
    \begin{enumerate}
        \item $[a_1,a_2]=[a_2,a_1]=[-a_1,a_2]$;一般有\\
              $[a_1,a_2,\ldots,a_i,\ldots,a_k]=[a_i,a_2,\ldots,a_1,\ldots,a_k]=[-a_1,a_2,\ldots,a_i,\ldots,a_k]$;
        \item $a_2|a_1\Rightarrow [a_1,a_2]=|a_1|$;\\
              $a_j|a_1(2\le j\le k)\Rightarrow [a_1,\ldots,a_k]=|a_1|$;
        \item 对任意的$d|a_1$,有$[a_1,a_2]=[a_1,a_2,d]$;$[a_1,\ldots,a_k]=[a_1,\ldots,a_k,d]$.
    \end{enumerate}
\end{theorem}
\begin{theorem}\label{theorem:7.ex02.10}
    设$m>0$,则$[ma_1,\ldots,ma_k]=m[a_1,\ldots,a_k]$.
\end{theorem}
\begin{prove}
    设$L=[ma_1,\ldots,ma_k], L'=[a_1,\ldots,a_k]$.
    由$ma_j|L(1\le j\le k)$推出$\displaystyle a_j\bigg|\frac{L}{m}(1\le j\le k)$,
    进而由最小公倍数定义知$L'\le\displaystyle\frac{L}{m}$.
    另一方面,由$a_j|L'(1\le j\le k)$推出$ma_j|mL'(1\le j\le k)$,
    进而由最小公倍数定义得$L\le mL'$.由此定理得证。
\end{prove}
\begin{theorem}\label{theorem:7.ex02.11}
    $a_j|c(1\le j\le k)\Leftrightarrow [a_1,\ldots,a_k]|c$.
\end{theorem}
\begin{prove}
    $[a_1,\ldots,a_k]|c\Rightarrow a_j|c(1\le j\le k)$是显然的。
    下面证$a_j|c(1\le j\le k)\Rightarrow [a_1,\ldots,a_k]|c$.
    设$L=[a_1,\ldots,a_k]$.由定理\ref{theorem:7.ex02.4}知,有$q,r$使得
    \begin{align}
        c=qL+r,\quad 0\le r<L\, .
    \end{align}
    由此及$a_j|c$推出$a_j|r(1\le j\le k)$,所以$r$是公倍数。
    进而由最小公倍数的定义及$0\le r<L$可得$r=0$,即$L|c$.
    结论表明:公倍数一定是最小公倍数的倍数。
\end{prove}
\begin{theorem}\label{theorem:7.ex02.12}
    设$D$为正整数,则$D=(a_1,\ldots,a_k)$的充要条件是
    \begin{enumerate}
        \item $D|a_j(1\le j\le k)$;
        \item 若$d|a_j(1\le j\le k)$,则$d|D$.
    \end{enumerate}
\end{theorem}
\begin{prove}
    {\sffamily 充分性}\quad 由第一个条件知$D$是$a_j(1\le j\le k)$的公因数,
    由第二个条件、定理\ref{theorem:7.ex02.3}(6)及$D\ge1$知,
    $a_j(1\le j\le k)$的任一公因数$d$有$|d|\le D$.
    因而由定义知$D=(a_1,\ldots,a_k)$.

        {\sffamily 必要性}\quad 设$d_1,\ldots,d_s$是$a_1,\ldots,a_k$的
    全体公因数,$L=[d_1,\ldots,d_s]$.由定理\ref{theorem:7.ex02.11}
    知$L|a_j(1\le j\le k)$,因此$L$满足了两个条件。
    由上面充分性的证明知$L=(a_1,\ldots,a_k)=D$.必要性得证。
    结论表明:公因数一定是最大公因数的因数。
\end{prove}
\begin{theorem}\label{theorem:7.ex02.13}
    设$m>0$,则$m(b_1,\ldots,b_k)=(mb_1,\ldots,mb_k)$.
\end{theorem}
\begin{prove}
    在定理\ref{theorem:7.ex02.8}中取$a_j=mb_j(1\le j\le k)$,
    由定理\ref{theorem:7.ex02.12}可得$m|(a_1,\ldots,a_k)$.
    因此\refeq{7.ex02.prove-theorem8-01}成立,即本定理结论成立。
\end{prove}
\begin{theorem}\label{theorem:7.ex02.14}
    \begin{enumerate}
        \item $(a_1,a_2,a_3,\ldots,a_k)=((a_1,a_2),a_3,\ldots,a_k)$;
        \item $(a_1,\ldots,a_{k+r})=((a_1,\ldots,a_k),(a_{k+1},\ldots,a_{k+r}))$.
    \end{enumerate}
\end{theorem}
\begin{prove}
    对于第一个结论:若$d|a_j(1\le j\le k)$,则由定理\ref{theorem:7.ex02.12}知,
    $d|(a_1,a_2)$,$d|a_j(3\le j\le k)$;反过来,若$d|(a_1,a_2)$,$d|a_j(3\le j\le k)$,
    则由定义知,$d|a_j(1\le j\le k)$.这就证明了
    \begin{align}
        \mathcal{D}(a_1,a_2,a_3,\ldots,a_k)=\mathcal{D}((a_1,a_2),a_3,\ldots,a_k)\, .
    \end{align}
    故第一个结论成立。由它可立即推出第二个结论。
\end{prove}
\begin{theorem}\label{theorem:7.ex02.15}
    设$(m,a)=1$,则$(m,ab)=(m,b)$.
\end{theorem}
\begin{prove}
    $m=0$时$a=\pm1$,结论显然成立。当$m\neq0$时,
    由定理\ref{theorem:7.ex02.6}、定理\ref{theorem:7.ex02.13}和定理\ref{theorem:7.ex02.14}可得
    \begin{align}
        (m,b)=(m,b(m,a))=(m,(mb,ab))=(m,mb,ab)=(m,ab)\, .
    \end{align}
    得证。
\end{prove}
\begin{theorem}\label{theorem:7.ex02.16}
    设$(m,a)=1$,那么,若$m|ab$,则$m|b$.
\end{theorem}
\begin{prove}
    由定理\ref{theorem:7.ex02.6}和定理\ref{theorem:7.ex02.15}得
    $|m|=(m,ab)=(m,b)$,于是$m|b$.
\end{prove}
% \begin{theorem}
%     $[a_1,a_2](a_1,a_2)=|a_1a_2|$.
% \end{theorem}
\begin{theorem}\label{theorem:7.ex02.17}
    设$a_1,\ldots,a_k\in\mathbb{Z}$不全为零,则有
    \begin{enumerate}
        \item $(a_1,\ldots,a_k)=\min\{s=a_1x_1+\cdots+a_kx_k:x_j\in\mathbb{Z}(1\le j\le k),s>0\}$,即
              $a_1,\ldots,a_k$的最大公因数等于$a_1,\ldots,a_k$的所有整系数线性组合
              构成的集合$S$中的最小正整数。
        \item 一定存在一组整数$x'_1,\ldots,x'_k$使得
              \begin{align}\label{eq:7.ex02.theorem17-02}
                  (a_1,\ldots,a_k)=a_1x'_1+\cdots+a_kx'_k\, .
              \end{align}
    \end{enumerate}
\end{theorem}
\begin{prove}
    由于$0<a_1^2+\cdots+a_k^2\in S$,所以集合$S$中有正整数,
    由定理\ref{theorem:7.ex02.1}知$S$中必有最小正整数,记为$s_0$.
    显然对任一公因数$d|a_j(1\le j \le k)$必有$d|s_0$,所以$|d|\le s_0$.
    另一方面,对任一$a_j$由定理\ref{theorem:7.ex02.4}知存在$q_j,r_j$满足
    \begin{align}
        a_j=q_js_0+r_j,\quad 0\le r_j<s_0\, .
    \end{align}
    显然$r_j\in S$.若$r_j>0$,则和$s_0$的最小性矛盾,所以$r_j=0$,
    即$s_0|a_j(1\le j \le k)$.所以$s_0$是最大公因数。$s_0$当然是
    \refeq{7.ex02.theorem17-02}右边的形式。
\end{prove}

% \subsection*{算术基本定理}
% \begin{theorem}
%     设$p$是质数,$p|a_1a_2$,则$p|a_1$或$p|a_2$至少有一个成立。
%     一般地,若$p|a_1\cdots a_k$,则$p|a_1,\ldots,p|a_k$至少有一个成立。
% \end{theorem}
% \begin{theorem}[\protect\keyindex{算术基本定理}{fundamental theorem of arithmetic}{}]
%     设$a>1$,则必有
%     \begin{align}\label{eq:7.ex02.arithmeticfundamental}
%         a=p_1p_2\cdots p_s\, ,
%     \end{align}
%     其中$p_j(1\le j\le s)$是质数,且在不计次序的意义下,
%     表示\refeq{7.ex02.arithmeticfundamental}是唯一的。
% \end{theorem}

\subsection{辗转相除法}\label{sub:辗转相除法}
\keyindex{辗转相除法}{Euclidean algorithm}{},
也称{\sffamily 欧几里得算法},是指下面求取最大公因数的算法。
它最早出现于欧几里得的《几何原本》中,我国则可追溯至约东汉出现的《九章算术》。
\begin{theorem}\label{theorem:7.ex02.18}
    给定$u_0,u_1\in\mathbb{Z}$,且$u_1\neq0,u_1\nmid u_0$.
    我们一定可以反复应用定理\ref{theorem:7.ex02.4}得到下面$k+1$个等式:
    \begin{align}\label{eq:7.ex02.EuclideanAlgorithm}
        u_0     & =q_0u_1+u_2,             &  & 0<u_2<|u_1|,\nonumber    \\
        u_1     & =q_1u_2+u_3,             &  & 0<u_3<u_2,\nonumber      \\
        u_2     & =q_2u_3+u_4,             &  & 0<u_4<u_3,\nonumber      \\
        \cdots  & \cdots\cdots\cdots\cdots &  & \cdots\cdots\cdots\cdots \\
        u_{k-2} & =q_{k-2}u_{k-1}+u_k,     &  & 0<u_k<u_{k-1},\nonumber  \\
        u_{k-1} & =q_{k-1}u_k+u_{k+1},     &  & 0<u_{k+1}<u_k,\nonumber  \\
        u_k     & =q_ku_{k+1}.             &  & \nonumber
    \end{align}
\end{theorem}
\begin{prove}
    对$u_0,u_1$应用定理\ref{theorem:7.ex02.4},由$u_1\nmid u_0$知
    必有第一式成立。同样地,如果$u_2\nmid u_1$就得到第二式。
    如果$u_2\nmid u_1$就证明定理对$k=1$成立。以此类推,就得到
    \begin{align}
        |u_1|>u_2>u_3\cdots>u_{j+1}>0
    \end{align}
    以及前面$j$个等式成立。若$u_{j+1}|u_j$,则定理对$k=j$成立;
    若$u_{j+1}\nmid u_j$,则继续对$u_j,u_{j+1}$用定理\ref{theorem:7.ex02.4}。
    由于小于$|u_1|$的正整数只有有限个,而1整除任一整数,
    所以该过程不能无限进行下去,一定会出现某个$k$,要么$1<u_{k+1}|u_k$,
    要么$1=u_{k+1}|u_k$,证毕。
\end{prove}
\begin{theorem}\label{theorem:7.ex02.19}
    在定理\ref{theorem:7.ex02.18}的条件和符号下,我们有
    \begin{enumerate}
        \item $u_{k+1}=(u_0,u_1)$;
        \item $d|u_0$且$d|u_1$的充要条件是$d|u_{k+1}$;
        \item 存在整数$x_0,x_1$,使$u_{k+1}=x_0u_0+x_1u_1$.
    \end{enumerate}
\end{theorem}
\begin{prove}
    利用定理\ref{theorem:7.ex02.6}(1)、(4),从\refeq{7.ex02.EuclideanAlgorithm}的
    最后一式开始依次往上推,可得
    \begin{align}
        u_{k+1} & =(u_{k+1},u_k)=(u_k,u_{k-1})=(u_{k-1},u_{k-2})=\cdots\nonumber \\
                & =(u_4,u_3)=(u_3,u_2)=(u_2,u_1)=(u_1,u_0)\, ,
    \end{align}
    这就得到了第一个结论。利用定理\ref{theorem:7.ex02.3}(2)、(3),
    从\refeq{7.ex02.EuclideanAlgorithm}立即推出第二个结论。
    由\refeq{7.ex02.EuclideanAlgorithm}的第$k$式知$u_{k+1}$可
    表示为$u_{k-1}$和$u_k$的整系数线性组合,
    利用\refeq{7.ex02.EuclideanAlgorithm}的第$k-1$式
    可消去该表示中的$u_k$,将$u_{k+1}$表示为$u_{k-2}$和$u_{k-1}$的
    整系数线性组合。以此类推利用\refeq{7.ex02.EuclideanAlgorithm}的
    第$k-2,k-3,\ldots,2,1$式,就能相应地消去$u_{k-1},u_{k-2},\ldots,u_3,u_2$,
    最后将$u_{k+1}$表示为$u_0$和$u_1$的整系数线性组合,即证明了第三个结论。
\end{prove}
% \begin{example}
%     利用辗转相除法求198和252的最大公因数,并将其表示为198和252的整系数线性组合。因为
%     \begin{align*}
%         252 & =1\times198+54\, , \\
%         198 & =3\times54+36\, ,  \\
%         54  & =1\times36+18\, ,  \\
%         36  & =2\times18\, ,
%     \end{align*}
%     于是$(252,198)=(198,54)=(54,36)=(36,18)=18$,且得
%     \begin{align*}
%         18 & =54-1\times36                 \\
%            & =54-(198-3\times54)           \\
%            & =-198+4\times54               \\
%            & =-198+4\times(252-1\times198) \\
%            & =4\times252-5\times198\, .
%     \end{align*}
% \end{example}

\keyindex{扩展欧几里得算法}{extended Euclidean algorithm}{}是
辗转相除法的扩展。对于给定的不全为零整数$a,b$(不妨设$b\neq0$),
它求解关于整数变量$x,y$的方程
\begin{align}\label{eq:7.ex02.ExtendedEuclideanAlgorithm}
    ax+by=(a,b)\, .
\end{align}
由定理\ref{theorem:7.ex02.17}知该方程一定有解。
如果$a$为负数,则可以转化为
\begin{align}
    |a|(-x)+by=(|a|,b)\, ,
\end{align}
$b$为负数时同理。因此我们只考虑$a,b$不小于零的情况。
在定理\ref{theorem:7.ex02.18}的记号下,原始的
辗转相除法求解$(a,b)$的递推过程是
\begin{align}
    u_0     & =a\, ,\nonumber                  \\
    u_1     & =b\, ,\nonumber                  \\
    u_2     & =u_0-q_0u_1\, ,\nonumber         \\
    u_3     & =u_1-q_1u_2\, ,\nonumber         \\
    \ldots\nonumber                            \\
    u_{k+1} & =u_{k-1}-q_{k-1}u_k\, ,\nonumber \\
    u_{k+2} & =u_k-q_ku_{k+1}=0\, .
\end{align}
最后一步得到$u_{k+2}=0$时算法终止,此时$u_{k+1}|u_k$,即$(a,b)=u_{k+1}$.

扩展欧几里得算法则还利用了以上步骤中的商$q_i$以求解\refeq{7.ex02.ExtendedEuclideanAlgorithm}:
它新引入了两组序列$s_i,t_i$,并初始化$s_0=1,s_1=0,t_0=0,t_1=1$,
在辗转相除法每步计算$u_{i+1}=u_{i-1}-q_{i-1}u_i$后额外计算
\begin{align}
    s_{i+1} & =s_{i-1}-q_{i-1}s_i\, , \\
    t_{i+1} & =t_{i-1}-q_{i-1}t_i\, ,
\end{align}
则当$u_{k+2}=0$算法终止时,求得\refeq{7.ex02.ExtendedEuclideanAlgorithm}的解
即为$x=s_{k+1},y=t_{k+1}$.
\begin{prove}
    当$i=0,1$时,显然可以验证
    \begin{align}\label{eq:7.ex02.ExtendedEuclidean-prove-01}
        as_i+bt_i=u_i
    \end{align}
    成立。若\refeq{7.ex02.ExtendedEuclidean-prove-01}对某个$i$成立,则有
    \begin{align}
        u_{i+1} & =u_{i-1}-q_{i-1}u_i\nonumber                          \\
                & =(as_{i-1}+bt_{i-1})-q_{i-1}(as_i+bt_i)\nonumber      \\
                & =a(s_{i-1}-q_{i-1}s_i)+b(t_{i-1}-q_{i-1}t_i)\nonumber \\
                & =as_{i+1}+bt_{i+1}\, ,
    \end{align}
    即\refeq{7.ex02.ExtendedEuclidean-prove-01}对$i+1$也成立。
    由数学归纳法,\refeq{7.ex02.ExtendedEuclidean-prove-01}对
    算法步骤中的所有$i$都成立,故
    \begin{align}
        as_{k+1}+bt_{k+1}=u_{k+1}=(a,b)\, ,
    \end{align}
    即求得\refeq{7.ex02.ExtendedEuclideanAlgorithm}的解为$x=s_{k+1},y=t_{k+1}$.
\end{prove}
\begin{example}
    以$a=240,b=46$为例演示扩展欧几里得法,具体步骤是
    \begin{align}
        \begin{array}{lrrrr}
            i & q_{i-2}     & u_i              & s_i             & t_i                \\
            0 & -           & 240              & 1               & 0                  \\
            1 & -           & 46               & 0               & 1                  \\
            2 & 240\div46=5 & 240-5\times46=10 & 1-5\times0=1    & 0-5\times1=-5      \\
            3 & 46\div10=4  & 46-4\times10=6   & 0-4\times1=-4   & 1-4\times(-5)=21   \\
            4 & 10\div6=1   & 10-1\times6=4    & 1-1\times(-4)=5 & -5-1\times21=-26   \\
            5 & 6\div4=1    & 6-1\times4=2     & -4-1\times5=-9  & 21-1\times(-26)=47 \\
            6 & 4\div2=2    & 4-2\times2=0     & -               & -
        \end{array}\nonumber
    \end{align}
    算法在$i=6$时终止,由$i=5$时的$u_i,s_i,t_i$可得$-9\times240+47\times46=(240,46)=2$.
\end{example}

\subsection{同余}\label{sub:同余}
\begin{definition}
    设$a,b,m\in\mathbb{Z}$且$m\neq0$,若$m|a-b$,则称$a$与$b$\keyindex{模$m$同余}{congruent modulo $m$}{},
    也称$a$同余于$b$模$m$、$b$是$a$对模$m$的剩余,记作
    \begin{align}\label{eq:7.ex02.congruent}
        a\equiv b\pmod{m}\, ,
    \end{align}
    其中$m$称为\keyindex{模}{modulus}{},称\refeq{7.ex02.congruent}为模$m$的同余式;
    否则称$a$不同余于$b$模$m$、$b$不是$a$对模$m$的剩余,记作
    \begin{align}
        a\not\equiv b\pmod{m}\, .
    \end{align}
\end{definition}

因为$m|a-b\Leftrightarrow -m|a-b$,所以\refeq{7.ex02.congruent}等价于$a\equiv b\pmod{-m}$.
由此,下文均假定模$m\ge1$.\refeq{7.ex02.congruent}中,
若$0\le b<m$,则称$b$是$a$对模$m$的最小非负剩余;
若$1\le b\le m$,则称$b$是$a$对模$m$的最小正剩余;
若$\displaystyle -\frac{m}{2}<b\le\frac{m}{2}$(或$\displaystyle -\frac{m}{2}\le b<\frac{m}{2}$),
则称$b$是$a$对模$m$的绝对最小剩余。
\begin{example}
    $m|a$可记为$a\equiv 0\pmod{m}$;偶数可记为$a\equiv 0\pmod{2}$;
    奇数可记为$a\equiv 1\pmod{2}$.
\end{example}

% \begin{theorem}
%     $a$与$b$模$m$同余的充要条件是$a$和$b$被$m$除后的最小非负余数相等,即若
%     \begin{align}
%         a & =q_1m+r_1, & 0\le r_1<m\, , \\
%         b & =q_2m+r_2, & 0\le r_2<m\, ,
%     \end{align}
%     则$r_1=r_2$.
% \end{theorem}
% \begin{prove}
%     因为$a-b=(q_1-q_2)m+(r_1-r_2)$,所以$m|a-b$的充要条件是$m|r_1-r_2$,
%     由此及$0\le |r_1-r_2|<m$即得$r_1=r_2$.
% \end{prove}

% 容易证明,$a$对模$m$的最小非负剩余、最小正剩余、绝对最小剩余
% 正好分别是$a$被$m$除后的最小非负余数、最小正余数、绝对最小余数。

\begin{theorem}\label{theorem:7.ex02.20}
    同余是一种等价关系,即有
    \begin{enumerate}
        \item $a\equiv a\pmod{m}$;
        \item $a\equiv b\pmod{m} \Leftrightarrow b\equiv a\pmod{m}$;
        \item $a\equiv b\pmod{m}, b\equiv c\pmod{m} \Rightarrow a\equiv c\pmod{m}$.
    \end{enumerate}
\end{theorem}
\begin{prove}
    由$m|a-a=0$,$m|a-b\Leftrightarrow m|b-a$,以及
    $m|a-b,m|b-c\Rightarrow m|(a-b)+(b-c)=a-c$,就推出这三个性质。
\end{prove}
\begin{theorem}\label{theorem:7.ex02.21}
    同余式可以相加,即若
    \begin{align}\label{eq:7.ex02.addcongruent}
        a\equiv b\pmod{m},\qquad c\equiv d\pmod{m}\, ,
    \end{align}
    则
    \begin{align}
        a+c\equiv b+d\pmod{m}\, .
    \end{align}
\end{theorem}
\begin{prove}
    由$m|a-b,m|c-d\Rightarrow m|(a-b)+(c-d)=(a+c)-(b+d)$,就证明该结论。
\end{prove}
\begin{theorem}\label{theorem:7.ex02.22}
    同余式可以相乘,即若\refeq{7.ex02.addcongruent}成立,则有
    \begin{align}
        ac\equiv bd\pmod{m}\, .
    \end{align}
\end{theorem}
\begin{prove}
    由$a=b+k_1m$,$c=d+k_2m$推出$ac=bd+(bk_2+dk_1+k_1k_2m)m$,就证明该结论。
\end{prove}
% \begin{theorem}
%     设$f(x)=a_nx^n+\cdots+a_0$,$g(x)=b_nx^n+\cdots+b_0$是
%     两个整系数多项式,满足
%     \begin{align}\label{eq:7.ex02.polynomialcongruent}
%         a_j\equiv b_j\pmod{m},\quad 0\le j\le n\, .
%     \end{align}
%     那么若$a\equiv b\pmod{m}$,则
%     \begin{align}
%         f(a)\equiv g(b)\pmod{m}\, .
%     \end{align}
% \end{theorem}
% \begin{definition}
%     把满足\refeq{7.ex02.polynomialcongruent}的这两个多项式
%     称作多项式$f(x)$与$g(x)$模$m$同余,记作
%     \begin{align}
%         f(x)\Equiv g(x)\pmod{m}\, .
%     \end{align}
% \end{definition}

% \begin{theorem}
%     设$d\ge1$, $d|m$,则$a\equiv b\pmod{m} \Rightarrow a\equiv b\pmod{d}$.
% \end{theorem}
% \begin{theorem}
%     设$d\neq0$,则$a\equiv b\pmod{m} \Leftrightarrow da\equiv db\pmod{|d|m}$.
% \end{theorem}

注意在模不变的条件下,同余式两边不能相约。
\begin{example}
    $6\times3\equiv6\times8\pmod{10}$,但是$3\not\equiv8\pmod{10}$.
\end{example}

\begin{theorem}\label{theorem:7.ex02.23}
    同余式$\displaystyle ca\equiv cb\pmod{m}\Leftrightarrow a\equiv b\pmod{\frac{m}{(c,m)}}$.
    特别地,当$(c,m)=1$时可得$a\equiv b\pmod{m}$,即此时可两边约去$c$.
\end{theorem}
\begin{prove}
    $ca\equiv cb\pmod{m}$即$m|c(a-b)$,这等价于
    \begin{align}
        \frac{m}{(c,m)}\bigg|\frac{c}{(c,m)}(a-b)\, .
    \end{align}
    由定理\ref{theorem:7.ex02.16}以及$\displaystyle\left(\frac{m}{(c,m)},\frac{c}{(c,m)}\right)=1$知
    这等价于
    \begin{align}
        \frac{m}{(c,m)}\bigg|a-b\, .
    \end{align}
    就证明了该结论。
\end{prove}
\begin{theorem}\label{theorem:7.ex02.24}
    若$m\ge1$,$(a,m)=1$,则存在$c$使得
    \begin{align}\label{eq:7.ex02.modularinverse}
        ca\equiv1\pmod{m}\, .
    \end{align}
    我们把$c$称作$a$对模$m$的逆,或\keyindex{模逆元}{modular multiplicative inverse}{},
    记作$a^{-1}\pmod{m}$或$a^{-1}$.
\end{theorem}
\begin{prove}
    由定理\ref{theorem:7.ex02.17}知,存在$x_0,y_0$,
    使得$ax_0+my_0=1$,取$c=x_0$即满足要求。
\end{prove}
$a$对模$m$的逆不是唯一的。若$c$是$a$对模$m$的逆,
则任一$\bar{c}\equiv c\pmod{m}$也必是$a$对模$m$的逆;
$a$对模$m$的任意两个逆$c_1,c_2$必有$c_1\equiv c_2\pmod{m}$;
若$(a,m)=1$,则$(a^{-1},m)=1$,及$(a^{-1})^{-1}\equiv a\pmod{m}$.
\begin{notation}
    下文中约定$a^{-1}\pmod{m}$或$a^{-1}$指
    任一取定的满足\refeq{7.ex02.modularinverse}的$c$.
\end{notation}

\begin{example}
    $a$对模7的逆(只列出了一个值):
    \begin{table}[htbp]
        \centering
        \begin{tabular}{c|cccccc}
            \toprule
            $a$              & 1 & 2 & 3 & 4 & 5 & 6 \\
            \midrule
            $a^{-1}\pmod{7}$ & 1 & 4 & 5 & 2 & 3 & 6 \\
            \bottomrule
        \end{tabular}
        \caption{$a$对模7的逆(只列出了一个值)。}
        \label{tab:7.ex02.modularinverse}
    \end{table}
\end{example}

\begin{theorem}\label{theorem:7.ex02.25}
    同余式组
    \begin{align}
        a\equiv b\pmod{m_j}\, \quad j=1,2,\ldots,k
    \end{align}
    同时成立的充要条件是
    \begin{align}
        a\equiv b\pmod{[m_1,\ldots,m_k]}\, .
    \end{align}
\end{theorem}
\begin{prove}
    由定理\ref{theorem:7.ex02.11}知,$m_j|a-b(j=1,2,\ldots,k)$同时成立的
    充要条件是$[m_1,\ldots,m_k]|a-b$,得证。
\end{prove}
\begin{definition}[同余类(剩余类)]
    由定理\ref{theorem:7.ex02.20}知,对于给定的模$m$,
    整数的同余关系是一个等价关系,因此全体整数可按对模$m$是否同余
    分为若干个两两不相交的集合,使得在同一个集合中的任意两个数
    对模$m$一定同余,而属于不同集合中的两个数对模$m$一定不同余。
    每一个这样的集合称为是模$m$的\keyindex{同余类}{congruence class}{},
    或模$m$的\keyindex{剩余类}{residue class}{}。
    我们把$r$所属的模$m$的同余类表示为$r\mod{m}$.
\end{definition}

\begin{theorem}\label{theorem:7.ex02.26}
    同余类具有以下性质:
    \begin{enumerate}
        \item $r\mod{m}=\{r+km:k\in\mathbb{Z}\}$;
        \item $r\mod{m}=s\mod{m}\Leftrightarrow r\equiv s\pmod{m}$;
        \item 对任意的$r,s$,要么$r\mod{m}=s\mod{m}$,要么$r\mod{m}$与$s\mod{m}$的交集为空集。
    \end{enumerate}
\end{theorem}
\begin{theorem}\label{theorem:7.ex02.27}
    对于给定的模$m$,有且恰有$m$个不同的模$m$的同余类,即
    \begin{align}\label{eq:7.ex02.theorem27-02}
        0\mod{m},\quad 1\mod{m},\quad \ldots,\quad (m-1)\mod{m}\, .
    \end{align}
\end{theorem}
\begin{prove}
    由定理\ref{theorem:7.ex02.26}(2)知这是$m$个两两不同的同余类。
    对每个整数$a$,由定理\ref{theorem:7.ex02.4}知
    \begin{align}
        a=qm+r,\quad 0\le r<m\, .
    \end{align}
    故由定理\ref{theorem:7.ex02.26}(1)知,$a\in r\mod{m}$,
    即必属于\refeq{7.ex02.theorem27-02}中的某个同余类。
\end{prove}
\begin{theorem}\label{theorem:7.ex02.28}
    同余具有以下性质:
    \begin{enumerate}
        \item 在任意取定的$m+1$个整数中,必有两个数对模$m$同余;
        \item 存在$m$个数两两对模$m$不同余。
    \end{enumerate}
\end{theorem}
\begin{prove}
    由定理\ref{theorem:7.ex02.27},对模$m$共有$m$个由\refeq{7.ex02.theorem27-02}给出的同余类,
    所以根据定理\ref{theorem:7.ex02.2},$m+1$个数中必有两个数属于同一个模$m$的同余类,
    这两个数就对模$m$同余,第一个结论得证。在每个同余类$r\mod{m}(0\le r<m)$中
    取定一个数$x_r$作代表,就得到$m$个两两对模$m$不同余的数$x_0,x_1,\ldots,x_{m-1}$,第二个结论得证。
\end{prove}

由定理\ref{theorem:7.ex02.28}可引进以下概念:
\begin{definition}
    一组数$y_1,\ldots,y_s$称为是模$m$的\keyindex{完全剩余系}{complete residue system}{},
    如果对任意的$a$有且仅有一个$y_j$满足$a\equiv y_j\pmod{m}$.
\end{definition}

\subsection{同余方程}\label{sub:同余方程}
\begin{definition}
    设整系数多项式
    \begin{align}
        f(x)=a_nx^n+\cdots+a_1x+a_0\, ,
    \end{align}
    我们称含有变量$x$的同余式
    \begin{align}\label{eq:7.ex02.congruenceequation}
        f(x)\equiv0\pmod{m}
    \end{align}
    为模$m$的\keyindex{同余方程}{congruence equation}{equation\ 方程}。
    若整数$c$满足
    \begin{align}
        f(c)\equiv0\pmod{m}\, ,
    \end{align}
    则称$c$是同余方程\refeq{7.ex02.congruenceequation}的\keyindex{解}{solution}{}。
\end{definition}

在上述定义中,显然同余类$c\mod{m}$中的任一整数也是
同余方程\refeq{7.ex02.congruenceequation}的解。
我们把这些解都看作是相同的,也常说同余类$c\mod{m}$是该方程的解,
写为$x\equiv c\pmod{m}$.当$c_1,c_2$均为该同余方程的解且对模$m$不同余时
才把它们看作是不同的解。我们把所有对模$m$两两不同余的解的个数
称为是同余方程\refeq{7.ex02.congruenceequation}
的\keyindex{解数}{number of solutions}{}。
因此我们只需要在模$m$的一组完全剩余系中来解模$m$的同余方程。
显然模$m$的同余方程的解数至多为$m$.
\begin{example}
    对于同余方程$4x^2+27x-12\equiv0\pmod{15}$,
    取模15的一个完全剩余系$-7,-6,\ldots,-1,0,1,\ldots,6,7$,
    直接代入验算知$x=-6,3$是解,所以该同余方程的解
    是$x\equiv -6,3\pmod{15}$,解数为2.
\end{example}
\begin{definition}
    设$m\nmid a$,称
    \begin{align}\label{eq:7.ex02.linearcongruence}
        ax\equiv b\pmod{m}
    \end{align}
    为模$m$的{\sffamily 一次同余方程}。
\end{definition}
\begin{example}
    同余方程$6x\equiv2\pmod{8}$的解是$x\equiv-1,3\pmod{8}$,解数为2.
\end{example}
\begin{theorem}\label{theorem:7.ex02.29}
    当$(a,m)=1$时,同余方程\refeq{7.ex02.linearcongruence}必有解,且其解数为1.
\end{theorem}
\begin{prove}
    当$(a,m)=1$时,由定理\ref{theorem:7.ex02.24}知,
    $a$对模$m$有逆$a^{-1}$(任取一个)满足
    \begin{align}
        aa^{-1}\equiv1\pmod{m}\, .
    \end{align}
    容易看出
    \begin{align}
        x_1=a^{-1}b
    \end{align}
    就满足同余方程\refeq{7.ex02.linearcongruence}。若还有解$x_2$,则有
    \begin{align}
        ax_2\equiv ax_1\pmod{m}\, ,
    \end{align}
    由此根据定理\ref{theorem:7.ex02.23}得
    \begin{align}
        x_2\equiv x_1\pmod{m}\, .
    \end{align}
    这就证明了解数为1.
\end{prove}
% \begin{theorem}
%     同余方程\refeq{7.ex02.linearcongruence}有解的充要条件是
%     \begin{align}\label{eq:7.ex02.conditionsolution}
%         (a,m)|b\, .
%     \end{align}
%     在有解时,其解数等于$(a,m)$;若$x_0$是它的解,则它的$(a,m)$个解是
%     \begin{align}
%         x\equiv x_0+\frac{m}{(a,m)}t\pmod{m},\quad t=0,\ldots,(a,m)-1\, .
%     \end{align}
% \end{theorem}
% \begin{theorem}\label{theorem:7.ex02.hassolutionlinear}
%     当$(a,m)=1$时,同余方程\refeq{7.ex02.linearcongruence}必有解,且其解数为1.
% \end{theorem}
% \begin{prove}
%     当$(a,m)=1$时,由定理\ref{theorem:7.ex02.modularinverse}知,
%     $a$对模$m$有逆$a^{-1}$(任取一个)满足
%     \begin{align}
%         aa^{-1}\equiv1\pmod{m}\, .
%     \end{align}
%     容易看出
%     \begin{align}
%         x_1=a^{-1}b
%     \end{align}
%     就满足同余方程\refeq{7.ex02.linearcongruence}。若还有解$x_2$,则有
%     \begin{align}
%         ax_2\equiv ax_1\pmod{m}\, ,
%     \end{align}
%     则从定理\ref{theorem:7.ex02.congruentreduce}推出
%     \begin{align}
%         x_2\equiv x_1\pmod{m}\, .
%     \end{align}
%     这就证明了解数为1.
% \end{prove}
% \begin{theorem}
%     同余方程\refeq{7.ex02.linearcongruence}有解的充要条件
%     是\refeq{7.ex02.conditionsolution}成立。在有解时,
%     它的解数等于$(a,m)$,以及若$x_0$是\refeq{7.ex02.linearcongruence}的解,
%     则它的$(a,m)$个解是
%     \begin{align}
%         x\equiv x_0+\frac{m}{(a,m)}t\pmod{m},\quad t=0,1,\ldots,(a,m)-1\, .
%     \end{align}
% \end{theorem}

\begin{definition}
    设$f_j(x), j=1,2,\ldots,k$是整系数多项式,我们把含有变量$x$的一组同余式
    \begin{align}\label{eq:7.ex02.congruencegroup}
        f_j(x)\equiv0\pmod{m_j},\quad 1\le j\le k\, ,
    \end{align}
    称为{\sffamily 同余方程组}。若整数$c$同时满足
    \begin{align}
        f_j(c)\equiv0\pmod{m_j},\quad 1\le j\le k\, ,
    \end{align}
    则称$c$是同余方程组\refeq{7.ex02.congruencegroup}的\keyindex{解}{solution}{}。
\end{definition}

显然在上述定义中,同余类
\begin{align}\label{eq:7.ex02.groupsolution}
    c\mod{m},\quad m=[m_1,\ldots,m_k]
\end{align}
中任一整数也是同余方程组\refeq{7.ex02.congruencegroup}的解,
我们把它们看作是相同的,也常说同余类\refeq{7.ex02.groupsolution}是
该同余方程组的一个解,写作$x\equiv c\pmod{m}$.
当$c_1,c_2$均为该同余方程组的解且对模$m$不同余时
才把它们看作是不同的解。我们把所有对模$m$两两不同余的解的个数
称为是同余方程组\refeq{7.ex02.congruencegroup}的\keyindex{解数}{number of solutions}{}。
因此我们只需要在模$m$的一组完全剩余系中来解该同余方程组,
它的解数至多为$m$.此外,只要同余方程组中任一一个方程无解,
则\refeq{7.ex02.congruencegroup}一定无解。
\begin{theorem}[\protect\keyindex{中国剩余定理}{Chinese remainder theorem}{}(CRT)]\label{theorem:7.ex02.30}
    也称{\sffamily 孙子定理}:设$m_1,\ldots,m_k$是两两互质的正整数,
    则对任意整数$a_1,\ldots,a_k$,一次同余方程组
    \begin{align}\label{eq:7.ex02.CRT}
        x\equiv a_j\pmod{m_j},\quad 1\le j\le k\, ,
    \end{align}
    必有解,且解数为1.事实上,该同余方程组的解是
    \begin{align}
        x\equiv M_1M_1^{-1}a_1+\ldots+M_kM_k^{-1}a_k\pmod{m}\, ,
    \end{align}
    这里$m=m_1m_2\cdots m_k$,$m=m_jM_j(1\le j\le k)$,以及$M_j^{-1}$是满足
    \begin{align}
        M_jM_j^{-1}\equiv1\pmod{m_j},\quad 1\le j\le k
    \end{align}
    的一个整数(即是$M_j$对模$m_j$的逆)。
\end{theorem}
\begin{prove}
    首先指出一个事实:若$x_0$满足同余方程组\refeq{7.ex02.CRT},
    且$x_0'$满足下面的另一同余方程组
    \begin{align}
        x\equiv a_j'\pmod{m_j},\quad 1\le j\le k\, ,
    \end{align}
    则$x_0+x_0'$一定是同余方程组
    \begin{align}
        x\equiv a_j+a_j'\pmod{m_j},\quad 1\le j\le k
    \end{align}
    的解。因此,我们可用下面的叠加方法来求同余方程组\refeq{7.ex02.CRT}的解。设
    \begin{align}\label{eq:7.ex02.proveCRT03}
        a_j^{(i)}=\left\{\begin{array}{ll}
            a_j, & \text{若}i=j\, ,     \\
            0,   & \text{若}i\neq j\, .
        \end{array}\right.
    \end{align}
    对每个固定的$i(1\le j\le k)$考虑同余方程组
    \begin{align}\label{eq:7.ex02.proveCRT01}
        x\equiv a_j^{(i)}\pmod{m_j},\quad 1\le j\le k\, .
    \end{align}
    注意到$j\neq i$时$a_j^{(i)}=0$,结合$m_j$两两互质,
    由这个方程组的第$1,\ldots,i-1,i+1,\ldots,k$个方程知
    \begin{align}
        x\equiv0\pmod{M_i}\, ,
    \end{align}
    即存在整数$y$使得
    \begin{align}\label{eq:7.ex02.proveCRT02}
        x=M_iy\, .
    \end{align}
    代入第$i$个方程得
    \begin{align}
        M_iy\equiv a_i\pmod{m_i}\, .
    \end{align}
    由定理\ref{theorem:7.ex02.29}的证明知
    \begin{align}
        y\equiv M_i^{-1}a_i\pmod{m_i}\, ,
    \end{align}
    即
    \begin{align}
        M_iy\equiv M_iM_i^{-1}a_i\pmod{m}\, .
    \end{align}
    由此及\refeq{7.ex02.proveCRT02}得
    \begin{align}
        x\equiv M_iM_i^{-1}a_i\pmod{m}\, .
    \end{align}
    容易验证,$M_iM_i^{-1}a_i$确是同余方程组\refeq{7.ex02.proveCRT01}的解,
    且由定理\ref{theorem:7.ex02.29}知解数为1.
    注意到由\refeq{7.ex02.proveCRT03}可得
    \begin{align}
        a_j^{(1)}+a_j^{(2)}+\cdots+a_j^{(k)}=a_j\, ,
    \end{align}
    所以$M_1M_1^{-1}a_1+\cdots+M_kM_k^{-1}a_k$一定是同余方程组\refeq{7.ex02.CRT}的解。
    若$c_1,c_2$均是同余方程组\refeq{7.ex02.CRT}的解,
    则必有
    \begin{align}
        c_1\equiv c_2\pmod{m_j},\quad 1\le j\le k\, .
    \end{align}
    又因为$m_1,\ldots,m_k$两两互质,所以
    \begin{align}
        m=m_1m_2\cdots m_k=[m_1,\ldots,m_k]\, .
    \end{align}
    利用定理\ref{theorem:7.ex02.25}结合上两式可得
    \begin{align}
        c_1\equiv c_2\pmod{m}\, ,
    \end{align}
    即同余方程组\refeq{7.ex02.CRT}的解数必为1.
\end{prove}
\begin{example}
    解同余方程组
    \begin{align}
        \left\{
        \begin{array}{l}
            x\equiv1\pmod{3}\, ,  \\
            x\equiv-1\pmod{5}\, , \\
            x\equiv2\pmod{7}\, ,  \\
            x\equiv-2\pmod{11}\, .
        \end{array}
        \right.
    \end{align}
    {\sffamily 解}\quad 取$m_1=3,m_2=5,m_3=7,m_4=11$,
    满足定理\ref{theorem:7.ex02.30}的条件。这时,
    $M_1=5\times7\times11,M_2=3\times7\times11,M_3=3\times5\times11,M_4=3\times5\times7$.
    由于$M_1\equiv(-1)\times1\times(-1)\equiv1\pmod{3}$,所以令
    \begin{align}
        1\equiv M_1M_1^{-1}\equiv M_1^{-1}\pmod{3}\, ,
    \end{align}
    因此可取$M_1^{-1}=1$.由于$M_2\equiv(-2)\times2\times1\equiv1\pmod{5}$,令
    \begin{align}
        1\equiv M_2M_2^{-1}\equiv M_2^{-1}\pmod{5}\, ,
    \end{align}
    因此可取$M_2^{-1}=1$.由于$M_3\equiv3\times5\times4\equiv4\pmod{7}$,令
    \begin{align}
        1\equiv M_3M_3^{-1}\equiv4M_3^{-1}\pmod{7}\, ,
    \end{align}
    因此可取$M_3^{-1}=2$.由$M_4\equiv3\times5\times7\equiv4\times7\equiv6\pmod{11}$,令
    \begin{align}
        1\equiv M_4M_4^{-1}\equiv6M_4^{-1}\pmod{11}\, ,
    \end{align}
    因此可取$M_4^{-1}=2$.进而由定理\ref{theorem:7.ex02.30}知
    同余方程组的解为
    \begin{align}
        x & \equiv(5\times7\times11)\times1\times1+(3\times7\times11)\times1\times(-1)\nonumber                               \\
          & \qquad+(3\times5\times11)\times2\times2+(3\times5\times7)\times2\times(-2)\pmod{3\times5\times7\times11}\nonumber \\
          & \equiv385-231+660-420\equiv394\pmod{1155}\, .
    \end{align}
\end{example}

至此我们介绍了Halton样本生成中求解一次同余方程组所需的全部背景知识。
代码片{\refcode{Compute Halton sample offset for currentPixel}{}}中
即采用了定理\ref{theorem:7.ex02.30}:其中{\ttfamily dimOffset}对应了$a_j$,
{\ttfamily\refvar{sampleStride}{} / \refvar{baseScales}{}}对应了$M_j$,
{\refvar{multInverse}{}}对应了$M_j^{-1}$,它由扩展欧几里得法求得,代码为
\begin{lstlisting}
static uint64_t `\initvar{multiplicativeInverse}{}`(int64_t a, int64_t n) {
    int64_t x, y;
    `\refvar{extendedGCD}{}`(a, n, &x, &y);
    return `\refvar{Mod}{}`(x, n);
}
\end{lstlisting}
\begin{lstlisting}
static void `\initvar{extendedGCD}{}`(uint64_t a, uint64_t b, int64_t *x, int64_t *y) {
    if (b == 0) {
        *x = 1;
        *y = 0;
        return;
    }
    int64_t d = a / b, xp, yp;
    `\refvar{extendedGCD}{}`(b, a % b, &xp, &yp);
    *x = yp;
    *y = xp - (d * yp);
}
\end{lstlisting}
