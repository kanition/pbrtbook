\section{三角形网格}\label{sec:三角形网格}
\keyindex{三角形}{triangle}{}是计算机图形学最常用的形状;
复杂场景会用上百万三角形建模以实现出色细节
(\reffig{3.11}展示了四百多万三角形的复杂三角网格图像)。
\begin{figure}[htb]
    \centering\includegraphics[width=\linewidth]{chap03/ganesha.png}
    \caption{象头神模型。该三角网格含有四百多万个独立三角形。
        它是使用结构光确定物体形状的3D扫描器用真实雕塑创建的。}
    \label{fig:3.11}
\end{figure}

尽管自然的表示是用{\ttfamily Triangle}形状实现,
每个三角形都存储其三个\keyindex{顶点}{vertex}{}的位置,
但内存更高效的表示是用顶点位置数组分开存储整个\keyindex{三角网格}{triangle mesh}{mesh\ 网格},
每个独立三角形只存储其三个顶点对该数组的三个偏移量。

为了理解该做法,考虑著名的\keyindex{欧拉-庞加莱公式}{Euler-Poincaré formula}{},
它将闭合离散\keyindex{网格}{mesh}{}的顶点数$V$、\keyindex{边}{edge}{}数$E$和\keyindex{面}{face}{}数$F$联系起来:
\begin{align*}
    V-E+F=2(1-g)\, ,
\end{align*}
其中$g\in\mathbb{N}$是网格的\keyindex{亏格}{genus}{}。
亏格通常是很小的数且能解释为网格的“手柄”数量(类似于茶杯把手)。
在三角网格上,边数和面数\sidenote{译者注:原文错写为顶点数,已修改。}还由恒等式联系
\begin{align*}
    E=\frac{3}{2}F\, .
\end{align*}
这可以看作把每条边分成两部分与两个相邻三角形关联。
有$3F$条这样的半边,所有同位的边对构成$E$条网格边。
对于巨大的闭合三角网格,亏格的整体影响通常可以忽略,
我们可以结合之前两个方程(以及$g=0$)得到
\begin{align*}
    F\approx2V\, .
\end{align*}
换句话说,面数大约是顶点数的两倍。
既然每个面引用三个顶点,每个顶点(平均)总共被引用六次。
因此当共享顶点时,每个三角形所需的总分摊存储为
偏移量的12字节内存(三个4字节32位整数偏移量)加上一个顶点存储的一半——6字节,
这里假设每个顶点用三个4字节浮点存储——每个三角形一共18字节。
这比每个三角形直接用36字节存储三个位置好得多。
当网格中有每个顶点的曲面法线或纹理坐标时,相对的存储节约会更好。

pbrt使用结构体\refvar{TriangleMesh}{}保存关于三角网格的共享信息。
\begin{lstlisting}
`\initcode{Triangle Declarations}{=}\initnext{TriangleDeclarations}`
struct `\initvar{TriangleMesh}{}` {
    `\refcode{TriangleMesh Public Methods}{}`
    `\refcode{TriangleMesh Data}{}`
};
\end{lstlisting}
\begin{lstlisting}
`\initcode{TriangleMesh Public Methods}{=}`
`\refvar{TriangleMesh}{}`(const `\refvar{Transform}{}` &ObjectToWorld, int nTriangles, const int *vertexIndices,
    int nVertices, const `\refvar{Point3f}{}` *P, const `\refvar{Vector3f}{}` *S, const `\refvar{Normal3f}{}` *N,
    const `\refvar{Point2f}{}` *uv, const std::shared_ptr<`\refvar{Texture}{}`<`\refvar{Float}{}`>> &alphaMask);
\end{lstlisting}

\refvar{TriangleMesh}{}构造函数的参数如下:
\begin{itemize}
    \item {\ttfamily ObjectToWorld}:网格的物体到世界的变换。
    \item {\ttfamily nTriangles}:网格中三角形总数。
    \item {\ttfamily vertexIndices}:指向顶点索引数组的指针。对于第{\ttfamily i}个三角形,其三个顶点位置为
          {\ttfamily P[vertexIndices[3*i]]}、{\ttfamily P[vertexIndices[3*i+1]]}和{\ttfamily P[vertexIndices[3*i+2]]}。
    \item {\ttfamily nVertices}:网格中顶点总数。
    \item {\ttfamily P}:{\ttfamily nVertices}个顶点位置的数组。
    \item {\ttfamily S}:可选切向量数组,网格中每个顶点都有一个。它们用于计算着色切线。
    \item {\ttfamily N}:可选法向量数组,网格中每个顶点都有一个。如果有,则它们在三角形面之间插值以计算着色法线。
    \item {\ttfamily UV}:可选参数值$(u,v)$数组,每个顶点一个。
    \item {\ttfamily alphaMask}:可选的\keyindex{$\alpha$掩模}{alpha mask}{}纹理,可用于截去部分三角形面。
\end{itemize}

pbrt中三角形在形状里有双重角色:不仅场景描述文件经常直接指定它们,
其他形状也常把自己细分为三角网格。
例如,细分曲面最终创建一个三角网格来近似光滑有限曲面。
再对这些三角形执行光线相交,而不是直接对细分曲面这样做(\refsub{细分})。

由于这第二种角色,创建三角网格的代码能指定三角形的参数化很重要。
如果三角形是通过求取参数化曲面在三个特定坐标值$(u,v)$处的位置来创建的,
这些$(u,v)$值应该被插值以计算三角形内光线交点处的$(u,v)$值。
显式指定$(u,v)$值对纹理贴图也很有用,创建三角网格的外部程序
可能想给网格赋值$(u,v)$坐标,这样纹理贴图以期望的方式把颜色赋值给网格曲面。

\refvar{TriangleMesh}{}构造函数赋值相关信息并存于成员变量中。
特别地,它自己复制{\ttfamily vertexIndices、P、N、S}和{\ttfamily UV},
允许调用者保留对传入数据的所有权。
\begin{lstlisting}
`\initcode{Triangle Method Definitions}{=}\initnext{TriangleMethodDefinitions}`
`\refvar{TriangleMesh}{}`::`\refvar{TriangleMesh}{}`(const `\refvar{Transform}{}` &ObjectToWorld,
        int nTriangles, const int *vertexIndices, int nVertices,
        const `\refvar{Point3f}{}` *P, const `\refvar{Vector3f}{}` *S, const `\refvar{Normal3f}{}` *N,
        const `\refvar{Point3f}{}` *UV,
        const std::shared_ptr<`\refvar{Texture}{}`<`\refvar{Float}{}`>> &alphaMask)
    : `\refvar{nTriangles}{}`(nTriangles), `\refvar{nVertices}{}`(nVertices), 
      `\refvar{vertexIndices}{}`(vertexIndices, vertexIndices + 3 * nTriangles),
      `\refvar{alphaMask}{}`(alphaMask) {
    `\refcode{Transform mesh vertices to world space}{}`
    `\refcode{Copy UV, N, and S vertex data, if present}{}`
}
\end{lstlisting}
\begin{lstlisting}
    `\initcode{TriangleMesh Data}{=}`
    const int `\initvar{nTriangles}{}`, `\initvar{nVertices}{}`;
    std::vector<int> `\initvar{vertexIndices}{}`;
    std::unique_ptr<`\refvar{Point3f}{}`[]> `\initvar[TriangleMesh::p]{p}{}`;
    std::unique_ptr<`\refvar{Normal3f}{}`[]> `\initvar[TriangleMesh::n]{n}{}`;
    std::unique_ptr<`\refvar{Vector3f}{}`[]> `\initvar[TriangleMesh::s]{s}{}`;
    std::unique_ptr<`\refvar{Point2f}{}`[]> `\initvar[TriangleMesh::uv]{uv}{}`;
    std::shared_ptr<`\refvar{Texture}{}`<`\refvar{Float}{}`>> `\initvar{alphaMask}{}`;
\end{lstlisting}

不像其他形状那样把形状描述留在物体空间内然后
将入射光线从世界空间变换到物体空间,
三角网格将形状变换到世界空间,
因此节约了把入射光线变换到物体空间的工作和
把相交处的几何表示变换出世界空间的工作。
这是个好主意,因为该操作可在启动后执行,
避免渲染中多次变换光线。
尽管也能对二次曲面使用该方法,但会更复杂(见本章末习题7了解更多信息)。
\begin{lstlisting}
`\initcode{Transform mesh vertices to world space}{=}`
`\refvar[TriangleMesh::p]{p}{}`.reset(new `\refvar{Point3f}{}`[`\refvar{nVertices}{}`]);
for (int i = 0; i < `\refvar{nVertices}{}`; ++i)
    `\refvar[TriangleMesh::p]{p}{}`[i] = ObjectToWorld(P[i]);
\end{lstlisting}

代码片\refcode{Copy UV, N, and S vertex data, if present}{}只是
分配合适数量的空间并复制合适的值。
如果有法线和切向量,则也变换到世界空间。
\begin{lstlisting}
`\initcode{Copy UV, N, and S vertex data, if present}{=}`
if (UV) {
    `\refvar[TriangleMesh::uv]{uv}{}`.reset(new `\refvar{Point2f}{}`[`\refvar{nVertices}{}`]);
    memcpy(`\refvar[TriangleMesh::uv]{uv}{}`.get(), UV, `\refvar{nVertices}{}` * sizeof(`\refvar{Point2f}{}`));
}
if (N) {
    `\refvar[TriangleMesh::n]{n}{}`.reset(new `\refvar{Normal3f}{}`[`\refvar{nVertices}{}`]);
    for (int i = 0; i < `\refvar{nVertices}{}`; ++i)
        `\refvar[TriangleMesh::n]{n}{}`[i] = ObjectToWorld(N[i]);
}
if (S) {
    `\refvar[TriangleMesh::s]{s}{}`.reset(new `\refvar{Vector3f}{}`[`\refvar{nVertices}{}`]);
    for (int i = 0; i < `\refvar{nVertices}{}`; ++i)
        `\refvar[TriangleMesh::s]{s}{}`[i] = ObjectToWorld(S[i]);
}
\end{lstlisting}

\subsection{三角形}\label{sub:三角形}
类\refvar{Triangle}{}实际上实现了\refvar{Shape}{}接口。它表示单个三角形。
\begin{lstlisting}
`\refcode{Triangle Declarations}{+=}\lastcode{TriangleDeclarations}`
class `\initvar{Triangle}{}` : public `\refvar{Shape}{}` {
public:
    `\refcode{Triangle Public Methods}{}`
private:
    `\refcode{Triangle Private Methods}{}`
    `\refcode{Triangle Private Data}{}`
};
\end{lstlisting}

\refvar{Triangle}{}不存储太多数据——只有一个指向其来处的父\refvar{TriangleMesh}{}指针
和一个指向网格中其三个顶点索引的指针。
\begin{lstlisting}
`\initcode{Triangle Public Methods}{=}`
`\refvar{Triangle}{}`(const `\refvar{Transform}{}` *ObjectToWorld, const `\refvar{Transform}{}` *WorldToObject,
         bool reverseOrientation,
         const std::shared_ptr<`\refvar{TriangleMesh}{}`> &mesh, int triNumber)
    : `\refvar{Shape}{}`(ObjectToWorld, WorldToObject, reverseOrientation),
      `\refvar{mesh}{}`(mesh) {
    `\refvar[Triangle::v]{v}{}` = &mesh->`\refvar{vertexIndices}{}`[3 * triNumber];
}
\end{lstlisting}

注意该实现存储了指向第一个顶点\emph{索引}的指针,
而非存储三个指向顶点自己的指针。
这以多一级间接性为代价减少了每个\refvar{Triangle}{}所需的存储量。
\begin{lstlisting}
`\initcode{Triangle Private Data}{=}`
std::shared_ptr<`\refvar{TriangleMesh}{}`> `\initvar{mesh}{}`;
const int *`\initvar[Triangle::v]{v}{}`;
\end{lstlisting}

因为pbrt中大量的其他形状表示会把自己转化为三角网格,
实用函数
\refvar{CreateTriangleMesh}{()}仔细创建
底层\refvar{TriangleMesh}{}以及网格中每个三角形的\refvar{Triangle}{}。
它返回三角形状的向量。
\begin{lstlisting}
`\refcode{Triangle Method Definitions}{+=}\lastnext{TriangleMethodDefinitions}`
std::vector<std::shared_ptr<`\refvar{Shape}{}`>> `\initvar{CreateTriangleMesh}{}`(
        const `\refvar{Transform}{}` *ObjectToWorld, const `\refvar{Transform}{}` *WorldToObject,
        bool reverseOrientation, int nTriangles,
        const int *vertexIndices, int nVertices, const `\refvar{Point3f}{}` *p,
        const `\refvar{Vector3f}{}` *s, const `\refvar{Normal3f}{}` *n, const `\refvar{Point2f}{}` *uv,
        const std::shared_ptr<`\refvar{Texture}{}`<`\refvar{Float}{}`>> &alphaMask) {
    std::shared_ptr<`\refvar{TriangleMesh}{}`> mesh = std::make_shared<`\refvar{TriangleMesh}{}`>(
        *ObjectToWorld, nTriangles, vertexIndices, nVertices, p, s, n, uv,
        alphaMask);
    std::vector<std::shared_ptr<`\refvar{Shape}{}`>> tris;
    for (int i = 0; i < nTriangles; ++i)
        tris.push_back(std::make_shared<`\refvar{Triangle}{}`>(ObjectToWorld,
            WorldToObject, reverseOrientation, mesh, i));
    return tris;
}
\end{lstlisting}

三角形的物体空间边界很容易通过计算包围其三个顶点的边界框求得。
因为顶点位置\refvar[TriangleMesh::p]{p}{}在构造函数中被变换到世界空间,
这里的实现要在计算其边界前把它们变换回物体空间。
\begin{lstlisting}
`\refcode{Triangle Method Definitions}{+=}\lastnext{TriangleMethodDefinitions}`
`\refvar{Bounds3f}{}` `\refvar{Triangle}{}`::`\initvar[Triangle::ObjectBound]{\refvar{ObjectBound}{}}{}`() const {
    `\refcode{Get triangle vertices in p0, p1, and p2}{}`
    return `\refvar[Union1]{Union}{}`(`\refvar{Bounds3f}{}`((*`\refvar{WorldToObject}{}`)(p0), (*`\refvar{WorldToObject}{}`)(p1)),
                 (*`\refvar{WorldToObject}{}`)(p2));
}
\end{lstlisting}
\begin{lstlisting}
`\initcode{Get triangle vertices in p0, p1, and p2}{=}`
const `\refvar{Point3f}{}` &p0 = `\refvar{mesh}{}`->`\refvar[TriangleMesh::p]{p}{}`[`\refvar[Triangle::v]{v}{}`[0]];
const `\refvar{Point3f}{}` &p1 = `\refvar{mesh}{}`->`\refvar[TriangleMesh::p]{p}{}`[`\refvar[Triangle::v]{v}{}`[1]];
const `\refvar{Point3f}{}` &p2 = `\refvar{mesh}{}`->`\refvar[TriangleMesh::p]{p}{}`[`\refvar[Triangle::v]{v}{}`[2]];
\end{lstlisting}

比起变换其物体空间边界框到世界空间,
\refvar{Triangle}{}形状是能计算更好的世界空间边界的形状之一。
其世界空间边界可以直接从世界空间顶点算得。
\begin{lstlisting}
`\refcode{Triangle Method Definitions}{+=}\lastnext{TriangleMethodDefinitions}`
`\refvar{Bounds3f}{}` `\refvar{Triangle}{}`::`\initvar[Triangle::WorldBound]{\refvar[Shape::WorldBound]{WorldBound}{}}{}`() const {
    `\refcode{Get triangle vertices in p0, p1, and p2}{}`
    return `\refvar[Union1]{Union}{}`(`\refvar{Bounds3f}{}`(p0, p1), p2); 
}
\end{lstlisting}

\subsection{三角形相交}\label{sub:三角形相交}
三角形状方法\refvar[Triangle::Intersect]{Intersect}{()}的结构
遵循之前的相交测试方法:应用几何测试以确定是否相交,
如果有,则计算更多关于相交的信息,并在给定的
\refvar{SurfaceInteraction}{}中返回。
\begin{lstlisting}
`\refcode{Triangle Method Definitions}{+=}\lastnext{TriangleMethodDefinitions}`
bool `\initvar{Triangle::Intersect}{}`(const `\refvar{Ray}{}` &ray, `\refvar{Float}{}` *tHit,
        `\refvar{SurfaceInteraction}{}` *isect, bool testAlphaTexture) const {
    `\refcode{Get triangle vertices in p0, p1, and p2}{}`
    `\refcode{Perform ray-triangle intersection test}{}`
    `\refcode{Compute triangle partial derivatives}{}`
    `\refcode{Compute error bounds for triangle intersection}{}`
    `\refcode{Interpolate (u,v) parametric coordinates and hit point}{}`
    `\refcode{Test intersection against alpha texture, if present}{}`
    `\refcode{Fill in SurfaceInteraction from triangle hit}{}`
    *tHit = t;
    return true;
}
\end{lstlisting}

pbrt的光线-三角形相交测试基于首先计算仿射变换
使得射线在变换后的坐标系里端点被变换为$(0,0,0)$且其方向沿$+z$轴。
在执行相机测试之前三角形顶点也变换到该坐标系。
下面我们将看到应用该坐标系变换会简化相交测试逻辑,
例如任何交点的$x$和$y$坐标必须为零。
之后在\refsub{稳定的三角形相交},我们会看到该变换让
拥有\keyindex{水密的}{watertight}{}光线-三角形相交算法成为可能,
这样刚好命中三角形边缘的棘手光线等的相交就永远不会被错误地报告为未命中了。
\begin{lstlisting}
`\initcode{Perform ray-triangle intersection test}{=}`
`\refcode{Transform triangle vertices to ray coordinate space}{}`
`\refcode{Compute edge function coefficients e0, e1, and e2}{}`
`\refcode{Fall back to double-precision test at triangle edges}{}`
`\refcode{Perform triangle edge and determinant tests}{}`
`\refcode{Compute scaled hit distance to triangle and test against ray t range}{}`
`\refcode{Compute barycentric coordinates and t value for triangle intersection}{}`
`\refcode{Ensure that computed triangle t is conservatively greater than zero}{}`
\end{lstlisting}

计算从世界空间到光线-三角形相交坐标空间的变换有三步:
平移$\bm T$、坐标\keyindex{置换}{permutation}{}$\bm P$和\keyindex{错切}{shear}{}$\bm S$.
下列实现没有为它们每个都计算显式的变换矩阵
再计算合成变换矩阵$\bm M=\bm S\bm P\bm T$来把顶点变换到坐标空间,
而是直接施加每步的变换,这最终是更高效的方法。
\begin{lstlisting}
`\initcode{Transform triangle vertices to ray coordinate space}{=}`
`\refcode{Translate vertices based on ray origin}{}`
`\refcode{Permute components of triangle vertices and ray direction}{}`
`\refcode{Apply shear transformation to translated vertex positions}{}`
\end{lstlisting}

将射线端点放置在坐标系的原点的平移是:
\begin{align*}
    \bm T=\left[\begin{array}{cccc}
            1 & 0 & 0 & -o_x \\
            0 & 1 & 0 & -o_y \\
            0 & 0 & 1 & -o_z \\
            0 & 0 & 0 & 1
        \end{array}\right]\, .
\end{align*}
该变换不需要显式施加到射线端点,但我们会将其施加于三角形三个顶点。
\begin{lstlisting}
`\initcode{Translate vertices based on ray origin}{=}`
`\refvar{Point3f}{}` p0t = p0 - `\refvar{Vector3f}{}`(ray.o);
`\refvar{Point3f}{}` p1t = p1 - `\refvar{Vector3f}{}`(ray.o);
`\refvar{Point3f}{}` p2t = p2 - `\refvar{Vector3f}{}`(ray.o);
\end{lstlisting}

接下来,置换空间三个维度使得$z$维的射线方向绝对值最大。
$x$和$y$维任意分配给另外两维。
例如,这步保证如果原始射线的$z$方向为零,
则非零幅度的维度映射到$+z$.

例如,如果光线方向在$x$有最大幅值,则置换为:
\begin{align*}
    \bm P=\left[\begin{array}{cccc}
            0 & 1 & 0 & 0 \\
            0 & 0 & 1 & 0 \\
            1 & 0 & 0 & 0 \\
            0 & 0 & 0 & 1
        \end{array}\right]\, .
\end{align*}

像之前那样,直接置换射线方向维度和平移后的三角形顶点最简单。
\begin{lstlisting}
`\initcode{Permute components of triangle vertices and ray direction}{=}`
int kz = `\refvar{MaxDimension}{}`(`\refvar[Vector3::Abs]{Abs}{}`(ray.d));
int kx = kz + 1; if (kx == 3) kx = 0;
int ky = kx + 1; if (ky == 3) ky = 0;
`\refvar{Vector3f}{}` d = `\refvar[Vector3::Permute]{Permute}{}`(ray.d, kx, ky, kz);
p0t = `\refvar[Point3::Permute]{Permute}{}`(p0t, kx, ky, kz);
p1t = `\refvar[Point3::Permute]{Permute}{}`(p1t, kx, ky, kz);
p2t = `\refvar[Point3::Permute]{Permute}{}`(p2t, kx, ky, kz);
\end{lstlisting}

最后,错切变换将射线方向与$+z$轴对齐:
\begin{align*}
    \bm S=\left[\begin{array}{rrrr}
            1 & 0 & \displaystyle-\frac{d_x}{d_z} & 0 \\
            0 & 1 & \displaystyle-\frac{d_y}{d_z} & 0 \\
            0 & 0 & \displaystyle\frac{1}{d_z}    & 0 \\
            0 & 0 & 0                             & 1
        \end{array}\right]\, .
\end{align*}
为了理解该变换如何工作,可考虑它对射线方向向量$[d_x\ d_y\ d_z\ 0]^{\mathrm{T}}$的操作。

现在,只有$x$和$y$维被错切;我们可以等到只有当光线确实与三角形相交时再错切$z$维。
\begin{lstlisting}
`\initcode{Apply shear transformation to translated vertex positions}{=}`
`\refvar{Float}{}` Sx = -d.x / d.z;
`\refvar{Float}{}` Sy = -d.y / d.z;
`\refvar{Float}{}` Sz = 1.f / d.z;
p0t.x += Sx * p0t.z;
p0t.y += Sy * p0t.z;
p1t.x += Sx * p1t.z;
p1t.y += Sy * p1t.z;
p2t.x += Sx * p2t.z;
p2t.y += Sy * p2t.z;
\end{lstlisting}

注意坐标置换和错切系数的计算只取决于给定的光线;
它们和三角形无关。在高性能光线追踪器里,
我们可能想一次性计算这些值并保存在类\refvar{Ray}{}里,
而不是为与光线相交的每个三角形重新计算它们。

三角形顶点变换到该坐标系后,我们现在的任务是求从原点出发
并沿$+z$的光线是否与变换后的三角形相交。
因为构造坐标系的方式,该问题等价于2D问题即
确定$x,y$坐标$(0,0)$是否在三角形的$xy$投影内(\reffig{3.12})。
\begin{figure}[htbp]
    \centering\input{Pictures/chap03/Triisectcoordinatesystem.tex}
    \caption{在光线-三角形相交坐标系中,从原点出发的光线沿$+z$轴射出。
        可通过只考虑光线和三角形顶点的$xy$投影执行相交测试,
        转而简化为确定2D点$(0,0)$是否在三角形内。}
    \label{fig:3.12}
\end{figure}

为了理解相交算法如何工作,首先回想\reffig{2.5}中两个向量
叉积的长度给出了它们定义的平行四边形面积。
在2D中,向量为$\bm a$和$\bm b$,则面积为
\begin{align*}
    a_xb_y-b_xa_y\, .
\end{align*}
该面积的一半是它们定义的三角形面积。因此我们看到在2D中,
顶点为$\bm p_0,\bm p_1$和$\bm p_2$的三角形面积为
\begin{align*}
    \frac{1}{2}((p_{1x}-p_{0x})(p_{2y}-p_{0y})-(p_{2x}-p_{0x})(p_{1y}-p_{0y}))\, .
\end{align*}
\reffig{3.13}以几何方式将其可视化。
\begin{figure}[htbp]
    \centering\input{Pictures/chap03/Triarea.tex}
    \caption{两边由向量$\bm v_1$和$\bm v_2$给定的三角形的面积
        是这里展示的平行四边形面积的一半。
        平行四边形面积由$\bm v_1$和$\bm v_2$的叉积长度给出。}
    \label{fig:3.13}
\end{figure}

我们将用三角形面积的该表达式定义有符号的\keyindex{边函数}{edge function}{}:
给定两个三角形顶点$\bm p_0$和$\bm p_1$,
则我们可以定义定向边函数$e$,它给出由$\bm p_0$和$\bm p_1$以及
给定的第三点$\bm p$指定的三角形面积的两倍(见\reffig{3.14})
\sidenote{译者注:原图符号和题注写反了,与正文矛盾,已修正。
    实际上这和$x,y$轴的排布有关,不论哪边取正,只要统一标准就行。}:
\begin{align}\label{eq:3.1}
    e(\bm p)=(p_{1x}-p_{0x})(p_y-p_{0y})-(p_x-p_{0x})(p_{1y}-p_{0y})\, .
\end{align}
\begin{figure}[htbp]
    \centering\input{Pictures/chap03/Triedgefunction.tex}
    \caption{边函数$e(\bm p)$用相对于两点$\bm p_0$和$\bm p_1$间的有向直线表征点。
        对于该直线左边的点$\bm p$边函数值为正,直线上的点为零,直线右边的点为负。
        光线-三角形相交算法使用的边函数是三点构造的三角形有符号面积的两倍。}
    \label{fig:3.14}
\end{figure}

对于直线左边的点边函数为正值,右边的点为负值。
因此若某点的边函数值对三角形三条边有相同符号,
它一定在三条边的同侧并因此必在三角形内。

得益于坐标系变换,我们要测试的点$\bm p$是$(0,0)$.
这简化了边函数表达式。例如,对于从$\bm p_1$到$\bm p_2$的边$e_0$,我们有:
\begin{align}\label{eq:3.2}
    e_0(\bm p) & =(p_{2x}-p_{1x})(p_y-p_{1y})-(p_x-p_{1x})(p_{2y}-p_{1y})\nonumber \\
               & =(p_{2x}-p_{1x})(-p_{1y})-(-p_{1x})(p_{2y}-p_{1y})\nonumber       \\
               & =p_{1x}p_{2y}-p_{2x}p_{1y}\, .
\end{align}

下面,我们将使用以边函数$e_i$对应
从顶点$\bm p_{(i+1)\mod{3}}$到$\bm p_{(i+2)\mod{3}}$的有向边的索引方案。
\begin{lstlisting}
`\initcode{Compute edge function coefficients e0, e1, and e2}{=}`
`\refvar{Float}{}` e0 = p1t.x * p2t.y - p1t.y * p2t.x;
`\refvar{Float}{}` e1 = p2t.x * p0t.y - p2t.y * p0t.x;
`\refvar{Float}{}` e2 = p0t.x * p1t.y - p0t.y * p1t.x;
\end{lstlisting}

在任一边函数值恰好为零的罕见情况下,
不可能确定光线是否命中了三角形,则用双精度浮点算术重新计算边方程。
(\refsub{稳定的三角形相交}更详细地讨论了对这步的需求。)
实现该计算的代码片\refcode{Fall back to double-precision test at triangle edges}{}只是
用{\ttfamily double}再次实现\refcode{Compute edge function coefficients e0, e1, and e2}{}。
\begin{lstlisting}
`\initcode{Fall back to double-precision test at triangle edges}{=}`
if (sizeof(`\refvar{Float}{}`) == sizeof(float) &&
(e0 == 0.0f || e1 == 0.0f || e2 == 0.0f)) {
    double p2txp1ty = (double)p2t.x * (double)p1t.y;
    double p2typ1tx = (double)p2t.y * (double)p1t.x;
    e0 = (float)(p2typ1tx - p2txp1ty);
    double p0txp2ty = (double)p0t.x * (double)p2t.y;
    double p0typ2tx = (double)p0t.y * (double)p2t.x;
    e1 = (float)(p0typ2tx - p0txp2ty);
    double p1txp0ty = (double)p1t.x * (double)p0t.y;
    double p1typ0tx = (double)p1t.y * (double)p0t.x;
    e2 = (float)(p1typ0tx - p1txp0ty);
}
\end{lstlisting}

给定三个边函数的值,我们有前两次机会确定没有相交。
首先,如果边函数的符号不同,则点$(0,0)$不在三条边的同侧,因此在三角形外。
然后,如果三个边函数值的和为零,则光线接近三角形的边缘
\sidenote{译者注:提示:因为此时三个边函数同号,如果和接近零,
    则它们必然分别也接近零。},
我们报告没有相交(对于闭合三角网格,光线则是命中相邻三角形)。
\begin{lstlisting}
`\initcode{Perform triangle edge and determinant tests}{=}`
if ((e0 < 0 || e1 < 0 || e2 < 0) && (e0 > 0 || e1 > 0 || e2 > 0))
    return false;
`\refvar{Float}{}` det = e0 + e1 + e2;
if (det == 0)
    return false;
\end{lstlisting}

因为光线从原点出发,有单位长度,且沿$+z$轴,交点的$z$坐标值等于相交处的参数$t$值。
为了计算该$z$值,我们先要继续对三角形顶点的$z$坐标施加错切变换。
有了这些值后,三角形中交点的\keyindex{重心坐标}{barycentric coordinate}{coordinate\ 坐标}可
用于在三角形间对其插值。它们由每个边函数值除以边函数值之和得到
\sidenote{译者注:提示:$b_i$表示了原点接近$\bm p_i$的程度。
    利用向量知识可以证明,2D中对于包围了点$O$(不在边缘上)的$\triangle P_0P_1P_2$,
    存在比例固定的非零系数$e_i$使得$e_0\overrightarrow{OP_0}+e_1\overrightarrow{OP_1}+e_2\overrightarrow{OP_2}=\bm 0$,
    且三角形面积满足$\displaystyle\frac{|e_0|}{S_{\triangle OP_1P_2}}=\frac{|e_1|}{S_{\triangle OP_2P_0}}=\frac{|e_2|}{S_{\triangle OP_0P_1}}$.}:
\begin{align*}
    b_i=\frac{e_i}{e_0+e_1+e_2}\, .
\end{align*}
因此$b_i$和为一。

插值的$z$值为
\begin{align*}
    z=b_0z_0+b_1z_1+b_2z_2\, ,
\end{align*}
其中$z_i$是光线-三角形相交坐标系中三个顶点的坐标。

为了节约计算$b_i$的浮点除法开销避免最后的$t$值超出有效$t$值范围,
这里的实现先用$e_i$插值$z_i$来计算$t$
(换句话说,还没执行除以$d=e_0+e_1+e_2$)。
若$d$的符号和插值的$t$的符号不同,
则最后的$t$值一定为负,因此不是有效相交。

同样,检查$t<t_{\max}$可用两种方式等价执行:
\begin{align*}
    \sum\limits_i{e_iz_i} & <t_{\max}(e_0+e_1+e_2)\quad\text{如果}e_0+e_1+e_2>0\, , \\
    \sum\limits_i{e_iz_i} & >t_{\max}(e_0+e_1+e_2)\quad\text{其他}\, .
\end{align*}
\begin{lstlisting}
`\initcode{Compute scaled hit distance to triangle and test against ray t range}{=}`
p0t.z *= Sz;
p1t.z *= Sz;
p2t.z *= Sz;
`\refvar{Float}{}` tScaled = e0 * p0t.z + e1 * p1t.z + e2 * p2t.z;
if (det < 0 && (tScaled >= 0 || tScaled < ray.tMax * det))
    return false;
else if (det > 0 && (tScaled <= 0 || tScaled > ray.tMax * det))
    return false;
\end{lstlisting}

我们现在知道有一处有效相交并继续花费浮点除法开销为相交处
计算实际的重心坐标以及实际的$t$值。
\begin{lstlisting}
`\initcode{Compute barycentric coordinates and t value for triangle intersection}{=}`
`\refvar{Float}{}` invDet = 1 / det;
`\refvar{Float}{}` b0 = e0 * invDet;
`\refvar{Float}{}` b1 = e1 * invDet;
`\refvar{Float}{}` b2 = e2 * invDet;
`\refvar{Float}{}` t = tScaled * invDet;
\end{lstlisting}

为了在三角网格上生成一致的切向量,如果提供了的话,
有必要用参数值$(u,v)$计算三角形顶点处的偏导数
$\displaystyle\frac{\partial \bm p}{\partial u}$和
$\displaystyle\frac{\partial \bm p}{\partial v}$.
尽管三角形上所有点的偏导数都一样,这里的实现
还是每次找到相交处就重算它们。
尽管这导致多余的计算,但对于大型三角网格节约的存储是显著的。

三角形可以描述为点集
\begin{align*}
    \bm p_{\mathrm{o}}+u\frac{\partial \bm p}{\partial u}+v\frac{\partial \bm p}{\partial v}\, ,
\end{align*}
对于某点$\bm p_{\mathrm{o}}$,$u$和$v$在三角形参数坐标范围内变动。
我们还知道三个顶点位置$\bm p_i, i=0,1,2$,
以及每个顶点的纹理坐标$(u_i,v_i)$.
由此推出$\bm p$的偏导数一定满足
\begin{align*}
    \bm p_i=\bm p_{\mathrm{o}}+u_i\frac{\partial \bm p}{\partial u}+v_i\frac{\partial \bm p}{\partial v}\, .
\end{align*}
换句话说,从2D$(u,v)$空间到三角形上的点有唯一的仿射映射
(即使三角形是在3D空间指定的,该映射也存在,因为三角形是平面的)。
为计算$\displaystyle\frac{\partial \bm p}{\partial u}$和
$\displaystyle\frac{\partial \bm p}{\partial v}$的表达式,
我们从作差$\bm p_0-\bm p_2$和$\bm p_1-\bm p_2$开始,得到矩阵方程
\sidenote{译者注:这里的点坐标按行向量处理。}
\begin{align*}
    \left[\begin{array}{cc}
            u_0-u_2 & v_0-v_2 \\
            u_1-u_2 & v_1-v_2
        \end{array}\right]\left[\begin{array}{c}
            \displaystyle\frac{\partial \bm p}{\partial u} \\
            \displaystyle\frac{\partial \bm p}{\partial v}
        \end{array}\right]=\left[\begin{array}{c}
            \bm p_0-\bm p_2 \\
            \bm p_1-\bm p_2
        \end{array}\right]\, .
\end{align*}
因此
\begin{align*}
    \left[\begin{array}{c}
            \displaystyle\frac{\partial \bm p}{\partial u} \\
            \displaystyle\frac{\partial \bm p}{\partial v}
        \end{array}\right]=\left[\begin{array}{cc}
            u_0-u_2 & v_0-v_2 \\
            u_1-u_2 & v_1-v_2
        \end{array}\right]^{-1}\left[\begin{array}{c}
            \bm p_0-\bm p_2 \\
            \bm p_1-\bm p_2
        \end{array}\right]\, .
\end{align*}
$2\times2$矩阵求逆很简单。$(u,v)$作差矩阵的逆为
\begin{align*}
    \frac{1}{(u_0-u_2)(v_1-v_2)-(v_0-v_2)(u_1-u_2)}\left[\begin{array}{cc}
            v_1-v_2    & -(v_0-v_2) \\
            -(u_1-u_2) & u_0-u_2
        \end{array}\right]\, .
\end{align*}
\begin{lstlisting}
`\initcode{Compute triangle partial derivatives}{=}`
`\refvar{Vector3f}{}` dpdu, dpdv;
`\refvar{Point2f}{}` uv[3];
`\refvar{GetUVs}{}`(uv);
`\refcode{Compute deltas for triangle partial derivatives}{}`
`\refvar{Float}{}` determinant = duv02[0] * duv12[1] - duv02[1] * duv12[0];
if (determinant == 0) {
    `\refcode{Handle zero determinant for triangle partial derivative matrix}{}`
} else {
    `\refvar{Float}{}` invdet = 1 / determinant;
    dpdu = ( duv12[1] * dp02 - duv02[1] * dp12) * invdet;
    dpdv = (-duv12[0] * dp02 + duv02[0] * dp12) * invdet;
}
\end{lstlisting}
\begin{lstlisting}
`\initcode{Compute deltas for triangle partial derivatives}{=}`
`\refvar{Vector2f}{}` duv02 = uv[0] - uv[2], duv12 = uv[1] - uv[2];
`\refvar{Vector3f}{}` dp02 = p0 - p2, dp12 = p1 - p2;
\end{lstlisting}

最后,有必要处理矩阵是奇异的而不能求逆的情况。
注意这只在用户提供的每个顶点参数化值是退化的时候发生。
这种情况下,\refvar{Triangle}{}只选择
关于三角形曲面法线的任一坐标系,确保它是标准正交的:
\begin{lstlisting}
`\initcode{Handle zero determinant for triangle partial derivative matrix}{=}`
`\refvar{CoordinateSystem}{}`(`\refvar{Normalize}{}`(`\refvar{Cross}{}`(p2 - p0, p1 - p0)), &dpdu, &dpdv);
\end{lstlisting}

为了计算交点和命中点的$(u,v)$参数坐标,
将重心插值公式用于顶点位置和顶点的$(u,v)$坐标。
我们将在\refsub{定界交点误差}看到,
比起用{\ttfamily t}代入参数射线方程,
它会给出交点更加精确的结果。
\begin{lstlisting}
`\initcode{Interpolate (u,v) parametric coordinates and hit point}{=}`
`\refvar{Point3f}{}` pHit = b0 * p0 + b1 * p1 + b2 * p2;
`\refvar{Point2f}{}` uvHit = b0 * uv[0] + b1 * uv[1] + b2 * uv[2];
\end{lstlisting}

实用例程\refvar{GetUVs}{()}返回三角形三个顶点的$(u,v)$坐标,
如果有的话就从
\refvar{Triangle}{}取,
否则如果网格没有指定显式$(u,v)$坐标就返回默认值。
\begin{lstlisting}
`\initcode{Triangle Private Methods}{=}`
void `\initvar{GetUVs}{}`(`\refvar{Point2f}{}` uv[3]) const {
    if (`\refvar{mesh}{}`->`\refvar[TriangleMesh::uv]{uv}{}`) {
        uv[0] = `\refvar{mesh}{}`->`\refvar[TriangleMesh::uv]{uv}{}`[`\refvar[Triangle::v]{v}{}`[0]];
        uv[1] = `\refvar{mesh}{}`->`\refvar[TriangleMesh::uv]{uv}{}`[`\refvar[Triangle::v]{v}{}`[1]];
        uv[2] = `\refvar{mesh}{}`->`\refvar[TriangleMesh::uv]{uv}{}`[`\refvar[Triangle::v]{v}{}`[2]];
    } else {
        uv[0] = `\refvar{Point2f}{}`(0, 0);
        uv[1] = `\refvar{Point2f}{}`(1, 0);
        uv[2] = `\refvar{Point2f}{}`(1, 1);
    }
}
\end{lstlisting}

在报告成功相交前,如果形状被赋值了$\alpha$掩模纹理,则用它测试交点。
该纹理可看作在三角形曲面上的1D函数,
在任何取值为零的点处都忽略相交,
高效地将三角形上的点当做不存在。
(第\refchap{纹理}更详细地定义了纹理接口和实现。)
$\alpha$掩模对于表示像叶子那样的物体很有用:
一片叶子可以建模为单个三角形,
$\alpha$掩模“裁去”边缘使之保留叶子形状。
该功能对其他形状没多大用,所以pbrt只为三角形支持。
\begin{lstlisting}
`\initcode{Test intersection against alpha texture, if present}{=}`
if (testAlphaTexture && `\refvar{mesh}{}`->`\refvar{alphaMask}{}`) {
    `\refvar{SurfaceInteraction}{}` isectLocal(pHit, `\refvar{Vector3f}{}`(0,0,0), uvHit,
        `\refvar{Vector3f}{}`(0,0,0), dpdu, dpdv, `\refvar{Normal3f}{}`(0,0,0), `\refvar{Normal3f}{}`(0,0,0),
        ray.`\refvar[Ray::time]{time}{}`, this);
    if (`\refvar{mesh}{}`->`\refvar{alphaMask}{}`->`\refvar[Texture::Evaluate]{Evaluate}{}`(isectLocal) == 0)
        return false;
}
\end{lstlisting}
现在我们一定有有效相交,可以更新传入相交例程的指针所指的值了。
不像其他形状的实现,这里初始化结构\refvar{SurfaceInteraction}{}的
代码不需要将偏导数变换到世界空间,
因为三角形的顶点已经变换到世界空间了。
像圆盘那样,三角形法线的偏导数也都是$(0,0,0)$,因为它是平的。
\begin{lstlisting}
`\initcode{Fill in SurfaceInteraction from triangle hit}{}`
*isect = `\refvar{SurfaceInteraction}{}`(pHit, pError, uvHit, -ray.`\refvar[Ray::d]{d}{}`, dpdu, dpdv,
    `\refvar{Normal3f}{}`(0, 0, 0), `\refvar{Normal3f}{}`(0, 0, 0), ray.`\refvar[Ray::time]{time}{}`, this);
`\refcode{Override surface normal in isect for triangle}{}`
if (`\refvar{mesh}{}`->`\refvar[TriangleMesh::n]{n}{}` || `\refvar{mesh}{}`->`\refvar[TriangleMesh::s]{s}{}`) {
    `\refcode{Initialize Triangle shading geometry}{}`
}
`\refcode{Ensure correct orientation of the geometric normal}{}`
\end{lstlisting}

\refvar{SurfaceInteraction}{}构造函数初始化几何法线{\ttfamily n}为
{\ttfamily dpdu}和{\ttfamily dpdv}的规范化叉积。
这对大多数形状都适用,但在三角网格的情况下
使用不依赖于底层纹理坐标的初始化更好:
遇到不良参数化即没有保全朝向的网格是很常见的,
这时几何法线可能有错误方向。

因此我们用边向量{\ttfamily dp02}和{\ttfamily dp12}的
规范化叉积初始化几何法线,它得到相同的法线要取决于由三角形顶点确切顺序
(也即三角形的\keyindex{缠绕顺序}{winding order}{})决定的可能的符号差。
3D建模包通常尽力确保网格中的三角形有一致的缠绕顺序,这让本方法更稳定。
\begin{lstlisting}
`\initcode{Override surface normal in isect for triangle}{=}`
isect->`\refvar[Interaction::n]{n}{}` = isect->`\refvar{shading}{}`.`\refvar[shading::n]{n}{}` = `\refvar{Normal3f}{}`(`\refvar{Normalize}{}`(`\refvar{Cross}{}`(dp02, dp12)));
\end{lstlisting}

当插值的法线可用时,我们将它们考虑为朝向信息最权威的来源。
这种情况下,如果{\ttfamily isect->\refvar[Interaction::n]{n}{}}和
插值的法线夹角大于90度,我们就翻转它的朝向。
\begin{lstlisting}
`\initcode{Ensure correct orientation of the geometric normal}{=}`
if (`\refvar{mesh}{}`->`\refvar[TriangleMesh::n]{n}{}`)
    isect->`\refvar[Interaction::n]{n}{}` = `\refvar{Faceforward}{}`(isect->`\refvar[Interaction::n]{n}{}`, isect->`\refvar{shading}{}`.`\refvar[shading::n]{n}{}`);
else if (`\refvar{reverseOrientation}{}` ^ `\refvar{transformSwapsHandedness}{}`)
    isect->`\refvar[Interaction::n]{n}{}` = isect->`\refvar{shading}{}`.`\refvar[shading::n]{n}{}` = -isect->`\refvar[Interaction::n]{n}{}`;  
\end{lstlisting}

\subsection{着色几何}\label{sub:着色几何}
有了\refvar{Triangle}{},用户可以提供网格顶点处的法向量和切向量,
它们用来插值求取三角形面上的点的法线和切线。
用插值法线的着色几何体可以让原本小面的三角网格显得比它们的几何形状更平滑。
如果提供了着色法线或着色切线,
就用它们初始化\refvar{SurfaceInteraction}{}里的着色几何体。
\begin{lstlisting}
`\initcode{Initialize Triangle shading geometry}{=}`
`\refcode{Compute shading normal ns for triangle}{}`
`\refcode{Compute shading tangent ss for triangle}{}`
`\refcode{Compute shading bitangent ts for triangle and adjust ss}{}`
`\refcode{Compute $\partial$n/$\partial$u and $\partial$n/$\partial$v for triangle shading geometry}{}`
isect->`\refvar{SetShadingGeometry}{}`(ss, ts, dndu, dndv, true);
\end{lstlisting}

给定交点的重心坐标,如果有的话,通过在合适的顶点法线间插值计算着色法线很简单。
\begin{lstlisting}
`\initcode{Compute shading normal ns for triangle}{=}`
`\refvar{Normal3f}{}` ns;
if (`\refvar{mesh}{}`->`\refvar[TriangleMesh::n]{n}{}`) ns = `\refvar{Normalize}{}`(b0 * `\refvar{mesh}{}`->`\refvar[TriangleMesh::n]{n}{}`[`\refvar[Triangle::v]{v}{}`[0]] +
                            b1 * `\refvar{mesh}{}`->`\refvar[TriangleMesh::n]{n}{}`[`\refvar[Triangle::v]{v}{}`[1]] + 
                            b2 * `\refvar{mesh}{}`->`\refvar[TriangleMesh::n]{n}{}`[`\refvar[Triangle::v]{v}{}`[2]]);
else
    ns = isect->`\refvar[Interaction::n]{n}{}`;
\end{lstlisting}

着色切线计算相同。
\begin{lstlisting}
`\initcode{Compute shading tangent ss for triangle}{=}`
`\refvar{Vector3f}{}` ss;
if (`\refvar{mesh}{}`->`\refvar[TriangleMesh::s]{s}{}`) ss = `\refvar{Normalize}{}`(b0 * `\refvar{mesh}{}`->`\refvar[TriangleMesh::s]{s}{}`[`\refvar[Triangle::v]{v}{}`[0]] +
                            b1 * `\refvar{mesh}{}`->`\refvar[TriangleMesh::s]{s}{}`[`\refvar[Triangle::v]{v}{}`[1]] + 
                            b2 * `\refvar{mesh}{}`->`\refvar[TriangleMesh::s]{s}{}`[`\refvar[Triangle::v]{v}{}`[2]]);
else
    ss = `\refvar{Normalize}{}`(isect->`\refvar[SurfaceInteraction::dpdu]{dpdu}{}`);
\end{lstlisting}

用{\ttfamily ns}和{\ttfamily ss}的叉积求\keyindex{双切向量}{bitangent vector}{vector\ 向量}{\ttfamily ts},
即得到一个正交于它俩的向量。
接下来,{\ttfamily ss}被{\ttfamily ts}和{\ttfamily ns}的叉积覆写;
这保证{\ttfamily ss}和{\ttfamily ts}的叉积是{\ttfamily ns}。
因此,如果提供了每个顶点的$\bm n$和$\bm s$值
且插值的$\bm n$和$\bm s$值并不完全正交,
则$\bm n$会被保留而$\bm s$被修改使得坐标系是正交的。
\begin{lstlisting}
`\initcode{Compute shading bitangent ts for triangle and adjust ss}{=}`
`\refvar{Vector3f}{}` ts = `\refvar{Cross}{}`(ns, ss);
if (ts.`\refvar{LengthSquared}{}`() > 0.f) {
    ts = `\refvar{Normalize}{}`(ts);
    ss = `\refvar{Cross}{}`(ts, ns);
}
else
    `\refvar{CoordinateSystem}{}`((`\refvar{Vector3f}{}`)ns, &ss, &ts);
\end{lstlisting}

计算着色法线偏导数$\displaystyle\frac{\partial\bm n}{\partial u}$
和$\displaystyle\frac{\partial\bm n}{\partial v}$的代码
几乎和计算偏导数$\displaystyle\frac{\partial\bm p}{\partial u}$
和$\displaystyle\frac{\partial\bm p}{\partial v}$的代码一样,
因此这里略去。

\subsection{表面积}\label{sub:表面积6}
利用平行四边形面积等于它沿两边的两个向量的叉积长度的事实,
方法\refvar[Triangle::Area]{Area}{()}计算三角形面积为
其边向量构成的平行四边形面积的一半(\reffig{3.13})。
\begin{lstlisting}
`\refcode{Triangle Method Definitions}{+=}\lastnext{TriangleMethodDefinitions}`
`\refvar{Float}{}` `\refvar{Triangle}{}`::`\initvar[Triangle::Area]{\refvar[Shape::Area]{Area}{}}{}`() const {
    `\refcode{Get triangle vertices in p0, p1, and p2}{}`
    return 0.5 * `\refvar{Cross}{}`(p1 - p0, p2 - p0).`\refvar{Length}{}`();
}
\end{lstlisting}