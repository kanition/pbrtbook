\section{路径-空间测量方程}\label{sec:路径-空间测量方程}

\subsection{采样相机}\label{sub:采样相机2}
\begin{lstlisting}
`\initcode{Compute image plane bounds at z=1 for PerspectiveCamera}{=}`
`\refvar{Point2i}{}` res = film->`\refvar{fullResolution}{}`;
`\refvar{Point3f}{}` pMin = `\refvar{RasterToCamera}{}`(`\refvar{Point3f}{}`(0, 0, 0));
`\refvar{Point3f}{}` pMax = `\refvar{RasterToCamera}{}`(`\refvar{Point3f}{}`(res.x, res.y, 0));
pMin /= pMin.z;
pMax /= pMax.z;
`\refvar[PerspectiveCamera::A]{A}{}` = std::abs((pMax.x - pMin.x) * (pMax.y - pMin.y));
\end{lstlisting}
\begin{lstlisting}
`\refcode{PerspectiveCamera Private Data}{+=}\lastcode{PerspectiveCameraPrivateData}`
`\refvar{Float}{}` `\initvar[PerspectiveCamera::A]{A}{}`;
\end{lstlisting}
\subsection{非对称散射}\label{sub:非对称散射}
\begin{lstlisting}
`\initcode{TransportMode Declarations}{=}`
enum class `\initvar{TransportMode}{}` { `\initvar{Radiance}{}`, `\initvar{Importance}{}` };
\end{lstlisting}


\subsection{折射引起的非对称性}\label{sub:折射引起的非对称性}
\begin{lstlisting}
`\initcode{Update scale to account for adjoint light transport}{=}`
if (`\refvar[FourierBSDF::mode]{mode}{}` == `\refvar{TransportMode}{}`::`\refvar{Radiance}{}` && muI * muO > 0) {
    float eta = muI > 0 ? 1 / `\refvar[FourierBSDF::bsdfTable]{bsdfTable}{}`.`\refvar[FourierBSDFTable::eta]{eta}{}` : `\refvar[FourierBSDF::bsdfTable]{bsdfTable}{}`.`\refvar[FourierBSDFTable::eta]{eta}{}`;
    scale *= eta * eta;
}
\end{lstlisting}