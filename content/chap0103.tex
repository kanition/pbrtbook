\section{pbrt:系统概述}\label{sec:pbrt:系统概述}

pbrt是使用标准的\keyindex{面向对象}{object-oriented}{}技术构建的:
重要实体都定义了抽象\keyindex{基类}{base class}{class类}(例如
抽象基类{\ttfamily Shape}定义了所有几何形状必须实现的接口,
光源的抽象基类{\ttfamily Light}也有相似设计)。
系统大部分都是纯粹由这些抽象基类提供的接口来实现的;
例如检查光源与着色点之间遮挡物体的代码
调用{\ttfamily Shape}的相交方法
而不要考虑场景中出现的特定类型的形状。
这种方式使得扩展系统变得很容易,
新增一种形状只需要实现一个完成{\ttfamily Shape}接口的类并链接到系统。

\begin{table}[h]
    \centering
    \begin{tabular}{l l l}
        \toprule
        \textbf{基类}                                           & \textbf{目录}           & \textbf{章节}         \\
        \midrule
        \hyperref[code:overview_Shape]{\ttfamily Shape}         & \ttfamily shapes/       & \refsec{基本形状接口} \\
        \hyperref[code:overview_Aggregate]{\ttfamily Aggregate} & \ttfamily accelerators/ & \refsec{聚合}         \\
        \hyperref[code:overview_Camera]{\ttfamily Camera}       & \ttfamily cameras/      & \refsec{相机模型}     \\
        \hyperref[code:overview_Sampler]{\ttfamily Sampler}     & \ttfamily samplers/     & \refsec{采样接口}     \\
        \ttfamily Filter                                        & \ttfamily filters/      & 0.296                 \\
        \ttfamily Material                                      & \ttfamily materials/    & 0.296                 \\
        \ttfamily Texture                                       & \ttfamily textures/     & 0.296                 \\
        \ttfamily Medium                                        & \ttfamily media/        & 0.296                 \\
        \ttfamily Light                                         & \ttfamily lights/       & 0.296                 \\
        \ttfamily Integrator                                    & \ttfamily integrators/  & 0.296                 \\
        \bottomrule
    \end{tabular}
    \caption{主要接口类型。}
    \label{tab:1.1}
\end{table}

的盛世嫡妃