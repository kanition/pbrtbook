\section{译者补充:初等数论基础}\label{sec:译者补充:初等数论基础}

\begin{remark}
    本节内容不是原书内容,而是译者根据\citet{ElementaryNumberTheory}
    所著著作补充的,请酌情参考和斧正。
\end{remark}

\begin{notation}
    本节我们重申以下记号:
    \begin{itemize}
        \item 用$\mathbb{N}$表示全体正整数构成的集合;$\mathbb{Z}$表示全体整数构成的集合。
        \item 若命题$p$能推出命题$q$,则记为$p\Rightarrow q$;若$p$与$q$等价,则记为$p\Leftrightarrow q$。
    \end{itemize}
\end{notation}

\begin{theorem}[\protect\keyindex{最小自然数原理}{least number principle}{}]
    设$T$是$\mathbb{N}$的一非空子集,则必有$t_0\in T$,
    使对任意的$t\in T$有$t_0\le t$,即$t_0$是$T$中最小的自然数。
\end{theorem}

\begin{theorem}[最大自然数原理]
    设$M$是$\mathbb{N}$的一非空子集,若$M$有上界(即存在$a\in \mathbb{N}$使
    对任意的$m\in M$有$m\le a$),则必有$m_0\in M$,使对任意的$m\in M$有$m\le m_0$,
    即$m_0$是$M$中最大的自然数。
\end{theorem}

\begin{theorem}[\protect\keyindex{归纳原理}{principle of induction}{}]
    设$S\subseteq \mathbb{N}$,且满足
    \begin{enumerate}
        \item 有$1\in S$;
        \item 对任意$n\in S$都有$n+1\in S$;
    \end{enumerate}
    则$S=\mathbb{N}$。
\end{theorem}

\begin{theorem}[\protect\keyindex{数学归纳法}{mathematical induction}{}]
    设$P(n)$是关于自然数$n$的命题,若
    \begin{enumerate}
        \item 当$n=1$时,$P(1)$成立;
        \item $P(n)$成立时必能推出$P(n+1)$成立;
    \end{enumerate}
    则$P(n)$对所有自然数$n$均成立。
\end{theorem}

\begin{theorem}[\protect 第二种数学归纳法]
    设$P(n)$是关于自然数$n$的命题,若
    \begin{enumerate}
        \item 当$n=1$时,$P(1)$成立;
        \item 设$n>1$,对所有自然数$m<n$都有$P(m)$成立时必能推出$P(n)$成立;
    \end{enumerate}
    则$P(n)$对所有自然数$n$均成立。
\end{theorem}

\begin{theorem}[\protect\keyindex{鸽巢原理}{pigeonhole principle}{}]
    对于某$n\in\mathbb{N}$,现有$n$个笼子和$n+1$只鸽子,
    所有的鸽子都被关在鸽笼里,那么至少有一个笼子有至少2只鸽子。
    也称\keyindex{狄利克雷抽屉原理}{Dirichlet's drawer principle}{}。
\end{theorem}

\begin{definition}
    设$a,b\in\mathbb{Z}$且$a\neq0$,若存在$q\in\mathbb{Z}$使得$b=aq$,
    则称$a$\keyindex{整除}{divide evenly}{}$b$,或说$b$能被$a$整除,记作$a|b$,
    并称$a$是$b$的\keyindex{因数}{divisor}{},也称{\sffamily 约数}、{\sffamily 除数},
    $b$是$a$的\keyindex{倍数}{multiple}{}。$a$不能整除$b$时记作$a\nmid b$。
\end{definition}

\begin{example}
    6能整除18,记作$6|18$,6是18的因数,18是6的倍数。
\end{example}

\begin{theorem}
    整除满足以下性质:
    \begin{enumerate}
        \item $a|b\Leftrightarrow -a|b \Leftrightarrow a|-b \Leftrightarrow |a|||b|$;
        \item $a|b$且$b|c \Rightarrow a|c$;
        \item $a|b$且$a|c \Leftrightarrow$对任意的$x,y\in\mathbb{Z}$有$a|bx+cy$;
        \item 设$m\neq0$,则$a|b\Leftrightarrow ma|mb$;
        \item $a|b$且$b|a\Rightarrow b=\pm a$;
        \item 设$b\neq0$,则$a|b\Rightarrow |a|\le|b|$。
    \end{enumerate}
\end{theorem}

\begin{corollary}
    非零整数的因数只有有限个。
\end{corollary}

\begin{definition}
    设$a,b,m\in\mathbb{Z}$且$m\neq0$,若$m|a-b$,则称$a$与$b$\keyindex{模$m$同余}{congruent modulo $m$}{},
    也称$a$同余于$b$模$m$、$b$是$a$对模$m$的剩余,记作
    \begin{align}\label{eq:7.ex02.01}
        a\equiv b(\mod m)\, ,
    \end{align}
    其中$m$称为\keyindex{模}{modulus}{},称\refeq{7.ex02.01}为模$m$的同余式;
    否则称$a$不同余于$b$模$m$、$b$不是$a$对模$m$的剩余,记作
    \begin{align}
        a\not\equiv b(\mod m)\, .
    \end{align}
\end{definition}

\begin{notation}
    因为$m|a-b\Leftrightarrow -m|a-b$,所以\refeq{7.ex02.01}等价于$a\equiv b(\mod (-m))$。
    由此,下文均假定模$m\ge1$。
\end{notation}