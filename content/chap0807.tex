\section{扩展阅读}\label{sec:扩展阅读08}
\citet{10.1145/360825.360839}为计算机图形学中的光泽曲面开发了早期经验反射模型。
尽管它既不互易也不能量守恒,但它是非朗伯体的首张合成图像的基石。
\citet{Torrance:67}阐述了Torrance-Sparrow微面模型;
\citet{10.1145/965141.563893}首次将它引入图形学,
\citet{10.1145/800224.806819,10.1145/357290.357293}则使用了它的变体。
Oren-Nayar朗伯模型在他们\citeyear{10.1145/192161.192213}年
的论文\citep{10.1145/192161.192213}中得到阐述。

\citet{10.1007/978-1-4612-3526-2}的书籍总结并阐述了
当时图形学中最好的基于物理的曲面反射模型。
该书详细讨论了曲面反射的物理性质,并附有许多原文指引和
从真实曲面测得的关于反射的实用数据表格。
\citet{Burley:2012:PBS}最近的论文包含了计算机图形学
反射模型近期工作的详尽注解书目。

\citet{heitz:hal-01024289}关于微面掩模遮挡函数的论文
对微面BSDF模型的整体介绍写得非常好,并附有关于这一话题细节许多有用的图示和解释。
参见\citet{1987BeckmannSpizzichino}与\citet{Trowbridge:75}的
论文分别对他们的微面分布函数的介绍。
\citet{10.1145/1722991.1722996}开发了各向异性Beckmann-Spizzichino分布函数;
参见\citet{heitz:hal-01024289}了解其他许多微面分布函数的各向异性版本。
计算机图形学的早期各向异性BRDF模型由\citet{10.1145/325334.325167}、
\citet{10.1145/97879.97909}开发。
