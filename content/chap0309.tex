\section{控制舍入误差}\label{sec:控制舍入误差}

\begin{remark}
    本节含有高级内容,第一次阅读时可以跳过。
\end{remark}

\subsection{浮点算术}\label{sub:浮点算术}
\begin{lstlisting}
`\initcode{Global Constants}{=}\initnext{GlobalConstants}`
static constexpr `\refvar{Float}{}` `\initvar{MaxFloat}{}` = std::numeric_limits<`\refvar{Float}{}`>::max();
static constexpr `\refvar{Float}{}` `\initvar{Infinity}{}` = std::numeric_limits<`\refvar{Float}{}`>::infinity();
\end{lstlisting}

\subsubsection*{实用例程}\label{subsub:实用例程}

\begin{lstlisting}
`\initcode{Global Inline Functions}{=}\initnext{GlobalInlineFunctions}`
inline uint32_t `\initvar{FloatToBits}{}`(float f) {
    uint32_t ui;
    memcpy(&ui, &f, sizeof(float));
    return ui;
}
\end{lstlisting}

\subsubsection*{算术运算}\label{subsub:算术运算}
\begin{lstlisting}
`\refcode{Global Constants}{+=}\lastnext{GlobalConstants}`
static constexpr `\refvar{Float}{}` `\initvar{MachineEpsilon}{}` =
       std::numeric_limits<`\refvar{Float}{}`>::epsilon() * 0.5;
\end{lstlisting}

\subsubsection*{误差传播}\label{subsub:误差传播}
\begin{lstlisting}
`\refcode{Global Inline Functions}{+=}\lastnext{GlobalInlineFunctions}`
inline constexpr `\refvar{Float}{}` `\initvar{gamma}{}`(int n) {
    return (n * `\refvar{MachineEpsilon}{}`) / (1 - n * `\refvar{MachineEpsilon}{}`);
}
\end{lstlisting}

\subsection{定界交点误差}\label{sub:定界交点误差}