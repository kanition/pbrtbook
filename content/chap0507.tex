\section{扩展阅读}\label{sec:扩展阅读05}

Meyer是最早严密研究图形学中光谱表示的学者之一
\citep{10.1145/800250.807502}\citep{10.1145/7529.7920}。
\citet{10.1007/978-1-4612-3526-2}总结了1989年光谱表示的最新进展,
\citet{GLASSNER1995}的《\citetitle{GLASSNER1995}》涵盖了20世纪90年代中期的话题。
\citet{773962}、\citet{773963}以及\citet{10.2312:egst.20021054}的研究论文
是关于该话题的优良资源。

\citet{10.1145/122718.122729}分析了三刺激表示用于光谱计算时引入的误差。
\citet{10.1145/166117.166142}以依赖于场景的方法基于选择基函数开发了技术:
通过看向场景中光源和反射物体的SPD,用特征向量分析可以找到
少量能准确表示场景SPD的光谱基函数。
\citet{10.1007/978-3-7091-6858-5_12}将场景中所有SPD投影到层次化基
(\keyindex{哈尔小波}{Haar wavelet}{wavelet\ 小波}
\sidenote{译者注:\keyindex{小波}{wavelet}{}变换指用有限长或快速衰减的“母小波”
    的振荡波形来表示信号。哈尔小波变换是最早提出的最简单小波变换。})上,
并说明了该自适应表示能用来保持在期望的误差界内。
\citet{10.2312:EGWR:EGWR02:117-124}通过在渲染前仔细调整提供给系统的颜色值
为提升来自于只用RGB的常规渲染系统的光谱结果开发了方法。

\citet{Sun2001}研究了另一个光谱表示的方法,
他把SPD分为光滑的基SPD和一组尖峰。
利用与分布每个部分相适配的基函数,每部分的表示都不同。
\citet{Drew:03}应用了一种“尖锐”基,它是自适应的但具有性质即
计算两个基中函数之积不用像许多其他基表示那样进行全矩阵乘法。

当用采样点表示(例如\refvar{SampledSpectrum}{})时,
很难知道对于精确结果需要多少样本。
\citet{Lehtonen:06}研究了这一问题并确定了
5纳米的样本间隔对于真实世界SPD足够了。

\citet{Evans:1999:10.20380/GI1999.07}为表示SPD引入了
分层\sidenote{译者注:原文stratified。}波长聚类:
思想是每次光谱计算都使用在代表性波长上少数固定数目的样本,
它们是依据光源的光谱分布选出的。
后续计算使用不同的波长,这样(基于少量样本的)单独计算会相对高效,
但是在大量次数计算的总和下,可以很好地覆盖波长的整体范围。
与该方法相关的思想是每次计算中只为单个波长算出结果并取全部结果的平均:
这是\citet{10.1145/256157.256158,}和\citet{Morley:2006:}用的方法。

\citet{Radziszewski2009}提出根据单个波长生成光传播路径的技术,
并用高效的SIMD指令在几个额外波长处追踪其贡献。
当模拟粗糙折射边界的色散时用多重重要性采样综合其贡献会得到减小的方差。
\citet{10.1111/cgf.12419}使用波长上的等间隔点样本并展示了
该方法可以怎样用于光子映射和介质渲染。

\citet{31468}写过关于将(例如用户从显示器上选择的)RGB值转化为SPD的欠约束问题的论文。
\citet{10.1080/10867651.1999.10487511}开发了
我们在\refsub{RGB颜色}实现的改进方法。
见\citet{10.1111/cgf.12676}了解该领域最近的工作,
包括精确执行这些转化时对所涉及复杂性的全面讨论。

\citet{9100708}关于辐射度学的书\sidenote{译者注:原文引用1994年第1版,此处引用的是第2版。}是对该话题的精彩介绍。
\citet{PREISENDORFER19653}也以易懂的方式介绍了辐射度学并
深入研究了辐射度学和物理光学的关系。
\citet{Nicodemus1977}仔细定义了BRDF、BSSRDF以及可从中推导出的各种数量。
见\citet{Moon1936}、\citet{Moon1948}以及\citet{10.1002/sapm193918151}了解
对辐射度学的经典早期介绍。
最后,18世纪中叶\citet{Lambert1760}早期关于光度学的开创性著作
已被Dilaura翻译为英文。

正确实现辐射度量学计算会很困难:
遗漏余弦项会算出与所期望的完全不同的结果。
调试这类问题会非常耗时。
\citet{10.1111/j.1467-8659.2010.01722.x}展示了
怎样用C++的类型系统与这些计算中每一项的单位关联,
例如这样使得尝试将辐亮度值加到另一个表示辐照度的值会触发编译时错误。