\section{扩展阅读}\label{sec:扩展阅读08}
\citet{10.1145/360825.360839}为计算机图形学中的光泽曲面开发了早期经验反射模型。
尽管它既不互易也不能量守恒,但它是非朗伯体的首张合成图像的基石。
\citet{Torrance:67}阐述了Torrance-Sparrow微面模型;
\citet{10.1145/965141.563893}首次将它引入图形学,
\citet{10.1145/800224.806819,10.1145/357290.357293}则使用了它的变体。
Oren-Nayar朗伯模型在他们\citeyear{10.1145/192161.192213}年
的论文\citep{10.1145/192161.192213}中得到阐述。

\citet{10.1007/978-1-4612-3526-2}的书籍总结并阐述了
当时图形学中最好的基于物理的曲面反射模型。
该书详细讨论了曲面反射的物理性质,并附有许多原文指引和
从真实曲面测得的关于反射的实用数据表格。
\citet{Burley:2012:PBS}最近的论文包含了计算机图形学
反射模型近期工作的详尽注解书目。

\citet{heitz:hal-01024289}关于微面掩模遮挡函数的论文
对微面BSDF模型的整体介绍写得非常好,并附有关于这一话题细节许多有用的图示和解释。
参见\citet{1987BeckmannSpizzichino}与\citet{Trowbridge:75}的
论文分别对他们的微面分布函数的介绍。
\citet{10.1145/1722991.1722996}开发了各向异性Beckmann-Spizzichino分布函数;
参见\citet{heitz:hal-01024289}了解其他许多微面分布函数的各向异性版本。
计算机图形学的早期各向异性BRDF模型由\citet{10.1145/325334.325167}、
\citet{10.1145/97879.97909}开发。

\citet{1138991}引入了\refeq{8.14}中的微面掩模遮挡函数,
他利用了微面上临近位置的高度之间没有关联的假设推导了该结果。
\citeauthor{1138991}还首次推导出\refeq{8.12}中的规范化约束
(\citet{10.1145/344779.344814}也独立推导出该结果)。
参见\citet{heitz:hal-01024289}对这些函数推导的进一步讨论。
\citet{Heitz01082013}还研究了更好地考虑两个方向间的相关性效应下
高斯微面曲面的更精确$G_2({\bm\omega}_{\mathrm{o}},{\bm\omega}_{\mathrm{i}})$函数
\sidenote{译者注:原文写作$G({\bm\omega}_{\mathrm{i}},{\bm\omega}_{\mathrm{o}})$,
这里笔者为上下文统一改写了形式。},
本章所用的对Beckmann-Spizzichino的$\Lambda({\bm\omega})$函数的
合理近似也来自\citet{heitz:hal-01024289},
它是从\citet{10.5555/2383847.2383874}所给的近似中推导出的。
我们对$\Lambda({\bm\omega})$函数的推导\refeq{8.13}也来自\citet{heitz:hal-01024289}
\sidenote{译者注:正文中的引用文献信息有所替换,因为笔者未能找到原书标注的如下文献:
Heitz, E. Derivation of the microfacet $\Lambda({\bm\omega})$ function. Personal communication.}。

\citet{10.1007/978-3-7091-6242-2_4}为透射建立了Cook-Torrance模型的一般形式,
最近\citet{10.5555/2383847.2383874}又重新审视了这个问题。
\citet{10.1145/1531326.1531338}开发了推测微面分布的方法
以匹配测量到的或期望的反射分布。令人瞩目的是,他们表明了
有可能制造出精确匹配期望的反射分布的实际物理曲面。
\citet{Simonot:09}开发了横跨Oren-Nayar与Torrance-Sparrow的模型:
微面被建模为朗伯反射体,其上一层则在完美透射与完美镜面反射体间变化。
然而,该模型没有考虑掩模遮挡效应且无法以解析形式计算。

本章的微面反射模型全都基于假设一个像素中有许多的微面可见以至于
可以用它们聚合后的统计行为来精确描述。
该假设对于现实世界中许多曲面而言并不成立,
其每个像素中只有相对少量的微面是可见的;
这样的曲面例子包括车漆和闪亮塑料\sidenote{译者注:原文glittery plastics。}。
\citet{10.1145/2601097.2601155}与\citet{10.1145/2601097.2601186}都
开发了对该情况建模的技术。

能够为分层材料找到BSDF会很有用,例如被铜绿锈蚀的金属基底曲面,或者涂有光泽涂料的木材。
\citet{10.1145/166117.166139}建模了分层皮肤,每层只考虑一次散射事件,
\citet{10.1145/237170.237278}用\keyindex{库贝尔卡-蒙克理论}{Kubelka-Munk theory}{}渲染了分层材料,
它考虑了层间的多次散射但假设了辐射分布不是随方向变化的函数。

\citet{10.1145/344779.344824}表明了蒙特卡罗积分可用于求解
\emph{叠加方程}\sidenote{译者注:原文adding equation。}以
高效计算分层材料的BSDF而无需这些简化。
叠加方程是\citet{10.1016/B978-0-12-710701-1.50002-1}
与\citet{MatrixMethodsforMultipleScatteringProblems}推导的
精确描述分层介质中多次散射效应的积分方程。
\citet{10.1145/1321261.1321292}通过大量简化假设更高效地渲染了分层材料,
\citet{10.1145/2601097.2601139}用如同这里\refvar{FourierBSDF}{}所实现的
傅里叶基表示高效计算了分层材料的散射。

大量学者研究了基于反射曲面小尺度几何特性建模的BRDF。
该工作包括\citet{10.1145/37401.37434}从
凹凸贴图\sidenote{译者注:原文bump map。}中计算BRDF、
\citet{fournier1992normal}的规范化分布模型,
以及\citet{10.1145/133994.134075}把蒙特卡罗光线追踪
应用于微观几何反射的统计建模并用\keyindex{球谐函数}{spherical harmonics}{}表示所得的结果。
最近,\citet{10.1145/2070781.2024179}开发的系统可以
建模微观几何、指定其底层BRDF并交互式地预览所得的宏观尺度BRDF。

数据采集技术的提升带来了真实世界BRDF详细数据数量的增长,甚至包括随空间变化的BRDF
(有时称作\keyindex{双向纹理函数}{bidirectional texture function}{}(BTF))
\citep{10.1145/300776.300778}。
\citet{10.5555/882404.882439,10.1145/882262.882343}建立了
测得的各向同性BRDF数据的早期数据库。
参见\citet{10.1111/j.1467-8659.2005.00830.x}对自2005年以来BRDF测量工作的综述。
\citet{10720740}测量了随时间变化的BRDF——例如,
油漆晾干、潮湿表面变干,或者灰尘堆积。
尽管大部分BRDF测法都基于测量由给定入射辐照度引起的反射辐射亮度,
但\citet{10.1145/2010324.1964939}表明了纺织物结构的CT成像
\sidenote{译者注:即\keyindex{计算机断层扫描}{computed tomography scan}{}。}
能得到非常精确的反射模型。
