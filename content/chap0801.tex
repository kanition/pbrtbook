\section{基本接口}\label{sec:基本接口}
我们将首先定义单个BRDF和BTDF函数的接口。
BRDF和BTDF共享共同的基类\refvar{BxDF}{}。
因为两者都有一样的接口,共享相同的基类减少了重复代码并
允许系统的一些部分和一般的\refvar{BxDF}{}配合而不用区分BRDF和BTDF。
\begin{lstlisting}
`\initcode{BxDF Declarations}{=}\initnext{BxDFDeclarations}`
class `\initvar{BxDF}{}` {
public:
    `\refcode{BxDF Interface}{}`
    `\refcode{BxDF Public Data}{}`
};
\end{lstlisting}
\refsec{BSDF}将要介绍的类\refvar{BSDF}{}

\begin{lstlisting}
`\initcode{BSDF Declarations}{=}\initnext{BSDFDeclarations}`
enum `\initvar{BxDFType}{}` {
    `\initvar[BSDFREFLECTION]{BSDF\_REFLECTION}{}` = 1 << 0,
    `\initvar[BSDFTRANSMISSION]{BSDF\_TRANSMISSION}{}` = 1 << 1,
    `\initvar[BSDFDIFFUSE]{BSDF\_DIFFUSE}{}` = 1 << 2,
    `\initvar[BSDFGLOSSY]{BSDF\_GLOSSY}{}` = 1 << 3,
    `\initvar[BSDFSPECULAR]{BSDF\_SPECULAR}{}` = 1 << 4,
    `\initvar[BSDFALL]{BSDF\_ALL}{}` = BSDF_DIFFUSE | BSDF_GLOSSY | BSDF_SPECULAR |
    BSDF_REFLECTION | BSDF_TRANSMISSION,
};
\end{lstlisting}

\begin{lstlisting}
`\initcode{BxDF Interface}{=}\initnext{BxDFInterface}`
`\refvar{BxDF}{}`(`\refvar{BxDFType}{}` type) : `\refvar[BxDF::type]{type}{}`(type) { }
\end{lstlisting}

\begin{lstlisting}
`\initcode{BxDF Public Data}{=}`
const `\refvar{BxDFType}{}` `\initvar[BxDF::type]{type}{}`;
\end{lstlisting}

\begin{lstlisting}
`\refcode{BxDF Interface}{+=}\lastnext{BxDFInterface}`
bool `\initvar{MatchesFlags}{}`(`\refvar{BxDFType}{}` t) const {
    return (`\refvar[BxDF::type]{type}{}` & t) == `\refvar[BxDF::type]{type}{}`;
}
\end{lstlisting}

\begin{lstlisting}
`\refcode{BxDF Interface}{+=}\lastnext{BxDFInterface}`
virtual `\refvar{Spectrum}{}` `\initvar[BxDF::f]{f}{}`(const `\refvar{Vector3f}{}` &wo, const `\refvar{Vector3f}{}` &wi) const = 0;
\end{lstlisting}