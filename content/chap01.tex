\chapterimage{Pictures/chap01/nightsnow-cut1368.png}

\chapter{绪论}\label{chap:绪论}
\setcounter{sidenote}{1}

\keyindex{渲染}{rendering}{render\ 渲染}是
由3D\keyindex{场景}{scene}{}描述生成图像的过程。
显然,这是一项十分庞大的任务,
有许多解决方案。\keyindex{基于物理的}{physically based}{physics\ 物理}
的技术采用模拟现实,
即运用物理学规律对光与物质的\keyindex{相互作用}{interaction}{}建模。
尽管基于物理的方法是实现渲染最容易想到的办法,
但它最近十年才在实践中得到广泛运用。
本章末的\refsec{基于物理的渲染简史}
将给出基于物理的渲染的简史
以及它近来在电影\keyindex{离线渲染}{offline rendering}{render\ 渲染}和
游戏\keyindex{交互式渲染}{interactive rendering}{render\ 渲染}方面的应用。

本书将介绍\emph{pbrt}这一基于\keyindex{光线追踪}{ray-tracing}{}算法的基于物理的渲染系统。
大多数计算机\keyindex{图形学}{graphics}{}书籍都主讲算法和理论,
偶尔附上一小段代码。
相反,本书将理论和一个功能齐全的渲染系统的完整实现结合起来。
系统的完整代码\footnote{\url{https://github.com/mmp/pbrt-v3}}
可在BSD许可证下获取。
在pbrt网站\href{https://pbrt.org}{\ttfamily pbrt.org}还可获取示例场景、渲染数据等更多信息。

\section{文学编程}\label{sec:文学编程}

在编写\TeX 排版系统时,Donald Knuth新提出一种简单但具有革命性的
编程方法论:\emph{程序应该写得更便于人类使用而不是更便于计算机理解}。
他将其称作\keyindex{文学编程}{literate programming}{}。
本书(包括本章)就是一个长长的文学程序。
这意味着在阅读本书的过程中,
你会读到pbrt渲染系统的\emph{完整}实现,
而不仅仅是高层叙述。

文学程序是由\keyindex{元语言}{metalanguage}{}
写成的,该语言把文档格式语言(例如\TeX 或HTML)
和编程语言(例如C++)结合起来。
两套分离的系统会这样处理程序:\keyindex{编排器}{weaver}{literate programming\ 文学编程}
把文学程序转换成适合排版的文档,\keyindex{整合器}{tangler}{literate programming\ 文学编程}
则生成可供编译的源码。
虽然我们的文学编程系统是自研的,
但很大程度上受到了Norman Ramsey的\emph{noweb}系统的影响。

文学编程元语言提供了两个重要功能。
第一个是把行文与源码结合起来。
这个功能让程序的说明和实际源码一样重要,
促使设计和文档做得更细致。
第二个是提供了与输入编译器的顺序完全不同的向读者展示程序代码的机制。
因此可以按逻辑顺序阐述程序。
每一段具有名称的代码块叫作\keyindex{代码片}{fragment}{},
每个代码片可以通过名称引用其他代码片。

例如,考虑一个负责初始化程序全部全局变量的函数
\footnote{本节的代码仅用作示例,不属于pbrt的一部分。}
{\ttfamily InitGlobals()}:
\begin{lstlisting}
void InitGlobals() {
    nMarbles = 25.7;
    shoeSize = 13;
    dielectric = true;
}
\end{lstlisting}

这个函数虽然很简短,但很难在没有任何上下文的情况下搞懂它。
比如为什么变量{\ttfamily nMarbles}采用浮点值?
刚看这段代码时,
就得在整个程序里寻找每个变量是在哪里声明的、怎么用的,
好弄清它的目的和合法值的含义。
尽管这样的系统结构对编译器来说没问题,
但人类阅读者更愿意看到
每个变量的初始化代码是分开呈现的,
而且最好紧挨着实际声明和使用这些变量的代码。

在文学程序中,可以把\refvar{InitGlobals}{()}写成这样:
\begin{lstlisting}
`\initcode{Function Definitions}{=}`
void `\initvar{InitGlobals}{()}` {
    `\refcode{Initialize Global Variables}{$\boxplus$}`
}
\end{lstlisting}

这就定义了称作\refcode{Function Definitions}{}的代码片,
包含了函数\refvar{InitGlobals}{()}的定义。
函数\refvar{InitGlobals}{()}自己又引用了另一
代码片\refcode{Initialize Global Variables}{}。
因为初始化的代码片还没有定义,
所以我们只知道这个函数可能会对全局变量赋值
(然而我们可以通过单击右边的加号\sidenote{译者注:本中译版改为直接点击代码片名称,以后省略加号。}向前跳转;
这样可以展开代码片最终全部的代码)。

现在有了代码片名称仅仅是有了正确的抽象层级,
因为还没有声明过任何变量。
之后在程序某处引入全局变量{\ttfamily shoeSize}时,
我们可以这样写:
\begin{lstlisting}
    `\initcode{Initialize Global Variables}{=}\initnext{InitializeGlobalVariables}`
    shoeSize = 13;
\end{lstlisting}

这里我们开始定义\refcode{Initialize Global Variables}{}的内容了。
当文学程序整合成待编译的源码时,
文学编程系统会把代码{\ttfamily shoeSize = 13;}
替换到函数\refvar{InitGlobals}{()}的定义内。
等号后的符号{\codecolor $\downarrow$}表示后续还有代码添加到该代码片。
点击它即可跳转到下一处。

后文我们也许又定义了另一个全局变量{\ttfamily dielectric},
可以这样把它的初始化添到代码片之后:
\begin{lstlisting}
    `\refcode{Initialize Global Variables}{+=}\lastcode{InitializeGlobalVariables}`
    dielectric = true;
\end{lstlisting}

代码片名后的符号{\codecolor +=}表示我们之前已经定义过该代码片了。
此外符号{\codecolor $\uparrow$}回链到
之前\refcode{Initialize Global Variables}{}添加代码的地方。

当整合时,这三个代码片转换为代码:
\begin{lstlisting}
void InitGlobals() {
    `\hypertarget{code:Initialize Global Variables}{\color[RGB]{115,48,11}\scriptsize\rmfamily// Initialize Global Variables}`
    shoeSize = 13;
    dielectric = true;
}
\end{lstlisting}

这样,我们可以把复杂函数分解为逻辑不同的部分,使之更容易理解。
例如我们可以这样把一个复杂函数写作一系列代码片:
\begin{lstlisting}
`\refcode{Function Definitions}{+=}`
void `\initvar{complexFunc}{(int x, int y, double *values)}` {
    `\refcode{Check validity of arguments}{}`
    if (x < y) {
        `\refcode{Swap parameter values}{}`
    }
    `\refcode{Do precomputation before loop}{}`
    `\refcode{Loop through and update values array}{}`
}
\end{lstlisting}

同样,编译时\refvar{complexFunc}{()}内每段代码片的内容都内联展开。
在文档中,我们可以依次介绍每个代码片的实现。
这种分解让我们每次只展示一小段代码,使之更易于理解。
这种编程风格的另一优点是,通过把函数分解为逻辑片,
每片有了单一且明确的目的,可以独立编写、验证、阅读。
一般我们尽量让每段代码片少于10行。

在某种意义上,文学编程系统只是个增强了的宏替换包,
完成重排程序源码的任务。
这变化看似微不足道,但事实上文学编程和其他软件构建系统方法迥然不同。



\section{逼真渲染和光线追踪算法}\label{sec:逼真渲染和光线追踪算法}

逼真渲染的目标是创建3D场景的图像且与同一场景的照片难以区分。
在我们介绍渲染流程之前要重点理解的是,
此处的{\itshape 难以区分}一词不是精确说法,
因为它涉及人类观察者,
不同观察者对同一图像的感知可能是不同的。
尽管本书会涉及少量感知问题,
但明确给出观察者的精确特性是非常困难且远未解决的问题。
绝大多数时候,我们都对针对光及其与物质相互作用的物理仿真感到满意,
并以我们对显示技术的理解尽可能向观察者展示最好的图像。

几乎所有逼真渲染系统都基于光线追踪算法。
光线追踪算法其实很简单;
它跟随光线\sidenote{译者注:原文a ray of light。
      此外会按个人理解把ray译作“光线”或“射线”,把light译作“光”或“光线”甚至“光源”。}路径
穿过场景与环境中的物体相互作用并反射。
虽然编写光线追踪器的方法有很多,
但所有这些系统都必须模拟至少一项以下对象和现象:
\begin{itemize}
      \item \keyindex{相机}{camera}{}: 相机模型决定了从哪里、怎样观察场景,
            包括场景的图像是怎样记录到传感器上的。
            许多渲染系统从相机处开始生成视线并追踪到场景中。
      \item \keyindex{光线-物体相交}{ray-object intersection}{}: 此外,我们需要确定
            交点处物体的特定属性,例如曲面法线或材质。
            多数光线追踪器都有测试光线与多个物体相交的功能,
            典型的例如沿光线返回最近交点。
      \item \keyindex{光源}{light source}{}: 没有光,渲染场景就没有意义。
            光线追踪器必须对整个场景的光分布建模,
            不仅包括灯光自身的位置,还包括它们向整个场景发散能量的方式。
      \item \keyindex{可见性}{visibility}{}:为了知道给定光是否在表面上一点积累能量,
            必须确认从该点到光源是否存在一条不中断的路径。
            幸运的是,在光线追踪器中这个问题很容易回答,
            因为我们可以构造从表面到光源\sidenote{译者注:原文为light,我按个人理解译作“光源”。}的射线,
            寻找最近的光线-物体交点,
            并比较交点距离和光源距离。
      \item \keyindex{表面散射}{surface scattering}{}:每个物体都必须提供外观描述,
            包括光如何与物体表面相互作用等信息,
            以及再辐射\sidenote{译者注:原文reradiated。}(或散射\sidenote{译者注:原文scattered。})光的性质。
            表面散射模型是典型的参数化模型,
            因此可以模拟各种外观。
      \item \keyindex{间接光传输}{indirect light transport}{light transport\ 光传输}\sidenote{译者注:这里把transport译作“传输”是为了
                  与下一段propagation译作“传播”区分开,但个人理解似乎就是“传播”的意思。}:因为
            光在一个物体上反射或折射后可能遇到另一个物体,
            所以通常有必要追踪从表面发出的额外光线来捕捉这种效应。
      \item \keyindex{光线传播}{ray propagation}{}:我们需要知道光在空间中沿光线传播时发生了什么。
            如果渲染真空中的场景,则光能量沿光线保持恒定。
            真正的真空虽然在地球上是罕见的,
            但对许多环境而言是合理的近似。
            更多复杂模型可用于追踪穿过雾、烟、大气等的光线。
\end{itemize}

本节将简要讨论以上每个仿真任务。
后续章节会展示pbrt底层仿真组件的高级接口,
了解贯穿主渲染循环的单个光线处理过程。
我们还会介绍基于Turner Whitted的
原始光线追踪算法的表面散射模型实现。

\subsection{相机}\label{sub:相机}

几乎每个人都用过\keyindex{相机}{camera}{},熟悉它的基本功能:
你想用一张图像记下世界(通常是按按钮或点击屏幕),
然后图像就被记录到\keyindex{胶片}{film}{}或电子传感器上。
\keyindex{针孔相机}{pinhole camera}{camera\ 相机}是最简单的拍照设备之一。
它由一端打有小孔的遮光盒组成(\reffig{1.1})。
当针孔未被遮挡时,光射进针孔落到固定在盒子另一端的相纸上。
虽然它很简单,但这种相机至今仍在使用,常用于艺术目的。
要在胶片上获得足够的光以形成图像需要非常长的曝光时间。
\begin{figure}[htbp]
      \centering\input{Pictures/chap01/PinholeCamera.tex}
      \caption{针孔相机}\label{fig:1.1}
\end{figure}

虽然大多数相机都比针孔相机复杂得多,
但针孔相机是仿真的便捷起点。
相机最重要的功能是定义会被记录到胶片上的场景部分。
在\reffig{1.1}中,我们可以看见从针孔到胶片边的连线
是如何构造出延伸到场景中的双锥体的。
不在该锥体内的物体不会在胶片上成像。
因为实际的相机成像形状比锥体更复杂,
所以我们把这个可能在胶片上成像的空间区域称为\keyindex{视见体}{viewing volume}{}。

针孔相机也可以看作是把胶片平面放置在针孔的\emph{前方}但距离不变(\reffig{1.2})。
注意从针孔到胶片的连线正好定义了和之前一样的视见体。
当然,这不是真实相机的实际构建方法,
但对于仿真目的而言它是个方便的抽象。
当胶片(或成像)平面在针孔前时,
针孔也常常改称作\keyindex{眼睛}{eye}{}。
\begin{figure}[htbp]
      \centering\input{Pictures/chap01/Filminfront.tex}
      \caption{当仿真针孔相机时,胶片放在针孔前的平面,针孔改称为\emph{眼睛}。}\label{fig:1.2}
\end{figure}

现在我们说到渲染的关键问题:
相机在图像中的每一点记录下的颜色值是什么?
回想最初的针孔相机,
显然光线只有沿着连接针孔与胶片上一点的向量
传播才能作用于胶片上的相应位置。
在把胶片放在眼睛前的仿真相机中,
我们关注沿着图像点传播到眼睛的
光量\sidenote{译者注:原文the amount of light,
      即“光的数量”。这里译为“光量”,可以笼统理解为光的强弱。}。

因此,相机仿真器的重要任务是取图像上一点
并生成\keyindex{射线}{ray}{},
沿这些射线的\keyindex{入射光}{incident light}{light\ 光}可作用于该图像位置。
因为射线由一个\keyindex{端点}{origin point}{point\ 点}和
一个\keyindex{方向向量}{direction vector}{vector\ 向量}组成,
所以该任务对\reffig{1.2}的针孔相机而言很简单:
它把针孔作为端点,
把从针孔到像平面\sidenote{译者注:原文near plane,按
      个人理解改译为“像平面”。}的向量
作为射线的方向。
对于含有多个透镜的复杂相机模型,
则图像上给定点相应的射线需要更多计算
(\refsec{逼真相机}介绍了这类模型的实现)。


\subsection{光线-物体相交}\label{sub:光线-物体相交}

相机每次生成射线时,
渲染器第一个任务就是确定
如果有的话,哪个物体和该射线最先在哪里\keyindex{相交}{intersect}{}。
该\keyindex{交点}{intersection point}{point\ 点}是沿射线可见的,
我们想要模拟光在该点与物体的相互作用。
为了找到相交处,我们必须把场景中的所有物体拿来和该射线测试,
并选出与该射线首先相交的那个。
给定一射线${\bm r}$,首先将其写成\keyindex{参数形式}{parametric form}{}:
\begin{align*}
      {\bm r}(t)={\bm o}+t{\bm d}\, ,
\end{align*}
其中${\bm o}$是射线端点,${\bm d}$是其方向向量,$t$是定义在$(0,\infty)$的\keyindex{参数}{parameter}{}。
我们可以通过指定参数$t$的值并代入上式求得射线上的一点。

通常很容易寻找射线${\bm r}$和由隐函数$F(x,y,z)=0$定义的\keyindex{曲面}{surface}{}的交点。
首先把射线方程代入\keyindex{隐式方程}{implicit equation}{equation\ 方程},
得到只含有参数$t$的新方程。
然后从中解出$t$并把最小正根代入射线方程得到所需的点。
例如,以\keyindex{原点}{origin}{point\ 点}为球心,$r$为\keyindex{半径}{radius}{}的\keyindex{球}{sphere}{}的隐式方程是
\begin{align*}
      x^2+y^2+z^2-r^2=0\, .
\end{align*}
代入射线方程,得到
\begin{align*}
      (o_x+td_x)^2+(o_y+td_y)^2+(o_z+td_z)^2-r^2=0\, .
\end{align*}
上式除了$t$外的所有值都是已知的,因此是易解的关于$t$的二次方程。
如果没有正根,则射线与球面错开了;
如果有,则最小正根给出了交点。

对于光线追踪器的其余部分。只有交点信息是不够的;
它还需要知道该点表明的特定属性。
首先,必须确定该点的材质表示
并传给光线追踪算法之后的步骤。
其次,还需要交点处额外的几何信息来对该点\keyindex{着色}{shade}{}。
例如,曲面\keyindex{法线}{normal}{}${\bm n}$是必需的。
虽然许多光线追踪器只对${\bm n}$操作,
但像pbrt这类更复杂的渲染系统还需要更多信息,
比如位置的各个\keyindex{偏导数}{partial derivative}{derivative\ 导数}以及
关于曲面局部参数化的曲面法线。

当然,绝大多数场景含有多个物体。
暴力法指依次用每个物体对射线测试,
从所有相交中选出$t$的最小正值来求得最近交点。
该方法虽然正确但对哪怕适度复杂的场景也很慢。
更好的方法是在光线相交处理中
并入一个\keyindex{加速结构}{acceleration structure}{}快速否决整组物体。
快速剔除无关几何体的能力意味着光线追踪常能以$O(I\log{N})$的时间运行
\sidenote{大$O$表示法常用来表示算法运行时间随问题规模增长的变化趋势,
称为\keyindex{渐进时间复杂度}{asymptotic time complexity}{},简称为时间复杂度。},
其中$I$是图像像素数目,$N$是场景中物体的数量
\footnote{虽然光线追踪的对数复杂度是
      常被认为是其主要优点,但该复杂度只在平均意义下是典型正确的。
      在计算几何文献中发表的许多光线追踪算法都保证有对数运行时间,
      但它们只在特定种类场景下才能工作,并且预处理开销和存储要求很高。
      \protect\citet{10.1007/BF02684409}提供了相关参考文献。
      但好在表现真实环境的场景通常不会遇到这种最坏的情形。
      实践中本书介绍的射线相交算法是次线性的,
      但在去掉大量预处理和存储开销时
      总是有可能构造出使光线追踪以$O(IN)$时间运行的最坏情形。}
(然而构建加速结构本身至少需要$O(N)$时间)。

pbrt为各种形状实现的几何接口将在第\refchap{形状}介绍,
加速接口和实现在第\refchap{图元和相交加速}。

\subsection{光分布}\label{sub:光分布}

光线-物体相交阶段给出了要着色的点和该点的局部几何信息。
回想我们的最终目标是找出从该点出发朝相机方向传播的光量。
为此需要知道有多少光\emph{到达}了该点。
这同时涉及到场景中光的\emph{几何}和\emph{辐射}分布。
对于非常简单的光源(例如\keyindex{点光源}{point light}{light source\ 光源}),
光的\keyindex{几何分布}{geometric distribution}{distribution\ 分布}只需
知道光源位置即可。
然而真实世界并不存在点光源,
所以基于物理的光照常基于\keyindex{面光源}{area light source}{light source\ 光源}。
这意味着光源和表面发光的几何体关联。
但本节我们用点光源说明光分布的构成;
光度量和分布的严格讨论是第\refchap{颜色和辐射度学}和第\refchap{光源}的主题。

我们常常想知道光在交点附近的微分面积上积累的功率(\reffig{1.3})。
假设点光源具有功率\sidenote{译者注:称为辐射通量或辐射功率。}
$\varPhi$并向所有方向均匀辐射光
\sidenote{译者注:原文使用正体字母$\Phi$,
      我把标量统一改为了斜体;
      此外原文正确使用了$\pi$的正体表示圆周率,
      但我因为字体环境冲突无法打出正体,
      所以全书被迫用的斜体。}。
这意味着围绕光源的单位球面在单位面积上的功率
为$\displaystyle\frac{\varPhi}{4\mathrm{\pi}}$
(这些度量将在\refsec{辐射度学}解释和形式化)。
\begin{figure}[htbp]
      \centering\input{Pictures/chap01/Basicreflectionsetting.tex}
      \caption{确定由点光源到达一点的单位面积功率的几何结构。
            该点到光源的距离记作$r$.}\label{fig:1.3}
\end{figure}

如果考虑两个这样的球面(\reffig{1.4}),
很明显大球上一点的单位面积功率必然比小球低,
因为相同的总功率分布在了更大的面积上。
特别地,到达半径为$r$的球面上一点的单位面积功率正比于$\displaystyle\frac{1}{r^2}$.
\begin{figure}[htbp]
      \centering\input{Pictures/chap01/Lightspheresequalpower.tex}
      \caption{因为点光源向所有方向均匀辐射光,
            所以以光源为球心的任意球面所积累的总功率相同。}
      \label{fig:1.4}
\end{figure}

此外可以看出,如果一小块曲面$\mathrm{d}A$相对于
从曲面指向光源的向量倾斜了角度$\theta$,
则$\mathrm{d}A$上积累的功率正比于$\cos{\theta}$.
综上所述,单位面积上的微分功率
(\keyindex{辐射照度}{irradiance}{}\sidenote{译者注:
      原文differential irradiance,鉴于其本身就是定义为功率对面积的微分,此处翻译略去了“微分”。})
$\mathrm{d}E$为
\begin{align*}
      \mathrm{d}E=\frac{\varPhi\cos{\theta}}{4\mathrm{\pi}r^2}\,.
\end{align*}

熟悉计算机图形学中基本光照的读者会注意到
该方程包含了两个熟悉的定律:
上述面的光照随倾斜按余弦衰减\sidenote{译者注:
      \keyindex{朗伯余弦定律}{Lambert's cosine law}{}。},
随距离按$\displaystyle\frac{1}{r^2}$衰减。

\subsection{可见性}\label{sub:可见性}

上节所述的光分布忽略了一个重要部分:\keyindex{阴影}{shadow}{}。
对于着色点,只有当该点到光源位置的路径畅通时,
该光源才会照亮该点(\reffig{1.5})。
\begin{figure}[htbp]
      \centering\input{Pictures/chap01/Twolightsoneblocker.tex}
      \caption{只有在视点处能无障碍地看到光源时,
            该光源才能对其表面积累能量。
            左边的光源照亮了点$\bm p$,但右边的不行。}\label{fig:1.5}
\end{figure}

幸运的是,在光线追踪器中很容易确定光对于着色点是否可见。
我们简单构造一条新射线,
其端点是表面上的点,方向指向光源。
这种特殊射线称为\keyindex{阴影射线}{shadow ray}{ray\ 射线}。
如果追踪这些射线穿过环境,
我们就能检查在射线原点与光源之间能否找到相交处,
方法是沿射线方向比较相交处和光源位置的相应参数值$t$.
如果光源与表面之间没有遮挡物,
就应考虑该光源的作用。

\subsection{表面散射}\label{sub:表面散射}
我们现在能计算两种对着色点至关重要的信息了:位置和入射光
\footnote{已熟悉渲染的读者可能会反对本节只考虑直接照明的讨论。但请放心,pbrt支持全局照明。}。
现在需要确定入射光是如何在表面上\keyindex{散射}{scattered}{}的。
特别地,我们关注沿着原来追踪寻找交点的射线散射回的光的能量大小,
因为该光线会通向相机(\reffig{1.6})。
\begin{figure}[htbp]
      \centering\input{Pictures/chap01/Surfacescatteringgeometry.tex}
      \caption{表面散射的几何结构。
            入射光沿${\bm \omega}_\mathrm{i}$方向与表面交于点$\bm p$并沿方向${\bm \omega}_\mathrm{o}$散射回相机。
            朝相机散射的光量由入射光能量与BRDF的积给定。}\label{fig:1.6}
\end{figure}

场景中每个物体都提供了\keyindex{材质}{material}{},
即对表面每点外观属性的描述。
这个描述由\keyindex{双向反射分布函数}{bidirectional reflectance distribution function}{}(BRDF)给定。
该函数告诉我们从\keyindex{入射}{incoming}{}方向${\bm \omega}_\mathrm{i}$
到\keyindex{出射}{outgoing}{}方向${\bm \omega}_\mathrm{o}$会反射多少能量。
我们把$\bm p$处的BRDF写作$f_{\mathrm{r}}({\bm p},{\bm \omega}_\mathrm{o},{\bm \omega}_\mathrm{i})$.
现在计算散射回相机的光量$L$就很直接了:
\begin{lstlisting}
for each light:
      if light is not blocked:
            incident_light = light.L(point)
            amount_reflected =
                  surface.BRDF(hit_point, camera_vector, light_vector)
            L += amount_reflected * incident_light
\end{lstlisting}
这里我们用符号$L$代表光照强弱;
在光度量中它用的单位和前面的$\mathrm{d}E$有一点不同。
$L$即\keyindex{辐射亮度}{radiance}{},是后文经常见到的光度量指标。

很容易从BRDF的表示推广到\keyindex{透射光}{transmitted light}{light\ 光}
(得到BTDF),或者得到到达表面任意一侧的光的一般散射。
描述一般散射的函数称作\keyindex{双向散射分布函数}{bidirectional scattering distribution function}{}(BSDF)。
pbrt支持各种BSDF模型;它们将于第\refchap{反射模型}介绍。
更复杂的还有\keyindex{双向散射表面反射分布函数}{bidirectional scattering surface reflectance distribution function}{}(BSSRDF),
它建模的光在离开表面时的点和进入时不同。
BSSRDF将在\refsub{BSSRDF}、\refsec{BSSRDF}和\refsec{使用扩散方程的次表面散射}介绍。

\subsection{间接光传输}\label{sub:间接光传输}

\citet{10.1145/358876.358882}在关于光线追踪
的原文中强调了其递归性质
是在渲染图像中引入间接\keyindex{镜面反射}{specular reflection}{reflection\ 反射}与\keyindex{透射}{transmission}{}的关键。
例如,如果来自相机的射线命中像镜子那样光滑的物体,
我们可以在交点处以曲面法线为对称轴反射该射线,
递归调用光线追踪程序找到到达镜面上该点的的光,
将其纳入相机原来的射线的考虑范围。
该技术也可用于追踪交于透明物体的透射光线。
很长一段时间大多数早期光线追踪示例都拿
镜子和玻璃球举例(\reffig{1.7}),
因为这种效果很难用其他渲染技术实现。
\begin{figure}[htbp]
      \centering
      \subfloat[Whitted光线追踪]{\includegraphics[width=\linewidth]{chap01/spheres-whitted.png}\label{fig:1.7.1}}\\
      \subfloat[随机渐进光子映射。]{\includegraphics[width=\linewidth]{chap01/spheres-sppm.png}\label{fig:1.7.2}}
      \caption{一个典型的早期光线追踪场景。
            注意镜面和玻璃物体的使用,
            它强调了算法处理这类表面的能力。
            (a)使用Whitted光线追踪渲染,
            (b)使用随机渐进光子映射(SPPM),
            \refsec{随机渐进光子映射}将介绍这一高级光传输算法。
            SPPM能准确模拟光通过球体的聚焦现象。}\label{fig:1.7}
\end{figure}

通常,从物体上一点到达相机的光量
\sidenote{译者注:此处指辐亮度。}
由物体的发光量(如果它自己就是光源)与反射光量之和决定。
它被形式化为\keyindex{光传输方程}{light transport equation}{light transport\ 光传输}
(也称作\keyindex{渲染方程}{rendering equation}{render\ 渲染}),
表示从点$\bm p$沿方向${\bm \omega}_\mathrm{o}$的
出射辐亮度$L_{\mathrm{o}}({\bm p},{\bm \omega}_\mathrm{o})$等于
该点沿该方向的发光亮度加上
点$\bm p$的邻域球面$S^2$所有方向上
经BSDF$f({\bm p},{\bm \omega}_\mathrm{o},{\bm \omega}_\mathrm{i})$和
余弦项调制的入射亮度:
\begin{align}
      L_{\mathrm{o}}({\bm p},{\bm \omega}_\mathrm{o})=L_{\mathrm{e}}({\bm p},{\bm \omega}_\mathrm{o})+\int_{S^2}f({\bm p},{\bm \omega}_\mathrm{o},{\bm \omega}_\mathrm{i})L_{\mathrm{i}}({\bm p},{\bm \omega}_\mathrm{i})|\cos{\theta_{\mathrm{i}}}| \,\mathrm{d}{\bm \omega}_\mathrm{i}\, .
      \label{eq:1.1}
\end{align}
\refsub{BRDF}和\refsec{光传输方程}将展示其更完整的推导。
除了最简单的场景,
解析地求解该方程是几乎不可能的,
所以必须简化假设或使用数值积分技术。

Whitted算法通过忽略绝大多数方向的入射光来简化积分,
只计算到光源方向以及
完美\keyindex{反射}{reflection}{}与\keyindex{折射}{refraction}{}方向
的$L_{\mathrm{i}}({\bm p},{\bm \omega}_\mathrm{i})$.
换句话说,它把积分变为少量方向上的求和。

Whitted的方法可以扩展到实现镜面和玻璃外的更多效果。
例如,追踪许多贴近\keyindex{镜面反射}{mirror reflection}{reflection\ 反射}方向
的递归光线并平均它们的作用,
可以近似得到\keyindex{光泽反射}{glossy reflection}{reflection\ 反射}。
事实上只要命中物体我们就\emph{一直}递归地追踪光线。
例如,随机选取反射方向${\bm \omega}_\mathrm{i}$并
计算BRDF$f_{\mathrm{r}}({\bm p},{\bm \omega}_\mathrm{o},{\bm \omega}_\mathrm{i})$对
这条新建光线赋权。
这个简单有效的办法可以得到非常逼真的图像,
因为它考虑了物体间的光所有的\keyindex{互反射}{interreflection}{reflection\ 反射}。
当然,我们需要知道何时停止递归,
何况完全随机选取方向会让渲染算法收敛到合理结果的速度变慢。
但是这些问题是可以解决的;
第\refchap{蒙特卡罗积分}和第\refchap{光传输III:双向方法}会对这些问题予以介绍。

当用该方法递归追踪光线时,
其实是把光线的一棵\keyindex{树}{tree}{}与图像的每个位置关联(\reffig{1.8}),
来自相机的射线是该树的根。
树中每条光线有关联的\keyindex{权重}{weight}{};
这允许我们对不反射100\%入射光的光滑表面建模。
\begin{figure}[htbp]
      \centering\input{Pictures/chap01/RayTree.tex}
      \caption{递归光线追踪与一整棵关于图像每处位置光线的树关联。}\label{fig:1.8}
\end{figure}

\subsection{光线传播}\label{sub:光线传播}

目前的讨论都假设光线是在\keyindex{真空}{vacuum}{}中传播的。
例如在描述点光源的光分布时,
我们假设光的功率朝着
以光源为球心的球面均匀发散且不随路线衰减
\sidenote{译者注:意思是没有传播介质吸收光的能量,
      和前文反比于半径平方不是一回事。}。
\keyindex{介质}{participating media}{}的出现,
例如烟、雾、尘会破坏该假设。
模拟这些效果很重要:
即使我们不渲染充满烟气的房间,
几乎所有室外场景也都会受到介质的实质影响。
例如,地球大气使得远处物体显得不够饱和(\reffig{1.9})。
\begin{figure}[htbp]
      \centering
      \subfloat[无大气散射]{\includegraphics[width=\linewidth]{chap01/ecosys-nofog.png}\label{fig:1.9.1}}\\
      \subfloat[有大气散射]{\includegraphics[width=\linewidth]{chap01/ecosys-fog.png}\label{fig:1.9.2}}
      \caption{地球大气随距离降低饱和度。
            (a)渲染场景没有模拟该现象,但(b)包含了大气模型。
            大气衰减程度是观察真实场景时重要的深度线索,
            为二次渲染增加了尺度感。}
      \label{fig:1.9}
\end{figure}

介质影响光沿路线传播的方式有两种。
第一种是介质可以通过吸收或沿不同方向散射
来\keyindex{熄灭}{extinguish}{}(或\keyindex{衰减}{attenuate}{})光。
可以通过计算射线端点与交点之间的\keyindex{透射率}{transmittance}{}$T$来
实现这一效果。
透射率表示交点处散射的光有多少成功到达射线端点。

介质也可以沿路线增强光。
在介质发光(例如火焰)或从其他方向把光散射回该射线时可发生该现象(\reffig{1.10})。
可以通过数值计算\keyindex{体积光传输方程}{volume light transport equation}{light transport\ 光传输}来寻求该量,
该方法还能计算光传输方程求得从表面反射回的光量。
介质的描述和体积渲染会留到第\refchap{体积散射}和\refchap{光传输II:体积渲染}。
现在我们就能计算介质效应并将其合并到光线所含的光量中了。
\begin{figure}[htbp]
      \centering
      \includegraphics[width=\linewidth]{chap01/spotfog.png}
      \caption{聚光灯通过雾气照在球上。
            注意因为介质增加了散射,聚光灯的光分布形状和球的阴影清晰可见。}
      \label{fig:1.10}
\end{figure}

\section{pbrt:系统概述}\label{sec:pbrt:系统概述}

pbrt是用标准的\keyindex{面向对象}{object-oriented}{}技术构建的:
重要实体都定义了抽象\keyindex{基类}{base class}{class\ 类}(如
抽象基类\refvar{Shape}{}定义了所有几何形状必须实现的接口,
光源的抽象基类\refvar{Light}{}也有相似设计)。
系统大部分都纯粹是由这些抽象基类提供的接口实现的;
如检查光源与着色点之间遮挡物体的代码
调用\refvar{Shape}{}的相交方法
而不用考虑场景中出现的特定类型的形状。
该方式让扩展系统变得很容易,
新增一种形状只需实现一个完成\refvar{Shape}{}接口的类并链接到系统。

pbrt用10个关键抽象基类写成,列于\reftab{1.1}。
向系统添加这些类的新实现很简单;
实现必须从适当的基类继承,
再编译和链接到可执行文件,
并且必须修改附录第\refchap{场景描述接口}中的对象创建例程
以创建解析场景描述文件所需要的对象。
\refsec{添加新对象的实现}
\sidenote{译者注:原书此处似乎链接错误,已纠正。}
将讨论这种扩展系统的方法的更多细节。

\begin{table}[htbp]
    \centering
    \begin{tabular}{l l l}
        \toprule
        \textbf{基类}         & \textbf{目录}           & \textbf{章节}                   \\
        \midrule
        \refvar{Shape}{}      & \ttfamily shapes/       & \refsec{基本形状接口}           \\
        \refvar{Aggregate}{}  & \ttfamily accelerators/ & \refsec{聚合}                   \\
        \refvar{Camera}{}     & \ttfamily cameras/      & \refsec{相机模型}               \\
        \refvar{Sampler}{}    & \ttfamily samplers/     & \refsec{采样接口}               \\
        \refvar{Filter}{}     & \ttfamily filters/      & \refsec{图像重建}               \\
        \refvar{Material}{}   & \ttfamily materials/    & \refsec{材质接口与实现}         \\
        \refvar{Texture}{}    & \ttfamily textures/     & \refsec{纹理接口与基本纹理}     \\
        \refvar{Medium}{}     & \ttfamily media/        & \refsec{介质}                   \\
        \refvar{Light}{}      & \ttfamily lights/       & \refsec{光源接口}               \\
        \refvar{Integrator}{} & \ttfamily integrators/  & \refsub{积分器接口与采样积分器} \\
        \bottomrule
    \end{tabular}
    \caption{主要接口类型。pbrt大部分由此处列出的10个关键抽象基类实现。
        每个的实现都很容易添加到系统中扩展其功能。}
    \label{tab:1.1}
\end{table}

pbrt源码发布于\href{https://pbrt.org/}{\ttfamily pbrt.org}
(大量场景示例\footnote{\url{https://pbrt.org/scenes-v3.html}}也可分开下载)。
所有的pbrt核心代码均在目录\href{https://github.com/mmp/pbrt-v3/tree/master/src/core}{\ttfamily src/core}内,
函数\refvar{main}{()}在
短文件\href{https://github.com/mmp/pbrt-v3/tree/master/src/main/pbrt.cpp}{\ttfamily main/pbrt.cpp}内。
抽象基类实例的各种实现在分开的目录下:
\href{https://github.com/mmp/pbrt-v3/tree/master/src/shapes}{\ttfamily src/shapes}
有基类\refvar{Shape}{}的实现,
\href{https://github.com/mmp/pbrt-v3/tree/master/src/materials}{\ttfamily src/materials}有基类\refvar{Material}{}的实现,以此类推。

本节有许多pbrt扩展版本渲染的图像。
其中从\reffig{1.11}到\reffig{1.14}
\sidenote{译者注:原书\reffig{1.14}引用文献似乎遗漏了链接,推测是\citet{10.1145/74333.74361}。}
都引人瞩目:
它们不仅令人过目不忘,而且每张都是渲染课程的学生在
最后的课程作业中为pbrt扩展新功能渲染得到的有趣图像。
这些是课程中最佳图像的一部分。
\begin{figure}[htbp]
    \centering\includegraphics[width=\linewidth]{chap01/nightsnow.jpg}
    \caption{Guillaume Poncin和Pramod Sharma用许多方法扩展了pbrt,
        实现了一系列复杂渲染算法,
        制作出这张斯坦福大学CS348b渲染竞赛获奖图像。
        树木由L系统程序化建模,
        辉光图像处理滤波器增加了树上灯光的真实感,
        雪由metaball程序化建模,
        次表面散射算法考虑了光在离开雪前在雪下传播了一段距离的影响,
        赋予了雪逼真的外观。}
    \label{fig:1.11}
\end{figure}
\begin{figure}[htbp]
    \centering\includegraphics[width=\linewidth]{chap01/icecave.png}
    \caption{Abe Davis、David Jacobs和Jongmin Baek渲染了这张惊艳的冰窟图像,
        夺得2009斯坦福大学CS348b渲染竞赛大奖。
        他们首先实现了对冰川作用,即雪多年落下、融化、再冻结形成分层冰层这一物理过程的仿真。
        然后他们模拟了融水径流对冰的侵蚀,生成了冰的几何模型。
        体积内的光散射由体积光子映射模拟;
        冰的蓝色完全取决于在冰体中对依赖于波长的光吸收的建模。}
    \label{fig:1.12}
\end{figure}
\begin{figure}[htbp]
    \centering\includegraphics[width=\linewidth]{chap01/cloth.png}
    \caption{Lingfeng Yang实现了双向纹理函数来模拟布料的外观,
        添加了解析的自阴影模型,
        渲染了这张2009斯坦福大学CS348b渲染竞赛一等奖图像。}
    \label{fig:1.13}
\end{figure}

\subsection{执行阶段}\label{sub:执行阶段}

pbrt在概念上可分为两个执行阶段。
首先,解析用户提供的场景描述文件。
场景描述是一个文本文件,
指定了构成场景的几何形状及其材质属性、
对其照明的光源、虚拟相机在场景中的摆放位置、
整个系统所用的各个算法的参数等。
输入文件的每个语句都直接映射到
附录第\refchap{场景描述接口}
\sidenote{译者注:原书此处似乎链接错误,已纠正。}
中的一个例程;
这些例程包含了描述场景的程序接口。
场景文件格式的文档详见pbrt网站\href{https://pbrt.org/}{\ttfamily pbrt.org}。

解析阶段的最终结果是类\refvar{Scene}{}和\refvar{Integrator}{}的实例。
\refvar{Scene}{}包含了场景内容(几何物体、光源等)的表示,
\refvar{Integrator}{}则实现渲染它的算法。称之为\keyindex{积分器}{integrator}{}就是因为
它的主要任务是计算\refeq{1.1}的积分。

一旦指定好场景,第二执行阶段就开始了,执行主渲染循环。
pbrt通常把绝大部分运行时间都花在这个阶段,
本书大部分都在讲解执行这一阶段的代码。
渲染循环由\refvar{Integrator::Render}{()}的
实现来执行,它是\refsub{主渲染循环}的重点。

本章将介绍\refvar{Integrator}{}的一个特定子类,
称为\refvar{SamplerIntegrator}{},
其{\ttfamily Render()}方法为大量建模成像过程的光线
确定到达虚拟胶片平面的光量。
在计算完所有这些胶片样本的贡献后,把最终图像写入文件。
内存里的场景描述数据被释放,
系统从场景描述文件恢复处理语句直到没有剩余,
如果需要,用户可以指定下一个要渲染的场景。

\begin{figure}[htbp]
    \centering\includegraphics[width=\linewidth]{chap01/furrydog.png}
    \caption{Jared Jacobs和Michael Turitzin为pbrt增加了
        \citet{10.1145/74333.74361}基于纹素的毛发渲染算法并渲染了该图像,
        狗毛和粗毛地毯都是用纹素毛发算法渲染的。}
    \label{fig:1.14}
\end{figure}

\reffig{1.15}和\reffig{1.16}由\emph{LuxCoreRender}渲染,
它是最初基于本书第一版pbrt源码的GPL许可的基于物理的渲染系统
(关于\emph{LuxCoreRender}的更多信息详见\url{https://luxcorerender.org}
\sidenote{译者注:原文的LuxRender现已更名为LuxCoreRender且
    迁移到了新网址,此处已更正。})。

\begin{figure}[htbp]
    \centering\includegraphics[width=\linewidth]{chap01/measure-one180-cut1260.png}
    \caption{Florent Boyer渲染了这个当代室内场景。
        该图像由\emph{LuxRender}渲染,它是最初
        基于pbrt源码的GPL许可的基于物理的渲染系统。
        建模和纹理由Blender完成。}
    \label{fig:1.15}
\end{figure}

\begin{figure}[htbp]
    \centering\includegraphics[width=\linewidth]{chap01/crown.png}
    \caption{Martin Lubich构建了这个奥地利皇冠的场景
        并用pbrt代码库的开源分支\emph{LuxRender}渲染了它。
        该场景由Blender建模,包含约一百八十万个顶点。
        它由具备基于真实世界光源测量数据的发射光谱的六个面光源照明,
        在四核CPU上对每个像素做1280次采样经73小时的计算完成渲染。
        包括可下载的Blender场景文件在内的更多信息
        详见Martin的网站\url{www.loramel.net}。}
    \label{fig:1.16}
\end{figure}

\subsection{场景表示}\label{sub:场景表示}
pbrt的函数\refvar{main}{()}可在
文件\href{https://github.com/mmp/pbrt-v3/tree/master/src/main/pbrt.cpp}{\ttfamily main/pbrt.cpp}内找到。
该函数很简单;
它首先循环读取{\ttfamily argv}中的命令行参数,
初始化结构体{\ttfamily Options}中的值并
保存参数中的文件名。
运行pbrt时带命令行参数{\ttfamily {-}{-}help}会
打印所有可指定的命令行选项。
解析命令行参数的代码片\refcode{Process command-line arguments}{}很简单,
故本书不再介绍\sidenote{译者注:我还是把它搬上来了。}。

选项结构体随后传入函数\refvar{pbrtInit}{()},做全系统初始化。
函数\refvar{main}{()}再解析给定场景描述,
创建\refvar{Scene}{}和\refvar{Integrator}{}。
\refvar{pbrtCleanup}{()}在系统完成所有渲染后退出前做最后的清理工作。

函数\refvar{pbrtInit}{()}和\refvar{pbrtCleanup}{()}出现在
页边空白处的迷你索引内,还注明了真正定义它们的页数。
每页的迷你索引都指向了所用的几乎所有函数、类、方法和成员变量的定义
\sidenote{译者注:在线版已经全部改用超链接了。翻译时尽力还原了在线版的跳转功能。试试看吧!}。
\begin{lstlisting}
`\initcode{Main program}{=}`
int `\initvar{main}{}`(int argc, char *argv[]) {
    Options options;
    std::vector<std::string> filenames;
    `\refcode{Process command-line arguments}{}`
    `\refvar{pbrtInit}{}`(options);
    `\refcode{Process scene description}{}`
    `\refvar{pbrtCleanup}{}`();
    return 0;
}
\end{lstlisting}
\begin{lstlisting}
`\initcode{Process command-line arguments}{=}`
for (int i = 1; i < argc; ++i) {
    if (!strcmp(argv[i], "--ncores") || !strcmp(argv[i], "--nthreads"))
        options.nThreads = atoi(argv[++i]);
    else if (!strcmp(argv[i], "--outfile")) options.imageFile = argv[++i];
    else if (!strcmp(argv[i], "--quick")) options.quickRender = true;
    else if (!strcmp(argv[i], "--quiet")) options.quiet = true;
    else if (!strcmp(argv[i], "--verbose")) options.verbose = true;
    else if (!strcmp(argv[i], "--help") || !strcmp(argv[i], "-h")) {
        printf("usage: pbrt [--nthreads n] [--outfile filename] [--quick] [--quiet] "
               "[--verbose] [--help] <filename.pbrt> ...\n");
        return 0;
    }
    else filenames.push_back(argv[i]);
}
\end{lstlisting}

如果运行pbrt时没有提供输入文件名,
则它会从标准输入读取场景描述。
否则它就遍历提供的文件名,依次处理每个文件。
\begin{lstlisting}
`\initcode{Process scene description}{=}`
if (filenames.size() == 0) {
    `\refcode{Parse scene from standard input}{}`
} else {
    `\refcode{Parse scene from input files}{}`
}
\end{lstlisting}
函数{\initvar{ParseFile}{()}}或从标准输入或磁盘文件读入并解析场景描述文件;
如果无法打开文件则返回{\ttfamily false}。
本书不介绍解析场景描述文件的机制;
解析器的实现可在文件\href{https://github.com/mmp/pbrt-v3/blob/master/src/core/parser.h}{\ttfamily core/parser.h}和
\href{https://github.com/mmp/pbrt-v3/blob/master/src/core/parser.cpp}{\ttfamily core/parser.cpp}中找到
\sidenote{译者注:此处已更正文件名,因为原书给出的文件名已经失效了。}。
想要了解解析子系统但不熟悉这些工具的读者可以参阅\citet{10.5555/136311}的著作。

我们遵循的UNIX习惯用法,以名为“{\ttfamily -}”的文件表示标准输入:
\begin{lstlisting}
`\initcode{Parse scene from standard input}{=}`
`\refvar{ParseFile}{}`("-");
\end{lstlisting}

如果无法打开特定的输入文件,则\refvar{Error}{()}例程会将此信息报告给用户。
\refvar{Error}{()}使用和{\ttfamily printf()}相同的格式化字符串语义。
\begin{lstlisting}
`\initcode{Parse scene from input files}{=}`
for (const std::string &f : filenames)
    if (!`\refvar{ParseFile}{}`(f))
        `\refvar{Error}{}`("Couldn't open scene file \"%s\"", f.c_str());
\end{lstlisting}

解析完场景文件后就创建表示场景中光源和几何图元的对象。
它们都存于\refvar{Scene}{}对象中,
由附录\refsub{世界端和渲染}的方法\refvar{RenderOptions::MakeScene}{()}创建。
类\refvar{Scene}{}在\href{https://github.com/mmp/pbrt-v3/tree/master/src/core/scene.h}{\ttfamily core/scene.h}中
声明并在\href{https://github.com/mmp/pbrt-v3/tree/master/src/core/scene.cpp}{\ttfamily core/scene.cpp}中定义。
\begin{lstlisting}
`\initcode{Scene Declarations}{=}`
class `\initvar{Scene}{}` {
public:
    `\refcode{Scene Public Methods}{}`
    `\refcode{Scene Public Data}{}`
private:
    `\refcode{Scene Private Data}{}`
};
\end{lstlisting}

\begin{lstlisting}
`\initcode{Scene Public Methods}{=}\initnext{ScenePublicMethods}`
`\refvar{Scene}{}`(std::shared_ptr<`\refvar{Primitive}{}`> `\refvar{aggregate}{}`,
      const std::vector<std::shared_ptr<`\refvar{Light}{}`>> &`\refvar{lights}{}`)
    : `\refvar{lights}{}`(`\refvar{lights}{}`), `\refvar{aggregate}{}`(`\refvar{aggregate}{}`) {
    `\refcode{Scene Constructor Implementation}{}`
}
\end{lstlisting}

场景中每个光源都由\refvar{Light}{}对象表示,
指定灯光的形状和发射能量的分布。
\refvar{Scene}{}用C++标准库中{\ttfamily shared\_ptr}实例
的一个{\ttfamily vector}来存储所有光源。
pbrt用共享指针\sidenote{译者注:原文shared pointers,是一种智能指针。}跟踪
其他实例对于对象的引用计数。
当最后一个持有引用的实例(例如这里的\refvar{Scene}{})被销毁时,
引用计数减到零,\refvar{Light}{}就可以安全释放了,而且是自动的。

尽管一些渲染器支持每个几何对象有单独的光源列表,
允许一个光源只照明场景中一部分物体,
但这种做法不符合pbrt采用的基于物理的渲染方法,
所以对于每个场景pbrt只支持单个全局列表。
系统的许多部分都需要获取光源,
所以\refvar{Scene}{}把它们置为公有成员变量。
\begin{lstlisting}
`\initcode{Scene Public Data}{=}`
std::vector<std::shared_ptr<`\refvar{Light}{}`>> `\initvar{lights}{}`;
\end{lstlisting}

场景中每个几何对象都由\refvar{Primitive}{}表示,结合了两个对象:
指定其几何结构的\refvar{Shape}{}和描述其外观
(例如物体的颜色、是否具有暗淡或光泽饰面)的\refvar{Material}{}。
所有几何图元都集中到\refvar{Scene}{}的
成员变量\refvar[aggregate]{Scene::aggregate}{}这一
单个\refvar{Primitive}{}聚合体中。
这个聚合体是一种特殊的图元,它自己持有许多对其他图元的引用。
因为它实现了\refvar{Primitive}{}接口,
所以从单个图元到系统其余部分似乎没什么区别。
聚合体的实现用加速的数据结构存储了所有场景图元,
减少对远离给定光线的图元做不必要的光线相交测试量。
\begin{lstlisting}
`\initcode{Scene Private Data}{=}\initnext{ScenePrivateData}`
std::shared_ptr<`\refvar{Primitive}{}`> `\initvar{aggregate}{}`;
\end{lstlisting}
构造函数把场景几何的边界框缓存到成员变量{\ttfamily worldBound}中。
\begin{lstlisting}
`\initcode{Scene Constructor Implementation}{=}\initnext{SceneConstructorImplementation}`
`\refvar{worldBound}{}` = aggregate->`\refvar[Primitive::WorldBound]{WorldBound}{}`();
\end{lstlisting}
\begin{lstlisting}
`\refcode{Scene Private Data}{+=}\lastcode{ScenePrivateData}`
`\refvar{Bounds3f}{}` `\initvar{worldBound}{}`;
\end{lstlisting}
可通过方法\refvar[Scene::WorldBound]{WorldBound}{()}获取该边界。
\begin{lstlisting}
`\refcode{Scene Public Methods}{+=}\lastcode{ScenePublicMethods}`
const `\refvar{Bounds3f}{}` &`\initvar[Scene::WorldBound]{WorldBound}{}`() const { return `\refvar{worldBound}{}`; }
\end{lstlisting}

一些光源实现发现在定义场景后开始渲染前做一些额外的初始化很有用。
\refvar{Scene}{}的构造函数调用其方法\refvar{Preprocess}{()}来允许它们这么做。
\begin{lstlisting}
`\refcode{Scene Constructor Implementation}{+=}\lastcode{SceneConstructorImplementation}`
for (const auto &light : `\refvar{lights}{}`)
    light->`\refvar{Preprocess}{}`(*this);
\end{lstlisting}

\refvar{Scene}{}类提供两个与光线-图元相交相关的方法。
它的\refvar[Scene::Intersect]{Intersect}{()}方法跟随给定光线到场景中并
返回表示光线是否与某一图元相交的布尔值。
如果是,它便把沿光线最近交点的信息填入提供的结构体\refvar{SurfaceInteraction}{}。
结构体\refvar{SurfaceInteraction}{}将在\refsec{图元接口与几何图元}定义。
\begin{lstlisting}
`\initcode{Scene Method Definitions}{=}\initnext{SceneMethodDefinitions}`
bool `\refvar{Scene}{}`::`\refvar[Primitive::Intersect]{\initvar[Scene::Intersect]{Intersect}{}}{}`(const `\refvar{Ray}{}` &ray, `\refvar{SurfaceInteraction}{}` *isect) const {
    return `\refvar{aggregate}{}`->`\refvar[Primitive::Intersect]{Intersect}{}`(ray, isect);
}
\end{lstlisting}
一个紧密相关的方法是\refvar{Scene::IntersectP}{()},
它沿光线检查相交的存在性但不返回任何关于这些相交处的信息。
因为这个例程不需要搜索最近的相交处或计算任何关于相交的额外信息,
所以它一般比\refvar{Scene::Intersect}{()}更高效。
这个例程用于阴影射线。
\begin{lstlisting}
`\refcode{Scene Method Definitions}{+=}\lastnext{SceneMethodDefinitions}`
bool `\refvar{Scene}{}`::`\refvar[Primitive::IntersectP]{\initvar[Scene::IntersectP]{IntersectP}{}}{}`(const `\refvar{Ray}{}` &ray) const {
    return `\refvar{aggregate}{}`->`\refvar[Primitive::IntersectP]{IntersectP}{}`(ray);
}
\end{lstlisting}

\subsection{积分器接口与采样积分器}\label{sub:积分器接口与采样积分器}
渲染一幅场景图像由实现了\refvar{Integrator}{}接口的类的实例负责。
抽象基类\refvar{Integrator}{}定义了任何积分器必须提供的方法\refvar[Integrator::Render]{Render}{()}。
本节我们将定义一个\refvar{Integrator}{}的实现,即\refvar{SamplerIntegrator}{}。
在\href{https://github.com/mmp/pbrt-v3/tree/master/src/core/integrator.h}{\ttfamily core/integrator.h}中
定义了基本积分器接口,
积分器使用的一些实用函数在\href{https://github.com/mmp/pbrt-v3/blob/master/src/core/integrator.cpp}{\ttfamily core/integrator.cpp}中。
不同积分器的实现在目录\href{https://github.com/mmp/pbrt-v3/tree/master/src/integrators}{\ttfamily integrators}中。
\begin{lstlisting}
`\initcode{Integrator Declarations}{=}`
class `\initvar{Integrator}{}` {
public:
    `\refcode{Integrator Interface}{}`
};
\end{lstlisting}

\refvar{Integrator}{}必须提供方法\refvar[Integrator::Render]{Render}{()};
它利用传入的\refvar{Scene}{}引用计算一幅场景图像,
或者更一般地说,一组场景光照的度量。
其接口专门保持高度一般性以便允许各种实现——
例如可以把\refvar{Integrator}{}实现成只度量
分布于场景中的一组稀疏位置,而不是生成常规的2D图像。
\begin{lstlisting}
`\initcode{Integrator Interface}{=}`
virtual void `\initvar[Integrator::Render]{Render}{}`(const `\refvar{Scene}{}` &scene) = 0;
\end{lstlisting}

本章中,我们的重点是\refvar{Integrator}{}的一个子类\refvar{SamplerIntegrator}{},
以及实现了\refvar{SamplerIntegrator}{}接口的\refvar{WhittedIntegrator}{}
(\refvar{SamplerIntegrator}{}
的其他实现会在第\refchap{光传输I:表面反射}和\refchap{光传输II:体积渲染}介绍;
第\refchap{光传输III:双向方法}的积分器直接从\refvar{Integrator}{}继承)。
\refvar{SamplerIntegrator}{}的名字源自其渲染过程是
由来自\refvar{Sampler}{}的\keyindex{样本}{sample}{}流驱动的;
每个这样的样本都标识了图像上的一点,
让积分器计算到达该点以构成图像的光量。
\begin{lstlisting}
`\initcode{SamplerIntegrator Declarations}{=}`
class `\initvar{SamplerIntegrator}{}` : public `\refvar{Integrator}{}` {
public:
    `\refcode{SamplerIntegrator Public Methods}{}`
protected:
    `\refcode{SamplerIntegrator Protected Data}{}`
private:
    `\refcode{SamplerIntegrator Private Data}{}`
};
\end{lstlisting}

\refvar{SamplerIntegrator}{}保存了指向\refvar{Sampler}{}的指针。
采样器看似存在感低,但它的实现会极大影响系统生成图像的质量。
首先,采样器负责选取光线要追踪的图像平面上的点。
其次,它负责为积分器提供用于计算光传输积分即\refeq{1.1}的值所需的采样位置。
例如,一些积分器需要对光源选取随机位置来计算来自面光源的照明。
生成分布良好的样本在渲染过程中很重要,会极大影响整体效率;
这是第\refchap{采样与重构}的主要内容。
\begin{lstlisting}
`\initcode{SamplerIntegrator Private Data}{=}`
std::shared_ptr<`\refvar{Sampler}{}`> `\initvar{sampler}{}`;
\end{lstlisting}

对象\refvar{Camera}{}控制视角和透镜参数,例如位置、朝向、焦点和视野。
\refvar{Camera}{}内的成员变量\refvar{Film}{}执行图像存储。
\refvar{Camera}{}类将在第\refchap{相机模型}介绍,
\refvar{Film}{}将在\refsec{胶片与成像管道}介绍。
\refvar{Film}{}负责把最终图像写入文件并可能在完成计算后在屏幕上将其显示出来。
\begin{lstlisting}
`\initcode{SamplerIntegrator Protected Data}{=}`
std::shared_ptr<const `\refvar{Camera}{}`> `\initvar{camera}{}`;
\end{lstlisting}

\refvar{SamplerIntegrator}{}构造函数在成员变量中保存了指向这些对象的指针。
它在\refvar{pbrtWorldEnd}{()}调用的
方法\refvar{RenderOptions::MakeIntegrator}{()}里创建,
而\refvar{pbrtWorldEnd}{()}在输入文件解析器
完成对输入文件的场景描述解析并准备渲染场景时被调用。
\begin{lstlisting}
`\initcode{SamplerIntegrator Public Methods}{=}\initnext{SamplerIntegratorPublicMethods}`
`\refvar{SamplerIntegrator}{}`(std::shared_ptr<const `\refvar{Camera}{}`> `\refvar{camera}{}`,
    std::shared_ptr<`\refvar{Sampler}{}`> `\refvar{sampler}{}`)
    : `\refvar{camera}{}`(`\refvar{camera}{}`), `\refvar{sampler}{}`(`\refvar{sampler}{}`) { }
\end{lstlisting}

\refvar{SamplerIntegrator}{}可选实现方法\refvar[SamplerIntegrator::Preprocess]{Preprocess}{()}。
它在\refvar{Scene}{}完全初始化后被调用,
并给积分器机会完成依赖于场景的计算,
例如分配依赖于场景光源数目的额外数据结构,
或者预计算场景中辐射分布的大致表示。
在这里不需要做任何事的实现可以将这个方法留为未实现状态。
\begin{lstlisting}
`\refcode{SamplerIntegrator Public Methods}{=}\lastnext{SamplerIntegratorPublicMethods}`
virtual void `\initvar[SamplerIntegrator::Preprocess]{Preprocess}{}`(const `\refvar{Scene}{}` &scene, `\refvar{Sampler}{}` &sampler) { }
\end{lstlisting}

\subsection{主渲染循环}\label{sub:主渲染循环}
完成\refvar{Scene}{}和\refvar{Integrator}{}的分配与初始化后,
调用方法\refvar{Integrator::Render}{()},
开始pbrt的第二执行阶段:主渲染循环。
在\refvar{SamplerIntegrator}{}对该方法的实现中,
对于图像平面上的每一个位置,
该方法用\refvar{Camera}{}和\refvar{Sampler}{}生成到场景中的光线,
再用方法{\ttfamily Li()}确定沿该光线到达图像平面的光量。
该值传入\refvar{Film}{},记录下光的贡献度。
\reffig{1.17}总结了该方法用到的主要类和它们之间的数据流。
\begin{figure}[htbp]
    \centering\input{Pictures/chap01/ClassRelationships.tex}
    \caption{\protect\href{https://github.com/mmp/pbrt-v3/tree/master/src/core/integrator.cpp}{\ttfamily core/integrator.cpp}里
    方法\protect\refvar{SamplerIntegrator::Render}{()}中主渲染循环的类关系。
    \protect\refvar{Sampler}{}提供了一采样值序列,每个图像样本取一个。
    \protect\refvar{Camera}{}把样本转为来自胶片平面的相应光线,
    方法{\ttfamily Li()}的实现计算沿光线到达胶片的辐亮度。
    样本和其辐亮度传给\protect\refvar{Film}{},
    保存它们在图像中的贡献。
    重复这个过程直到\protect\refvar{Sampler}{}提供足够多样本生成最终图像。}
    \label{fig:1.17}
\end{figure}
\begin{lstlisting}
`\initcode{SamplerIntegrator Method Definitions}{=}\initnext{SamplerIntegratorMethodDefinitions}`
void `\initvar{SamplerIntegrator::Render}{}`(const `\refvar{Scene}{}` &scene) {
    `\refvar[SamplerIntegrator::Preprocess]{Preprocess}{}`(scene, *sampler);
    `\refcode{Render image tiles in parallel}{}`
    `\refcode{Save final image after rendering}{}`
}
\end{lstlisting}

为了让多处理核系统能并行渲染,图像会被分解为像素小块。
每个图块可并行独立处理。
\refsec{并行化}详细介绍的函数\refvar{ParallelFor}{()}实现了并行{\ttfamily for}循环,
其中多个迭代可并行运行。
C++ lambda表达式提供循环体。
这里使用在2D域上循环的\refvar{ParallelFor}{()}的变体来在图块间迭代。
\begin{lstlisting}
`\initcode{Render image tiles in parallel}{=}`
`\refcode{Compute number of tiles, nTiles, to use for parallel rendering}{}`
`\refvar{ParallelFor2D}{}`(
    [&](`\refvar{Point2i}{}` tile) {
        `\refcode{Render section of image corresponding to tile}{}`
    }, `\refvar{nTiles}{}`);
\end{lstlisting}

决定图块要做多大时应权衡两点:负载平衡和每块的开销。
一方面,我们希望图块明显比系统的处理器多:
考虑四核计算机和仅仅四个图块。
一般某些图块耗时可能比其他的少;
负责场景相对简单的图像部分的图块通常比相对复杂部分的图块消耗更少处理时间。
因此,如果图块数目等于处理器数目,
一些处理器就会先完工并坐等负责最耗时图块的那个处理器。
\reffig{1.18}说明了该问题。
它展示了渲染\reffig{1.7}的光泽球场景时执行时间关于图块数的分布。
最长耗时是最短耗时的151倍。
\begin{figure}[htbp]
    \centering\input{Pictures/chap01/bucket-time.tex}
    \caption{渲染\protect\reffig{1.7}场景中每个图块的耗时直方图。横轴表示时间秒数。
        注意执行时间变动范围很宽,说明图像不同部分所需计算量差别极大。}
    \label{fig:1.18}
\end{figure}

另一方面,图块太小也会很低效。
处理核决定接下来该执行哪次迭代时会有一小部分固定开销;
图块越多,这类开销耗时就越多。

为了简化,pbrt固定使用$16\times16$的图块;
除了分辨率非常低的图像,
这个粒度\sidenote{译者注:原文granularity。}对绝大多数图像都适用。
我们隐含假设小图像的情况对于渲染的最大效率并不重要。
\refvar{Film}{}的方法\refvar{GetSampleBounds}{()}返回像素范围
以供在其上生成必需的样本来渲染图像。
计算\refvar{nTiles}{}时加上\refvar{tileSize}{-1}使得
许多图块在样本框沿某轴不能被16整除时向下一个更大整数取整。
这意味着\refvar{ParallelFor}{()}调用的lambda函数必须能
处理包含无用像素的局部图块。
\begin{lstlisting}
`\initcode{Compute number of tiles, nTiles, to use for parallel rendering}{=}`
`\refvar{Bounds2i}{}` sampleBounds = `\refvar{camera}{}`->`\refvar{film}{}`->`\refvar{GetSampleBounds}{}`();
`\refvar{Vector2i}{}` sampleExtent = sampleBounds.`\refvar{Diagonal}{}`();
const int `\initvar{tileSize}{}` = 16;
`\refvar{Point2i}{}` `\initvar{nTiles}{}`((sampleExtent.x + tileSize - 1) / tileSize,
               (sampleExtent.y + tileSize - 1) / tileSize);
\end{lstlisting}

当附录\refsub{并行的for循环}定义的并行{\ttfamily for}循环实现决定
在某一处理器上执行一次循环迭代时,
会带上图块的坐标调用lambda。
它启动时会做一点设置工作,决定负责哪部分胶片平面,
并在使用\refvar{Sampler}{}生成图像样本、
用\refvar{Camera}{}决定离开胶片平面的相应光线以及
用方法{\ttfamily Li()}计算沿这些光线到达胶片平面的辐亮度之前
为一些临时数据分配空间。
\begin{lstlisting}
`\initcode{Render section of image corresponding to tile}{=}`
`\refcode{Allocate MemoryArena for tile}{}`
`\refcode{Get sampler instance for tile}{}`
`\refcode{Compute sample bounds for tile}{}`
`\refcode{Get FilmTile for tile}{}`
`\refcode{Loop over pixels in tile to render them}{}`
`\refcode{Merge image tile into Film}{}`
\end{lstlisting}

方法{\ttfamily Li()}的实现一般需要为每次辐亮度计算临时分配少量内存。
系统常规内存分配例程(例如{\ttfamily malloc()}或{\ttfamily new})
必须维护和同步精巧的内部数据结构以跟踪处理器间的空闲内存区域集合,
而大量进行这样的分配容易让其崩溃。
简单的实现可能让大量计算时间花在内存分配器上。

为了解决这个问题,我们把类\refvar{MemoryArena}{}的一个实例传给方法{\ttfamily Li()}。
\refvar{MemoryArena}{}的实例管理内存池以启用比标准库例程更高性能的分配。

arena\sidenote{译者注:arena,竞技场。}内存池总是整体释放,
消解了对复杂内部数据结构的需求。
该类的实例只能用于单个线程——不允许没有额外同步的并行存取。
我们为每个循环迭代创建可直接使用的唯一\refvar{MemoryArena}{},
也保证了arena只被单个线程访问。
\begin{lstlisting}
`\initcode{Allocate MemoryArena for tile}{=}`
`\refvar{MemoryArena}{}` arena;
\end{lstlisting}

多数\refvar{Sampler}{}实现发现保留一些状态很有用,
例如当前渲染的像素坐标。
这意味着多个处理线程不能兼用单个\refvar{Sampler}{}。
因此,\refvar{Sampler}{}提供了方法\refvar{Clone}{()}来
创建给定\refvar{Sampler}{}的新实例;
它取用的种子也用于伪随机数生成器的一些实现,
这样每个图块就不会生成一样的伪随机数序列
(注意不是所有\refvar{Sampler}{}都使用伪随机数;
不需要时会忽略种子)。
\begin{lstlisting}
`\initcode{Get sampler instance for tile}{=}`
int seed = tile.y * nTiles.x + tile.x;
std::unique_ptr<`\refvar{Sampler}{}`> `\initvar{tileSampler}{}` = sampler->`\refvar{Clone}{}`(seed);
\end{lstlisting}

接下来,本次循环迭代的像素采样范围会基于图块索引计算出来。
该计算中必须考虑两个问题:
第一,要采样的全部像素边界可能不等于全图分辨率。
例如,用户可能指定采样一个像素子集的“裁窗”。
第二,如果图像分辨率不是16的倍数,则图像右边和底部的图块不是完整的$16\times16$.
\begin{lstlisting}
`\initcode{Compute sample bounds for tile}{=}`
int x0 = sampleBounds.pMin.x + tile.x * tileSize;
int x1 = std::min(x0 + tileSize, sampleBounds.pMax.x);
int y0 = sampleBounds.pMin.y + tile.y * tileSize;
int y1 = std::min(y0 + tileSize, sampleBounds.pMax.y);
`\refvar{Bounds2i}{}` `\initvar{tileBounds}{}`(`\refvar{Point2i}{}`(x0, y0), `\refvar{Point2i}{}`(x1, y1));
\end{lstlisting}

最后,从\refvar{Film}{}获取\refvar{FilmTile}{}。
该类提供了一个小内存缓冲区来存储当前图块的像素值。
它的存储对循环迭代是私有的,
所以可以更新像素值而不用担心其他线程同时修改同一像素。
一旦该图块完成渲染工作就被合并到胶片的存储中;
然后序列化对图像的并发更新。
\begin{lstlisting}
`\initcode{Get FilmTile for tile}{=}`
std::unique_ptr<`\refvar{FilmTile}{}`> `\initvar{filmTile}{}` =
    `\refvar{camera}{}`->`\refvar{film}{}`->`\refvar{GetFilmTile}{}`(`\refvar{tileBounds}{}`);
\end{lstlisting}

现在可以进行渲染了。
范围{\ttfamily for}循环自动使用
由类\refvar{Bounds2}{}提供的迭代器遍历图块中所有像素。
复制的\refvar{Sampler}{}会被通知要开始为当前像素生成样本了,
样本被依次处理直到\refvar{StartNextSample}{()}返回{\ttfamily false}
(如第\refchap{采样与重构}所述,每个像素取多个样本能大幅提高最终图像的质量)。

\begin{lstlisting}
`\initcode{Loop over pixels in tile to render them}{=}`
for (`\refvar{Point2i}{}` pixel : tileBounds) {
    tileSampler->`\refvar{StartPixel}{}`(pixel);
    do {
        `\refcode{Initialize CameraSample for current sample}{}`
        `\refcode{Generate camera ray for current sample}{}`
        `\refcode{Evaluate radiance along camera ray}{}`
        `\refcode{Add camera ray's contribution to image}{}`
        `\refcode{Free MemoryArena memory from computing image sample value}{}`
    } while (tileSampler->`\refvar{StartNextSample}{}`());
}
\end{lstlisting}

结构体\refvar{CameraSample}{}记录了相机应为之生成光线的相应胶片位置。
它还存储了时间和透镜位置采样值,
分别用于渲染移动物体场景和模拟非针孔光圈的相机模型。
\begin{lstlisting}
`\initcode{Initialize CameraSample for current sample}{=}`
`\refvar{CameraSample}{}` cameraSample = tileSampler->`\refvar{GetCameraSample}{}`(pixel);
\end{lstlisting}

\refvar{Camera}{}接口提供了两种方法生成光线:\refvar[GenerateRay]{Camera::GenerateRay}{()}返回
给定图像采样位置的光线,\refvar[GenerateRayDifferential]{Camera::GenerateRayDifferential}{()}返回\keyindex{射线差分}{ray differential}{},
它合并了\refvar{Camera}{}为那些在$x$和$y$方向都偏离图像平面一像素的样本生成的光线信息。
射线差分用于第\refchap{纹理}一些纹理函数以获取更好的结果,
让计算纹理随像素空间变化有多快成为可能,这是纹理抗锯齿的关键因素。

返回射线差分后,方法\refvar{ScaleDifferentials}{()}对其缩放
以兼顾每个像素取多个样本时胶片平面上样本之间的实际间距。

相机还返回与光线关联的浮点权重。
对于单个相机模型,每条光线都平等赋权,
但更精确建模透镜系统成像过程的相机模型生成的一些光线可能比另一些作用更大。
这样的相机模型也许模拟了到达胶片平面边缘的光比中心更少的效应,称为\keyindex{渐晕}{vignetting}{}。
返回的权重之后会用于缩放光线对图像的作用。
\begin{lstlisting}
`\initcode{Generate camera ray for current sample}{=}`
`\refvar{RayDifferential}{}` ray;
`\refvar{Float}{}` rayWeight = `\refvar{camera}{}`->`\refvar{GenerateRayDifferential}{}`(cameraSample, &ray);
ray.`\refvar{ScaleDifferentials}{}`(1 / std::sqrt(tileSampler->`\refvar{samplesPerPixel}{}`));
\end{lstlisting}

注意大写浮点类型\refvar{Float}{}:它取决于pbrt的编译选项,
是{\ttfamily float}或{\ttfamily double}的别名。
\refsec{主要包含文件}提供了这些设计选项的更多细节。

给定一条光线,下一个任务是确定沿其到达图像平面的辐亮度。
方法\refvar{Li}{()}负责这项任务。
\begin{lstlisting}
`\initcode{Evaluate radiance along camera ray}{=}`
`\refvar{Spectrum}{}` L(0.f);
if (rayWeight > 0)
    L = `\refvar{Li}{}`(ray, scene, *tileSampler, arena);
`\refcode{Issue warning if unexpected radiance value is returned}{}`
\end{lstlisting}
\begin{lstlisting}
`\initcode{Issue warning if unexpected radiance value is returned}{=}`
if (L.HasNaNs()) {
    `\refvar{Error}{}`("Not-a-number radiance value returned "
    "for image sample. Setting to black.");
    L = `\refvar{Spectrum}{}`(0.f);
}
else if (L.y() < -1e-5) {
    `\refvar{Error}{}`("Negative luminance value, %f, returned "
    "for image sample. Setting to black.", L.y());
    L = `\refvar{Spectrum}{}`(0.f);
}
else if (std::isinf(L.y())) {
    `\refvar{Error}{}`("Infinite luminance value returned "
    "for image sample. Setting to black.");
    L = `\refvar{Spectrum}{}`(0.f);
}
\end{lstlisting}
\refvar{Li}{()}是纯虚方法,返回给定光线起点的入射量;
每个\refvar{SamplerIntegrator}{}的子类必须提供该方法的一个实现。
\refvar{Li}{()}的参数有:
\begin{itemize}
    \item {\ttfamily ray}:计算入射量所沿的光线。
    \item {\ttfamily scene}:要渲染的\refvar{Scene}{}。实现会向场景索取关于光照和几何等的信息。
    \item {\ttfamily sampler}:通过蒙特卡罗积分求解光传输方程的样本生成器。
    \item {\ttfamily arena}:为积分器高效临时分配内存的\refvar{MemoryArena}{}。
          积分器应假设它用{\ttfamily arena}分配的任何内存都会在方法\refvar{Li}{()}返回后不久就被释放,
          因此不应使用{\ttfamily arena}分配任何超出当前光线所需的驻留更久的内存。
    \item {\ttfamily depth}:光线从相机出发直到当前调用\refvar{Li}{()}之间已反射的次数。
\end{itemize}

该方法返回的\refvar{Spectrum}{}表示光线起点的入射量:
\begin{lstlisting}
`\refcode{SamplerIntegrator Public Methods}{+=}\lastcode{SamplerIntegratorPublicMethods}`
virtual `\refvar{Spectrum}{}` `\initvar{Li}{}`(const `\refvar{RayDifferential}{}` &ray, const `\refvar{Scene}{}` &scene,
    `\refvar{Sampler}{}` &sampler, `\refvar{MemoryArena}{}` &arena, int depth = 0) const = 0;
\end{lstlisting}

渲染过程中一个常见bug是计算出不可能的辐亮度。
例如,除以零得到的辐亮度值等于IEEE浮点无穷
\sidenote{译者注:可搜索IEEE Standard for Floating-Point Arithmetic或IEEE 754了解详情。}
或“not a number”值。
渲染器监控这种和具有负数贡献光谱等可能的情况,
并在遇到时打印错误信息。
此处我们不介绍执行这些的
代码片\refcode{Issue warning if unexpected radiance value is returned}{}
\sidenote{译者注:我把它补充上来了。后续被原文省略的代码如果被补充上来,我就删掉这样的表述了。}。
如果你对其细节感兴趣可以查看\href{https://github.com/mmp/pbrt-v3/tree/master/src/core/integrator.cpp}{\ttfamily core/integrator.cpp}里的实现。

知道到达光线原点的辐亮度后,就能更新图像了:
方法\refvar[AddSample]{FilmTile::AddSample}{()}
依据样本的结果更新图块里的像素。
样本值如何记录到胶片的细节将在\refsec{图像重建}和\refsec{胶片与成像管道}解释。
\begin{lstlisting}
`\initcode{Add camera ray's contribution to image}{=}`
filmTile->`\refvar{AddSample}{}`(cameraSample.`\refvar{pFilm}{}`, L, rayWeight);
\end{lstlisting}

在处理完一个样本后,\refvar{MemoryArena}{}内所有已经分配过的内存都会在调用
\refvar[Reset]{MemoryArena::Reset}{()}
时一起释放(关于如何用\refvar{MemoryArena}{}分配内存
来表示交点处BSDF的解释详见\refsub{BSDF内存管理})。
\begin{lstlisting}
`\initcode{Free MemoryArena memory from computing image sample value}{=}`
arena.`\refvar{Reset}{}`();
\end{lstlisting}

一旦计算完图块内所有样本的辐亮度值,
\refvar{FilmTile}{}就会被传递给\refvar{Film}{}的方法\refvar{MergeFilmTile}{()},
它负责把图块的像素贡献加到最终图像上。
注意函数{\ttfamily std::move()}用于把{\ttfamily unique\_ptr}的所有权
转让给\refvar{MergeFilmTile}{()}。
\begin{lstlisting}
`\initcode{Merge image tile into Film}{=}`
`\refvar{camera}{}`->`\refvar{film}{}`->`\refvar{MergeFilmTile}{}`(std::move(filmTile));
\end{lstlisting}

所有循环迭代完成后,\refvar{SamplerIntegrator}{}的
方法\refvar[SamplerIntegrator::Render]{Render}{()}调用\refvar{Film}{}的
方法\refvar[Film::WriteImage]{WriteImage}{()}将图像输出写入到文件。
\begin{lstlisting}
`\initcode{Save final image after rendering}{=}`
`\refvar{camera}{}`->`\refvar{film}{}`->`\refvar[Film::WriteImage]{WriteImage}{}`();
\end{lstlisting}

\subsection{Whitted光线追踪积分器}\label{sub:Whitted光线追踪积分器}
第\refchap{光传输I:表面反射}和\refchap{光传输II:体积渲染}包含了
许多基于不同精度层级的各种算法的不同积分器实现。
这里我们将介绍一种基于Whitted光线追踪算法的积分器。
该积分器精确计算来自镜面表面例如玻璃、镜子和水的反射和透射光,
但它不考虑其他类型的间接光效应例如光从墙面反射照亮房间。
类\refvar{WhittedIntegrator}{}可在pbrt发行
文件\href{https://github.com/mmp/pbrt-v3/tree/master/src/integrators/whitted.h}{\ttfamily integrators/whitted.h}和
\href{https://github.com/mmp/pbrt-v3/tree/master/src/integrators/whitted.cpp}{\ttfamily integrators/whitted.cpp}中找到。
\begin{lstlisting}
`\initcode{WhittedIntegrator Declarations}{=}`
class `\initvar{WhittedIntegrator}{}` : public `\refvar{SamplerIntegrator}{}` {
public:
    `\refcode{WhittedIntegrator Public Methods}{}`
private:
    `\refcode{WhittedIntegrator Private Data}{}`
};
\end{lstlisting}
\begin{lstlisting}
`\initcode{WhittedIntegrator Public Methods}{=}`
`\refvar{WhittedIntegrator}{}`(int maxDepth, std::shared_ptr<const `\refvar{Camera}{}`> camera,
    std::shared_ptr<`\refvar{Sampler}{}`> sampler)
    : `\refvar{SamplerIntegrator}{}`(camera, sampler), `\refvar{maxDepth}{}`(maxDepth) { }
\end{lstlisting}

Whitted积分器工作时递归地计算沿反射和折射光线方向的辐亮度。
它在达到预设最大深度\refvar[maxDepth]{WhittedIntegrator::maxDepth}{}时停止递归。
默认最大递归深度为5.
没有这项终止准则,递归可能无法停止(例如想象一下镜厅场景)。
这个成员变量在\refvar{WhittedIntegrator}{}构造函数内
基于场景描述文件设置的参数进行初始化。
\begin{lstlisting}
`\initcode{WhittedIntegrator Private Data}{=}`
const int `\initvar{maxDepth}{}`;
\end{lstlisting}

如同\refvar{SamplerIntegrator}{}的实现,
\refvar{WhittedIntegrator}{}必须提供方法\refvar{Li}{()}的一种实现,
返回到达给定光线起点的辐亮度。
\reffig{1.19}总结了在表面积分期间所用的主要类间的数据流。
\begin{figure}[htbp]
    \centering\input{Pictures/chap01/SurfaceIntegrationClassRelationships.tex}
    \caption{表面积分类间关系。
        \protect\refvar{SamplerIntegrator}{}中的主渲染循环
        计算一条相机光线并将其传给方法\protect\refvar{Li}{()},
        返回沿该光线到达光线起点的辐亮度。
        找到最近相交处后,它计算交点处的材料属性,以BSDF的形式表示它们。
        然后再用场景中的灯光来确定照明。
        同样,它们给出了计算交点处沿光线反射回去的辐亮度所需的信息。}
    \label{fig:1.19}
\end{figure}
\begin{lstlisting}
`\initcode{WhittedIntegrator Method Definitions}{=}`
`\refvar{Spectrum}{}` `\initvar{WhittedIntegrator::Li}{}`(const `\refvar{RayDifferential}{}` &ray,
    const `\refvar{Scene}{}` &scene, `\refvar{Sampler}{}` &sampler, `\refvar{MemoryArena}{}` &arena,
    int depth) const {
    `\refvar{Spectrum}{}` L(0.);
    `\refcode{Find closest ray intersection or return background radiance}{}`
    `\refcode{Compute emitted and reflected light at ray intersection point}{}`
    return L;
}
\end{lstlisting}

第一步是找到光线与场景中形状的首个相交处。方法
\refvar{Scene::Intersect}{()}
取一条光线并返回表示它是否和形状相交的布尔值。
对于找到相交处的光线,它用相交处的几何信息
初始化提供的\refvar{SurfaceInteraction}{}。

如果没有找到相交,那可能因为光源没有关联几何体而就沿光线携带辐射。
这类光源的一个例子是\refvar{InfiniteAreaLight}{},
它能表示来自天空的光照。
方法\refvar{Light::Le}{()}允许这样的光源返回其沿给定光线的辐亮度。
\begin{lstlisting}
`\initcode{Find closest ray intersection or return background radiance}{=}`
`\refvar{SurfaceInteraction}{}` isect;
if (!scene.`\refvar[Scene::Intersect]{Intersect}{}`(ray, &isect)) {
    for (const auto &light : scene.`\refvar{lights}{}`)
        L += light->`\refvar[Light::Le]{Le}{}`(ray);
    return L;
}
\end{lstlisting}

否则就找到了可行的相交处。
积分器必须确定光在交点处如何被形状表面散射,
确定有多少光照从光源到达该点,
并用\refeq{1.1}的近似计算有多少光沿视察方向离开表面。
因为这个积分器忽略了诸如烟或雾等介质的影响,
所以离开交点的辐亮度和到达该光线起点的辐亮度相同。
\begin{lstlisting}
`\initcode{Compute emitted and reflected light at ray intersection point}{=}`
`\refcode{Initialize common variables for Whitted integrator}{}`
`\refcode{Compute scattering functions for surface interaction}{}`
`\refcode{Compute emitted light if ray hit an area light source}{}`
`\refcode{Add contribution of each light source}{}`
if (depth + 1 < `\refvar{maxDepth}{}`) {
    `\refcode{Trace rays for specular reflection and refraction}{}`
}
\end{lstlisting}

\reffig{1.20}展示了在接下来的代码片中经常用到的一些量。
$\bm n$是交点处的表面法向量,
从命中点回指向光线起点的规范化方向存于${\bm \omega}_\mathrm{o}$中;
\protect\refvar{Camera}{}负责规范化生成光线的方向分量,
所以这里无需再规范化了。
本书中规范化后的方向用符号$\bm \omega$表示,
在pbrt代码中我们用{\ttfamily wo}表示散射光的出射方向${\bm \omega}_\mathrm{o}$.
\begin{figure}[htbp]
    \centering\input{Pictures/chap01/Whittedintegrationsetting.tex}
    \caption{Whitted积分器的几何设置。
        $\bm p$是光线交点,$\bm n$是其表面法向量。
        我们要计算反射辐亮度的方向为${\bm \omega}_\mathrm{o}$;
        它是指向与入射光相反方向的向量。}
    \label{fig:1.20}
\end{figure}
\begin{lstlisting}
`\initcode{Initialize common variables for Whitted integrator}{=}`
`\refvar{Normal3f}{}` n = isect.`\refvar{shading}{}`.`\refvar[shading::n]{n}{}`;
`\refvar{Vector3f}{}` wo = isect.`\refvar[Interaction::wo]{wo}{}`;
\end{lstlisting}

如果找到相交处,就需要确定表面材质如何散射光照。
负责这项任务的是方法\refvar[SurfaceInteraction::ComputeScatteringFunctions]{ComputeScatteringFunctions}{()},
它求取纹理函数来确定表面属性
然后再初始化该点处BSDF(也可能是BSSRDF)的表示。
该方法一般需要为构成该表示的对象分配内存;
因为这些内存只需要对当前光线可用,
所以为其提供\refvar{MemoryArena}{}完成分配。
\begin{lstlisting}
`\initcode{Compute scattering functions for surface interaction}{=}`
isect.`\refvar[SurfaceInteraction::ComputeScatteringFunctions]{ComputeScatteringFunctions}{}`(ray, arena);
\end{lstlisting}

万一光线命中自发光几何体(例如面光源),
积分器会通过调用方法
\refvar{SurfaceInteraction::Le}{()}计算所有出射辐亮度。
这给出了\refeq{1.1}光传输方程的第一项。
如果物体不发光,该方法则返回黑光谱。
\begin{lstlisting}
`\initcode{Compute emitted light if ray hit an area light source}{=}`
L += isect.`\refvar[SurfaceInteraction::Le]{Le}{}`(wo);
\end{lstlisting}

对于每个光源,积分器调用方法\refvar[SampleLi]{Light::Sample\_Li}{()}计算
从该光源落到表面上待着色点的辐亮度。
该方法也返回从待着色点指向光源的方向向量存于变量{\ttfamily wi}中
(表示入射方向${\bm \omega}_\mathrm{i}$)
\footnote{当考虑表面位置上的光散射时,pbrt遵循的惯例
    是${\bm \omega}_\mathrm{i}$总表示关心的量(这里是辐亮度)到来的方向,
    而不是\refvar{Integrator}{}接近表面的方向。}。


该方法返回的光谱不考虑其他一些形状可能挡住来自光源的光照阻止其到达待着色点的可能性。
代替的是,它返回一个\refvar{VisibilityTester}{}对象
来确定是否有任何图元挡住从光源到表面点的路线。
它通过追踪待着色点和光源之间的阴影射线验证路径是畅通的来完成该测试。
pbrt的代码是按该方式组织的,
所以它能避免追踪不必要的阴影射线:
这样能首先保证如果光没被遮挡
那落到表面的光\emph{将会}沿方向${\bm \omega}_\mathrm{o}$散射。
例如,如果表面不是透射的,
那到达表面背面的光对反射无作用。

方法\refvar[SampleLi]{Light::Sample\_Li}{()}也在变量{\ttfamily pdf}里
为采样方向${\bm \omega}_\mathrm{i}$的光返回了概率密度。
该值用于复杂面光源的蒙特卡罗积分,
那时光是从多个方向到达一点的,尽管这里只采样一个方向;
对于点光源那般的简单光源,{\ttfamily pdf}的值为1.
如何在渲染中算出和使用这个概率密度的细节
是第\refchap{蒙特卡罗积分}和\refchap{光传输I:表面反射}的主题;
最后,光的作用必须除以{\ttfamily pdf},这由此处的实现完成。

如果达到的辐亮度非零且BSDF表明
一些从方向${\bm \omega}_\mathrm{i}$来的入射光确实
散射到了出射方向${\bm \omega}_\mathrm{o}$,
则积分器将辐亮度值$L_\mathrm{i}$乘以BSDF即$f$的值以及余弦项。
余弦项用函数\refvar{AbsDot}{()}计算,
它返回两向量点积的绝对值。
如果向量规范化了,例如此处的${\bm \omega}_\mathrm{i}$和$\bm n$,
则结果等于两者间夹角的余弦绝对值(\refsub{点积与叉积})。

该积表示该光对\refeq{1.1}光传输方程积分的贡献,
并被加到总反射辐亮度$L$上。
考虑了所有光照后,积分器便算出了\emph{直接光照}的总量——
从发光物体直接到达表面的光
(相对于场景中从其他物体反射过来到达该点的光而言)。

\begin{lstlisting}
`\initcode{Add contribution of each light source}{=}`
for (const auto &light : scene.`\refvar{lights}{}`) {
    `\refvar{Vector3f}{}` wi;
    `\refvar{Float}{}` pdf;
    `\refvar{VisibilityTester}{}` visibility;
    `\refvar{Spectrum}{}` Li = light->`\refvar[SampleLi]{Sample\_Li}{}`(isect, sampler.`\refvar{Get2D}{}`(), &wi,
        &pdf, &visibility);
    if (Li.`\refvar{IsBlack}{}`() || pdf == 0) continue;
    `\refvar{Spectrum}{}` f = isect.`\refvar{bsdf}{}`->`\refvar[BSDF::f]{f}{}`(wo, wi);
    if (!f.`\refvar{IsBlack}{}`() && visibility.`\refvar{Unoccluded}{}`(scene))
        L += f * Li * `\refvar{AbsDot}{}`(wi, n) / pdf;
}
\end{lstlisting}

该积分器也处理诸如镜子或玻璃等完美镜面散射的光。
它非常简单地利用镜面性质寻找反射方向(\reffig{1.21})
并用\keyindex{斯涅尔定律}{Snell's law}{}\sidenote{译者注:即折射定律,可参考译者补充的\refsub{光学背景知识}。}寻找
折射方向(\refsec{镜面反射与透射})。
然后积分器会递归地跟随新方向的适当光线
并将其贡献加到最初从相机看到的该点处反射辐亮度中。
镜面反射效应和透射的计算是分开的实用方法处理的,
因此这些函数易于被其他\refvar{SamplerIntegrator}{}实现再利用。
\begin{figure}[htbp]
    \centering\input{Pictures/chap01/Perfectspecularreflection.tex}
    \caption{因为是完美镜面反射故反射光线与曲面法线的夹角与入射光线相等。}
    \label{fig:1.21}
\end{figure}
\begin{lstlisting}
`\initcode{Trace rays for specular reflection and refraction}{=}`
L += `\refvar{SpecularReflect}{}`(ray, isect, scene, sampler, arena, depth);
L += `\refvar{SpecularTransmit}{}`(ray, isect, scene, sampler, arena, depth);
\end{lstlisting}
\begin{lstlisting}
`\refcode{SamplerIntegrator Method Definitions}{+=}\lastcode{SamplerIntegratorMethodDefinitions}`
`\refvar{Spectrum}{}` `\refvar{SamplerIntegrator}{}`::`\initvar{SpecularReflect}{}`(const `\refvar{RayDifferential}{}` &ray,
const `\refvar{SurfaceInteraction}{}` &isect, const `\refvar{Scene}{}` &scene,
`\refvar{Sampler}{}` &sampler, `\refvar{MemoryArena}{}` &arena, int depth) const {
    `\refcode{Compute specular reflection direction wi and BSDF value}{}`
    `\refcode{Return contribution of specular reflection}{}`
}
\end{lstlisting}

方法\refvar{SpecularReflect}{()}和\refvar{SpecularTransmit}{()}中,方法
\refvar[BSDF::Samplef]{BSDF::Sample\_f}{()}
对给定出射方向返回入射光线方向并给出光散射的方式。
该方法是本书最后几章所述的蒙特卡罗光传输算法的基础之一。
这里我们只用它寻找关于完美镜面反射或折射的出射方向,
并用标志指示\refvar[BSDF::Samplef]{BSDF::Sample\_f}{()}忽略其他类型的反射。
尽管\refvar[BSDF::Samplef]{BSDF::Sample\_f}{()}能
为概率积分算法采样离开表面的随机方向,
但随机性被约束为和\refvar{BSDF}{}的散射属性一致。
在完美镜面反射或折射中,只有一个可能的方向,所以根本没有随机性。

在这些函数中调用\refvar{BSDF}{}会用所选方向初始化{\ttfamily wi}
并返回方向对$({\bm \omega}_\mathrm{o},{\bm \omega}_\mathrm{i})$的BSDF值。
如果BSDF值非零,则积分器用方法\refvar[Li]{SamplerIntegrator::Li}{()}获取
沿${\bm \omega}_\mathrm{i}$的入射辐亮度,
这里是依次解析为方法\refvar{WhittedIntegrator::Li}{()}。
\begin{lstlisting}
`\initcode{Compute specular reflection direction wi and BSDF value}{=}`
`\refvar{Vector3f}{}` wo = isect.`\refvar[Interaction::wo]{wo}{}`, wi;
`\refvar{Float}{}` pdf;
`\refvar{BxDFType}{}` type = `\refvar{BxDFType}{}`(`\refvar[BSDFREFLECTION]{BSDF\_REFLECTION}{}` | `\refvar[BSDFSPECULAR]{BSDF\_SPECULAR}{}`);
`\refvar{Spectrum}{}` f = isect.bsdf->`\refvar[BSDF::Samplef]{Sample\_f}{}`(wo, &wi, sampler.`\refvar{Get2D}{}`(), &pdf, type);
\end{lstlisting}

为了用射线差分对反射或折射看到的纹理做抗锯齿,
需要知道反射和透射是如何影响屏幕空间光线的覆盖区的。
之后的\refsub{镜面反射和透射的射线差分}会定义
为这些光线计算射线差分的代码片。
给定完全初始化后的射线差分,递归调用\refvar{Li}{()}得到入射辐亮度,
并依据\refeq{1.1}用BSDF值和余弦项缩放再除以PDF。
\begin{lstlisting}
`\initcode{Return contribution of specular reflection}{=}`
const `\refvar{Normal3f}{}` &ns = isect.`\refvar{shading}{}`.`\refvar[shading::n]{n}{}`;
if (pdf > 0 && !f.`\refvar{IsBlack}{}`() && `\refvar{AbsDot}{}`(wi, ns) != 0) {
    `\refcode{Compute ray differential rd for specular reflection}{}`
    return f * `\refvar{Li}{}`(rd, scene, sampler, arena, depth + 1) * `\refvar{AbsDot}{}`(wi, ns) /
        pdf;
}
else
    return `\refvar{Spectrum}{}`(0.f);
\end{lstlisting}

方法{\initvar{SpecularTransmit}{()}}本质上和\refvar{SpecularReflect}{()}相同,
只是如果有的话,比起\refvar{SpecularReflect}{()}使用\refvar[BSDFREFLECTION]{BSDF\_REFLECTION}{}分量,
它只需要BSDF的\refvar[BSDFTRANSMISSION]{BSDF\_TRANSMISSION}{}光谱分量。
因此这里我们不再赘述它的实现了。

\input{content/chap0104.tex}

\input{content/chap0105.tex}

\input{content/chap0106.tex}

\input{content/chap0107.tex}

\input{content/chap0108.tex}

\input{content/chap0109.tex}