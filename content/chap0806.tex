\section{傅里叶基BSDF}\label{sec:傅里叶基BSDF}
尽管像Torrance-Sparrow和Oren-Nayar这样的模型可以准确地表示许多材料,
但一些材料所拥有的BRDF形态并不能很好地匹配这些模型
(例如带有光滑或粗糙涂料或面料的金属等分层材料,它们通常是部分反光的)。
针对这类材料的一个办法是把它们的BSDF值存进一个
巨大的3D或4D\keyindex{查找表}{lookup table}{},
但是该方法需要难以接受的存储量——例如,
如果${\bm\omega}_{\mathrm{i}}$和${\bm\omega}_{\mathrm{o}}$都按1度间隔
在球面角内采样,则以4D查找表的形式表示相应各向异性BSDF需要超过十亿个样本点。

因此,我们很需要有一个更紧凑的方式来依然准确表示BSDF。
本节介绍的\refvar{FourierBSDF}{}利用傅里叶基,
以缩放的余弦项之和表示BSDF。该表示既准确又节约空间,并能与蒙特卡罗积分很好地配合
(见\refsub{傅里叶BSDF})。\reffig{8.24}展示了用这种表示渲染的两个龙模型的例子。
\begin{figure}[htbp]
    \centering
    \includegraphics[width=\linewidth]{Pictures/chap08/dragons-fourier.png}
    \caption{用\refvar{FourierBSDF}{}模型渲染的龙模型。
        左边龙表面的BSDF建模了粗糙黄金的外观;右边的则是镀铜(感谢Christian Schüller提供模型)。}
    \label{fig:8.24}
\end{figure}

这里我们不会讨论BSDF是如何转换成这种表示的,但我们会关注它在渲染中的应用。
详见本章末的“扩展阅读”一节了解这些问题的更多细节以及
pbrt发行中路径\href{https://pbrt.org/scenes-v3}{\ttfamily bsdfs}下各种以此格式表示的BSDF。

\refvar{FourierBSDF}{}用一对入射和出射方向的球面坐标对BSDF进行参数化来表示各向同性的BSDF,
其中$\mu_{\mathrm{i}}$和$\mu_{\mathrm{o}}$分别表示入射和出射天顶角的余弦,
$\varphi_{\mathrm{i}}$和$\varphi_{\mathrm{o}}$是方位角:
\begin{align*}
    f({\bm\omega}_{\mathrm{i}},{\bm\omega}_{\mathrm{o}})
    =f(\mu_{\mathrm{i}},\varphi_{\mathrm{i}},\mu_{\mathrm{o}},\varphi_{\mathrm{o}})\, .
\end{align*}
各向同性的假设意味着该函数可以重写为更简单的依赖于天顶角余弦
和方位角之差$\varphi=\varphi_{\mathrm{i}}-\varphi_{\mathrm{o}}$的形式:
\begin{align*}
    f({\bm\omega}_{\mathrm{i}},{\bm\omega}_{\mathrm{o}})
    =f(\mu_{\mathrm{i}},\mu_{\mathrm{o}},\varphi_{\mathrm{i}}-\varphi_{\mathrm{o}})
    =f(\mu_{\mathrm{i}},\mu_{\mathrm{o}},\varphi)\, .
\end{align*}
各向同性BSDF通常也是关于方位角的偶函数,即:
\begin{align}\label{eq:8.20}
    f(\mu_{\mathrm{i}},\mu_{\mathrm{o}},\varphi)=f(\mu_{\mathrm{i}},\mu_{\mathrm{o}},-\varphi)\, .
\end{align}
有了这些性质,BSDF和余弦衰减的乘积就可以表示为关于方位角之差的傅里叶级数:
\begin{align}\label{eq:8.21}
    f(\mu_{\mathrm{i}},\mu_{\mathrm{o}},\varphi_{\mathrm{i}}-\varphi_{\mathrm{o}})|\mu_{\mathrm{i}}|
    =\sum\limits_{k=0}^{m-1}a_k(\mu_{\mathrm{i}},\mu_{\mathrm{o}})\cos(k(\varphi_{\mathrm{i}}-\varphi_{\mathrm{o}}))\, .
\end{align}

注意为什么\refeq{8.21}只需余弦项而不用正弦项。
函数值$a_0(\mu_{\mathrm{i}},\mu_{\mathrm{o}}),\dots,a_{m-1}(\mu_{\mathrm{i}},\mu_{\mathrm{o}})$表示
一对特定天顶角余弦值的傅里叶系数。

接着,函数$a_k$在其输入参数上进行离散化。我们选一个天顶角余弦值
集合$\bar{\mu}=\{\mu_0,\dots,\mu_{n-1}\}$并为每对$0\le i,j<n$存储$a_k(\mu_i,\mu_j)$的值。
因此,我们可以把每个$a_k$看作一个$n\times n$矩阵,
而整个BRDF表示就是$m$个这样的矩阵的集合构成的。
每一个都描述了材料对入射光照的响应中不同的方位振荡频率。

计算\refeq{8.21}至满意的精度所需的最高阶数$m$是不同的:
它取决于特定的天顶角,所以对于给定的一对方向依据BSDF的复杂度来调整系数$a_k$的数量是值得的。
这种做法对于压缩该表示是非常重要的。

为了看出可以调整系数数量的价值,考虑几乎完美的镜面反射:
当$\mu_{\mathrm{i}}\approx\mu_{\mathrm{o}}$时,需要许多系数以精确表示镜面瓣,
它对于几乎所有的方位角之差$\varphi=\varphi_{\mathrm{i}}-\varphi_{\mathrm{o}}$都是零,
但对$\varphi\approx\pi$附近的一小部分方向集合取值非常大,此时入射和出射方向几乎相反。
然而,当$\mu_{\mathrm{i}}$和$\mu_{\mathrm{o}}$没有对齐时,
只需要一项来表示BSDF取零(或者有可忽略的值)
\footnote{\citet{10.1145/2601097.2601139}展示了这种自适应性能够
用51MiB以1\%的$L^2$相对误差表示Beckmann粗糙度$\alpha=0.01$的光洁镜面,
而给所有方向对中的任何一对方向都用所需的最大阶数$m$时则需要28GiB才能达到相同的误差。}。
对于更平滑的BSDF,大部分或所有的$\mu_{\mathrm{i}}$与$\mu_{\mathrm{o}}$角度对都需要
多个系数$a_k$来精确表示它们的$\varphi$分布,但它们的平滑性意味着对于每个$a_k$通常不需要太多的系数。
\refvar{FourierBSDF}{}表示利用了该性质并只存储达到期望的精度所需的系数稀疏集。
因此,对于现实BSDF数据的大部分类型,\refeq{8.21}的表示都足够紧凑;典型的只要几兆字节。

\refvar{FourierBSDFTable}{}是一个辅助结构体,用于存放按该方式表达的BSDF的所有数据。
它几乎就是一个简单的{\ttfamily struct}以存放可被调用代码直接获取的数据,但它还提供一些实用方法。
\begin{lstlisting}
`\refcode{BSDF Declarations}{+=}\lastnext{BSDFDeclarations}`
struct `\initvar{FourierBSDFTable}{}` {
    `\refcode{FourierBSDFTable Public Data}{}`
    `\refcode{FourierBSDFTable Public Methods}{}`
};
\end{lstlisting}

方法\refvar[FourierBSDFTable::Read]{Read}{()}为存储在给定文件中的BSDF初始化结构体。
它成功时返回{\ttfamily true}或在读取文件遇到错误时返回{\ttfamily false}。
\begin{lstlisting}
`\initcode{FourierBSDFTable Public Methods}{=}\initnext{FourierBSDFTablePublicMethods}`
static bool `\initvar[FourierBSDFTable::Read]{Read}{}`(const std::string &filename, `\refvar{FourierBSDFTable}{}` *table);
\end{lstlisting}

如果BSDF表示两种不同介质间边界处的散射,则\refvar{FourierBSDFTable::eta}{}
成员变量给出曲面边界上的相对折射率(\refsub{镜面透射}),\refvar[FourierBSDFTable::mMax]{mMax}{}给出对
任意一对方向$\mu_{\mathrm{i}},\mu_{\mathrm{o}}$的最大阶数$m$;
例如在申请临时缓存保存系数$a_k$时,该上限很有用。
\begin{lstlisting}
`\initcode{FourierBSDFTable Public Data}{=}\initnext{FourierBSDFTablePublicData}`
`\refvar{Float}{}` `\initvar[FourierBSDFTable::eta]{eta}{}`;
int `\initvar[FourierBSDFTable::mMax]{mMax}{}`;
\end{lstlisting}

\refvar[FourierBSDFTable::nChannels]{nChannels}{}给出了可用的光谱通道数目。
在本实现中,它要么是1表示单色BSDF,要么是3表示RGB颜色的BSDF。
这里,三通道版本实际上存储着光亮度、红色、蓝色,
而不是红色、绿色、蓝色——直接表示光亮度对于\refsub{傅里叶BSDF}定义的
蒙特卡罗采样过程会变得很有用,因为它提供了关于所有颜色通道上的函数的聚合信息。
不久我们就会看到对应的绿色可以很容易从光照度、红色与蓝色中算出。
\begin{lstlisting}
`\refcode{FourierBSDFTable Public Data}{+=}\lastnext{FourierBSDFTablePublicData}`
int `\initvar[FourierBSDFTable::nChannels]{nChannels}{}`;
\end{lstlisting}

天顶角被离散化为\refvar[FourierBSDFTable::nMu]{nMu}{}个方向,
它保存于数组\refvar[FourierBSDFTable::mu]{mu}{}中。
\refvar[FourierBSDFTable::mu]{mu}{}按从低到高保存,
所以可以用二分查找来求得最接近给定角度$\mu_{\mathrm{i}}$或$\mu_{\mathrm{o}}$的项。
\begin{lstlisting}
`\refcode{FourierBSDFTable Public Data}{+=}\lastnext{FourierBSDFTablePublicData}`
int `\initvar[FourierBSDFTable::nMu]{nMu}{}`;
`\refvar{Float}{}` *`\initvar[FourierBSDFTable::mu]{mu}{}`;
\end{lstlisting}

为了计算\refeq{8.21},我们需要知道对应于方向${\bm\omega}_{\mathrm{i}}$和${\bm\omega}_{\mathrm{o}}$的
目标傅里叶阶数$m$与所有系数$a_0,\dots,a_{m-1}$.
现在简单起见,我们将基本思路表达成在小于或等于$\mu_{\mathrm{i}}$和$\mu_{\mathrm{o}}$时,
仿佛只有最接近的方向\refvar[FourierBSDFTable::mu]{mu}{}的系数才会被使用,
但其实后面的实现是在方向附近的多个\refvar[FourierBSDFTable::mu]{mu}{}值的系数之间插值。

傅里叶表示的阶数$m$总是受限于\refvar[FourierBSDFTable::mMax]{mMax}{}但
会随着入射和出射天顶角余弦$\mu_{\mathrm{i}}$和$\mu_{\mathrm{o}}$而变化:
需要多大阶数可通过查询一个\refvar[FourierBSDFTable::nMu]{nMu}{}$\times$\refvar[FourierBSDFTable::nMu]{nMu}{}整数
矩阵\refvar[FourierBSDFTable::m]{m}{}来确定。
\begin{lstlisting}
`\refcode{FourierBSDFTable Public Data}{+=}\lastnext{FourierBSDFTablePublicData}`
int *`\initvar[FourierBSDFTable::m]{m}{}`;
\end{lstlisting}

为了给特定角度集求得$m$,我们先在离散化方向\refvar[FourierBSDFTable::mu]{mu}{}中
进行二分查找以给出偏移量{\ttfamily oi}和{\ttfamily oo}使得
\begin{align*}
    \text{{\ttfamily mu}}[\text{\ttfamily oi}]\le&\mu_{\mathrm{i}}<\text{{\ttfamily mu}}[\text{\ttfamily oi}+1]\, ,\\
    \text{{\ttfamily mu}}[\text{\ttfamily oo}]\le&\mu_{\mathrm{o}}<\text{{\ttfamily mu}}[\text{\ttfamily oo}+1]\, .
\end{align*}
利用这些索引,所需的阶数可以从{\ttfamily\refvar[FourierBSDFTable::m]{m}{}[oo * \refvar[FourierBSDFTable::nMu]{nMu}{} + oi]}获取。

所有离散方向对\refvar[FourierBSDFTable::mu]{mu}{}的全部系数$a_k$都打包进数组\refvar[FourierBSDFTable::a]{a}{}。
因为最大阶数(由此以及系数数量)会变化,甚至对于一给定方向对按照BSDF的特性会取零,
所以寻找数组\refvar[FourierBSDFTable::a]{a}{}中的偏移量是个两步过程:
\begin{enumerate}
    \item 首先,偏移量{\ttfamily oi}和{\ttfamily oo}用于
    在数组\refvar[FourierBSDFTable::aOffset]{aOffset}{}内索引以获取
    对\refvar[FourierBSDFTable::a]{a}{}的偏移量:
    {\ttfamily offset = \refvar[FourierBSDFTable::aOffset]{aOffset}{}[oo * \refvar[FourierBSDFTable::nMu]{nMu}{} + oi]}
    (因此数组\refvar[FourierBSDFTable::aOffset]{aOffset}{}共有\refvar[FourierBSDFTable::nMu]{nMu}{}*\refvar[FourierBSDFTable::nMu]{nMu}{}项)。
    \item 接着,从\refvar[FourierBSDFTable::a]{a}{}{\ttfamily [offset]}开始的$m$个系数
    为对应的方向对给出了$a_k$的值。对于三颜色通道的情况,\refvar[FourierBSDFTable::a]{a}{}{\ttfamily [offset]}后的
    前$m$个系数编码了光亮度的系数,接着的$m$个对应红色通道,之后是蓝色的。
\end{enumerate}
\begin{lstlisting}
`\refcode{FourierBSDFTable Public Data}{+=}\lastnext{FourierBSDFTablePublicData}`
int *`\initvar[FourierBSDFTable::aOffset]{aOffset}{}`;
`\refvar{Float}{}` *`\initvar[FourierBSDFTable::a]{a}{}`;
\end{lstlisting}

\refvar[FourierBSDFTable::GetAk]{GetAk}{()}是个小巧的便利方法,
为入射和出射方向余弦给定对数组\refvar[FourierBSDFTable::mu]{mu}{}的偏移量,
则为该方向返回系数的阶数$m$和指向这些系数的指针。
\begin{lstlisting}
`\refcode{FourierBSDFTable Public Methods}{+=}\lastcode{FourierBSDFTablePublicMethods}`
const `\refvar{Float}{}` *`\initvar[FourierBSDFTable::GetAk]{GetAk}{}`(int offsetI, int offsetO, int *mptr) const {
    *mptr = `\refvar[FourierBSDFTable::m]{m}{}`[offsetO * `\refvar[FourierBSDFTable::nMu]{nMu}{}` + offsetI];
    return `\refvar[FourierBSDFTable::a]{a}{}` + `\refvar[FourierBSDFTable::aOffset]{aOffset}{}`[offsetO * `\refvar[FourierBSDFTable::nMu]{nMu}{}` + offsetI];
}
\end{lstlisting}

类\refvar{FourierBSDF}{}提供了\refvar{FourierBSDFTable}{}表示和\refvar{BxDF}{}接口间的桥梁。
该类的实例由类\refvar{FourierMaterial}{}创建。
\begin{lstlisting}
`\refcode{BxDF Declarations}{+=}\lastcode{BxDFDeclarations}`
class `\initvar{FourierBSDF}{}` : public `\refvar{BxDF}{}` {
public:
    `\refcode{FourierBSDF Public Methods}{}`
private:
    `\refcode{FourierBSDF Private Data}{}`
};
\end{lstlisting}
\begin{lstlisting}
`\initcode{FourierBSDF Public Methods}{=}`
`\refvar{FourierBSDF}{}`(const `\refvar{FourierBSDFTable}{}` &bsdfTable, `\refvar{TransportMode}{}` mode)
    : `\refvar{BxDF}{}`(`\refvar{BxDFType}{}`(`\refvar[BSDFREFLECTION]{BSDF\_REFLECTION}{}` | `\refvar[BSDFTRANSMISSION]{BSDF\_TRANSMISSION}{}` | `\refvar[BSDFGLOSSY]{BSDF\_GLOSSY}{}`)),
      `\refvar[FourierBSDF::bsdfTable]{bsdfTable}{}`(bsdfTable), `\refvar[FourierBSDF::mode]{mode}{}`(mode) { }
\end{lstlisting}

类\refvar{FourierBSDF}{}只存储了指向表的{\ttfamily const}引用;
该表巨大以至于我们完全不想再为每个\refvar{FourierBSDF}{}实例复制一份单独的副本。
这里只需要读取权限,所以该方法不会造成任何问题
(\refvar{FourierMaterial}{}负责\refvar{FourierBSDFTable}{}的存储)。
\begin{lstlisting}
`\initcode{FourierBSDF Private Data}{=}`
const `\refvar{FourierBSDFTable}{}` &`\initvar[FourierBSDF::bsdfTable]{bsdfTable}{}`;
const `\refvar{TransportMode}{}` `\initvar[FourierBSDF::mode]{mode}{}`;
\end{lstlisting}

求BSDF的值就是计算余弦$\mu_{\mathrm{i}}$和$\mu_{\mathrm{o}}$、
寻找对应的系数$a_k$和最大阶数,然后求\refeq{8.21}的值的事了。
\begin{lstlisting}
`\refcode{BxDF Method Definitions}{+=}\lastnext{BxDFMethodDefinitions}`
`\refvar{Spectrum}{}` `\refvar{FourierBSDF}{}`::`\initvar[FourierBSDF::f]{f}{}`(const `\refvar{Vector3f}{}` &wo, const `\refvar{Vector3f}{}` &wi) const { 
    `\refcode{Find the zenith angle cosines and azimuth difference angle}{}`
    `\refcode{Compute Fourier coefficients $a_k$ for ($\mu$i,$\mu$o)}{}`
    `\refcode{Evaluate Fourier expansion for angle $\varphi$}{}`
}
\end{lstlisting}

\refvar{FourierBSDF}{}中如何表示方向有一个重要的习惯差异:
入射方向${\bm\omega}_{\mathrm{i}}$相比于pbrt中的一般方法是取了反的。
当执行其他计算例如用该表示为分层材料计算BSDF时这个差异很有帮助
\footnote{例如,这种习惯暗含着,对于不变地穿过某介质的光线,
如果我们把$a_k(\mu_{\mathrm{i}},\mu_{\mathrm{o}})$看作矩阵,
则我们会有一个对角矩阵,其非零项是对应于$\delta$分布的傅里叶系数,
该分布对于所有$\varphi\ne0$都取零。该性质反过来让这些计算排序的表示更容易处理。}。
\begin{lstlisting}
`\initcode{Find the zenith angle cosines and azimuth difference angle}{=}`
`\refvar{Float}{}` muI = `\refvar{CosTheta}{}`(-wi), muO = `\refvar{CosTheta}{}`(wo);
`\refvar{Float}{}` cosPhi = `\refvar{CosDPhi}{}`(-wi, wo);
\end{lstlisting}
这样重建的BSDF就非常平滑,这里的实现在分别紧邻$\mu_{\mathrm{i}}$和$\mu_{\mathrm{o}}$的
四个量化方向\refvar[FourierBSDFTable::mu]{mu}{}的乘积的系数$a_k$上进行插值。
插值是用\keyindex{张量积}{tensor-product}{}样条执行的,
被采样函数值的权重对于每个参数是独立计算的,然后乘在一起。
因此每个最终的傅里叶系数$a_k$算得为
\begin{align}\label{eq:8.22}
    a_k=\sum\limits_{a=0}^{3}\sum\limits_{b=0}^{3}a_k(o_{\mathrm{i}}+a,o_{\mathrm{o}}+b)w_{\mathrm{i}}(a)w_{\mathrm{o}}(b)\, ,
\end{align}
其中$a_k(i,j)$为量化方向$\mu_{\mathrm{i}},\mu_{\mathrm{o}}$给出了第$k$个傅里叶系数,
$w_{\mathrm{i}}$和$w_{\mathrm{o}}$则是样条权重。
即使方向$\mu_{\mathrm{i}}$的离散化相对粗糙,该插值也能保证足够的平滑性;
如何计算这些权重的细节会在接下来的几页介绍。
\begin{lstlisting}
`\initcode{Compute Fourier coefficients $a_k$ for ($\mu$i,$\mu$o)}{=}`
`\refcode{Determine offsets and weights for $\mu$i and $\mu$o}{}`
`\refcode{Allocate storage to accumulate $a_k$ coefficients}{}`
`\refcode{Accumulate weighted sums of nearby $a_k$ coefficients}{}`
\end{lstlisting}

对于每个方向$\mu_{\mathrm{i}}$和$\mu_{\mathrm{o}}$,
\refvar[FourierBSDFTable::GetWeightsAndOffset]{GetWeightsAndOffset}{()}
给出了要插值的前四个\refvar[FourierBSDFTable::mu]{mu}{}值的偏移量
以及含有四个浮点权重的数组。
\begin{lstlisting}
`\initcode{Determine offsets and weights for $\mu$i and $\mu$o}{=}`
int offsetI, offsetO;
`\refvar{Float}{}` weightsI[4], weightsO[4];
if (!`\refvar[FourierBSDF::bsdfTable]{bsdfTable}{}`.`\refvar[FourierBSDFTable::GetWeightsAndOffset]{GetWeightsAndOffset}{}`(muI, &offsetI, weightsI) ||
    !`\refvar[FourierBSDF::bsdfTable]{bsdfTable}{}`.`\refvar[FourierBSDFTable::GetWeightsAndOffset]{GetWeightsAndOffset}{}`(muO, &offsetO, weightsO))
    return `\refvar{Spectrum}{}`(0.f);
\end{lstlisting}

不同的向量$a_k$在被插值的$4\times4$个方向范围内可能有不同的阶数$m$.
因此,这里的实现为$a_k$值申请存储时用最大可能的阶数$m$乘以通道数作为大小。
对于多通道的情况,这里申请的数组{\ttfamily ak}的
前\refvar[FourierBSDF::bsdfTable]{bsdfTable}{}{\ttfamily .}\refvar[FourierBSDFTable::mMax]{mMax}{}项
会用于首个通道,接下来的\refvar[FourierBSDFTable::mMax]{mMax}{}项用于第二个通道,以此类推
(因此在十六个方向上的最大阶数小于\refvar[FourierBSDFTable::mMax]{mMax}{}的常见情况下,
通常数组{\ttfamily ak}中会有些没用到的空间)。
所有这些存储都初始化为零,所以后续代码可直接把\refeq{8.22}的项加到{\ttfamily ak}的对应项中。
\begin{lstlisting}
`\initcode{Allocate storage to accumulate $a_k$ coefficients}{=}`
`\refvar{Float}{}` *ak = `\refvar{ALLOCA}{}`(`\refvar{Float}{}`, `\refvar[FourierBSDF::bsdfTable]{bsdfTable}{}`.`\refvar[FourierBSDFTable::mMax]{mMax}{}` * `\refvar[FourierBSDF::bsdfTable]{bsdfTable}{}`.`\refvar[FourierBSDFTable::nChannels]{nChannels}{}`);
memset(ak, 0, `\refvar[FourierBSDF::bsdfTable]{bsdfTable}{}`.`\refvar[FourierBSDFTable::mMax]{mMax}{}` * `\refvar[FourierBSDF::bsdfTable]{bsdfTable}{}`.`\refvar[FourierBSDFTable::nChannels]{nChannels}{}` * sizeof(`\refvar{Float}{}`));
\end{lstlisting}

有了结果的权重、偏移量和存储,现在可以执行插值了。
\begin{lstlisting}
`\initcode{Accumulate weighted sums of nearby $a_k$ coefficients}{=}`
int mMax = 0;
for (int b = 0; b < 4; ++b) {
    for (int a = 0; a < 4; ++a) {
        `\refcode{Add contribution of (a, b) to $a_k$ values}{}`
    }
}
\end{lstlisting}

有了权重和起始偏移量,把\refeq{8.22}求和中每一项加起来就是按偏移量
从表中取出对应的系数并把它们加到{\ttfamily ak}的累积之和中。
\begin{lstlisting}
`\initcode{Add contribution of (a, b) to $a_k$ values}{=}`
`\refvar{Float}{}` weight = weightsI[a] * weightsO[b];
if (weight != 0) {
    int m;
    const `\refvar{Float}{}` *ap = `\refvar[FourierBSDF::bsdfTable]{bsdfTable}{}`.`\refvar[FourierBSDFTable::GetAk]{GetAk}{}`(offsetI + a, offsetO + b, &m);
    mMax = std::max(mMax, m);
    for (int c = 0; c < `\refvar[FourierBSDF::bsdfTable]{bsdfTable}{}`.`\refvar[FourierBSDFTable::nChannels]{nChannels}{}`; ++c)
        for (int k = 0; k < m; ++k)
            ak[c * `\refvar[FourierBSDF::bsdfTable]{bsdfTable}{}`.`\refvar[FourierBSDFTable::mMax]{mMax}{}` + k] += weight * ap[c * m + k];
}
\end{lstlisting}

有了{\ttfamily ak}中的最终加权系数,调用一次\refvar{Fourier}{()}就为首个颜色通道计算出BSDF值。
傅里叶重建中的误差可以让它自己变成负值,所以返回的值必须被截断到零。

回想\refeq{8.21}中系数$a_k$表示余弦加权的BSDF。
该余弦因子必须从方法\refvar[FourierBSDF::f]{f}{()}返回的值中去掉;
项{\ttfamily scale}编码了这一因子。
\begin{lstlisting}
`\initcode{Evaluate Fourier expansion for angle $\varphi$}{=}`
`\refvar{Float}{}` Y = std::max((`\refvar{Float}{}`)0, `\refvar{Fourier}{}`(ak, mMax, cosPhi));
`\refvar{Float}{}` scale = muI != 0 ? (1 / std::abs(muI)) : (`\refvar{Float}{}`)0;
`\refcode{Update scale to account for adjoint light transport}{}`
if (`\refvar[FourierBSDF::bsdfTable]{bsdfTable}{}`.`\refvar[FourierBSDFTable::nChannels]{nChannels}{}` == 1)
    return `\refvar{Spectrum}{}`(Y * scale);
else {
    `\refcode{Compute and return RGB colors for tabulated BSDF}{}`
}
\end{lstlisting}

就像镜面透射那样,辐射从某折射率的介质穿射到另一种时会被缩减,
但这种缩减并不适用于起始于相机的光线。\refsec{路径-空间测量方程}将提供处理这种调整的
代码片\refcode{Update scale to account for adjoint light transport}{}的定义和讨论。

如前所述,当有三个颜色通道时,首个通道编码光照度,接下来两个分别是红色和蓝色。
为了理解如何计算绿色通道值,考虑函数\refvar{RGBToXYZ}{()}的实现,
它假设基色来自sRGB标准,用下面的式子从红、绿、蓝颜色分量中算出$y_{\lambda}$:
\begin{align*}
    y_{\lambda}=0.212671r+0.715160g+0.072169b\, .
\end{align*}
现在的情况下,我们知道$y_{\lambda}$、$r$和$b$. 求解$g$,我们可以得到:
\begin{align*}
    g=1.39829y_{\lambda}-0.100913b-0.297375r\, .
\end{align*}
像之前那样,任何因为傅里叶重建误差造成的颜色系数负值都必须截断到零。
\begin{lstlisting}
`\initcode{Compute and return RGB colors for tabulated BSDF}{=}`
`\refvar{Float}{}` R = `\refvar{Fourier}{}`(ak + 1 * `\refvar[FourierBSDF::bsdfTable]{bsdfTable}{}`.`\refvar[FourierBSDFTable::mMax]{mMax}{}`, mMax, cosPhi);
`\refvar{Float}{}` B = `\refvar{Fourier}{}`(ak + 2 * `\refvar[FourierBSDF::bsdfTable]{bsdfTable}{}`.`\refvar[FourierBSDFTable::mMax]{mMax}{}`, mMax, cosPhi);
`\refvar{Float}{}` G = 1.39829f * Y - 0.100913f * B - 0.297375f * R;
`\refvar{Float}{}` rgb[3] = { R * scale, G * scale, B * scale };
return `\refvar{Spectrum}{}`::`\refvar[RGBSpectrum::FromRGB]{FromRGB}{}`(rgb).`\refvar[CoefficientSpectrum::Clamp]{Clamp}{}`();
\end{lstlisting}

我们现在将定义函数\refvar{Fourier}{()},它接收系数数组$a_k$、
最大阶数$m$以及角度$\varphi$的余弦。它计算\refeq{8.21}中
余弦的加权和,且可以利用现在已知的$a_k$写成更简单的形式:
\begin{align}\label{eq:8.23}
    f(\varphi)=\sum\limits_{k=0}^{m-1}a_k\cos(k\varphi)\, .
\end{align}
该函数的实现在求和时对求和项用了双精度以最小化浮点舍入误差的影响。
\begin{lstlisting}
`\initcode{Fourier Interpolation Definitions}{=}\initnext{FourierInterpolationDefinitions}`
`\refvar{Float}{}` `\initvar{Fourier}{}`(const `\refvar{Float}{}` *a, int m, double cosPhi) {
    double value = 0.0;
    `\refcode{Initialize cosine iterates}{}`
    for (int k = 0; k < m; ++k) {
        `\refcode{Add the current summand and update the cosine iterates}{}`
    }
    return value;
}
\end{lstlisting}

随着系数数量增加,\refeq{8.23}的朴素计算方法会涉及对三角函数的大量调用。
这会导致严重的性能问题:当前CPU架构调用一次{\ttfamily std::cos()}就要几百个处理器周期。
因此,为余弦使用\keyindex{倍角公式}{multiple angle formula}{}是值得的
\sidenote{译者注:将$k\varphi$拆为$(k-1)\varphi+\varphi$,
$(k-2)\varphi$拆为$(k-1)\varphi-\varphi$,再利用和角差角公式即可得证。}:
\begin{align}\label{eq:8.24}
    \cos(k\varphi)=2\cos\varphi\cos((k-1)\varphi)-\cos((k-2)\varphi)\, .
\end{align}
该式把\refeq{8.23}中加数$k$的余弦用加数$k-1$和$k-2$的形式表达。

实现从当下和之前两个余弦变量的声明开始,对应索引$k=-1$和$k=0$的值。
这里重要的是还用双精度计算$\cos(k\varphi)$的值;
否则在用倍角公式时,一旦{\ttfamily m}有上千的值,
用32位{\ttfamily float}累积的浮点舍入误差就会变得明显。
\begin{lstlisting}
`\initcode{Initialize cosine iterates}{=}`
double cosKMinusOnePhi = cosPhi;
double cosKPhi = 1;
\end{lstlisting}
然后循环体把当前系数与余弦值的积加到累积和中并计算下次迭代的余弦值。
\begin{lstlisting}
`\initcode{Add the current summand and update the cosine iterates}{=}`
value += a[k] * cosKPhi;
double cosKPlusOnePhi = 2 * cosPhi * cosKPhi - cosKMinusOnePhi;
cosKMinusOnePhi = cosKPhi;
cosKPhi = cosKPlusOnePhi;
\end{lstlisting}

\subsection{样条插值}\label{sub:样条插值}
最后要解释的细节是用于重建系数$a_k$的基于样条的插值是如何起作用的。
这里的实现使用\keyindex{Catmull-Rom样条}{Catmull-Rom spline}{spline样条},
它在一维上可以表示为四个控制点的加权和,
所用的权重和特定控制点取决于计算它的值时在曲线路径上的参数化位置。
样条会穿过给定控制点并沿路径走出相当光滑的曲线。

为了理解如何计算这些权重,我们首先假设有一个
在位置$x_0,x_1,\dots,x_k$上的函数值集$f$及其导数$f'$.
对于每个区间$[x_i,x_{i+1}]$,我们想用三次多项式
\begin{align}\label{eq:8.25}
    p_i(x)=ax^3+bx^2+cx+d\, ,
\end{align}
来逼近该函数,且使其在采样处匹配该函数及其导数,
即$p_i(x_i)=f(x_i),\,p_i(x_{i+1})=f(x_{i+1}),\,p'_i(x_i)=f'(x_i)$且$p'_i(x_{i+1})=f'(x_{i+1})$.
简单起见,让我们先关注第一个区间并进一步假设$[x_0,x_1]=[0,1]$.
求解系数$a,b,c$和$d$得\sidenote{译者注:读者不妨推导一下。}
\begin{align*}
    a&=f'(x_0)+f'(x_1)+2f(x_0)-2f(x_1)\, ,\\
    b&=3f(x_1)-3f(x_0)-2f'(x_0)-f'(x_1)\, ,\\
    c&=f'(x_0)\, ,\\
    d&=f(x_0)\, .
\end{align*}
注意所有系数都是函数值和导数的线性组合,这能让我们把\refeq{8.25}重写为
\begin{align}\label{eq:8.26}
    p(x)=&(2x^3-3x^2+1)f(x_0)\nonumber\\
    &+(-2x^3+3x^2)f(x_1)\nonumber\\
    &+(x^3-2x^2+x)f'(x_0)\nonumber\\
    &+(x^3-x^2)f'(x_1)\, .
\end{align}
这种插值很方便但不幸的是它仍然限制太多,因为我们
通常不能指望能获取到导数信息:反射模型可解析的导数一般都很繁琐,
而测量的数据则完全无法提供这些。因此我们基于两个相邻的
函数值$f(x_{i-1})$和$f(x_{i+1})$利用中心差分估计每个$f(x_i)$处的导数。
于是估计的导数为
\begin{align*}
    f'(x_0)\approx\frac{f(x_1)-f(x_{-1})}{x_1-x_{-1}}=\frac{f(x_1)-f(x_{-1})}{1-x_{-1}}\, .
\end{align*}
类似地,$f(x_1)$处的导数可用两个相邻函数值估计为
\begin{align*}
    f'(x_1)\approx\frac{f(x_2)-f(x_0)}{x_2-x_0}=\frac{f(x_2)-f(x_0)}{x_2}\, .
\end{align*}
如果把这两个表达式代入\refeq{8.26}再按$f$的项整理,我们将有
\begin{align*}
    p(x)=&\frac{x^3-2x^2+x}{x_{-1}-1}f(x_{-1})\\
    &+\left(2x^3-3x^2+1-\frac{x^3-x^2}{x_2}\right)f(x_0)\\
    &+\left(-2x^3+3x^2+\frac{x^3-2x^2+x}{1-x_{-1}}\right)f(x_1)\\
    &+\frac{x^3-x^2}{x_2}f(x_2)\, ,
\end{align*}
注意这些权重独立于函数值:因此我们也能把该插值表达为
\begin{align}\label{eq:8.27}
    p(x)=w_0f(x_{-1})+w_1f(x_0)+w_2f(x_1)+w_3f(x_2)\, ,
\end{align}
其中
\begin{align}\label{eq:8.28}
    w_0&=\frac{x^3-2x^2+x}{x_{-1}-1}\, ,\nonumber\\
    w_1&=2x^3-3x^2+1-\frac{x^3-x^2}{x_2}=(2x^3-3x^2+1)-w_3\, ,\nonumber\\
    w_2&=-2x^3+3x^2+\frac{x^3-2x^2+x}{1-x_{-1}}=(-2x^3+3x^2)-w_0\, ,\nonumber\\
    w_3&=\frac{x^3-x^2}{x_2}\, .
\end{align}

函数\refvar{CatmullRomWeights}{()}接收变量{\ttfamily x}和
插值节点数量以及它们的位置作为参数。它不以任何方式使用函数值
而是计算索引{\ttfamily offset}以及对应于\refeq{8.28}中四个权重的数组。

计算这些权重的代码在BSDF重建任务之外也很有用,
因此它定义在文件\href{https://github.com/mmp/pbrt-v3/blob/master/src/core/interpolation.h}{\ttfamily core/interpolation.h}
与\href{https://github.com/mmp/pbrt-v3/blob/master/src/core/interpolation.cpp}{\ttfamily core/interpolation.cpp}中。
\begin{lstlisting}
`\initcode{Spline Interpolation Definitions}{=}\initnext{SplineInterpolationDefinitions}`
bool `\initvar{CatmullRomWeights}{}`(int size, const `\refvar{Float}{}` *nodes, `\refvar{Float}{}` x,
                       int *offset, `\refvar{Float}{}` *weights) {
    `\refcode{Return false if x is out of bounds}{}`
    `\refcode{Search for the interval idx containing x}{}`
    `\refcode{Compute the $t$ parameter and powers}{}`
    `\refcode{Compute initial node weights $w_1$ and $w_2$}{}`
    `\refcode{Compute first node weight $w_0$}{}`
    `\refcode{Compute last node weight $w_3$}{}`
    return true;
}
\end{lstlisting}
当{\ttfamily x}超出函数定义域时第一条语句返回错误。
注意到把条件逻辑用否定形式书写似乎有点奇怪:但这样做我们还能捕获NaN参数,
按惯例它让比较操作的值为{\ttfamily false}。
\begin{lstlisting}
`\initcode{Return false if x is out of bounds}{=}`
if (!(x >= nodes[0] && x <= nodes[size-1]))
    return false;
\end{lstlisting}

辅助函数\refvar{FindInterval}{()}通过二分搜索高效地定位包含{\ttfamily x}的区间。
有了这一结果,我们现在可以