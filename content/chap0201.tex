\section{坐标系}\label{sec:坐标系}

按计算机图形学中的经典做法,
pbrt用三个\keyindex{坐标}{coordinate}{}值$x,y$和$z$表示\keyindex{点}{point}{}、
\keyindex{向量}{vector}{}和\keyindex{法向量}{normal vector}{vector\ 向量}。
这些值须在\keyindex{坐标系}{coordinate system}{}下才有意义,
即定义了空间的原点并给出三个线性独立的向量定义$x,y$和$z$坐标轴。
总之,这样的原点和三个向量称为\keyindex{标架}{frame}{}\sidenote{译者注:坐标系的同义词。}。
给定3D中的任意一点或方向,其$(x,y,z)$坐标值取决于它和坐标系的关系。
\reffig{2.1}以2D形式给出例子说明了这点。
\begin{figure}[htbp]
    \centering\input{Pictures/chap02/CoordSysPoint.tex}
    \caption{2D中,点$\bm p$的坐标$(x,y)$由该点与特定2D坐标系的关系定义。
        这里展示了两个坐标系;
        该点关于实线坐标系的坐标为$(8,8)$,但关于虚线坐标系的坐标为$(2,-4)$.
        两种情况下,2D点$\bm p$都处于空间中的同一绝对位置。}
    \label{fig:2.1}
\end{figure}

在一般的$n$维情形下,坐标系原点$\bm p_\mathrm{o}$及其
$n$个线性独立的\keyindex{基向量}{basis vector}{vector\ 向量}定义了一个$n$维\keyindex{仿射空间}{affine space}{}。
该空间内所有向量$\bm v$都能表示为基向量的线性组合。
给定向量$\bm v$和基向量$\bm v_i$,存在唯一一组标量值$s_i$使得
\begin{align*}
    \bm v=s_1\bm v_1+\cdots+s_n\bm v_n\, .
\end{align*}
标量$s_i$是$\bm v$相对于基$\{\bm v_1,\bm v_2,\ldots \bm v_n\}$的\keyindex{表示}{representation}{},
即我们用向量保存的坐标值。
同样,对于所有点$\bm p$,存在唯一的标量$s_i$使得该点
可以用原点$\bm p_\mathrm{o}$和基向量表示:
\begin{align*}
    \bm p=\bm p_\mathrm{o}+s_1\bm v_1+\cdots+s_n\bm v_n\, .
\end{align*}

因此,尽管3D中的点和向量都用$x,y$和$z$表示,
它们却是不同的数学对象,不能随便互换。

以坐标系形式定义点和向量造成了一个悖论:
我们需要一个点和一组向量定义坐标系,
但我们只能在特定坐标系下才能有意义地谈论点和向量。
因此,在三维中我们需要一个原点为$(0,0,0)$、
基向量为$(1,0,0)$、$(0,1,0)$以及$(0,0,1)$
的\keyindex{标准坐标系}{standard frame}{frame\ 标架}。
其他所有坐标系都参照这个规范坐标系定义,
该坐标系称为\keyindex{世界空间}{world space}{}。

\subsection{坐标系惯用手}\label{sub:坐标系惯用手}
如\reffig{2.2}所示,三个坐标轴有两种排布方式。
给定垂直的$x$和$y$坐标轴,$z$轴可以指向两个方向中的一个。
这两种选择分别称为\keyindex{左手}{left-handed}{}和\keyindex{右手}{right-handed}{}。
任选一种都行但暗含了整个系统中的一些几何操作要如何实现。
pbrt使用左手坐标系。
\begin{figure}[htbp]
    \centering\input{Pictures/chap02/LeftRightHanded.tex}
    \caption{(左图)在左手坐标系中,当$x$轴朝右且$y$轴朝上时,$z$轴指入页面。
        (右图)在右手坐标系中,$z$轴指出页面。}
    \label{fig:2.2}
\end{figure}