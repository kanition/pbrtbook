\section{图元接口与几何图元}\label{sec:图元接口与几何图元}

抽象基类\refvar{Primitive}{}是pbrt的几何处理与着色子系统之间的桥梁。
\begin{lstlisting}
`\initcode{Primitive Declarations}{=}`
class `\initvar{Primitive}{}` {
public:
    `\refcode{Primitive Interface}{}`
};
\end{lstlisting}

\refvar{Primitive}{}接口中有许多几何例程,它们全都与\refvar{Shape}{}相应的方法类似。
首先,方法\refvar{Primitive::WorldBound}{()}返回世界空间中装入图元几何体的框。
这样的框有许多用处;最重要之一是把\refvar{Primitive}{}放入加速数据结构。
\begin{lstlisting}
`\initcode{Primitive Interface}{=}\initnext{PrimitiveInterface}`
virtual `\refvar{Bounds3f}{}` `\initvar[Primitive::WorldBound]{WorldBound}{()}` const = 0;
\end{lstlisting}

接下来两个方法提供了光线相交测试。两个基类的差别之一是\linebreak
\refvar{Shape::Intersect}{()}在一个\refvar{Float}{*}输出变量中
返回沿射线到相交处的参数距离,而
\refvar{Primitive::Intersect}{()}则负责在找到相交处时用该值更新\refvar[tMax]{Ray::tMax}{}。
\begin{lstlisting}
`\refcode{Primitive Interface}{+=}\lastnext{PrimitiveInterface}`
virtual bool `\initvar[Primitive::Intersect]{Intersect}{}`(const `\refvar{Ray}{}` &r, `\refvar{SurfaceInteraction}{}` *) const = 0;
virtual bool `\initvar[Primitive::IntersectP]{IntersectP}{}`(const `\refvar{Ray}{}` &r) const = 0;
\end{lstlisting}

找到相交后,\refvar{Primitive}{}的方法\refvar[Primitive::Intersect]{Intersect}{()}也负责
初始化额外的\linebreak\refvar{SurfaceInteraction}{}成员变量,
包括指向射线命中的\refvar{Primitive}{}的指针。
\begin{lstlisting}
`\refcode{SurfaceInteraction Public Data}{+=}\lastnext{SurfaceInteractionPublicData}`
const `\refvar{Primitive}{}` *`\initvar{primitive}{}` = nullptr;
\end{lstlisting}

\refvar{Primitive}{}对象有一些方法也与非几何性质有关。
首先,如果该图元本身是光源,
则\refvar[GetAreaLight]{Primitive::GetAreaLight}{()}返回
一个指向描述图元发射分布的\refvar{AreaLight}{}的指针。
如果该图元不发光,则该方法应返回{\ttfamily nullptr}。
\begin{lstlisting}
`\refcode{Primitive Interface}{+=}\lastnext{PrimitiveInterface}`
virtual const `\refvar{AreaLight}{}` *`\initvar{GetAreaLight}{}`() const = 0;
\end{lstlisting}

\refvar{GetMaterial}{()}返回指向赋给该图元的材质实例的指针。
如果返回{\ttfamily nullptr},则应该忽略与该图元的光线相交;
该图元只用于描述介质的一块空间。
通过比较其\refvar{Material}{}指针,该方法也用于检查两条光线是否相交于同一个物体。
\begin{lstlisting}
`\refcode{Primitive Interface}{+=}\lastnext{PrimitiveInterface}`
virtual const `\refvar{Material}{}` *`\initvar{GetMaterial}{}`() const = 0;
\end{lstlisting}

第三个与材质有关的方法\refvar[Primitive::ComputeScatteringFunctions]{ComputeScatteringFunctions}{()}初始化
曲面上交点处材质光散射性质的表示。
\refvar{BSDF}{}对象(\refsec{BSDF}介绍)描述了
交点处的局部光散射性质。
如果可用,该方法还初始化一个\refvar{BSSRDF}{}描述
图元内部的次表面散射——光照进入表面的点离退出的点很远。
尽管次表面光传输对诸如金属、布料或塑料的物体外观影响很小,
但它在生物材料如皮肤、粘稠液体如牛奶等的光散射机制中占主体。
\refvar{BSSRDF}{}由\refchap{光传输II:体积渲染}讨论的
扩展的光线追踪算法支持。

除了\refvar{MemoryArena}{}为\refvar{BSDF}{}和/或\refvar{BSSRDF}{}分配内存外,该方法还接收枚举量\linebreak
\refvar{TransportMode}{}表示找到该交点的光路是从相机开始还是从光源开始的;
正如\refsec{路径-空间测量方程}讨论的,
这些细节对怎样求取材质模型的某些部分给出了重要假设。
参数{\ttfamily allowMultipleLobes}控制着怎样表示一些类型的BRDF的细节;
它将于\refsec{材质接口与实现}讨论。
\refsub{BSDF内存管理}更详细地讨论\refvar{MemoryArena}{}用于\refvar{BSDF}{}内存分配。
\begin{lstlisting}
`\refcode{Primitive Interface}{+=}\lastcode{PrimitiveInterface}`
virtual void `\initvar[Primitive::ComputeScatteringFunctions]{ComputeScatteringFunctions}{}`(`\refvar{SurfaceInteraction}{}` *isect,
    `\refvar{MemoryArena}{}` &arena, `\refvar{TransportMode}{}` mode,
    bool allowMultipleLobes) const = 0;
\end{lstlisting}

该点的\refvar{BSDF}{}和\refvar{BSSRDF}{}指针
存于传入\refvar[Primitive::ComputeScatteringFunctions]{ComputeScatteringFunctions}{()}的\linebreak
\refvar{SurfaceInteraction}{}中。
\begin{lstlisting}
`\refcode{SurfaceInteraction Public Data}{+=}\lastnext{SurfaceInteractionPublicData}`
`\refvar{BSDF}{}` *`\initvar{bsdf}{}` = nullptr;
`\refvar{BSSRDF}{}` *`\initvar{bssrdf}{}` = nullptr;
\end{lstlisting}

\subsection{几何图元}\label{sub:几何图元}
类\refvar{GeometricPrimitive}{}表示场景中的单个形状(例如一个球体)。


\subsection{TransformedPrimitive:物体实例化与动画基元}\label{sub:TransformedPrimitive:物体实例化与动画基元}