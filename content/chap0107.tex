\section{基于物理的渲染简史}\label{sec:基于物理的渲染简史}

20世纪70年代早期计算机图形学中,
最重要的问题是解决可见性算法和几何表示那样的基础问题。
那时兆字节的RAM还是稀有昂贵的奢侈品,
每秒能执行百万次浮点运算的计算机要花费数十万美元,
因此计算机图形学中可能达到的复杂度相应地受到限制,
任何为了渲染而尝试精确模拟物理的做法都是不切实际的。

随着计算机越来越强大和廉价,
考虑计算需求更大的渲染方法成为可能,
这也让基于物理的方法变得可行。
\keyindex{布林定律}{Blinn's law}{}简洁地解释了这一过程:
“随着技术进步,渲染时间保持不变。”

Jim Blinn简单的表述抓住了一条重要的约束:
给定某数量必须要渲染的图像
(对研究论文可能是几张,对故事片则超过十万张),
每张只可能花费这么多处理时间。
有人只有特定数量的计算资源,
有人必须在特定时间内完成渲染,
因此有必要限制每张图像的最大计算量。

布林定律也表达了他的观察:
人们想要渲染的图像和他们能渲染的图像还存在差距:
随着计算机越来越快,
内容创作者会持续利用增加的计算能力
和更精巧的渲染算法渲染更加复杂的场景,
而不只是更快地渲染和以前一样的场景。
渲染会持续消耗其可用的全部计算力。

\subsection{研究}\label{sub:研究}
20世纪80年代图形学研究者开始认真考虑基于物理的渲染方法。
