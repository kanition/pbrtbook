\section{胶片与成像管道}\label{sec:胶片与成像管道}

\subsection{胶片类}\label{sub:胶片类}

\begin{lstlisting}
`\initcode{Film Declarations}{=}\initnext{FilmDeclarations}`
class `\initvar{Film}{}` {
public:
    `\refcode{Film Public Methods}{}`
    `\refcode{Film Public Data}{}`
private:
    `\refcode{Film Private Data}{}`
    `\refcode{Film Private Methods}{}`
};
\end{lstlisting}

\begin{lstlisting}
`\initcode{Film Method Definitions}{=}\initnext{FilmMethodDefinitions}`
`\refvar{Film}{}`::`\refvar{Film}{}`(const `\refvar{Point2i}{}` &resolution, const `\refvar{Bounds2f}{}` &cropWindow,
        std::unique_ptr<Filter> filt, Float `\refvar{diagonal}{}`,
        const std::string &`\refvar{filename}{}`, Float `\refvar{scale}{}`)
    : `\refvar{fullResolution}{}`(resolution), `\refvar{diagonal}{}`(`\refvar{diagonal}{}` * .001),
    `\refvar{filter}{}`(std::move(filt)), `\refvar{filename}{}`(`\refvar{filename}{}`), `\refvar{scale}{}`(`\refvar{scale}{}`) {
    `\refcode{Compute film image bounds}{}`
    `\refcode{Allocate film image storage}{}`
    `\refcode{Precompute filter weight table}{}`
}
\end{lstlisting}

\begin{lstlisting}
`\initcode{Film Public Data}{=}\initnext{FilmPublicData}`
const `\refvar{Point2i}{}` `\initvar{fullResolution}{}`;
const Float `\initvar{diagonal}{}`;
std::unique_ptr<Filter> `\initvar{filter}{}`;
const std::string `\initvar{filename}{}`;
\end{lstlisting}

\begin{lstlisting}
`\initcode{Film Private Data}{=}\initnext{FilmPrivateData}`
struct `\initvar{Pixel}{}` {
    Float `\initvar[Pixel:xyz]{xyz}{}`[3] = { 0, 0, 0 };
    Float `\initvar[Pixel:filterWeightSum]{filterWeightSum}{}` = 0;
    AtomicFloat `\initvar[Pixel:splatXYZ]{splatXYZ}{}`[3];
    Float `\initvar[Pixel:pad]{pad}{}`;
};
std::unique_ptr<`\refvar{Pixel}{}`[]> `\initvar{pixels}{}`;
\end{lstlisting}

\begin{lstlisting}
`\refcode{Film Method Definitions}{+=}\lastnext{FilmMethodDefinitions}`
Bounds2i `\refvar{Film}{}`::`\initvar{GetSampleBounds}{()}` const {
    `\refvar{Bounds2f}{}` floatBounds(
        `\refvar{Floor}{}`(`\refvar{Point2f}{}`(croppedPixelBounds.pMin) + `\refvar{Vector2f}{}`(0.5f, 0.5f) -
              `\refvar{filter}{}`->`\refvar[Filter::radius]{radius}{}`),
        `\refvar{Ceil}{}`( `\refvar{Point2f}{}`(croppedPixelBounds.pMax) - `\refvar{Vector2f}{}`(0.5f, 0.5f) +
              `\refvar{filter}{}`->`\refvar[Filter::radius]{radius}{}`));
    return (Bounds2i)floatBounds;
}
\end{lstlisting}

\begin{lstlisting}
`\refcode{Film Private Data}{+=}\lastcode{FilmPrivateData}`
const Float `\initvar{scale}{}`;
\end{lstlisting}

\subsection{为胶片提供像素值}\label{sub:为胶片提供像素值}
\begin{lstlisting}
`\refcode{Film Method Definitions}{+=}\lastnext{FilmMethodDefinitions}`
std::unique_ptr<FilmTile> `\refvar{Film}{}`::`\initvar{GetFilmTile}{}`(
        const Bounds2i &sampleBounds) {
    `\refcode{Bound image pixels that samples in sampleBounds contribute to}{}`
    return std::unique_ptr<FilmTile>(new FilmTile(tilePixelBounds,
        filter->radius, filterTable, filterTableWidth));
}
\end{lstlisting}

\begin{lstlisting}
`\refcode{Film Declarations}{+=}\lastcode{FilmDeclarations}`
class `\initvar{FilmTile}{}` {
public:
    `\refcode{FilmTile Public Methods}{}`
private:
    `\refcode{FilmTile Private Data}{}`
};
\end{lstlisting}