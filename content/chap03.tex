\chapterimage{Pictures/chap03/killeroo-control-684x1368.png}
\chapter{形状}\label{chap:形状}
\setcounter{sidenote}{1}

本章中,我们将阐述pbrt对例如球体和三角形等几何图元的抽象。
光线追踪器中对几何形状的仔细抽象是干净系统设计的关键部分,
而且形状是面向对象方法的理想选择。
所有几何图元都实现一个公共接口,
渲染器剩余部分可以使用该接口而无需底层形状的任何细节。
这样能分离pbrt的几何与着色系统。

pbrt将图元的细节隐藏在两层抽象后。
类\refvar{Shape}{}提供对图元原始几何属性的访问,
例如其表面面积和边界框,并且提供光线相交例程。
类\refvar{Primitive}{}封装了关于图元的额外非几何信息,例如其材质属性。
然后渲染器剩余部分只处理\refvar{Primitive}{}抽象接口。
本章将关注只与几何相关的类\refvar{Shape}{};
\refvar{Primitive}{}接口是第\refchap{图元和相交加速}的关键内容。

\section{基本形状接口}\label{sec:基本形状接口}

\label{code:overview_Shape}
\begin{lstlisting}
`\refcode{Shape Declarations}{=}`
class Shape {
    public:
    `\refcode{Shape Interface}{}`
    `\refcode{Shape Public Data}{}`
    };
\end{lstlisting}

\section{球体}\label{sec:球体}

\keyindex{球}{sphere}{}是称为\keyindex{二次}{quadrics}{}曲面——
由$x,y$和$z$的二次多项式描述的通用类型曲面的一种特殊情况。


\section{圆柱体}\label{sec:圆柱体}

另一个有用的二次曲面是\keyindex{圆柱体}{cylinder}{};
pbrt提供以$z$轴为中心的圆柱体\refvar{Shape}{}。
实现在文件\href{https://github.com/mmp/pbrt-v3/tree/master/src/shapes/cylinder.h}{\ttfamily shapes/cylinder.h}和
\href{https://github.com/mmp/pbrt-v3/tree/master/src/shapes/cylinder.cpp}{\ttfamily shapes/cylinder.cpp}内。
用户为圆柱体提供最小和最大$z$值,以及半径和最大扫掠值$\varphi$(\reffig{3.6})。
\begin{figure}[htbp]
    \centering%LaTeX with PSTricks extensions
%%Creator: Inkscape 1.0.1 (3bc2e813f5, 2020-09-07)
%%Please note this file requires PSTricks extensions
\psset{xunit=.35pt,yunit=.35pt,runit=.35pt}
\begin{pspicture}(448.67001343,346.17999268)
{
\newrgbcolor{curcolor}{0.50196081 0.50196081 0.50196081}
\pscustom[linewidth=1,linecolor=curcolor]
{
\newpath
\moveto(247.69,277.90999268)
\curveto(247.69,272.81999268)(231.64,268.68999268)(211.84,268.68999268)
\curveto(192.04,268.68999268)(176,272.81999268)(176,277.90999268)
\curveto(176,282.99999268)(192.05,287.12999268)(211.84,287.12999268)
\curveto(222.98,287.12999268)(232.93,285.82999268)(239.5,283.77999268)
}
}
{
\newrgbcolor{curcolor}{0 0 0}
\pscustom[linewidth=1,linecolor=curcolor]
{
\newpath
\moveto(211.83000183,101.72999573)
\lineto(211.83000183,314.95999336)
}
}
{
\newrgbcolor{curcolor}{0 0 0}
\pscustom[linestyle=none,fillstyle=solid,fillcolor=curcolor]
{
\newpath
\moveto(217.34,310.04999268)
\lineto(211.83,314.30999268)
\lineto(206.33,310.04999268)
\lineto(211.83,323.05999268)
\closepath
}
}
{
\newrgbcolor{curcolor}{0.65098041 0.65098041 0.65098041}
\pscustom[linestyle=none,fillstyle=solid,fillcolor=curcolor]
{
\newpath
\moveto(216.13,311.60999268)
\lineto(211.83,321.74999268)
\lineto(211.83,314.93999268)
\closepath
}
}
{
\newrgbcolor{curcolor}{0.40000001 0.40000001 0.40000001}
\pscustom[linestyle=none,fillstyle=solid,fillcolor=curcolor]
{
\newpath
\moveto(207.53,311.60999268)
\lineto(211.83,321.74999268)
\lineto(211.83,314.93999268)
\closepath
}
}
{
\newrgbcolor{curcolor}{0 0 0}
\pscustom[linewidth=1,linecolor=curcolor]
{
\newpath
\moveto(211.83000183,102.07998657)
\lineto(132.74000549,23.97998047)
}
}
{
\newrgbcolor{curcolor}{0 0 0}
\pscustom[linestyle=none,fillstyle=solid,fillcolor=curcolor]
{
\newpath
\moveto(132.37,31.34999268)
\lineto(133.2,24.43999268)
\lineto(140.1,23.50999268)
\lineto(126.97,18.28999268)
\closepath
}
}
{
\newrgbcolor{curcolor}{0.65098041 0.65098041 0.65098041}
\pscustom[linestyle=none,fillstyle=solid,fillcolor=curcolor]
{
\newpath
\moveto(132.1,29.38999268)
\lineto(127.91,19.20999268)
\lineto(132.75,23.98999268)
\closepath
}
}
{
\newrgbcolor{curcolor}{0.40000001 0.40000001 0.40000001}
\pscustom[linestyle=none,fillstyle=solid,fillcolor=curcolor]
{
\newpath
\moveto(138.14,23.27999268)
\lineto(127.91,19.20999268)
\lineto(132.75,23.98999268)
\closepath
}
}
{
\newrgbcolor{curcolor}{0 0 0}
\pscustom[linewidth=1,linecolor=curcolor]
{
\newpath
\moveto(211.57000732,102.31999207)
\lineto(424.79998779,102.31999207)
}
}
{
\newrgbcolor{curcolor}{0 0 0}
\pscustom[linestyle=none,fillstyle=solid,fillcolor=curcolor]
{
\newpath
\moveto(419.89,96.80999268)
\lineto(424.15,102.31999268)
\lineto(419.89,107.81999268)
\lineto(432.91,102.31999268)
\closepath
}
}
{
\newrgbcolor{curcolor}{0.65098041 0.65098041 0.65098041}
\pscustom[linestyle=none,fillstyle=solid,fillcolor=curcolor]
{
\newpath
\moveto(421.45,98.01999268)
\lineto(431.59,102.31999268)
\lineto(424.78,102.31999268)
\closepath
}
}
{
\newrgbcolor{curcolor}{0.40000001 0.40000001 0.40000001}
\pscustom[linestyle=none,fillstyle=solid,fillcolor=curcolor]
{
\newpath
\moveto(421.45,106.61999268)
\lineto(431.59,102.31999268)
\lineto(424.78,102.31999268)
\closepath
}
}
{
\newrgbcolor{curcolor}{0 0 0}
\pscustom[linewidth=1,linecolor=curcolor]
{
\newpath
\moveto(350.83000183,277.73999023)
\curveto(350.83000183,286.35502645)(316.96111637,294.12183088)(265.01762877,297.41847591)
\curveto(213.07424711,300.71511422)(153.28645653,298.89232333)(113.53009079,292.80059904)
\curveto(73.77372505,286.70887475)(61.87766115,277.5478076)(83.39248224,269.58871089)
\curveto(104.90734722,261.62959795)(155.59577355,256.439991)(211.82000732,256.439991)
\curveto(268.0442411,256.439991)(318.73266743,261.62959795)(340.2475324,269.58871089)
\curveto(361.7623535,277.5478076)(349.86628959,286.70887475)(310.10992386,292.80059904)
\curveto(270.35355812,298.89232333)(210.56576754,300.71511422)(158.62238587,297.41847591)
\curveto(106.67889828,294.12183088)(72.81001282,286.35502645)(72.81001282,277.73999023)
\curveto(72.81001282,269.12495402)(106.67889828,261.35814959)(158.62238587,258.06150455)
\curveto(210.56576754,254.76486625)(270.35355812,256.58765714)(310.10992386,262.67938143)
\curveto(349.86628959,268.77110572)(361.7623535,277.93217287)(340.2475324,285.89126957)
\curveto(318.73266743,293.85038252)(268.0442411,299.03998947)(211.82000732,299.03998947)
\curveto(155.59577355,299.03998947)(104.90734722,293.85038252)(83.39248224,285.89126957)
\curveto(61.87766115,277.93217287)(73.77372505,268.77110572)(113.53009079,262.67938143)
\curveto(153.28645653,256.58765714)(213.07424711,254.76486625)(265.01762877,258.06150455)
\curveto(316.96111637,261.35814959)(350.83000183,269.12495402)(350.83000183,277.73999023)
\closepath
}
}
{
\newrgbcolor{curcolor}{0 0 0}
\pscustom[linewidth=1,linecolor=curcolor]
{
\newpath
\moveto(211.74000549,278.00999451)
\lineto(291.57000732,295.15999222)
}
}
{
\newrgbcolor{curcolor}{0 0 0}
\pscustom[linewidth=1,linecolor=curcolor]
{
\newpath
\moveto(211.88999939,278.00999451)
\lineto(350.79000854,278.00999451)
}
}
{
\newrgbcolor{curcolor}{0 0 0}
\pscustom[linewidth=1,linecolor=curcolor]
{
\newpath
\moveto(350.72999573,140.75)
\curveto(350.72999573,149.36503621)(316.86111027,157.13184065)(264.91762267,160.42848568)
\curveto(212.97424101,163.72512399)(153.18645043,161.90233309)(113.43008469,155.8106088)
\curveto(73.67371895,149.71888452)(61.77765504,140.55781737)(83.29247614,132.59872066)
\curveto(104.80734112,124.63960772)(155.49576745,119.45000076)(211.72000122,119.45000076)
\curveto(267.94423499,119.45000076)(318.63266133,124.63960772)(340.1475263,132.59872066)
\curveto(361.6623474,140.55781737)(349.76628349,149.71888452)(310.00991775,155.8106088)
\curveto(270.25355201,161.90233309)(210.46576143,163.72512399)(158.52237977,160.42848568)
\curveto(106.57889217,157.13184065)(72.71000671,149.36503621)(72.71000671,140.75)
\curveto(72.71000671,132.13496379)(106.57889217,124.36815935)(158.52237977,121.07151432)
\curveto(210.46576143,117.77487601)(270.25355201,119.59766691)(310.00991775,125.6893912)
\curveto(349.76628349,131.78111548)(361.6623474,140.94218263)(340.1475263,148.90127934)
\curveto(318.63266133,156.86039228)(267.94423499,162.04999924)(211.72000122,162.04999924)
\curveto(155.49576745,162.04999924)(104.80734112,156.86039228)(83.29247614,148.90127934)
\curveto(61.77765504,140.94218263)(73.67371895,131.78111548)(113.43008469,125.6893912)
\curveto(153.18645043,119.59766691)(212.97424101,117.77487601)(264.91762267,121.07151432)
\curveto(316.86111027,124.36815935)(350.72999573,132.13496379)(350.72999573,140.75)
\closepath
}
}
{
\newrgbcolor{curcolor}{0 0 0}
\pscustom[linewidth=1,linecolor=curcolor]
{
\newpath
\moveto(72.76999664,277.96999359)
\lineto(72.76999664,140.83999634)
}
}
{
\newrgbcolor{curcolor}{0 0 0}
\pscustom[linewidth=1,linecolor=curcolor]
{
\newpath
\moveto(350.79998779,278.11999512)
\lineto(350.79998779,140.22999573)
}
}
{
\newrgbcolor{curcolor}{0 0 0}
\pscustom[linewidth=1,linecolor=curcolor]
{
\newpath
\moveto(68.01000214,140.95999146)
\lineto(33.52000046,140.95999146)
}
}
{
\newrgbcolor{curcolor}{0 0 0}
\pscustom[linewidth=1,linecolor=curcolor]
{
\newpath
\moveto(68.01000214,277.78999329)
\lineto(33.52000046,277.78999329)
}
}
{
\newrgbcolor{curcolor}{0 0 0}
\pscustom[linestyle=none,fillstyle=solid,fillcolor=curcolor]
{
\newpath
\moveto(282.46993563,265.46999408)
\curveto(282.39181063,265.07936908)(282.23556063,264.49343158)(282.23556063,264.37624408)
\curveto(282.23556063,263.94655658)(282.58712313,263.71218158)(282.97774813,263.71218158)
\curveto(283.29024813,263.71218158)(283.71993563,263.90749408)(283.91524813,264.41530658)
\curveto(283.95431063,264.49343158)(284.77462313,267.89186908)(284.89181063,268.36061908)
\curveto(285.08712313,269.18093158)(285.55587313,270.89968158)(285.67306063,271.60280658)
\curveto(285.79024813,271.91530658)(286.49337313,273.08718158)(287.07931063,273.63405658)
\curveto(287.27462313,273.79030658)(288.01681063,274.45436908)(289.07149813,274.45436908)
\curveto(289.73556063,274.45436908)(290.08712313,274.14186908)(290.12618563,274.14186908)
\curveto(289.38399813,274.02468158)(288.83712313,273.43874408)(288.83712313,272.77468158)
\curveto(288.83712313,272.38405658)(289.11056063,271.91530658)(289.77462313,271.91530658)
\curveto(290.43868563,271.91530658)(291.14181063,272.50124408)(291.14181063,273.39968158)
\curveto(291.14181063,274.25905658)(290.36056063,275.00124408)(289.07149813,275.00124408)
\curveto(287.46993563,275.00124408)(286.37618563,273.79030658)(285.90743563,273.08718158)
\curveto(285.67306063,274.21999408)(284.77462313,275.00124408)(283.60274813,275.00124408)
\curveto(282.46993563,275.00124408)(282.00118563,274.02468158)(281.76681063,273.59499408)
\curveto(281.33712313,272.73561908)(281.02462313,271.25124408)(281.02462313,271.17311908)
\curveto(281.02462313,270.89968158)(281.25899813,270.89968158)(281.29806063,270.89968158)
\curveto(281.57149813,270.89968158)(281.57149813,270.93874408)(281.72774813,271.48561908)
\curveto(282.15743563,273.24343158)(282.66524813,274.45436908)(283.56368563,274.45436908)
\curveto(283.95431063,274.45436908)(284.30587313,274.25905658)(284.30587313,273.32155658)
\curveto(284.30587313,272.77468158)(284.22774813,272.50124408)(283.91524813,271.21218158)
\closepath
\moveto(282.46993563,265.46999408)
}
}
{
\newrgbcolor{curcolor}{0 0 0}
\pscustom[linestyle=none,fillstyle=solid,fillcolor=curcolor]
{
\newpath
\moveto(442.23157013,104.08531158)
\curveto(442.38782013,104.71031158)(442.97375763,107.01499908)(444.69250763,107.01499908)
\curveto(444.80969513,107.01499908)(445.43469513,107.01499908)(445.94250763,106.70249908)
\curveto(445.23938263,106.54624908)(444.77063263,105.96031158)(444.77063263,105.33531158)
\curveto(444.77063263,104.94468658)(445.04407013,104.47593658)(445.70813263,104.47593658)
\curveto(446.25500763,104.47593658)(447.03625763,104.90562408)(447.03625763,105.92124908)
\curveto(447.03625763,107.21031158)(445.59094513,107.56187408)(444.73157013,107.56187408)
\curveto(443.28625763,107.56187408)(442.42688263,106.23374908)(442.11438263,105.68687408)
\curveto(441.48938263,107.32749908)(440.16125763,107.56187408)(439.41907013,107.56187408)
\curveto(436.84094513,107.56187408)(435.39563263,104.35874908)(435.39563263,103.73374908)
\curveto(435.39563263,103.46031158)(435.66907013,103.46031158)(435.70813263,103.46031158)
\curveto(435.90344513,103.46031158)(435.98157013,103.53843658)(436.02063263,103.73374908)
\curveto(436.88000763,106.38999908)(438.52063263,107.01499908)(439.38000763,107.01499908)
\curveto(439.84875763,107.01499908)(440.70813263,106.78062408)(440.70813263,105.33531158)
\curveto(440.70813263,104.55406158)(440.27844513,102.91343658)(439.38000763,99.39781158)
\curveto(438.98938263,97.87437408)(438.09094513,96.81968658)(436.99719513,96.81968658)
\curveto(436.84094513,96.81968658)(436.29407013,96.81968658)(435.74719513,97.13218658)
\curveto(436.37219513,97.28843658)(436.91907013,97.79624908)(436.91907013,98.49937408)
\curveto(436.91907013,99.16343658)(436.37219513,99.35874908)(436.02063263,99.35874908)
\curveto(435.23938263,99.35874908)(434.65344513,98.73374908)(434.65344513,97.91343658)
\curveto(434.65344513,96.78062408)(435.86438263,96.27281158)(436.95813263,96.27281158)
\curveto(438.63782013,96.27281158)(439.53625763,98.03062408)(439.57532013,98.14781158)
\curveto(439.88782013,97.24937408)(440.78625763,96.27281158)(442.27063263,96.27281158)
\curveto(444.84875763,96.27281158)(446.25500763,99.47593658)(446.25500763,100.10093658)
\curveto(446.25500763,100.37437408)(446.05969513,100.37437408)(445.98157013,100.37437408)
\curveto(445.74719513,100.37437408)(445.70813263,100.25718658)(445.63000763,100.10093658)
\curveto(444.80969513,97.40562408)(443.13000763,96.81968658)(442.34875763,96.81968658)
\curveto(441.37219513,96.81968658)(440.98157013,97.60093658)(440.98157013,98.46031158)
\curveto(440.98157013,99.00718658)(441.09875763,99.55406158)(441.37219513,100.64781158)
\closepath
\moveto(442.23157013,104.08531158)
}
}
{
\newrgbcolor{curcolor}{0 0 0}
\pscustom[linestyle=none,fillstyle=solid,fillcolor=curcolor]
{
\newpath
\moveto(124.36640263,18.72960658)
\curveto(124.48359013,19.08116908)(124.48359013,19.12023158)(124.48359013,19.31554408)
\curveto(124.48359013,19.74523158)(124.13202763,19.97960658)(123.74140263,19.97960658)
\curveto(123.50702763,19.97960658)(123.11640263,19.82335658)(122.88202763,19.47179408)
\curveto(122.84296513,19.31554408)(122.60859013,18.57335658)(122.53046513,18.10460658)
\curveto(122.33515263,17.47960658)(122.17890263,16.77648158)(122.02265263,16.11241908)
\lineto(120.88984013,11.62023158)
\curveto(120.81171513,11.26866908)(119.71796513,9.51085658)(118.07734013,9.51085658)
\curveto(116.82734013,9.51085658)(116.55390263,10.60460658)(116.55390263,11.54210658)
\curveto(116.55390263,12.67491908)(116.98359013,14.23741908)(117.80390263,16.42491908)
\curveto(118.19452763,17.44054408)(118.31171513,17.71398158)(118.31171513,18.22179408)
\curveto(118.31171513,19.31554408)(117.53046513,20.25304408)(116.28046513,20.25304408)
\curveto(113.89765263,20.25304408)(112.99921513,16.62023158)(112.99921513,16.42491908)
\curveto(112.99921513,16.15148158)(113.23359013,16.15148158)(113.27265263,16.15148158)
\curveto(113.54609013,16.15148158)(113.54609013,16.22960658)(113.66327763,16.62023158)
\curveto(114.36640263,18.96398158)(115.34296513,19.70616908)(116.20234013,19.70616908)
\curveto(116.39765263,19.70616908)(116.82734013,19.70616908)(116.82734013,18.92491908)
\curveto(116.82734013,18.29991908)(116.55390263,17.63585658)(116.39765263,17.16710658)
\curveto(115.38202763,14.51085658)(114.95234013,13.10460658)(114.95234013,11.93273158)
\curveto(114.95234013,9.70616908)(116.51484013,8.96398158)(117.99921513,8.96398158)
\curveto(118.97577763,8.96398158)(119.79609013,9.39366908)(120.49921513,10.09679408)
\curveto(120.18671513,8.80773158)(119.87421513,7.55773158)(118.89765263,6.22960658)
\curveto(118.23359013,5.40929408)(117.29609013,4.66710658)(116.16327763,4.66710658)
\curveto(115.81171513,4.66710658)(114.67890263,4.74523158)(114.24921513,5.72179408)
\curveto(114.63984013,5.72179408)(114.99140263,5.72179408)(115.30390263,6.03429408)
\curveto(115.57734013,6.22960658)(115.81171513,6.58116908)(115.81171513,7.04991908)
\curveto(115.81171513,7.83116908)(115.14765263,7.90929408)(114.91327763,7.90929408)
\curveto(114.32734013,7.90929408)(113.50702763,7.51866908)(113.50702763,6.30773158)
\curveto(113.50702763,5.05773158)(114.60077763,4.12023158)(116.16327763,4.12023158)
\curveto(118.70234013,4.12023158)(121.28046513,6.38585658)(121.98359013,9.19835658)
\closepath
\moveto(124.36640263,18.72960658)
}
}
{
\newrgbcolor{curcolor}{0 0 0}
\pscustom[linestyle=none,fillstyle=solid,fillcolor=curcolor]
{
\newpath
\moveto(208.32254713,328.08682158)
\curveto(209.65067213,329.53213408)(210.39285963,330.15713408)(211.29129713,330.93838408)
\curveto(211.29129713,330.93838408)(212.81473463,332.26650908)(213.71317213,333.16494658)
\curveto(216.09598463,335.46963408)(216.64285963,336.68057158)(216.64285963,336.79775908)
\curveto(216.64285963,337.03213408)(216.40848463,337.03213408)(216.36942213,337.03213408)
\curveto(216.17410963,337.03213408)(216.13504713,336.99307158)(215.97879713,336.75869658)
\curveto(215.23660963,335.54775908)(214.72879713,335.15713408)(214.14285963,335.15713408)
\curveto(213.51785963,335.15713408)(213.24442213,335.54775908)(212.85379713,335.97744658)
\curveto(212.38504713,336.52432158)(211.95535963,337.03213408)(211.13504713,337.03213408)
\curveto(209.26004713,337.03213408)(208.12723463,334.72744658)(208.12723463,334.18057158)
\curveto(208.12723463,334.06338408)(208.20535963,333.90713408)(208.40067213,333.90713408)
\curveto(208.63504713,333.90713408)(208.67410963,334.02432158)(208.75223463,334.18057158)
\curveto(209.22098463,335.35244658)(210.66629713,335.35244658)(210.86160963,335.35244658)
\curveto(211.36942213,335.35244658)(211.83817213,335.19619658)(212.42410963,335.00088408)
\curveto(213.43973463,334.61025908)(213.71317213,334.61025908)(214.33817213,334.61025908)
\curveto(213.43973463,333.55557158)(211.36942213,331.75869658)(210.90067213,331.36807158)
\lineto(208.63504713,329.25869658)
\curveto(206.95535963,327.57900908)(206.05692213,326.17275908)(206.05692213,325.97744658)
\curveto(206.05692213,325.74307158)(206.33035963,325.74307158)(206.36942213,325.74307158)
\curveto(206.56473463,325.74307158)(206.60379713,325.78213408)(206.76004713,326.05557158)
\curveto(207.34598463,326.95400908)(208.08817213,327.61807158)(208.90848463,327.61807158)
\curveto(209.45535963,327.61807158)(209.72879713,327.38369658)(210.35379713,326.68057158)
\curveto(210.74442213,326.13369658)(211.21317213,325.74307158)(211.91629713,325.74307158)
\curveto(214.41629713,325.74307158)(215.86160963,328.90713408)(215.86160963,329.57119658)
\curveto(215.86160963,329.68838408)(215.74442213,329.84463408)(215.54910963,329.84463408)
\curveto(215.31473463,329.84463408)(215.27567213,329.68838408)(215.19754713,329.49307158)
\curveto(214.61160963,327.89150908)(213.01004713,327.42275908)(212.18973463,327.42275908)
\curveto(211.72098463,327.42275908)(211.25223463,327.57900908)(210.74442213,327.73525908)
\curveto(209.88504713,328.04775908)(209.49442213,328.16494658)(208.98660963,328.16494658)
\curveto(208.94754713,328.16494658)(208.55692213,328.16494658)(208.32254713,328.08682158)
\closepath
\moveto(208.32254713,328.08682158)
}
}
{
\newrgbcolor{curcolor}{0 0 0}
\pscustom[linestyle=none,fillstyle=solid,fillcolor=curcolor]
{
\newpath
\moveto(3.87147013,135.85697268)
\curveto(5.19959513,137.30228518)(5.94178263,137.92728518)(6.84022013,138.70853518)
\curveto(6.84022013,138.70853518)(8.36365763,140.03666018)(9.26209513,140.93509768)
\curveto(11.64490763,143.23978518)(12.19178263,144.45072268)(12.19178263,144.56791018)
\curveto(12.19178263,144.80228518)(11.95740763,144.80228518)(11.91834513,144.80228518)
\curveto(11.72303263,144.80228518)(11.68397013,144.76322268)(11.52772013,144.52884768)
\curveto(10.78553263,143.31791018)(10.27772013,142.92728518)(9.69178263,142.92728518)
\curveto(9.06678263,142.92728518)(8.79334513,143.31791018)(8.40272013,143.74759768)
\curveto(7.93397013,144.29447268)(7.50428263,144.80228518)(6.68397013,144.80228518)
\curveto(4.80897013,144.80228518)(3.67615763,142.49759768)(3.67615763,141.95072268)
\curveto(3.67615763,141.83353518)(3.75428263,141.67728518)(3.94959513,141.67728518)
\curveto(4.18397013,141.67728518)(4.22303263,141.79447268)(4.30115763,141.95072268)
\curveto(4.76990763,143.12259768)(6.21522013,143.12259768)(6.41053263,143.12259768)
\curveto(6.91834513,143.12259768)(7.38709513,142.96634768)(7.97303263,142.77103518)
\curveto(8.98865763,142.38041018)(9.26209513,142.38041018)(9.88709513,142.38041018)
\curveto(8.98865763,141.32572268)(6.91834513,139.52884768)(6.44959513,139.13822268)
\lineto(4.18397013,137.02884768)
\curveto(2.50428263,135.34916018)(1.60584513,133.94291018)(1.60584513,133.74759768)
\curveto(1.60584513,133.51322268)(1.87928263,133.51322268)(1.91834513,133.51322268)
\curveto(2.11365763,133.51322268)(2.15272013,133.55228518)(2.30897013,133.82572268)
\curveto(2.89490763,134.72416018)(3.63709513,135.38822268)(4.45740763,135.38822268)
\curveto(5.00428263,135.38822268)(5.27772013,135.15384768)(5.90272013,134.45072268)
\curveto(6.29334513,133.90384768)(6.76209513,133.51322268)(7.46522013,133.51322268)
\curveto(9.96522013,133.51322268)(11.41053263,136.67728518)(11.41053263,137.34134768)
\curveto(11.41053263,137.45853518)(11.29334513,137.61478518)(11.09803263,137.61478518)
\curveto(10.86365763,137.61478518)(10.82459513,137.45853518)(10.74647013,137.26322268)
\curveto(10.16053263,135.66166018)(8.55897013,135.19291018)(7.73865763,135.19291018)
\curveto(7.26990763,135.19291018)(6.80115763,135.34916018)(6.29334513,135.50541018)
\curveto(5.43397013,135.81791018)(5.04334513,135.93509768)(4.53553263,135.93509768)
\curveto(4.49647013,135.93509768)(4.10584513,135.93509768)(3.87147013,135.85697268)
\closepath
\moveto(3.87147013,135.85697268)
}
}
{
\newrgbcolor{curcolor}{0 0 0}
\pscustom[linestyle=none,fillstyle=solid,fillcolor=curcolor]
{
\newpath
\moveto(26.46958414,135.32508944)
\curveto(26.46958414,136.84852694)(25.68833414,137.74696444)(23.85239664,137.74696444)
\curveto(22.44614664,137.74696444)(21.50864664,137.00477694)(21.03989664,136.10633944)
\curveto(20.68833414,137.35633944)(19.75083414,137.74696444)(18.50083414,137.74696444)
\curveto(17.05552164,137.74696444)(16.15708414,136.96571444)(15.64927164,136.02821444)
\lineto(15.64927164,137.74696444)
\lineto(13.07114664,137.55165194)
\lineto(13.07114664,136.92665194)
\curveto(14.24302164,136.92665194)(14.39927164,136.80946444)(14.39927164,135.95008944)
\lineto(14.39927164,131.41883944)
\curveto(14.39927164,130.67665194)(14.20395914,130.67665194)(13.07114664,130.67665194)
\lineto(13.07114664,130.05165194)
\curveto(13.11020914,130.05165194)(14.32114664,130.12977694)(15.06333414,130.12977694)
\curveto(15.68833414,130.12977694)(16.89927164,130.05165194)(17.05552164,130.05165194)
\lineto(17.05552164,130.67665194)
\curveto(15.92270914,130.67665194)(15.76645914,130.67665194)(15.76645914,131.41883944)
\lineto(15.76645914,134.58290194)
\curveto(15.76645914,136.37977694)(17.21177164,137.23915194)(18.34458414,137.23915194)
\curveto(19.55552164,137.23915194)(19.71177164,136.30165194)(19.71177164,135.40321444)
\lineto(19.71177164,131.41883944)
\curveto(19.71177164,130.67665194)(19.55552164,130.67665194)(18.42270914,130.67665194)
\lineto(18.42270914,130.05165194)
\curveto(18.46177164,130.05165194)(19.67270914,130.12977694)(20.41489664,130.12977694)
\curveto(21.03989664,130.12977694)(22.25083414,130.05165194)(22.40708414,130.05165194)
\lineto(22.40708414,130.67665194)
\curveto(21.27427164,130.67665194)(21.11802164,130.67665194)(21.11802164,131.41883944)
\lineto(21.11802164,134.58290194)
\curveto(21.11802164,136.37977694)(22.56333414,137.23915194)(23.69614664,137.23915194)
\curveto(24.90708414,137.23915194)(25.06333414,136.30165194)(25.06333414,135.40321444)
\lineto(25.06333414,131.41883944)
\curveto(25.06333414,130.67665194)(24.90708414,130.67665194)(23.77427164,130.67665194)
\lineto(23.77427164,130.05165194)
\curveto(23.81333414,130.05165194)(25.02427164,130.12977694)(25.76645914,130.12977694)
\curveto(26.39145914,130.12977694)(27.60239664,130.05165194)(27.75864664,130.05165194)
\lineto(27.75864664,130.67665194)
\curveto(26.62583414,130.67665194)(26.46958414,130.67665194)(26.46958414,131.41883944)
\closepath
\moveto(26.46958414,135.32508944)
}
}
{
\newrgbcolor{curcolor}{0 0 0}
\pscustom[linestyle=none,fillstyle=solid,fillcolor=curcolor]
{
\newpath
\moveto(32.16780726,140.79383944)
\curveto(32.16780726,141.30165194)(31.73811976,141.80946444)(31.15218226,141.80946444)
\curveto(30.64436976,141.80946444)(30.17561976,141.37977694)(30.17561976,140.79383944)
\curveto(30.17561976,140.16883944)(30.68343226,139.77821444)(31.15218226,139.77821444)
\curveto(31.73811976,139.77821444)(32.16780726,140.20790194)(32.16780726,140.79383944)
\closepath
\moveto(29.51155726,137.55165194)
\lineto(29.51155726,136.92665194)
\curveto(30.60530726,136.92665194)(30.76155726,136.80946444)(30.76155726,135.95008944)
\lineto(30.76155726,131.41883944)
\curveto(30.76155726,130.67665194)(30.60530726,130.67665194)(29.47249476,130.67665194)
\lineto(29.47249476,130.05165194)
\curveto(29.51155726,130.05165194)(30.72249476,130.12977694)(31.42561976,130.12977694)
\curveto(32.05061976,130.12977694)(32.67561976,130.09071444)(33.26155726,130.05165194)
\lineto(33.26155726,130.67665194)
\curveto(32.24593226,130.67665194)(32.08968226,130.67665194)(32.08968226,131.41883944)
\lineto(32.08968226,137.74696444)
\closepath
\moveto(29.51155726,137.55165194)
}
}
{
\newrgbcolor{curcolor}{0 0 0}
\pscustom[linestyle=none,fillstyle=solid,fillcolor=curcolor]
{
\newpath
\moveto(43.11862055,135.32508944)
\curveto(43.11862055,136.84852694)(42.37643305,137.74696444)(40.50143305,137.74696444)
\curveto(39.05612055,137.74696444)(38.15768305,136.96571444)(37.64987055,136.02821444)
\lineto(37.64987055,137.74696444)
\lineto(35.07174555,137.55165194)
\lineto(35.07174555,136.92665194)
\curveto(36.24362055,136.92665194)(36.39987055,136.80946444)(36.39987055,135.95008944)
\lineto(36.39987055,131.41883944)
\curveto(36.39987055,130.67665194)(36.20455805,130.67665194)(35.07174555,130.67665194)
\lineto(35.07174555,130.05165194)
\curveto(35.11080805,130.05165194)(36.32174555,130.12977694)(37.06393305,130.12977694)
\curveto(37.68893305,130.12977694)(38.89987055,130.05165194)(39.05612055,130.05165194)
\lineto(39.05612055,130.67665194)
\curveto(37.92330805,130.67665194)(37.76705805,130.67665194)(37.76705805,131.41883944)
\lineto(37.76705805,134.58290194)
\curveto(37.76705805,136.37977694)(39.21237055,137.23915194)(40.34518305,137.23915194)
\curveto(41.55612055,137.23915194)(41.71237055,136.30165194)(41.71237055,135.40321444)
\lineto(41.71237055,131.41883944)
\curveto(41.71237055,130.67665194)(41.55612055,130.67665194)(40.42330805,130.67665194)
\lineto(40.42330805,130.05165194)
\curveto(40.46237055,130.05165194)(41.67330805,130.12977694)(42.41549555,130.12977694)
\curveto(43.04049555,130.12977694)(44.25143305,130.05165194)(44.40768305,130.05165194)
\lineto(44.40768305,130.67665194)
\curveto(43.27487055,130.67665194)(43.11862055,130.67665194)(43.11862055,131.41883944)
\closepath
\moveto(43.11862055,135.32508944)
}
}
{
\newrgbcolor{curcolor}{0 0 0}
\pscustom[linestyle=none,fillstyle=solid,fillcolor=curcolor]
{
\newpath
\moveto(5.36712013,272.68059158)
\curveto(6.69524513,274.12590408)(7.43743263,274.75090408)(8.33587013,275.53215408)
\curveto(8.33587013,275.53215408)(9.85930763,276.86027908)(10.75774513,277.75871658)
\curveto(13.14055763,280.06340408)(13.68743263,281.27434158)(13.68743263,281.39152908)
\curveto(13.68743263,281.62590408)(13.45305763,281.62590408)(13.41399513,281.62590408)
\curveto(13.21868263,281.62590408)(13.17962013,281.58684158)(13.02337013,281.35246658)
\curveto(12.28118263,280.14152908)(11.77337013,279.75090408)(11.18743263,279.75090408)
\curveto(10.56243263,279.75090408)(10.28899513,280.14152908)(9.89837013,280.57121658)
\curveto(9.42962013,281.11809158)(8.99993263,281.62590408)(8.17962013,281.62590408)
\curveto(6.30462013,281.62590408)(5.17180763,279.32121658)(5.17180763,278.77434158)
\curveto(5.17180763,278.65715408)(5.24993263,278.50090408)(5.44524513,278.50090408)
\curveto(5.67962013,278.50090408)(5.71868263,278.61809158)(5.79680763,278.77434158)
\curveto(6.26555763,279.94621658)(7.71087013,279.94621658)(7.90618263,279.94621658)
\curveto(8.41399513,279.94621658)(8.88274513,279.78996658)(9.46868263,279.59465408)
\curveto(10.48430763,279.20402908)(10.75774513,279.20402908)(11.38274513,279.20402908)
\curveto(10.48430763,278.14934158)(8.41399513,276.35246658)(7.94524513,275.96184158)
\lineto(5.67962013,273.85246658)
\curveto(3.99993263,272.17277908)(3.10149513,270.76652908)(3.10149513,270.57121658)
\curveto(3.10149513,270.33684158)(3.37493263,270.33684158)(3.41399513,270.33684158)
\curveto(3.60930763,270.33684158)(3.64837013,270.37590408)(3.80462013,270.64934158)
\curveto(4.39055763,271.54777908)(5.13274513,272.21184158)(5.95305763,272.21184158)
\curveto(6.49993263,272.21184158)(6.77337013,271.97746658)(7.39837013,271.27434158)
\curveto(7.78899513,270.72746658)(8.25774513,270.33684158)(8.96087013,270.33684158)
\curveto(11.46087013,270.33684158)(12.90618263,273.50090408)(12.90618263,274.16496658)
\curveto(12.90618263,274.28215408)(12.78899513,274.43840408)(12.59368263,274.43840408)
\curveto(12.35930763,274.43840408)(12.32024513,274.28215408)(12.24212013,274.08684158)
\curveto(11.65618263,272.48527908)(10.05462013,272.01652908)(9.23430763,272.01652908)
\curveto(8.76555763,272.01652908)(8.29680763,272.17277908)(7.78899513,272.32902908)
\curveto(6.92962013,272.64152908)(6.53899513,272.75871658)(6.03118263,272.75871658)
\curveto(5.99212013,272.75871658)(5.60149513,272.75871658)(5.36712013,272.68059158)
\closepath
\moveto(5.36712013,272.68059158)
}
}
{
\newrgbcolor{curcolor}{0 0 0}
\pscustom[linestyle=none,fillstyle=solid,fillcolor=curcolor]
{
\newpath
\moveto(27.96523414,272.14870834)
\curveto(27.96523414,273.67214584)(27.18398414,274.57058334)(25.34804664,274.57058334)
\curveto(23.94179664,274.57058334)(23.00429664,273.82839584)(22.53554664,272.92995834)
\curveto(22.18398414,274.17995834)(21.24648414,274.57058334)(19.99648414,274.57058334)
\curveto(18.55117164,274.57058334)(17.65273414,273.78933334)(17.14492164,272.85183334)
\lineto(17.14492164,274.57058334)
\lineto(14.56679664,274.37527084)
\lineto(14.56679664,273.75027084)
\curveto(15.73867164,273.75027084)(15.89492164,273.63308334)(15.89492164,272.77370834)
\lineto(15.89492164,268.24245834)
\curveto(15.89492164,267.50027084)(15.69960914,267.50027084)(14.56679664,267.50027084)
\lineto(14.56679664,266.87527084)
\curveto(14.60585914,266.87527084)(15.81679664,266.95339584)(16.55898414,266.95339584)
\curveto(17.18398414,266.95339584)(18.39492164,266.87527084)(18.55117164,266.87527084)
\lineto(18.55117164,267.50027084)
\curveto(17.41835914,267.50027084)(17.26210914,267.50027084)(17.26210914,268.24245834)
\lineto(17.26210914,271.40652084)
\curveto(17.26210914,273.20339584)(18.70742164,274.06277084)(19.84023414,274.06277084)
\curveto(21.05117164,274.06277084)(21.20742164,273.12527084)(21.20742164,272.22683334)
\lineto(21.20742164,268.24245834)
\curveto(21.20742164,267.50027084)(21.05117164,267.50027084)(19.91835914,267.50027084)
\lineto(19.91835914,266.87527084)
\curveto(19.95742164,266.87527084)(21.16835914,266.95339584)(21.91054664,266.95339584)
\curveto(22.53554664,266.95339584)(23.74648414,266.87527084)(23.90273414,266.87527084)
\lineto(23.90273414,267.50027084)
\curveto(22.76992164,267.50027084)(22.61367164,267.50027084)(22.61367164,268.24245834)
\lineto(22.61367164,271.40652084)
\curveto(22.61367164,273.20339584)(24.05898414,274.06277084)(25.19179664,274.06277084)
\curveto(26.40273414,274.06277084)(26.55898414,273.12527084)(26.55898414,272.22683334)
\lineto(26.55898414,268.24245834)
\curveto(26.55898414,267.50027084)(26.40273414,267.50027084)(25.26992164,267.50027084)
\lineto(25.26992164,266.87527084)
\curveto(25.30898414,266.87527084)(26.51992164,266.95339584)(27.26210914,266.95339584)
\curveto(27.88710914,266.95339584)(29.09804664,266.87527084)(29.25429664,266.87527084)
\lineto(29.25429664,267.50027084)
\curveto(28.12148414,267.50027084)(27.96523414,267.50027084)(27.96523414,268.24245834)
\closepath
\moveto(27.96523414,272.14870834)
}
}
{
\newrgbcolor{curcolor}{0 0 0}
\pscustom[linestyle=none,fillstyle=solid,fillcolor=curcolor]
{
\newpath
\moveto(37.76501976,271.56277084)
\curveto(37.76501976,272.46120834)(37.76501976,273.12527084)(36.94470726,273.78933334)
\curveto(36.24158226,274.37527084)(35.42126976,274.64870834)(34.36658226,274.64870834)
\curveto(32.72595726,274.64870834)(31.55408226,274.02370834)(31.55408226,272.96902084)
\curveto(31.55408226,272.38308334)(31.94470726,272.10964584)(32.41345726,272.10964584)
\curveto(32.88220726,272.10964584)(33.23376976,272.46120834)(33.23376976,272.92995834)
\curveto(33.23376976,273.20339584)(33.07751976,273.59402084)(32.60876976,273.71120834)
\curveto(33.23376976,274.14089584)(34.24939476,274.14089584)(34.32751976,274.14089584)
\curveto(35.30408226,274.14089584)(36.39783226,273.51589584)(36.39783226,272.03152084)
\lineto(36.39783226,271.52370834)
\curveto(35.42126976,271.48464584)(34.28845726,271.40652084)(32.99939476,270.93777084)
\curveto(31.43689476,270.39089584)(30.96814476,269.41433334)(30.96814476,268.63308334)
\curveto(30.96814476,267.14870834)(32.76501976,266.71902084)(34.01501976,266.71902084)
\curveto(35.38220726,266.71902084)(36.20251976,267.50027084)(36.59314476,268.16433334)
\curveto(36.63220726,267.46120834)(37.10095726,266.79714584)(37.92126976,266.79714584)
\curveto(37.96033226,266.79714584)(39.64001976,266.79714584)(39.64001976,268.43777084)
\lineto(39.64001976,269.41433334)
\lineto(39.05408226,269.41433334)
\lineto(39.05408226,268.47683334)
\curveto(39.05408226,268.28152084)(39.05408226,267.46120834)(38.39001976,267.46120834)
\curveto(37.76501976,267.46120834)(37.76501976,268.28152084)(37.76501976,268.47683334)
\closepath
\moveto(36.39783226,269.33620834)
\curveto(36.39783226,267.65652084)(34.91345726,267.18777084)(34.13220726,267.18777084)
\curveto(33.23376976,267.18777084)(32.41345726,267.77370834)(32.41345726,268.63308334)
\curveto(32.41345726,269.60964584)(33.23376976,270.93777084)(36.39783226,271.05495834)
\closepath
\moveto(36.39783226,269.33620834)
}
}
{
\newrgbcolor{curcolor}{0 0 0}
\pscustom[linestyle=none,fillstyle=solid,fillcolor=curcolor]
{
\newpath
\moveto(45.81897102,270.78152084)
\lineto(45.70178352,270.93777084)
\curveto(45.70178352,271.01589584)(46.83459602,272.22683334)(46.99084602,272.38308334)
\curveto(47.61584602,273.08620834)(48.20178352,273.75027084)(49.60803352,273.75027084)
\lineto(49.60803352,274.37527084)
\curveto(49.10022102,274.33620834)(48.59240852,274.29714584)(48.12365852,274.29714584)
\curveto(47.61584602,274.29714584)(46.91272102,274.33620834)(46.40490852,274.37527084)
\lineto(46.40490852,273.75027084)
\curveto(46.67834602,273.71120834)(46.75647102,273.51589584)(46.75647102,273.35964584)
\curveto(46.75647102,273.32058334)(46.75647102,273.04714584)(46.48303352,272.77370834)
\lineto(45.27209602,271.40652084)
\lineto(43.82678352,273.04714584)
\curveto(43.63147102,273.24245834)(43.63147102,273.32058334)(43.63147102,273.39870834)
\curveto(43.63147102,273.63308334)(43.86584602,273.75027084)(44.10022102,273.75027084)
\lineto(44.10022102,274.37527084)
\curveto(43.47522102,274.33620834)(42.81115852,274.29714584)(42.14709602,274.29714584)
\curveto(41.63928352,274.29714584)(40.97522102,274.33620834)(40.46740852,274.37527084)
\lineto(40.46740852,273.75027084)
\curveto(41.24865852,273.75027084)(41.71740852,273.75027084)(42.18615852,273.24245834)
\lineto(44.33459602,270.74245834)
\curveto(44.37365852,270.70339584)(44.49084602,270.58620834)(44.49084602,270.54714584)
\curveto(44.49084602,270.50808334)(43.16272102,269.06277084)(43.00647102,268.86745834)
\curveto(42.34240852,268.16433334)(41.75647102,267.53933334)(40.38928352,267.50027084)
\lineto(40.38928352,266.87527084)
\curveto(40.89709602,266.91433334)(41.28772102,266.95339584)(41.83459602,266.95339584)
\curveto(42.34240852,266.95339584)(43.04553352,266.91433334)(43.55334602,266.87527084)
\lineto(43.55334602,267.50027084)
\curveto(43.35803352,267.53933334)(43.20178352,267.65652084)(43.20178352,267.89089584)
\curveto(43.20178352,268.24245834)(43.39709602,268.47683334)(43.67053352,268.75027084)
\lineto(44.92053352,270.11745834)
\lineto(46.40490852,268.35964584)
\curveto(46.75647102,268.00808334)(46.75647102,267.96902084)(46.75647102,267.85183334)
\curveto(46.75647102,267.53933334)(46.36584602,267.50027084)(46.28772102,267.50027084)
\lineto(46.28772102,266.87527084)
\curveto(46.44397102,266.87527084)(47.69397102,266.95339584)(48.24084602,266.95339584)
\curveto(48.78772102,266.95339584)(49.37365852,266.91433334)(49.92053352,266.87527084)
\lineto(49.92053352,267.50027084)
\curveto(48.74865852,267.50027084)(48.63147102,267.61745834)(48.20178352,268.04714584)
\closepath
\moveto(45.81897102,270.78152084)
}
}
{
\newrgbcolor{curcolor}{0 0 0}
\pscustom[linestyle=none,fillstyle=solid,fillcolor=curcolor]
{
\newpath
\moveto(126.71388763,273.01617908)
\curveto(126.63576263,272.66461658)(126.59670013,272.62555408)(126.59670013,272.50836658)
\curveto(126.59670013,271.96149158)(127.06545013,271.80524158)(127.33888763,271.80524158)
\curveto(127.45607513,271.80524158)(128.00295013,271.88336658)(128.23732513,272.46930408)
\curveto(128.31545013,272.66461658)(128.43263763,273.48492908)(129.13576263,277.00055408)
\curveto(129.33107513,277.00055408)(129.52638763,276.96149158)(129.95607513,276.96149158)
\curveto(134.09670013,276.96149158)(137.92482513,280.86774158)(137.92482513,284.81305408)
\curveto(137.92482513,286.76617908)(136.94826263,288.25055408)(135.07326263,288.25055408)
\curveto(131.47951263,288.25055408)(129.95607513,283.40680408)(128.47170013,278.56305408)
\curveto(125.77638763,279.07086658)(124.37013763,280.43805408)(124.37013763,282.23492908)
\curveto(124.37013763,282.93805408)(124.95607513,285.67242908)(126.44045013,287.39117908)
\curveto(126.67482513,287.62555408)(126.67482513,287.66461658)(126.67482513,287.74274158)
\curveto(126.67482513,287.82086658)(126.59670013,287.97711658)(126.36232513,287.97711658)
\curveto(125.65920013,287.97711658)(123.74513763,284.34430408)(123.74513763,281.96149158)
\curveto(123.74513763,279.61774158)(125.38576263,277.82086658)(128.04201263,277.19586658)
\closepath
\moveto(130.19045013,278.40680408)
\curveto(129.95607513,278.40680408)(129.91701263,278.40680408)(129.72170013,278.44586658)
\curveto(129.40920013,278.44586658)(129.40920013,278.44586658)(129.40920013,278.52399158)
\curveto(129.40920013,278.56305408)(129.83888763,280.86774158)(129.87795013,281.21930408)
\curveto(130.65920013,284.42242908)(132.61232513,286.80524158)(134.83888763,286.80524158)
\curveto(136.55763763,286.80524158)(137.22170013,285.47711658)(137.22170013,284.26617908)
\curveto(137.22170013,281.45367908)(134.01857513,278.40680408)(130.19045013,278.40680408)
\closepath
\moveto(130.19045013,278.40680408)
}
}
{
\newrgbcolor{curcolor}{0 0 0}
\pscustom[linestyle=none,fillstyle=solid,fillcolor=curcolor]
{
\newpath
\moveto(153.12357086,278.77335834)
\curveto(153.12357086,280.29679584)(152.34232086,281.19523334)(150.50638336,281.19523334)
\curveto(149.10013336,281.19523334)(148.16263336,280.45304584)(147.69388336,279.55460834)
\curveto(147.34232086,280.80460834)(146.40482086,281.19523334)(145.15482086,281.19523334)
\curveto(143.70950836,281.19523334)(142.81107086,280.41398334)(142.30325836,279.47648334)
\lineto(142.30325836,281.19523334)
\lineto(139.72513336,280.99992084)
\lineto(139.72513336,280.37492084)
\curveto(140.89700836,280.37492084)(141.05325836,280.25773334)(141.05325836,279.39835834)
\lineto(141.05325836,274.86710834)
\curveto(141.05325836,274.12492084)(140.85794586,274.12492084)(139.72513336,274.12492084)
\lineto(139.72513336,273.49992084)
\curveto(139.76419586,273.49992084)(140.97513336,273.57804584)(141.71732086,273.57804584)
\curveto(142.34232086,273.57804584)(143.55325836,273.49992084)(143.70950836,273.49992084)
\lineto(143.70950836,274.12492084)
\curveto(142.57669586,274.12492084)(142.42044586,274.12492084)(142.42044586,274.86710834)
\lineto(142.42044586,278.03117084)
\curveto(142.42044586,279.82804584)(143.86575836,280.68742084)(144.99857086,280.68742084)
\curveto(146.20950836,280.68742084)(146.36575836,279.74992084)(146.36575836,278.85148334)
\lineto(146.36575836,274.86710834)
\curveto(146.36575836,274.12492084)(146.20950836,274.12492084)(145.07669586,274.12492084)
\lineto(145.07669586,273.49992084)
\curveto(145.11575836,273.49992084)(146.32669586,273.57804584)(147.06888336,273.57804584)
\curveto(147.69388336,273.57804584)(148.90482086,273.49992084)(149.06107086,273.49992084)
\lineto(149.06107086,274.12492084)
\curveto(147.92825836,274.12492084)(147.77200836,274.12492084)(147.77200836,274.86710834)
\lineto(147.77200836,278.03117084)
\curveto(147.77200836,279.82804584)(149.21732086,280.68742084)(150.35013336,280.68742084)
\curveto(151.56107086,280.68742084)(151.71732086,279.74992084)(151.71732086,278.85148334)
\lineto(151.71732086,274.86710834)
\curveto(151.71732086,274.12492084)(151.56107086,274.12492084)(150.42825836,274.12492084)
\lineto(150.42825836,273.49992084)
\curveto(150.46732086,273.49992084)(151.67825836,273.57804584)(152.42044586,273.57804584)
\curveto(153.04544586,273.57804584)(154.25638336,273.49992084)(154.41263336,273.49992084)
\lineto(154.41263336,274.12492084)
\curveto(153.27982086,274.12492084)(153.12357086,274.12492084)(153.12357086,274.86710834)
\closepath
\moveto(153.12357086,278.77335834)
}
}
{
\newrgbcolor{curcolor}{0 0 0}
\pscustom[linestyle=none,fillstyle=solid,fillcolor=curcolor]
{
\newpath
\moveto(162.9233374,278.18742084)
\curveto(162.9233374,279.08585834)(162.9233374,279.74992084)(162.1030249,280.41398334)
\curveto(161.3998999,280.99992084)(160.5795874,281.27335834)(159.5248999,281.27335834)
\curveto(157.8842749,281.27335834)(156.7123999,280.64835834)(156.7123999,279.59367084)
\curveto(156.7123999,279.00773334)(157.1030249,278.73429584)(157.5717749,278.73429584)
\curveto(158.0405249,278.73429584)(158.3920874,279.08585834)(158.3920874,279.55460834)
\curveto(158.3920874,279.82804584)(158.2358374,280.21867084)(157.7670874,280.33585834)
\curveto(158.3920874,280.76554584)(159.4077124,280.76554584)(159.4858374,280.76554584)
\curveto(160.4623999,280.76554584)(161.5561499,280.14054584)(161.5561499,278.65617084)
\lineto(161.5561499,278.14835834)
\curveto(160.5795874,278.10929584)(159.4467749,278.03117084)(158.1577124,277.56242084)
\curveto(156.5952124,277.01554584)(156.1264624,276.03898334)(156.1264624,275.25773334)
\curveto(156.1264624,273.77335834)(157.9233374,273.34367084)(159.1733374,273.34367084)
\curveto(160.5405249,273.34367084)(161.3608374,274.12492084)(161.7514624,274.78898334)
\curveto(161.7905249,274.08585834)(162.2592749,273.42179584)(163.0795874,273.42179584)
\curveto(163.1186499,273.42179584)(164.7983374,273.42179584)(164.7983374,275.06242084)
\lineto(164.7983374,276.03898334)
\lineto(164.2123999,276.03898334)
\lineto(164.2123999,275.10148334)
\curveto(164.2123999,274.90617084)(164.2123999,274.08585834)(163.5483374,274.08585834)
\curveto(162.9233374,274.08585834)(162.9233374,274.90617084)(162.9233374,275.10148334)
\closepath
\moveto(161.5561499,275.96085834)
\curveto(161.5561499,274.28117084)(160.0717749,273.81242084)(159.2905249,273.81242084)
\curveto(158.3920874,273.81242084)(157.5717749,274.39835834)(157.5717749,275.25773334)
\curveto(157.5717749,276.23429584)(158.3920874,277.56242084)(161.5561499,277.67960834)
\closepath
\moveto(161.5561499,275.96085834)
}
}
{
\newrgbcolor{curcolor}{0 0 0}
\pscustom[linestyle=none,fillstyle=solid,fillcolor=curcolor]
{
\newpath
\moveto(170.97728866,277.40617084)
\lineto(170.86010116,277.56242084)
\curveto(170.86010116,277.64054584)(171.99291366,278.85148334)(172.14916366,279.00773334)
\curveto(172.77416366,279.71085834)(173.36010116,280.37492084)(174.76635116,280.37492084)
\lineto(174.76635116,280.99992084)
\curveto(174.25853866,280.96085834)(173.75072616,280.92179584)(173.28197616,280.92179584)
\curveto(172.77416366,280.92179584)(172.07103866,280.96085834)(171.56322616,280.99992084)
\lineto(171.56322616,280.37492084)
\curveto(171.83666366,280.33585834)(171.91478866,280.14054584)(171.91478866,279.98429584)
\curveto(171.91478866,279.94523334)(171.91478866,279.67179584)(171.64135116,279.39835834)
\lineto(170.43041366,278.03117084)
\lineto(168.98510116,279.67179584)
\curveto(168.78978866,279.86710834)(168.78978866,279.94523334)(168.78978866,280.02335834)
\curveto(168.78978866,280.25773334)(169.02416366,280.37492084)(169.25853866,280.37492084)
\lineto(169.25853866,280.99992084)
\curveto(168.63353866,280.96085834)(167.96947616,280.92179584)(167.30541366,280.92179584)
\curveto(166.79760116,280.92179584)(166.13353866,280.96085834)(165.62572616,280.99992084)
\lineto(165.62572616,280.37492084)
\curveto(166.40697616,280.37492084)(166.87572616,280.37492084)(167.34447616,279.86710834)
\lineto(169.49291366,277.36710834)
\curveto(169.53197616,277.32804584)(169.64916366,277.21085834)(169.64916366,277.17179584)
\curveto(169.64916366,277.13273334)(168.32103866,275.68742084)(168.16478866,275.49210834)
\curveto(167.50072616,274.78898334)(166.91478866,274.16398334)(165.54760116,274.12492084)
\lineto(165.54760116,273.49992084)
\curveto(166.05541366,273.53898334)(166.44603866,273.57804584)(166.99291366,273.57804584)
\curveto(167.50072616,273.57804584)(168.20385116,273.53898334)(168.71166366,273.49992084)
\lineto(168.71166366,274.12492084)
\curveto(168.51635116,274.16398334)(168.36010116,274.28117084)(168.36010116,274.51554584)
\curveto(168.36010116,274.86710834)(168.55541366,275.10148334)(168.82885116,275.37492084)
\lineto(170.07885116,276.74210834)
\lineto(171.56322616,274.98429584)
\curveto(171.91478866,274.63273334)(171.91478866,274.59367084)(171.91478866,274.47648334)
\curveto(171.91478866,274.16398334)(171.52416366,274.12492084)(171.44603866,274.12492084)
\lineto(171.44603866,273.49992084)
\curveto(171.60228866,273.49992084)(172.85228866,273.57804584)(173.39916366,273.57804584)
\curveto(173.94603866,273.57804584)(174.53197616,273.53898334)(175.07885116,273.49992084)
\lineto(175.07885116,274.12492084)
\curveto(173.90697616,274.12492084)(173.78978866,274.24210834)(173.36010116,274.67179584)
\closepath
\moveto(170.97728866,277.40617084)
}
}
\end{pspicture}

    \caption{圆柱体形状基本设置。它半径为$r$并沿$z$轴覆盖一定范围。
        通过指定最大值$\varphi$可扫掠出部分圆柱体。}
    \label{fig:3.6}
\end{figure}
\begin{lstlisting}
`\initcode{Cylinder Declarations}{=}`
class `\initvar{Cylinder}{}` : public `\refvar{Shape}{}` {
public:
    `\refcode{Cylinder Public Methods}{}`
protected:
    `\refcode{Cylinder Private Data}{}`
};
\end{lstlisting}

参数形式下,圆柱体由下列方程描述:
\begin{align*}
    \varphi & =u\varphi_{\max}\, ,               \\
    x       & =r\cos\varphi\, ,                  \\
    y       & =r\sin\varphi\, ,                  \\
    z       & =z_{\min}+v(z_{\max}-z_{\min})\, .
\end{align*}

\reffig{3.7}展示了两个圆柱体的渲染图像。像球体图像那样,
左边圆柱体是完整圆柱体,右边是部分圆柱体,因为它的$\varphi_{\max}$值小于$2\pi$。
\begin{figure}[htbp]
    \centering\includegraphics[width=\linewidth]{chap03/twocylinders.png}
    \caption{两个圆柱体。左边是完整圆柱体,右边是部分圆柱体。}
    \label{fig:3.7}
\end{figure}

\begin{lstlisting}
`\initcode{Cylinder Public Methods}{=}`
`\refvar{Cylinder}{}`(const `\refvar{Transform}{}` *ObjectToWorld, const `\refvar{Transform}{}` *WorldToObject,
         bool reverseOrientation, `\refvar{Float}{}` radius, `\refvar{Float}{}` zMin, `\refvar{Float}{}` zMax,
         `\refvar{Float}{}` phiMax)
    : `\refvar{Shape}{}`(ObjectToWorld, WorldToObject, reverseOrientation),
      `\refvar[Cylinder::radius]{radius}{}`(radius), `\refvar[Cylinder::zMin]{zMin}{}`(std::min(zMin, zMax)),
      `\refvar[Cylinder::zMax]{zMax}{}`(std::max(zMin, zMax)),
      `\refvar[Cylinder::phiMax]{phiMax}{}`(`\refvar{Radians}{}`(`\refvar{Clamp}{}`(phiMax, 0, 360))) { }
\end{lstlisting}
\begin{lstlisting}
`\initcode{Cylinder Private Data}{=}`
const `\refvar{Float}{}` `\initvar[Cylinder::radius]{radius}{}`, `\initvar[Cylinder::zMin]{zMin}{}`, `\initvar[Cylinder::zMax]{zMax}{}`, `\initvar[Cylinder::phiMax]{phiMax}{}`;
\end{lstlisting}

\subsection{边界}\label{sub:边界3}
像球那样,圆柱体边界方法用$z$的范围计算保守边界框但不考虑最大的$\varphi$。
\begin{lstlisting}
`\initcode{Cylinder Method Definitions}{=}\initnext{CylinderMethodDefinitions}`
`\refvar{Bounds3f}{}` `\refvar{Cylinder}{}`::`\initvar[Cylinder::ObjectBound]{\refvar{ObjectBound}{}}{}`() const {
    return `\refvar{Bounds3f}{}`(`\refvar{Point3f}{}`(-`\refvar[Cylinder::radius]{radius}{}`, -`\refvar[Cylinder::radius]{radius}{}`, `\refvar[Cylinder::zMin]{zMin}{}`),
                    `\refvar{Point3f}{}`( `\refvar[Cylinder::radius]{radius}{}`,  `\refvar[Cylinder::radius]{radius}{}`, `\refvar[Cylinder::zMax]{zMax}{}`));
}
\end{lstlisting}

\subsection{相交测试}\label{sub:相交测试3}
光线-圆柱体相交公式可以通过把射线方程代入圆柱体的隐式方程得到,和球体的情况一样。
以$z$轴为中心$r$为半径的无限长圆柱体的隐式方程为
\begin{align*}
    x^2+y^2-r^2=0\, .
\end{align*}
代入射线方程\refeq{2.3},我们有
\begin{align*}
    (o_x+td_x)^2+(o_y+td_y)^2=r^2\, .
\end{align*}
我们将其展开并求二次式方程$at^2+bt+c=0$的系数得
\begin{align*}
    a & =d_x^2+d_y^2\, ,      \\
    b & =2(d_xo_x+d_yo_y)\, , \\
    c & =o_x^2+o_y^2-r^2\, .
\end{align*}
\begin{lstlisting}
`\initcode{Compute quadratic cylinder coefficients}{=}`
`\refcode{Initialize EFloat ray coordinate values}{}`
`\refvar{EFloat}{}` a = dx * dx + dy * dy;
`\refvar{EFloat}{}` b = 2 * (dx * ox + dy * oy);
`\refvar{EFloat}{}` c = ox * ox + oy * oy - `\refvar{EFloat}{}`(`\refvar[Cylinder::radius]{radius}{}`) * `\refvar{EFloat}{}`(`\refvar[Cylinder::radius]{radius}{}`);
\end{lstlisting}

\section{圆盘}\label{sec:圆盘}

\keyindex{圆盘}{disk}{}是一种有趣的二次曲面,
因为它有特别简单的相交例程避免求解二次方程。

{\noindent\hfil$=========$\hfil{\color{red}{施工分割线}}\hfil$=========$\
\section{控制舍入误差}\label{sec:控制舍入误差}

\begin{remark}
    本节含有高级内容,第一次阅读时可以跳过。
\end{remark}

到目前为止,我们都是根据基于实数的理想化算术运算
纯粹地讨论光线——形状相交算法。该方法已经让我们走得很远了,
尽管有个重要事实是计算机只能表示有限的数量,
因此它实际上不能表示所有实数。
计算机用浮点数代替实数,它有固定的存储要求。
然而,因为结果可能无法在特定量的内存中表示,
每执行一次浮点运算就可能引入误差。

该误差的累积对相交测试的精度有些许影响。
首先,它可能造成完全错过有效的相交——
例如,一个精确值是正数的相交处$t$值算成了负的。
而且,算得的光线——形状交点可能在形状实际曲面的上面或下面。
这导致一个问题:当从算得的交点开始为阴影射线和反射光线追踪新光线时,
若射线端点在实际曲面之下,我们可能求得一次与曲面错误的再相交。
反之,若端点在曲面上面离得太远,阴影和反射可能会脱钩(见\reffig{3.39})。
\begin{figure}[htbp]
    \centering%LaTeX with PSTricks extensions
%%Creator: Inkscape 1.0.1 (3bc2e813f5, 2020-09-07)
%%Please note this file requires PSTricks extensions
\psset{xunit=.5pt,yunit=.5pt,runit=.5pt}
\begin{pspicture}(384.51998901,248.1499939)
{
\newrgbcolor{curcolor}{0 0 0}
\pscustom[linewidth=1,linecolor=curcolor,linestyle=dashed,dash=2]
{
\newpath
\moveto(219.8500061,133.36999512)
\lineto(190.47000122,209.74999237)
}
}
{
\newrgbcolor{curcolor}{0 0 0}
\pscustom[linewidth=1,linecolor=curcolor]
{
\newpath
\moveto(236.88000488,89.4099884)
\lineto(220.97000122,130.44999695)
}
}
{
\newrgbcolor{curcolor}{0 0 0}
\pscustom[linestyle=none,fillstyle=solid,fillcolor=curcolor]
{
\newpath
\moveto(227.88,127.8599939)
\lineto(221.2,129.8399939)
\lineto(217.61,123.8799939)
\lineto(218.04,137.9999939)
\closepath
}
}
{
\newrgbcolor{curcolor}{0.65098041 0.65098041 0.65098041}
\pscustom[linestyle=none,fillstyle=solid,fillcolor=curcolor]
{
\newpath
\moveto(226.19,128.8799939)
\lineto(218.51,136.7799939)
\lineto(220.98,130.4299939)
\closepath
}
}
{
\newrgbcolor{curcolor}{0.40000001 0.40000001 0.40000001}
\pscustom[linestyle=none,fillstyle=solid,fillcolor=curcolor]
{
\newpath
\moveto(218.18,125.7699939)
\lineto(218.51,136.7799939)
\lineto(220.98,130.4299939)
\closepath
}
}
{
\newrgbcolor{curcolor}{0 0 0}
\pscustom[linewidth=1,linecolor=curcolor,linestyle=dashed,dash=2]
{
\newpath
\moveto(304.55,139.9099939)
\lineto(261.9,24.0399939)
\lineto(236.11,91.4999939)
}
}
{
\newrgbcolor{curcolor}{0 0 0}
\pscustom[linewidth=1,linecolor=curcolor]
{
\newpath
\moveto(320.82998657,184.15999222)
\lineto(307.3500061,147.51999664)
}
}
{
\newrgbcolor{curcolor}{0 0 0}
\pscustom[linestyle=none,fillstyle=solid,fillcolor=curcolor]
{
\newpath
\moveto(303.88,154.0299939)
\lineto(307.57,148.1299939)
\lineto(314.21,150.2299939)
\lineto(304.55,139.9099939)
\closepath
}
}
{
\newrgbcolor{curcolor}{0.65098041 0.65098041 0.65098041}
\pscustom[linestyle=none,fillstyle=solid,fillcolor=curcolor]
{
\newpath
\moveto(304.47,152.1499939)
\lineto(305,141.1499939)
\lineto(307.35,147.5399939)
\closepath
}
}
{
\newrgbcolor{curcolor}{0.40000001 0.40000001 0.40000001}
\pscustom[linestyle=none,fillstyle=solid,fillcolor=curcolor]
{
\newpath
\moveto(312.54,149.1799939)
\lineto(305,141.1499939)
\lineto(307.35,147.5399939)
\closepath
}
}
{
\newrgbcolor{curcolor}{0 0 0}
\pscustom[linewidth=1,linecolor=curcolor,linestyle=dashed,dash=2]
{
\newpath
\moveto(132.72999573,89.11999512)
\lineto(188.08999634,207.56999207)
}
}
{
\newrgbcolor{curcolor}{0 0 0}
\pscustom[linewidth=1,linecolor=curcolor]
{
\newpath
\moveto(110,40.87998962)
\lineto(129.27999878,81.78999329)
}
}
{
\newrgbcolor{curcolor}{0 0 0}
\pscustom[linestyle=none,fillstyle=solid,fillcolor=curcolor]
{
\newpath
\moveto(132.16,74.9999939)
\lineto(129,81.1999939)
\lineto(122.21,79.6899939)
\lineto(132.73,89.1199939)
\closepath
}
}
{
\newrgbcolor{curcolor}{0.65098041 0.65098041 0.65098041}
\pscustom[linestyle=none,fillstyle=solid,fillcolor=curcolor]
{
\newpath
\moveto(131.74,76.9199939)
\lineto(132.18,87.9299939)
\lineto(129.27,81.7699939)
\closepath
}
}
{
\newrgbcolor{curcolor}{0.40000001 0.40000001 0.40000001}
\pscustom[linestyle=none,fillstyle=solid,fillcolor=curcolor]
{
\newpath
\moveto(123.96,80.5899939)
\lineto(132.18,87.9299939)
\lineto(129.27,81.7699939)
\closepath
}
}
{
\newrgbcolor{curcolor}{0 0 0}
\pscustom[linewidth=1,linecolor=curcolor,linestyle=dashed,dash=2]
{
\newpath
\moveto(73.42,114.5799939)
\lineto(102.58,25.0299939)
\lineto(110,40.8699939)
}
}
{
\newrgbcolor{curcolor}{0 0 0}
\pscustom[linewidth=1,linecolor=curcolor]
{
\newpath
\moveto(58.33000183,160.93999481)
\lineto(70.91000366,122.28999329)
}
}
{
\newrgbcolor{curcolor}{0 0 0}
\pscustom[linestyle=none,fillstyle=solid,fillcolor=curcolor]
{
\newpath
\moveto(64.16,125.2499939)
\lineto(70.71,122.8999939)
\lineto(74.63,128.6599939)
\lineto(73.42,114.5799939)
\closepath
}
}
{
\newrgbcolor{curcolor}{0.65098041 0.65098041 0.65098041}
\pscustom[linestyle=none,fillstyle=solid,fillcolor=curcolor]
{
\newpath
\moveto(65.78,124.1399939)
\lineto(73.02,115.8299939)
\lineto(70.91,122.3099939)
\closepath
}
}
{
\newrgbcolor{curcolor}{0.40000001 0.40000001 0.40000001}
\pscustom[linestyle=none,fillstyle=solid,fillcolor=curcolor]
{
\newpath
\moveto(73.96,126.7999939)
\lineto(73.02,115.8299939)
\lineto(70.91,122.3099939)
\closepath
}
}
{
\newrgbcolor{curcolor}{0.98823529 0.93333334 0.12941177}
\pscustom[linestyle=none,fillstyle=solid,fillcolor=curcolor]
{
\newpath
\moveto(195.71,230.6399939)
\lineto(194.14,220.9699939)
\lineto(199.09,229.4199939)
\lineto(195.76,220.1999939)
\lineto(202.18,227.6099939)
\lineto(197.22,219.1599939)
\lineto(204.89,225.2599939)
\lineto(198.46,217.8599939)
\lineto(207.12,222.4499939)
\lineto(199.44,216.3599939)
\lineto(208.79,219.2699939)
\lineto(200.13,214.6999939)
\lineto(209.86,215.8499939)
\lineto(200.5,212.9499939)
\lineto(210.28,212.2899939)
\lineto(200.54,211.1599939)
\lineto(210.03,208.7099939)
\lineto(200.25,209.3899939)
\lineto(209.13,205.2399939)
\lineto(199.64,207.6999939)
\lineto(207.61,201.9899939)
\lineto(198.74,206.1499939)
\lineto(205.52,199.0699939)
\lineto(197.56,204.7999939)
\lineto(202.93,196.5999939)
\lineto(196.16,203.6799939)
\lineto(199.92,194.6299939)
\lineto(194.57,202.8499939)
\lineto(196.61,193.2599939)
\lineto(192.86,202.3099939)
\lineto(193.1,192.5199939)
\lineto(191.08,202.1099939)
\lineto(189.51,192.4299939)
\lineto(189.29,202.2299939)
\lineto(185.97,192.9999939)
\lineto(187.55,202.6799939)
\lineto(182.6,194.2199939)
\lineto(185.93,203.4399939)
\lineto(179.5,196.0299939)
\lineto(184.47,204.4899939)
\lineto(176.8,198.3899939)
\lineto(183.23,205.7799939)
\lineto(174.57,201.1999939)
\lineto(182.25,207.2799939)
\lineto(172.89,204.3699939)
\lineto(181.56,208.9399939)
\lineto(171.83,207.7899939)
\lineto(181.19,210.6899939)
\lineto(171.41,211.3599939)
\lineto(181.15,212.4899939)
\lineto(171.66,214.9299939)
\lineto(181.44,214.2599939)
\lineto(172.56,218.4099939)
\lineto(182.04,215.9399939)
\lineto(174.08,221.6599939)
\lineto(182.95,217.4899939)
\lineto(176.17,224.5699939)
\lineto(184.13,218.8499939)
\lineto(178.76,227.0499939)
\lineto(185.53,219.9599939)
\lineto(181.77,229.0099939)
\lineto(187.12,220.7999939)
\lineto(185.08,230.3799939)
\lineto(188.83,221.3299939)
\lineto(188.59,231.1299939)
\lineto(190.61,221.5399939)
\lineto(192.17,231.2099939)
\lineto(192.4,221.4199939)
\closepath
}
}
{
\newrgbcolor{curcolor}{0 0 0}
\pscustom[linewidth=0.30000001,linecolor=curcolor]
{
\newpath
\moveto(195.71,230.6399939)
\lineto(194.14,220.9699939)
\lineto(199.09,229.4199939)
\lineto(195.76,220.1999939)
\lineto(202.18,227.6099939)
\lineto(197.22,219.1599939)
\lineto(204.89,225.2599939)
\lineto(198.46,217.8599939)
\lineto(207.12,222.4499939)
\lineto(199.44,216.3599939)
\lineto(208.79,219.2699939)
\lineto(200.13,214.6999939)
\lineto(209.86,215.8499939)
\lineto(200.5,212.9499939)
\lineto(210.28,212.2899939)
\lineto(200.54,211.1599939)
\lineto(210.03,208.7099939)
\lineto(200.25,209.3899939)
\lineto(209.13,205.2399939)
\lineto(199.64,207.6999939)
\lineto(207.61,201.9899939)
\lineto(198.74,206.1499939)
\lineto(205.52,199.0699939)
\lineto(197.56,204.7999939)
\lineto(202.93,196.5999939)
\lineto(196.16,203.6799939)
\lineto(199.92,194.6299939)
\lineto(194.57,202.8499939)
\lineto(196.61,193.2599939)
\lineto(192.86,202.3099939)
\lineto(193.1,192.5199939)
\lineto(191.08,202.1099939)
\lineto(189.51,192.4299939)
\lineto(189.29,202.2299939)
\lineto(185.97,192.9999939)
\lineto(187.55,202.6799939)
\lineto(182.6,194.2199939)
\lineto(185.93,203.4399939)
\lineto(179.5,196.0299939)
\lineto(184.47,204.4899939)
\lineto(176.8,198.3899939)
\lineto(183.23,205.7799939)
\lineto(174.57,201.1999939)
\lineto(182.25,207.2799939)
\lineto(172.89,204.3699939)
\lineto(181.56,208.9399939)
\lineto(171.83,207.7899939)
\lineto(181.19,210.6899939)
\lineto(171.41,211.3599939)
\lineto(181.15,212.4899939)
\lineto(171.66,214.9299939)
\lineto(181.44,214.2599939)
\lineto(172.56,218.4099939)
\lineto(182.04,215.9399939)
\lineto(174.08,221.6599939)
\lineto(182.95,217.4899939)
\lineto(176.17,224.5699939)
\lineto(184.13,218.8499939)
\lineto(178.76,227.0499939)
\lineto(185.53,219.9599939)
\lineto(181.77,229.0099939)
\lineto(187.12,220.7999939)
\lineto(185.08,230.3799939)
\lineto(188.83,221.3299939)
\lineto(188.59,231.1299939)
\lineto(190.61,221.5399939)
\lineto(192.17,231.2099939)
\lineto(192.4,221.4199939)
\closepath
}
}
{
\newrgbcolor{curcolor}{0 0 0}
\pscustom[linewidth=1,linecolor=curcolor]
{
\newpath
\moveto(36.70000076,54.98999023)
\lineto(345.42999268,54.98999023)
}
}
{
\newrgbcolor{curcolor}{0 0 0}
\pscustom[linestyle=none,fillstyle=solid,fillcolor=curcolor]
{
\newpath
\moveto(104.71000051,25.27999878)
\curveto(104.71000051,27.3737707)(102.17872502,28.4219561)(100.69838415,26.94161524)
\curveto(99.21804329,25.46127437)(100.26622869,22.92999887)(102.36000061,22.92999887)
\curveto(104.45377253,22.92999887)(105.50195793,25.46127437)(104.02161707,26.94161524)
\curveto(102.5412762,28.4219561)(100.01000071,27.3737707)(100.01000071,25.27999878)
\curveto(100.01000071,23.18622686)(102.5412762,22.13804146)(104.02161707,23.61838232)
\curveto(105.50195793,25.09872318)(104.45377253,27.62999868)(102.36000061,27.62999868)
\curveto(100.26622869,27.62999868)(99.21804329,25.09872318)(100.69838415,23.61838232)
\curveto(102.17872502,22.13804146)(104.71000051,23.18622686)(104.71000051,25.27999878)
\closepath
}
}
{
\newrgbcolor{curcolor}{0 0 0}
\pscustom[linewidth=1,linecolor=curcolor]
{
\newpath
\moveto(104.71000051,25.27999878)
\curveto(104.71000051,27.3737707)(102.17872502,28.4219561)(100.69838415,26.94161524)
\curveto(99.21804329,25.46127437)(100.26622869,22.92999887)(102.36000061,22.92999887)
\curveto(104.45377253,22.92999887)(105.50195793,25.46127437)(104.02161707,26.94161524)
\curveto(102.5412762,28.4219561)(100.01000071,27.3737707)(100.01000071,25.27999878)
\curveto(100.01000071,23.18622686)(102.5412762,22.13804146)(104.02161707,23.61838232)
\curveto(105.50195793,25.09872318)(104.45377253,27.62999868)(102.36000061,27.62999868)
\curveto(100.26622869,27.62999868)(99.21804329,25.09872318)(100.69838415,23.61838232)
\curveto(102.17872502,22.13804146)(104.71000051,23.18622686)(104.71000051,25.27999878)
\closepath
}
}
{
\newrgbcolor{curcolor}{0 0 0}
\pscustom[linestyle=none,fillstyle=solid,fillcolor=curcolor]
{
\newpath
\moveto(264.26000357,25.27999878)
\curveto(264.26000357,27.3737707)(261.72872807,28.4219561)(260.24838721,26.94161524)
\curveto(258.76804634,25.46127437)(259.81623174,22.92999887)(261.91000366,22.92999887)
\curveto(264.00377558,22.92999887)(265.05196098,25.46127437)(263.57162012,26.94161524)
\curveto(262.09127926,28.4219561)(259.56000376,27.3737707)(259.56000376,25.27999878)
\curveto(259.56000376,23.18622686)(262.09127926,22.13804146)(263.57162012,23.61838232)
\curveto(265.05196098,25.09872318)(264.00377558,27.62999868)(261.91000366,27.62999868)
\curveto(259.81623174,27.62999868)(258.76804634,25.09872318)(260.24838721,23.61838232)
\curveto(261.72872807,22.13804146)(264.26000357,23.18622686)(264.26000357,25.27999878)
\closepath
}
}
{
\newrgbcolor{curcolor}{0 0 0}
\pscustom[linewidth=1,linecolor=curcolor]
{
\newpath
\moveto(264.26000357,25.27999878)
\curveto(264.26000357,27.3737707)(261.72872807,28.4219561)(260.24838721,26.94161524)
\curveto(258.76804634,25.46127437)(259.81623174,22.92999887)(261.91000366,22.92999887)
\curveto(264.00377558,22.92999887)(265.05196098,25.46127437)(263.57162012,26.94161524)
\curveto(262.09127926,28.4219561)(259.56000376,27.3737707)(259.56000376,25.27999878)
\curveto(259.56000376,23.18622686)(262.09127926,22.13804146)(263.57162012,23.61838232)
\curveto(265.05196098,25.09872318)(264.00377558,27.62999868)(261.91000366,27.62999868)
\curveto(259.81623174,27.62999868)(258.76804634,25.09872318)(260.24838721,23.61838232)
\curveto(261.72872807,22.13804146)(264.26000357,23.18622686)(264.26000357,25.27999878)
\closepath
}
}
{
\newrgbcolor{curcolor}{0 0 0}
\pscustom[linewidth=1,linecolor=curcolor]
{
\newpath
\moveto(225.3999939,66.95999146)
\lineto(258.75,85.07998657)
}
}
{
\newrgbcolor{curcolor}{1 1 1}
\pscustom[linestyle=none,fillstyle=solid,fillcolor=curcolor]
{
\newpath
\moveto(111.72000265,38.94999695)
\curveto(111.72000265,41.04376887)(109.18872715,42.09195427)(107.70838629,40.6116134)
\curveto(106.22804543,39.13127254)(107.27623082,36.59999704)(109.37000275,36.59999704)
\curveto(111.46377467,36.59999704)(112.51196006,39.13127254)(111.0316192,40.6116134)
\curveto(109.55127834,42.09195427)(107.02000284,41.04376887)(107.02000284,38.94999695)
\curveto(107.02000284,36.85622503)(109.55127834,35.80803963)(111.0316192,37.28838049)
\curveto(112.51196006,38.76872135)(111.46377467,41.29999685)(109.37000275,41.29999685)
\curveto(107.27623082,41.29999685)(106.22804543,38.76872135)(107.70838629,37.28838049)
\curveto(109.18872715,35.80803963)(111.72000265,36.85622503)(111.72000265,38.94999695)
\closepath
}
}
{
\newrgbcolor{curcolor}{0 0 0}
\pscustom[linewidth=1,linecolor=curcolor]
{
\newpath
\moveto(111.72000265,38.94999695)
\curveto(111.72000265,41.04376887)(109.18872715,42.09195427)(107.70838629,40.6116134)
\curveto(106.22804543,39.13127254)(107.27623082,36.59999704)(109.37000275,36.59999704)
\curveto(111.46377467,36.59999704)(112.51196006,39.13127254)(111.0316192,40.6116134)
\curveto(109.55127834,42.09195427)(107.02000284,41.04376887)(107.02000284,38.94999695)
\curveto(107.02000284,36.85622503)(109.55127834,35.80803963)(111.0316192,37.28838049)
\curveto(112.51196006,38.76872135)(111.46377467,41.29999685)(109.37000275,41.29999685)
\curveto(107.27623082,41.29999685)(106.22804543,38.76872135)(107.70838629,37.28838049)
\curveto(109.18872715,35.80803963)(111.72000265,36.85622503)(111.72000265,38.94999695)
\closepath
}
}
{
\newrgbcolor{curcolor}{1 1 1}
\pscustom[linestyle=none,fillstyle=solid,fillcolor=curcolor]
{
\newpath
\moveto(240.07000113,87.34999084)
\curveto(240.07000113,89.44376277)(237.53872563,90.49194816)(236.05838476,89.0116073)
\curveto(234.5780439,87.53126644)(235.6262293,84.99999094)(237.72000122,84.99999094)
\curveto(239.81377314,84.99999094)(240.86195854,87.53126644)(239.38161768,89.0116073)
\curveto(237.90127682,90.49194816)(235.37000132,89.44376277)(235.37000132,87.34999084)
\curveto(235.37000132,85.25621892)(237.90127682,84.20803353)(239.38161768,85.68837439)
\curveto(240.86195854,87.16871525)(239.81377314,89.69999075)(237.72000122,89.69999075)
\curveto(235.6262293,89.69999075)(234.5780439,87.16871525)(236.05838476,85.68837439)
\curveto(237.53872563,84.20803353)(240.07000113,85.25621892)(240.07000113,87.34999084)
\closepath
}
}
{
\newrgbcolor{curcolor}{0 0 0}
\pscustom[linewidth=1,linecolor=curcolor]
{
\newpath
\moveto(240.07000113,87.34999084)
\curveto(240.07000113,89.44376277)(237.53872563,90.49194816)(236.05838476,89.0116073)
\curveto(234.5780439,87.53126644)(235.6262293,84.99999094)(237.72000122,84.99999094)
\curveto(239.81377314,84.99999094)(240.86195854,87.53126644)(239.38161768,89.0116073)
\curveto(237.90127682,90.49194816)(235.37000132,89.44376277)(235.37000132,87.34999084)
\curveto(235.37000132,85.25621892)(237.90127682,84.20803353)(239.38161768,85.68837439)
\curveto(240.86195854,87.16871525)(239.81377314,89.69999075)(237.72000122,89.69999075)
\curveto(235.6262293,89.69999075)(234.5780439,87.16871525)(236.05838476,85.68837439)
\curveto(237.53872563,84.20803353)(240.07000113,85.25621892)(240.07000113,87.34999084)
\closepath
}
}
\end{pspicture}

    \caption{可能在图像中造成可见错误的舍入误差问题几何设置。
        左边的入射光线与曲面相交。在左图中,算得的交点(黑圆圈)略低于曲面
        且阴影射线端点过低的“epsilon”偏移可能导致错误的自相交,
        因为阴影射线端点(白圆圈)仍在曲面之下;因此错误地认定光源被遮挡了。
        右图中,太高的“epsilon”导致错过了有效相交,
        因为射线端点通过了遮挡面。}
    \label{fig:3.39}
\end{figure}

在光线追踪中解决该问题的典型实践是将生成的射线偏移固定的“射线epsilon”值
\sidenote{译者注:epsilon即希腊字母$\epsilon$。},
忽略沿射线$\bm p+t\bm d$比某个$t_{\min}$还近的任何相交。
\reffig{3.40}展示了为什么该方法需要很高的$t_{\min}$值才能高效工作:
如果生成的射线相对于曲面非常倾斜,
则在离射线很远处可能会发生错误的射线相交。
不幸的是,大的$t_{\min}$值会造成射线端点相对远离原始交点,
这又反过来造成错过附近的有效相交,导致阴影和反射丢失细节。
\begin{figure}[htbp]
    \centering%LaTeX with PSTricks extensions
%%Creator: Inkscape 1.0.1 (3bc2e813f5, 2020-09-07)
%%Please note this file requires PSTricks extensions
\psset{xunit=.5pt,yunit=.5pt,runit=.5pt}
\begin{pspicture}(311.58999634,120.58000183)
{
\newrgbcolor{curcolor}{0 0 0}
\pscustom[linewidth=1,linecolor=curcolor]
{
\newpath
\moveto(0,32.56000519)
\lineto(308.73001099,32.56000519)
}
}
{
\newrgbcolor{curcolor}{0 0 0}
\pscustom[linestyle=none,fillstyle=solid,fillcolor=curcolor]
{
\newpath
\moveto(68.00000143,2.84999847)
\curveto(68.00000143,4.9437704)(65.46872593,5.99195579)(63.98838507,4.51161493)
\curveto(62.50804421,3.03127407)(63.5562296,0.49999857)(65.65000153,0.49999857)
\curveto(67.74377345,0.49999857)(68.79195884,3.03127407)(67.31161798,4.51161493)
\curveto(65.83127712,5.99195579)(63.30000162,4.9437704)(63.30000162,2.84999847)
\curveto(63.30000162,0.75622655)(65.83127712,-0.29195884)(67.31161798,1.18838202)
\curveto(68.79195884,2.66872288)(67.74377345,5.19999838)(65.65000153,5.19999838)
\curveto(63.5562296,5.19999838)(62.50804421,2.66872288)(63.98838507,1.18838202)
\curveto(65.46872593,-0.29195884)(68.00000143,0.75622655)(68.00000143,2.84999847)
\closepath
}
}
{
\newrgbcolor{curcolor}{0 0 0}
\pscustom[linewidth=1,linecolor=curcolor]
{
\newpath
\moveto(68.00000143,2.84999847)
\curveto(68.00000143,4.9437704)(65.46872593,5.99195579)(63.98838507,4.51161493)
\curveto(62.50804421,3.03127407)(63.5562296,0.49999857)(65.65000153,0.49999857)
\curveto(67.74377345,0.49999857)(68.79195884,3.03127407)(67.31161798,4.51161493)
\curveto(65.83127712,5.99195579)(63.30000162,4.9437704)(63.30000162,2.84999847)
\curveto(63.30000162,0.75622655)(65.83127712,-0.29195884)(67.31161798,1.18838202)
\curveto(68.79195884,2.66872288)(67.74377345,5.19999838)(65.65000153,5.19999838)
\curveto(63.5562296,5.19999838)(62.50804421,2.66872288)(63.98838507,1.18838202)
\curveto(65.46872593,-0.29195884)(68.00000143,0.75622655)(68.00000143,2.84999847)
\closepath
}
}
{
\newrgbcolor{curcolor}{0 0 0}
\pscustom[linewidth=1,linecolor=curcolor,linestyle=dashed,dash=2]
{
\newpath
\moveto(129.72999573,15.99000549)
\lineto(311.48999023,53.27999878)
}
}
{
\newrgbcolor{curcolor}{0 0 0}
\pscustom[linewidth=1,linecolor=curcolor,linestyle=dashed,dash=2]
{
\newpath
\moveto(120.30999756,14.06000519)
\lineto(129.72999573,15.99000549)
}
}
{
\newrgbcolor{curcolor}{0 0 0}
\pscustom[linewidth=1,linecolor=curcolor]
{
\newpath
\moveto(65.65000153,2.84999847)
\lineto(112.37999725,12.43000031)
}
}
{
\newrgbcolor{curcolor}{0 0 0}
\pscustom[linestyle=none,fillstyle=solid,fillcolor=curcolor]
{
\newpath
\moveto(108.67,6.05000183)
\lineto(111.74,12.30000183)
\lineto(106.46,16.84000183)
\lineto(120.31,14.06000183)
\closepath
}
}
{
\newrgbcolor{curcolor}{0.65098041 0.65098041 0.65098041}
\pscustom[linestyle=none,fillstyle=solid,fillcolor=curcolor]
{
\newpath
\moveto(109.96,7.55000183)
\lineto(119.03,13.80000183)
\lineto(112.36,12.43000183)
\closepath
}
}
{
\newrgbcolor{curcolor}{0.40000001 0.40000001 0.40000001}
\pscustom[linestyle=none,fillstyle=solid,fillcolor=curcolor]
{
\newpath
\moveto(108.23,15.97000183)
\lineto(119.03,13.80000183)
\lineto(112.36,12.43000183)
\closepath
}
}
{
\newrgbcolor{curcolor}{0 0 0}
\pscustom[linewidth=1,linecolor=curcolor,linestyle=dashed,dash=2]
{
\newpath
\moveto(43.90000153,75.87000275)
\lineto(65.65000153,2.84999847)
}
}
{
\newrgbcolor{curcolor}{0 0 0}
\pscustom[linewidth=1,linecolor=curcolor]
{
\newpath
\moveto(30.62000084,120.44000183)
\lineto(41.58000183,83.6400032)
}
}
{
\newrgbcolor{curcolor}{0 0 0}
\pscustom[linestyle=none,fillstyle=solid,fillcolor=curcolor]
{
\newpath
\moveto(34.91,86.77000183)
\lineto(41.4,84.26000183)
\lineto(45.46,89.92000183)
\lineto(43.9,75.87000183)
\closepath
}
}
{
\newrgbcolor{curcolor}{0.65098041 0.65098041 0.65098041}
\pscustom[linestyle=none,fillstyle=solid,fillcolor=curcolor]
{
\newpath
\moveto(36.5,85.62000183)
\lineto(43.52,77.13000183)
\lineto(41.58,83.66000183)
\closepath
}
}
{
\newrgbcolor{curcolor}{0.40000001 0.40000001 0.40000001}
\pscustom[linestyle=none,fillstyle=solid,fillcolor=curcolor]
{
\newpath
\moveto(44.74,88.08000183)
\lineto(43.52,77.13000183)
\lineto(41.58,83.66000183)
\closepath
}
}
{
\newrgbcolor{curcolor}{1 1 1}
\pscustom[linestyle=none,fillstyle=solid,fillcolor=curcolor]
{
\newpath
\moveto(212.12000418,32.54000092)
\curveto(212.12000418,34.63377284)(209.58872868,35.68195823)(208.10838782,34.20161737)
\curveto(206.62804695,32.72127651)(207.67623235,30.19000101)(209.77000427,30.19000101)
\curveto(211.8637762,30.19000101)(212.91196159,32.72127651)(211.43162073,34.20161737)
\curveto(209.95127987,35.68195823)(207.42000437,34.63377284)(207.42000437,32.54000092)
\curveto(207.42000437,30.44622899)(209.95127987,29.3980436)(211.43162073,30.87838446)
\curveto(212.91196159,32.35872532)(211.8637762,34.89000082)(209.77000427,34.89000082)
\curveto(207.67623235,34.89000082)(206.62804695,32.35872532)(208.10838782,30.87838446)
\curveto(209.58872868,29.3980436)(212.12000418,30.44622899)(212.12000418,32.54000092)
\closepath
}
}
{
\newrgbcolor{curcolor}{0 0 0}
\pscustom[linewidth=1,linecolor=curcolor]
{
\newpath
\moveto(212.12000418,32.54000092)
\curveto(212.12000418,34.63377284)(209.58872868,35.68195823)(208.10838782,34.20161737)
\curveto(206.62804695,32.72127651)(207.67623235,30.19000101)(209.77000427,30.19000101)
\curveto(211.8637762,30.19000101)(212.91196159,32.72127651)(211.43162073,34.20161737)
\curveto(209.95127987,35.68195823)(207.42000437,34.63377284)(207.42000437,32.54000092)
\curveto(207.42000437,30.44622899)(209.95127987,29.3980436)(211.43162073,30.87838446)
\curveto(212.91196159,32.35872532)(211.8637762,34.89000082)(209.77000427,34.89000082)
\curveto(207.67623235,34.89000082)(206.62804695,32.35872532)(208.10838782,30.87838446)
\curveto(209.58872868,29.3980436)(212.12000418,30.44622899)(212.12000418,32.54000092)
\closepath
}
}
\end{pspicture}

    \caption{如果算得的交点(实心圆)低于曲面且生成的射线是斜的,
        在与射线端点有一定距离的地方可能会发生错误的再相交(空心圆)。
        如果用沿射线的最小$t$值消除附近的相交,
        需要相对大的$t_{\min}$才能处理好倾斜射线。}
    \label{fig:3.40}
\end{figure}

本节中,我们将介绍浮点算术基本思想并描述分析浮点计算误差的技术。
然后我们将这些方法用于本章之前介绍的光线——形状算法
并展示怎样计算带有有界误差的光线交点。
这将允许我们保守地定位射线端点,这样就永远不会求得错误的自相交,
而又保留了与实际交点极其接近的射线端点使得错误脱靶被最小化。
反过来也不需要额外的“射线epsilon”值。

\subsection{浮点算术}\label{sub:浮点算术}
计算必须在容纳于有限量内存的数字的有限表示上执行;
计算机上无法表示实数的无限集合。
一种这样的有限表示是定点,例如给定一个16位整数,
有人可能通过除以256将其映射为正实数。
这允许我们表示值之间具有相等间距$\displaystyle\frac{1}{256}$的
范围$\displaystyle\left[0,\frac{65535}{256}\right]=\left[0,255+\frac{255}{256}\right]$。
\keyindex{定点数}{fixed-point number}{}可以用整数算术运算高效实现
(该特性使其在早期不支持浮点计算的个人计算机上很流行),
但是它们受制于许多缺点:其中,它们能表示的最大数字是受限的,
且不能精确表示非常小的接近于零的数。

计算机上实数的另一种表示是\keyindex{浮点数}{floating-point number}{}。
它用\keyindex{符号}{sign}{}、\keyindex{有效数字}{significand}{}\footnote{单词“\protect\keyindex{尾数}{mantissa}{}”
    常用来代替“有效数字”,但浮点纯粹主义者注意到“尾数”在对数上下文中
    有不同含义而因此更偏爱“有效数字”。这里我们遵循该用法。}和\keyindex{指数}{exponent}{}表示数字:
本质上和\keyindex{科学计数法}{scientific notation}{}相同但用固定数量的数字表示有效数字和指数。
(下文中,我们将只讨论以2为底的数字。)
这种表示能够用固定数量的存储对极大范围的数字进行表示和执行计算。

用浮点算术的程序员通常知道浮点是不精确的;
有时这种看法导致了浮点算术是不可预测的观念。
本节中我们将看到浮点算术有精心设计的基础反而
能计算特定计算中引入误差的保守边界。
对于光线追踪计算,该误差常意外地小。

现代CPU和GPU几乎都基于电气与电子工程师协会
\sidenote{译者注:即Institute of Electrical and Electronics Engineers (IEEE),
    是电气工程与电子工程以及相关学科的专业协会,成立于1963年1月,总部在美国纽约。
    其范围已经扩展到电气、电子、通信、计算机工程、计算机科学与信息技术等诸多领域,
    是世界上最大的技术专业组织。}
颁布的标准\parencite*{10.1109/IEEESTD.1985.82928,10.1109/IEEESTD.2008.4610935}实现浮点算术模型。
(今后我们说的浮点特指IEEE 754规定的32位浮点数。)
IEEE 754技术标准规定了内存中浮点数的格式以及
精度和浮点计算舍入的特定规则;
正是这些规则使得能对给定浮点值中出现的误差进行严格推导。

\subsubsection*{浮点表示}
IEEE标准规定32位浮点用1位符号、8位指数和23位有效数字表示。
用了8位的指数$e$范围为从0到255;
实际用的指数$e_{\mathrm{b}}$是通过偏置$e$算得的:
\begin{align*}
    e_{\mathrm{b}}=e-127\, .
\end{align*}

当存储\keyindex{规范化的}{normalized}{}浮点值时有效数字实际有24位精度。
当有效数字和指数表示规范化的数字时,有效数字中没有前导零。
在\keyindex{二进制}{binary}{}中,这意味着有效数字开头的数字必须是一;
反过来,没必要显式存储该值。
因此,隐式前导的1位和编码有效数字小数部分的23位给出了总共24位的精度。

给定符号$s=\pm 1$、有效数字$m$和指数$e$,相应的浮点值为
\begin{align*}
    s\times 1.m\times2^{e-127}\, .
\end{align*}

例如,浮点数6.5可以通过规范化的有效数字写作$1.101_2\times2^2$,
其中下标2表示以2为底的值\sidenote{译者注:即二进制。}
(如果非整数的二进制数不够直观,可以注意
小数点右边第一个数表示$2^{-1}$,以此类推)。
因此,我们有
\begin{align*}
    (1\times2^0+1\times2^{-1}+0\times2^{-2}+1\times2^{-3})\times2^2=1.625\times2^2=6.5\, .
\end{align*}
$e_{\mathrm{b}}=2$,所以$e=129=10000001_2$且$m=10100000000000000000000_2$。

浮点在内存中的布局是符号位在32位值的最高位
(负号用一位编码),然后是指数和有效数字。
因此,对于值6.5其内存中的二进制表示是
\begin{align*}
    0\ 10000001\ 10100000000000000000000=40\mathrm{d}00000_{16}\, .
\end{align*}

同样,浮点值1.0有$m=0\ldots0_2$和$e_{\mathrm{b}}=0$,所以$e=127=01111111_2$,它的二进制表示为
\begin{align*}
    0\ 01111111\ 00000000000000000000000=3\mathrm{f}800000_{16}\, .
\end{align*}

该\keyindex{十六进制}{hexadecimal}{}值值得记住,因为调试时它常出现于内存转储。

该表示隐含了整个范围内两个相邻的二的幂次之间
可表示的浮点数之间的间隔是均匀的(它对应于有效数字位增一)。
在范围$[2^e,2^{e+1})$内,间隔为
\begin{align}\label{eq:3.6}
    2^{e-23}\, .
\end{align}
因此对1和2之间的浮点数,$e=0$,
浮点值间的间隔为$2^{-23}\approx1.19209\times10^{-7}$。
该间隔也称为\keyindex{最后一位上的单位值}{unit in last place}{}(ulp)\sidenote{译者注:也叫“最小精度单位”。}的大小;
注意一个ulp的大小由相应浮点值决定——更大的数的ulp比更小的数的ulp相对更大。

按我们目前描述的表示是不可能恰好将零表示为浮点数的。
这事显然不可接受,所以最小指数$e=0$,
或说$e_{\mathrm{b}}=-127$,被留出来特殊对待。
对于该指数,浮点值解释为有效数字中没有隐式前导一位,
这意味全零位的有效数字会得到
\begin{align*}
    s\times0.0\ldots0_2\times2^{-127}=0\, .
\end{align*}

去掉有效数字前导一位也能表示\sidenote{译者注:我完善了这两个式子。}\keyindex{非规范化的}{denormalized}{}数:
如果总是出现前导一,则最小的32位浮点是
\begin{align*}
    1.{\underbrace{0\ldots0}_{\text{23个0}}}\ _2\times2^{-127}\approx5.8774718\times10^{-39}\, .
\end{align*}
没有前导一位,最小值是
\begin{align*}
    0.\underbrace{0\ldots0}_{\text{22个0}}1_2\times2^{-126}=2^{-23}\times2^{-126}\approx1.4012985\times10^{-45}\, .
\end{align*}

有了一些表示这些小值的能力可以避免需要将非常小的数舍入为零。

注意该表示同时有“正”和“负”零值。
该细节对程序员大多是透明的。
例如,标准保证了比较{\ttfamily -0.0 == 0.0}为真,
即使这两值在内存中的表示不同。

最大指数,$e=255$,也保留作特殊对待。
因此,可以表示的最大规范化浮点值有$e=254$(或$e_{\mathrm{b}}=127$)且约为\sidenote{译者注:我完善了该式。}
\begin{align*}
    1.{\underbrace{1\ldots1}_{\text{23个1}}}\ _2\times2^{127}=(2-2^{-23})\times2^{127}\approx3.402823\times10^{38}\, .
\end{align*}

对于$e=255$,若有效数字位全是零,则该值依据符号位对应正或负无穷。
例如,在浮点中执行像1/0的计算会得到无穷值。
对无穷的算术运算得到无穷。
比较时,正无穷大于任何非无穷值,负无穷类似。

常数\refvar{MaxFloat}{}和\refvar{Infinity}{}分别初始化为可表示的最大和“无穷”浮点值。
我们令其可在单独的常数中获取,这样使用这些值的代码
就不需要用唠叨的C++标准库调用来获取它们的值了。
\begin{lstlisting}
`\initcode{Global Constants}{=}\initnext{GlobalConstants}`
static constexpr `\refvar{Float}{}` `\initvar{MaxFloat}{}` = std::numeric_limits<`\refvar{Float}{}`>::max();
static constexpr `\refvar{Float}{}` `\initvar{Infinity}{}` = std::numeric_limits<`\refvar{Float}{}`>::infinity();
\end{lstlisting}

对于$e=255$,非零有效数字位对应
特殊的NaN值\sidenote{译者注:原文误写为$e_b=255$,已修正。},
它由诸如取负数平方根或尝试计算0/0的运算得到。
NaN随计算传播:\keyindex{运算对象}{operand}{}之一
为NaN本身的任何算术运算总是返回NaN。
因此,如果NaN出现于一长串计算中,
我们就知道该方式中的某处出错了。
在调试构建中,pbrt有许多\refvar{Assert}{()}语句检查NaN值,
因为我们几乎从不希望它们出现在事件的常规过程中。
任何与NaN值的比较返回假;
因此检查{\ttfamily !(x == x)}用来检查值是否不是数字
\footnote{这是编译器不得对包含浮点值的表达式执行
看似明显且安全的代数简化的少数几个地方之一——
这个特别的比较不得简化为{\ttfamily false}。
启用编译器的“快速数学”或“执行不安全的数学优化”标志
可能会允许执行这些优化。但是错误行为可能引入pbrt中。}。
为了清楚起见,我们用C++标准库函数{\ttfamily std::isnan()}来检查NaN值。

\subsubsection*{实用例程}
对于某些底层运算,能将浮点值解释为其组成位以及将表示浮点值的数位
转换为实际的{\ttfamily float}或{\ttfamily double}很有用。

一个自然的方法是取指向要转换的值的指针并将其强制转换为另一类型:
{\ttfamily\newline\noindent
float f = ...;\newline\noindent
uint32\_t bits = *((uint32\_t *)\&f);\newline
}
然而,现代版本的C++规定将一种{\ttfamily float}指针强制转换为
不同类型{\ttfamily uint32\_t}是非法的
(该限制允许编译器在分析两个指针是否可能指向
同一内存位置时进行更激进的优化,禁止在寄存器中保存值)。

另一常见方法是对两类元素使用{\ttfamily union},赋予一种类型并按另一种读取:
{\ttfamily\newline\noindent
union FloatBits \{\newline\noindent
\indent float f;\newline\noindent
\indent uint32\_t ui;\newline\noindent
\};\newline\noindent
FloatBits fb;\newline\noindent
fb.f = ...;\newline\noindent
uint32\_t bits = fb.ui;
}

这也是非法的:C++标准说从{\ttfamily union}读取和最后一次赋值时不同的元素是未定义行为。

可以用{\ttfamily memcpy()}将指向源类型的指针复制到指向目标类型的指针来正确执行这些转换。
\begin{lstlisting}
`\initcode{Global Inline Functions}{=}\initnext{GlobalInlineFunctions}`
inline uint32_t `\initvar{FloatToBits}{}`(float f) {
    uint32_t ui;
    memcpy(&ui, &f, sizeof(float));
    return ui;
}
\end{lstlisting}
\begin{lstlisting}
`\refcode{Global Inline Functions}{+=}\lastnext{GlobalInlineFunctions}`
inline float `\initvar{BitsToFloat}{}`(uint32_t ui) {
    float f;
    memcpy(&f, &ui, sizeof(uint32_t));
    return f;
}
\end{lstlisting}

尽管调用函数{\ttfamily memcpy()}以避免这些问题可能看起来太昂贵了,
但实际中好的编译器会将其变为无操作而只是将寄存器或内存中的内容重新解释为另一类型
(pbrt中还有这些函数在{\ttfamily double}和{\ttfamily uint64\_t}之间
转换的类似版本,所以这里就不介绍了)。

这些转换可用于实现函数即把浮点值向上或向下调整到相邻更大或更小的可表示浮点值
\footnote{这些函数等价于{\ttfamily std::nextafter(v, Infinity)}和{\ttfamily std::nextafter(v, -Infinity)}但更加高效,
因为它们不负责处理NaN值或浮点信号异常。}。
它们对我们接下来的代码中需要的某些保守的舍入操作很有用。
多亏浮点在内存中表示的特殊性,这些操作很高效。
\begin{lstlisting}
`\refcode{Global Inline Functions}{+=}\lastnext{GlobalInlineFunctions}`
inline float `\initvar{NextFloatUp}{}`(float v) {
    `\refcode{Handle infinity and negative zero for NextFloatUp()}{}`
    `\refcode{Advance v to next higher float}{}`
}
\end{lstlisting}

有两种重要的特殊情况:如果{\ttfamily v}为正无穷,则该函数就返回没变的{\ttfamily v}。
在继续执行有效数字的代码之前让负零向前跳到正零。
这一步必须显式处理,因为-0.0和0.0的位模式不相邻。
\begin{lstlisting}
`\initcode{Handle infinity and negative zero for NextFloatUp()}{=}`
if (std::isinf(v) && v > 0.)
    return v;
if (v == -0.f)
    v = 0.f;
\end{lstlisting}

概念上,给定一浮点值,我们想对有效数字增加一,
如果结果\keyindex{溢出}{overflow}{},
则有效数字重置为零且指数增加一。
意外的是对浮点在内存中的整数表示加一实现了这点:
因为指数在有效数字之上的高位,所以如果有效数字全是一,
则有效数字低位加一会将一一路带到指数去,
否则就在当前指数下推进到相邻更大的有效数字。
还要注意当增加最大可表示的有限浮点值数位表示时,
会得到正的浮点无穷数位模式。

对于负值,从数位表示减一类似地推进到相邻值。
\begin{lstlisting}
`\initcode{Advance v to next higher float}{=}`
uint32_t ui = `\refvar{FloatToBits}{}`(v);
if (v >= 0) ++ui;
else        --ui;
return `\refvar{BitsToFloat}{}`(ui);
\end{lstlisting}

这里没有介绍函数{\initvar{NextFloatDown}{()}}了,
它遵循相同的逻辑但高效地取反。
pbrt也提供了这些函数的{\ttfamily double}版本。

\subsubsection*{算术运算}
IEEE 754提供了关于浮点算术的重要保证:
具体而言,它保证了加法、减法、乘法、除法和平方根
在相同输入下给出相同结果且这些结果的浮点数最接近于
在无限精度算术下执行底层计算的结果
\footnote{IEEE浮点允许用户选一种数字舍入模式,
    但我们这里假设用默认的——舍入到最近的偶数。}。
值得注意的是这在有限精度数字计算机上是完全可能的;
IEEE 754的成就之一是证明了这种级别的精度是可能的且
能在硬件上很高效地实现。

用圆圈运算符表示浮点算术运算符,用sqrt表示浮点平方根,
这些精度保证可以写作:
\begin{align}
    a\oplus b        & =\mathrm{round}(a+b)\, ,\nonumber      \\
    a\ominus b       & =\mathrm{round}(a-b)\, ,\nonumber      \\
    a\otimes b       & =\mathrm{round}(a*b)\, ,\label{eq:3.7} \\
    a\oslash b       & =\mathrm{round}(a/b)\, ,\nonumber      \\
    \mathrm{sqrt}(a) & =\mathrm{round}(\sqrt{a})\, ,\nonumber
\end{align}
其中$\mathrm{round}(x)$表示将实数舍入到最接近的浮点值的结果。

舍入误差的界可以表示为实数区间:例如
对于加法,我们可以说舍入的结果在与某个$\epsilon$有关的区间内
\begin{align}
    a\oplus b & =\mathrm{round}(a+b)\in(a+b)(1\pm\epsilon)\nonumber \\
              & =[(a+b)(1-\epsilon),(a+b)(1+\epsilon)]\, ,
    \label{eq:3.8}
\end{align}
该舍入引入的误差量不超过在$a+b$处的浮点间隔的一半——
如果它超过浮点间隔的一半,则它会以更小误差舍入到另一个不同的浮点数(\reffig{3.41})。
\begin{figure}[htbp]
    \centering%LaTeX with PSTricks extensions
%%Creator: Inkscape 1.0.1 (3bc2e813f5, 2020-09-07)
%%Please note this file requires PSTricks extensions
\psset{xunit=.5pt,yunit=.5pt,runit=.5pt}
\begin{pspicture}(290.97000122,48.74321747)
{
\newrgbcolor{curcolor}{0 0 0}
\pscustom[linestyle=none,fillstyle=solid,fillcolor=curcolor]
{
\newpath
\moveto(175.22998953,38.20321655)
\curveto(175.22998953,40.29698848)(172.69871403,41.34517387)(171.21837317,39.86483301)
\curveto(169.73803231,38.38449215)(170.7862177,35.85321665)(172.87998962,35.85321665)
\curveto(174.97376155,35.85321665)(176.02194694,38.38449215)(174.54160608,39.86483301)
\curveto(173.06126522,41.34517387)(170.52998972,40.29698848)(170.52998972,38.20321655)
\curveto(170.52998972,36.10944463)(173.06126522,35.06125923)(174.54160608,36.5416001)
\curveto(176.02194694,38.02194096)(174.97376155,40.55321646)(172.87998962,40.55321646)
\curveto(170.7862177,40.55321646)(169.73803231,38.02194096)(171.21837317,36.5416001)
\curveto(172.69871403,35.06125923)(175.22998953,36.10944463)(175.22998953,38.20321655)
\closepath
}
}
{
\newrgbcolor{curcolor}{0 0 0}
\pscustom[linewidth=1,linecolor=curcolor]
{
\newpath
\moveto(175.22998953,38.20321655)
\curveto(175.22998953,40.29698848)(172.69871403,41.34517387)(171.21837317,39.86483301)
\curveto(169.73803231,38.38449215)(170.7862177,35.85321665)(172.87998962,35.85321665)
\curveto(174.97376155,35.85321665)(176.02194694,38.38449215)(174.54160608,39.86483301)
\curveto(173.06126522,41.34517387)(170.52998972,40.29698848)(170.52998972,38.20321655)
\curveto(170.52998972,36.10944463)(173.06126522,35.06125923)(174.54160608,36.5416001)
\curveto(176.02194694,38.02194096)(174.97376155,40.55321646)(172.87998962,40.55321646)
\curveto(170.7862177,40.55321646)(169.73803231,38.02194096)(171.21837317,36.5416001)
\curveto(172.69871403,35.06125923)(175.22998953,36.10944463)(175.22998953,38.20321655)
\closepath
}
}
{
\newrgbcolor{curcolor}{0 0 0}
\pscustom[linewidth=1,linecolor=curcolor]
{
\newpath
\moveto(289.54998779,38.20321655)
\lineto(1.5,38.20321655)
}
}
{
\newrgbcolor{curcolor}{0 0 0}
\pscustom[linewidth=1,linecolor=curcolor]
{
\newpath
\moveto(1.5,47.74321747)
\lineto(1.5,28.65321732)
}
}
{
\newrgbcolor{curcolor}{0 0 0}
\pscustom[linewidth=1,linecolor=curcolor]
{
\newpath
\moveto(73.48999786,47.74321747)
\lineto(73.48999786,28.65321732)
}
}
{
\newrgbcolor{curcolor}{0 0 0}
\pscustom[linewidth=1,linecolor=curcolor]
{
\newpath
\moveto(289.47000122,47.74321747)
\lineto(289.47000122,28.65321732)
}
}
{
\newrgbcolor{curcolor}{0 0 0}
\pscustom[linewidth=1,linecolor=curcolor]
{
\newpath
\moveto(145.47999573,47.74321747)
\lineto(145.47999573,28.65321732)
}
}
{
\newrgbcolor{curcolor}{0 0 0}
\pscustom[linewidth=1,linecolor=curcolor]
{
\newpath
\moveto(217.47999573,47.74321747)
\lineto(217.47999573,28.65321732)
}
}
{
\newrgbcolor{curcolor}{0 0 0}
\pscustom[linewidth=1,linecolor=curcolor]
{
\newpath
\moveto(172.7499996,24.85321747)
\lineto(172.7499996,20.18321747)
\lineto(145.6299996,20.18321747)
\lineto(145.6299996,24.99321747)
}
}
{
\newrgbcolor{curcolor}{0 0 0}
\pscustom[linestyle=none,fillstyle=solid,fillcolor=curcolor]
{
\newpath
\moveto(159.9560781,9.9687526)
\curveto(157.4873281,9.3750026)(155.5498281,6.7812526)(155.5498281,4.3750026)
\curveto(155.5498281,2.4375026)(156.8310781,1.0000026)(158.7060781,1.0000026)
\curveto(161.0185781,1.0000026)(162.6748281,4.1562526)(162.6748281,6.9062526)
\curveto(162.6748281,8.7187526)(161.8935781,9.7187526)(161.2060781,10.5937526)
\curveto(160.4873281,11.5000026)(159.2998281,13.0000026)(159.2998281,13.8750026)
\curveto(159.2998281,14.3125026)(159.7060781,14.7812526)(160.3935781,14.7812526)
\curveto(161.0185781,14.7812526)(161.3935781,14.5312526)(161.8310781,14.2500026)
\curveto(162.2373281,14.0000026)(162.6123281,13.7500026)(162.9248281,13.7500026)
\curveto(163.4248281,13.7500026)(163.7060781,14.2187526)(163.7060781,14.5625026)
\curveto(163.7060781,15.0000026)(163.3935781,15.0312526)(162.6748281,15.2187526)
\curveto(161.6435781,15.4375026)(161.3623281,15.4375026)(161.0498281,15.4375026)
\curveto(159.4873281,15.4375026)(158.7685781,14.5625026)(158.7685781,13.3750026)
\curveto(158.7685781,12.2812526)(159.3623281,11.1875026)(159.9560781,9.9687526)
\closepath
\moveto(160.2060781,9.5312526)
\curveto(160.7060781,8.5937526)(161.2998281,7.5312526)(161.2998281,6.0937526)
\curveto(161.2998281,4.7812526)(160.5498281,1.4375026)(158.7060781,1.4375026)
\curveto(157.6123281,1.4375026)(156.7685781,2.2812526)(156.7685781,3.8125026)
\curveto(156.7685781,5.0625026)(157.5185781,8.8125026)(160.2060781,9.5312526)
\closepath
\moveto(160.2060781,9.5312526)
}
}
\end{pspicture}

    \caption{IEEE标准规定浮点计算必须实现为假设以无限精度的实数
        执行计算再舍入到最接近的可表示浮点。
        这里,无限精度得到的实数表示为实心点,
        它附近可表示的浮点表示为数轴上的刻度。
        我们可以看到舍入到最近浮点引入的误差$\delta$不超过
        浮点之间间隔的一半。}
    \label{fig:3.41}
\end{figure}

对于32位浮点,我们可以用\refeq{3.6}确定
在$a+b$处的浮点间隔(即该值处的ulp)上界为$(a+b)2^{-23}$,
所以间隔一半的上界为$(a+b)2^{-24}$,所以$|\epsilon|\le2^{-24}$。
该界称为\keyindex{机器$\epsilon$}{machine epsilon}{}
\footnote{不幸的是,C和C++标准用它们自己的特殊方式定义了机器$\epsilon$,
    即数字1之上一个ulp的大小。对于32位浮点,该值为$2^{-23}$,
    是数值分析用的术语机器$\epsilon$的两倍大。}。
对于32位浮点,$\epsilon_{\mathrm{m}}=2^{-24}\approx5.960464\times10^{-8}$。
\begin{lstlisting}
`\refcode{Global Constants}{+=}\lastnext{GlobalConstants}`
static constexpr `\refvar{Float}{}` `\initvar{MachineEpsilon}{}` =
       std::numeric_limits<`\refvar{Float}{}`>::epsilon() * 0.5;
\end{lstlisting}
因此我们有
\begin{align*}
    a\oplus b & =\mathrm{round}(a+b)\in(a+b)(1\pm\epsilon_{\mathrm{m}})\nonumber     \\
              & =[(a+b)(1-\epsilon_{\mathrm{m}}),(a+b)(1+\epsilon_{\mathrm{m}})]\, ,
\end{align*}

类似的关系对其他算术运算符和平方根运算符成立
\footnote{该界假设计算中没有上溢或\protect\keyindex{下溢}{underflow}{};
    可以很容易处理这些可能的情况\citep[p.56]{doi:10.1137/1.9780898718027}但
    一般对于我们这里的应用并不重要。}。

可以从\refeq{3.7}直接得到许多有用的性质。
对于浮点数$x$,
\begin{itemize}
    \item $1\otimes x=x$。
    \item $x\oslash x=1$。
    \item $x\oplus 0=x$。
    \item $x\ominus x=0$。
    \item $2\otimes x$和$x\oslash 2$是准确的;计算最终结果没有执行舍入。
          更一般地,任何乘以或除以二的幂都得到准确结果(假设没有上溢或下溢)。
    \item $x\oslash 2^i=x\otimes 2^{-i}$对所有整数$i$成立,假设$2^i$不溢出。
\end{itemize}

所有这些性质都是从结果必须是与实际结果最接近的浮点值这一原则推出的;
当结果可以准确表示时,必须算得准确结果。

\subsubsection*{误差传播}
利用IEEE浮点算术的保证,可以开发方法分析并界定给定浮点计算的误差。
关于该话题的更多细节,详见\citet{doi:10.1137/1.9780898718027}的优秀书籍
以及\citet{10.5555/1096474}的早期经典
\sidenote{译者注:笔者将引用换成了1994年新版。}。

在这项工作中有两种有用的误差度量:绝对的和相对的。
如果我们执行某浮点计算并得到舍入的结果$\tilde{a}$,
我们说$\tilde{a}$和以实数进行该计算的结果之间的
差的大小为\keyindex{绝对误差}{absolute error}{}$\delta_{\mathrm{a}}$:
\begin{align*}
    \delta_{\mathrm{a}}=|\tilde{a}-a|\, .
\end{align*}

\keyindex{相对误差}{relative error}{}$\delta_{\mathrm{r}}$是
绝对误差和精确结果的比值:
\begin{align}\label{eq:3.9}
    \delta_{\mathrm{r}}=\left|\frac{\tilde{a}-a}{a}\right|=\left|\frac{\delta_{\mathrm{a}}}{a}\right|\, ,
\end{align}
只要$a\neq0$。利用相对误差定义,
我们可将算得的值$\tilde{a}$写作准确结果$a$的扰动:
\begin{align*}
    \tilde{a}=a\pm\delta_{\mathrm{a}}=a(1\pm\delta_{\mathrm{r}})\, .
\end{align*}

作为这些思想的首个应用,考虑计算四个表示为浮点的数$a,b,c$和$d$的和。
如果我们将该和算为{\ttfamily r = (((a + b) + c) + d)},\refeq{3.8}给出
\begin{align*}
    (((a\oplus b)\oplus c)\oplus d) & \in((((a+b)(1\pm\epsilon_{\mathrm{m}}))+c)(1\pm\epsilon_{\mathrm{m}})+d)(1\pm\epsilon_{\mathrm{m}}) \\
                                    & =(a+b)(1\pm\epsilon_{\mathrm{m}})^3+c(1\pm\epsilon_{\mathrm{m}})^2+d(1\pm\epsilon_{\mathrm{m}})\, .
\end{align*}
因$\epsilon_{\mathrm{m}}$很小,$\epsilon_{\mathrm{m}}$的高次幂可被额外项$\epsilon_{\mathrm{m}}$限定,
所以我们可将$(1\pm\epsilon_{\mathrm{m}})^n$限为
\begin{align*}
    (1\pm\epsilon_{\mathrm{m}})^n\le1\pm(n+1)\epsilon_{\mathrm{m}}\, .
\end{align*}
(实际情况是,$1\pm n\epsilon_{\mathrm{m}}$几乎界定了这些项,
因为$\epsilon_{\mathrm{m}}$的更高次幂变小得很快,但上面是完全保守的界。)

该界让我们把加法的结果化简为:
\begin{align*}
      & (a+b)(1\pm4\epsilon_{\mathrm{m}})+c(1\pm3\epsilon_{\mathrm{m}})+d(1\pm2\epsilon_{\mathrm{m}})    \\
    = & a+b+c+d+[\pm4\epsilon_{\mathrm{m}}(a+b)\pm3\epsilon_{\mathrm{m}}c\pm2\epsilon_{\mathrm{m}}d]\, .
\end{align*}

方括号内的项给出了绝对误差:其大小限定为
\begin{align}\label{eq:3.10}
    \pm4\epsilon_{\mathrm{m}}|a+b|\pm3\epsilon_{\mathrm{m}}|c|\pm2\epsilon_{\mathrm{m}}|d|\, .
\end{align}

因此,如果我们按上述括号把四个浮点数加在一起,
我们可以确定最后舍入的结果与假设我们
用无限精度实数相加得到的结果之间的差被\refeq{3.10}界定。
给定$a,b,c$和$d$的具体值很容易计算该误差界。

这个结果很有趣;我们看到$a+b$的大小对误差界作相对较大的贡献,
尤其是和$d$相比的时候
(这个结果给出了一种情况,即为什么如果大量浮点数相加时,
将他们从小到大排序一般会给出比任意顺序有更低最终误差的结果)。

这里我们的分析隐含假设编译器会根据定义该和的表达式生成指令。
编译器应遵循给定浮点表达式的形式以避免破坏仔细设计的能最小化舍入误差的计算。
这又有某些用整数表达式时有效的转换不能安全用于浮点计算的情况。

如果我们把表达式改为算术等价的{\ttfamily float r = (a + b) + (c + d)}会怎么样?
相应的浮点计算为
\begin{align*}
    ((a\oplus b)\oplus(c\oplus d))\, .
\end{align*}

如果我们采用运用\refeq{3.8}的相同过程,展开项,
将高次项$(1\pm\epsilon_{\mathrm{m}})^n$转换为$1\pm(n+1)\epsilon_{\mathrm{m}}$,
我们得到绝对误差界为
\begin{align*}
    3\epsilon_{\mathrm{m}}|a+b|+3\epsilon_{\mathrm{m}}|c+d|\, ,
\end{align*}
它在$|a+b|$相对较大时低于第一种算法,但若$|d|$相对较大则可能更高。

这种计算误差的方法称为\keyindex{前向误差分析}{forward error analysis}{};
给定计算输入,我们可用很机械的过程提供结果误差的保守边界。
结果中推导的界可能夸大了实际误差——
实际中误差项的符号通常是混合的,所以当它们相加时会有抵消
\footnote{一些数值分析员用的经验法则是,因为中间结果误差抵消,
    实际误差的ulp数通常接近于ulp的边界数的平方根。}
\sidenote{译者注:这句脚注看不懂。}。
另一种方法是\keyindex{后向误差分析}{backward error analysis}{},
把算得结果当做准确的并提供所给结果相同时对输入扰动的界。
当分析数值算法的稳定性时该方法更有用,
但不太适合于推导我们这里关注的几何计算的保守误差界。

用$1\pm(n+1)\epsilon_{\mathrm{m}}$作为$(1\pm\epsilon_{\mathrm{m}})^n$的
保守边界还是有点不满意,因为它仅是加上整个$\epsilon_{\mathrm{m}}$项以
保守地界定各个$\epsilon_{\mathrm{m}}$高次项的和。
\citet[3.1节]{doi:10.1137/1.9780898718027}给出了
一个更紧致界定误差项$1\pm\epsilon_{\mathrm{m}}$之积的方法
\sidenote{译者注:笔者无法阅读到该文献,
    但认为可以利用级数展开和二项式展开证明相应不等式成立。}。
如果我们有$(1\pm\epsilon_{\mathrm{m}})^n$,
则可以证明该值被$1+\theta_n$界定,其中
\begin{align}\label{eq:3.11}
    |\theta_n|\le\frac{n\epsilon_{\mathrm{m}}}{1-n\epsilon_{\mathrm{m}}}\, ,
\end{align}
只要$n\epsilon_{\mathrm{m}}<1$(当然是我们考虑计算的情况)
\sidenote{译者注:更确切说是$0\le n\epsilon_{\mathrm{m}}<1$。}。
注意到对于合理$n$值该表达式的分母会小于一,
所以它只是稍稍增大$n\epsilon_{\mathrm{m}}$以获得保守边界。

我们用$\gamma_n$表示该界:
\begin{align*}
    \gamma_n=\frac{n\epsilon_{\mathrm{m}}}{1-n\epsilon_{\mathrm{m}}}\, .
\end{align*}

计算该值的函数声明为{\ttfamily constexpr},
这样任何带有编译时常量的调用都会被替换为相应的浮点返回值。
\begin{lstlisting}
`\refcode{Global Inline Functions}{+=}\lastnext{GlobalInlineFunctions}`
inline constexpr `\refvar{Float}{}` `\initvar{gamma}{}`(int n) {
    return (n * `\refvar{MachineEpsilon}{}`) / (1 - n * `\refvar{MachineEpsilon}{}`);
}
\end{lstlisting}

使用$\gamma$符号,四个值之和的误差被界定为
\begin{align*}
    |a+b|\gamma_3+|c|\gamma_2+|d|\gamma_1\, .
\end{align*}

该方法的一个优点是$(1\pm\epsilon_{\mathrm{m}})^n$项的商也可以用$\gamma$函数定界。
给定
\begin{align*}
    \frac{(1\pm\epsilon_{\mathrm{m}})^m}{(1\pm\epsilon_{\mathrm{m}})^n}\, ,
\end{align*}
该区间以$1\pm\gamma_{m+n}$为界
\sidenote{译者注:我没有理解为何有这个结论,希望读者提供帮助。}。
因此,$\gamma$可通过除法用于合并方程两边的$\epsilon_{\mathrm{m}}$项;
在下面一些推导中这会很有用
(注意因为$1\pm\epsilon_{\mathrm{m}}$项表示区间,消去它们是错的:
\begin{align*}
    \frac{(1\pm\epsilon_{\mathrm{m}})^m}{(1\pm\epsilon_{\mathrm{m}})^n}\neq(1\pm\epsilon_{\mathrm{m}})^{m-n}\, ;
\end{align*}
必须换为边界$\gamma_{m+n}$)。

给定一些本身带有一定量误差的计算输入,
观察该误差如何被带到各种基本算术运算中是有益的。
给定都带有之前运算累积误差的
两个值$a(1\pm\gamma_i)$和$b(1\pm\gamma_j)$,考虑它们的积。
利用$\otimes$的定义,结果在区间
\begin{align*}
    a(1\pm\gamma_i)\otimes b(1\pm\gamma_j)\in ab(1\pm\gamma_{i+j+1})
\end{align*}
中,其中我们用了直接从\refeq{3.11}推得的关系$(1\pm\gamma_i)(1\pm\gamma_j)\in(1\pm\gamma_{i+j})$。

该结果的相对误差界定为:
\begin{align*}
    \left|\frac{ab\gamma_{i+j+1}}{ab}\right|=\gamma_{i+j+1}\, ,
\end{align*}
因此最终误差大约为乘积值处ulp的$\displaystyle\frac{1}{2}(i+j+1)$——和我们希望的乘法误差一样好
(除法的情况也一样好)。

不幸的是,加减法中相对误差可能大幅增加。
使用相同运算值定义,考虑
\begin{align*}
    a(1\pm\gamma_i)\oplus b(1\pm\gamma_j)\, ,
\end{align*}
它在区间$a(1\pm\gamma_{i+1})+b(1\pm\gamma_{j+1})$内,
所以绝对误差定界为$|a|\gamma_{i+1}+|b|\gamma_{j+1}$。

如果$a$和$b$同号,则绝对误差定界为$|a+b|\gamma_{i+j+1}$且
相对误差约为算得值附近ulp的$\displaystyle\frac{1}{2}(i+j+1)$。

然而,如果$a$和$b$异号(或者等价地,它们同号但做减法),
则相对误差可能很高。考虑$a\approx-b$的情况:相对误差为
\begin{align*}
    \frac{|a|\gamma_{i+1}+|b|\gamma_{j+1}}{a+b}\approx\frac{2|a|\gamma_{i+j+1}}{a+b}\, .
\end{align*}
分子的大小与原始值$|a|$成正比但除以一个非常小的数,因此相对误差很高。
这种相对误差的大幅增加称为\keyindex{灾难性抵消}{catastrophic cancellation}{}。
等价地,我们从绝对误差取决于$|a|$值大小的事实中感受到了问题所在,
尽管现在问题在于一个远小于$a$的值。

\subsubsection*{运行误差分析}
除了用代数算出误差边界外,我们还可以让计算机在执行计算时为我们完成这项工作。
该方法称为\keyindex{运行误差分析}{running error analysis}{}。
其背后的思想很简单:每次执行浮点运算时,我们也基于\refeq{3.7}计算给出区间的项
以算出目前已积累误差的运行边界。
尽管该方法比推导直接给出误差边界的表达式有更高的运行时开销,
但当推导变得很难时它会很方便。

pbrt提供了简单的类\refvar{EFloat}{},
绝大部分就像常规的{\ttfamily float}那样
但使用运算符重载提供了浮点的所有常规算术运算并计算它们的误差边界。

和\refchap{几何与变换}的类\refvar{Interval}{}相似,
\refvar{EFloat}{}跟踪一个描述感兴趣的值不确定性的区间。
与\refvar{Interval}{}相比,\refvar{EFloat}{}的区间是
由于中间浮点算术的误差产生的而不是输入参数的不确定性。
\begin{lstlisting}
`\initcode{EFloat Public Methods}{=}\initnext{EFloatPublicMethods}`
`\initvar{EFloat}{}`() { }
`\refvar{EFloat}{}`(float v, float err = 0.f) : `\refvar[EFloat::v]{v}{}`(v), `\refvar[EFloat::err]{err}{}`(err) {
    `\refcode{Store high-precision reference value in EFloat}{}`
}
\end{lstlisting}

\begin{lstlisting}
`\initcode{EFloat Private Data}{=}\initnext{EFloatPrivateData}`
float `\initvar[EFloat::v]{v}{}`;
float `\initvar[EFloat::err]{err}{}`;
\end{lstlisting}

在调试构建中,\refvar{EFloat}{}还维护\refvar[EFloat::v]{v}{}的高精度版本
用作参考值以计算相对误差的精确近似。
在优化构建中,我们一般不为计算该额外值花费开销。
\begin{lstlisting}
`\initcode{Store high-precision reference value in EFloat}{=}`
#ifndef NDEBUG
ld = v;
#endif // NDEBUG
\end{lstlisting}
\begin{lstlisting}
`\refcode{EFloat Private Data}{+=}\lastcode{EFloatPrivateData}`
#ifndef NDEBUG
long double `\initvar[EFloat::ld]{ld}{}`;
#endif // NDEBUG
\end{lstlisting}

该类的加法运算实现本质上是相关定义的实现。我们有:
\begin{align*}
    (a\pm\delta_a)\oplus(b\pm\delta_b) & =((a\pm\delta_a)+(b\pm\delta_b))(1\pm\gamma_1)                                          \\
                                       & =a+b+(\pm\delta_a\pm\delta_b\pm(a+b)\gamma_1\pm\gamma_1\delta_a\pm\gamma_1\delta_b)\, .
\end{align*}
所以(括号里的)绝对误差定界为
\begin{align*}
    \delta_a+\delta_b+\gamma_1(|a+b|+\delta_a+\delta_b)\, .
\end{align*}
\begin{lstlisting}
`\refcode{EFloat Public Methods}{+=}\lastnext{EFloatPublicMethods}` 
`\refvar{EFloat}{}` operator+(`\refvar{EFloat}{}` f) const {
    `\refvar{EFloat}{}` r;
    r.`\refvar[EFloat::v]{v}{}` = `\refvar[EFloat::v]{v}{}` + f.`\refvar[EFloat::v]{v}{}`;
#ifndef NDEBUG
    r.`\refvar[EFloat::ld]{ld}{}` = `\refvar[EFloat::ld]{ld}{}` + f.`\refvar[EFloat::ld]{ld}{}`;
#endif  // DEBUG
    r.`\refvar[EFloat::err]{err}{}` = `\refvar[EFloat::err]{err}{}` + f.`\refvar[EFloat::err]{err}{}` +
        `\refvar{gamma}{}`(1) * (std::abs(`\refvar[EFloat::v]{v}{}` + f.`\refvar[EFloat::v]{v}{}`) + `\refvar[EFloat::err]{err}{}` + f.`\refvar[EFloat::err]{err}{}`);
    return r;
}
\end{lstlisting}

\refvar{EFloat}{}的其他算术运算实现是类似的。

注意该实现忽略了计算误差本身也受到舍入误差影响的问题。
如果这有问题,我们可以切换浮点舍入模式使误差边界总是向正无穷大舍入,
但这会是开销很大的操作,因为它在当前处理器上引发完全的管道刷新
\sidenote{译者注:原文a full pipeline flush。}。
这里我们用默认舍入模式:后文中,当它们用于解决该问题时,
误差边界扩展一个ulp。

\refvar{EFloat}{}中的{\ttfamily float}值可通过类型转换运算符获取;
它有修饰符{\ttfamily explicit}以要求调用者
用显式{\ttfamily (float)}转换来提取浮点值。
使用显式转换降低了无意从\refvar{EFloat}{}到\refvar{Float}{}以及
倒回的风险而丢失积累的误差边界。
\begin{lstlisting}
`\refcode{EFloat Public Methods}{+=}\lastnext{EFloatPublicMethods}`
explicit operator float() const { return `\refvar[EFloat::v]{v}{}`; }
\end{lstlisting}

如果用\refvar{EFloat}{}而不是浮点类型变量执行一系列计算,
则计算中的任何地方都可以调用方法\refvar{GetAbsoluteError}{()}求得
算得值的绝对误差边界。
\begin{lstlisting}
`\refcode{EFloat Public Methods}{+=}\lastnext{EFloatPublicMethods}`
float `\initvar{GetAbsoluteError}{}`() const { return `\refvar[EFloat::err]{err}{}`; }
\end{lstlisting}

误差区间边界可通过方法\refvar{UpperBound}{()}和\refvar{LowerBound}{()}获取。
它们的实现分别使用\refvar{NextFloatUp}{()}和\refvar{NextFloatDown}{()}将
返回的值扩展一个ulp,保证区间是保守的。
\begin{lstlisting}
`\refcode{EFloat Public Methods}{+=}\lastnext{EFloatPublicMethods}`
float `\initvar{UpperBound}{}`() const { return `\refvar{NextFloatUp}{}`(`\refvar[EFloat::v]{v}{}` + `\refvar[EFloat::err]{err}{}`); }
float `\initvar{LowerBound}{}`() const { return `\refvar{NextFloatDown}{}`(`\refvar[EFloat::v]{v}{}` - `\refvar[EFloat::err]{err}{}`); }
\end{lstlisting}

在调试构建中,可以用方法获取相对误差和维护在\refvar[EFloat::ld]{ld}{}中的精确值。
\begin{lstlisting}
`\refcode{EFloat Public Methods}{+=}\lastcode{EFloatPublicMethods}`
#ifndef NDEBUG
float `\initvar{GetRelativeError}{}`() const { return std::abs((`\refvar[EFloat::ld]{ld}{}` - `\refvar[EFloat::v]{v}{}`)/`\refvar[EFloat::ld]{ld}{}`); }
long double `\initvar{PreciseValue}{}`() const { return `\refvar[EFloat::ld]{ld}{}`; }
#endif
\end{lstlisting}

pbrt还提供了函数\refvar{Quadratic}{()}的变种,
它在可能有误差的系数上运算并返回{\ttfamily t0}和{\ttfamily t1}值以及误差边界。
其实现和常规函数\refvar{Quadratic}{()}一样,只是使用了\refvar{EFloat}{}。
\begin{lstlisting}
`\initcode{EFloat Inline Functions}{=}`
inline bool `\initvar[Quadratic:2]{\refvar{Quadratic}{}}{}`(`\refvar{EFloat}{}` A, `\refvar{EFloat}{}` B, `\refvar{EFloat}{}` C,
                      `\refvar{EFloat}{}` *t0, `\refvar{EFloat}{}` *t1);
\end{lstlisting}

有了浮点误差基本知识,我们现在专注用这些工具提供稳定的相交运算。

\subsection{保守的光线——边界框相交}
浮点舍入误差可能造成光线——边界框相交测试错失光线实际上与框相交了的情况。

\subsection{稳定的三角形相交}\label{sub:稳定的三角形相交}

\subsection{定界交点误差}\label{sub:定界交点误差}

\section{译者补充:微分几何基础}\label{sec:译者补充:微分几何基础}
\begin{remark}
    本节内容不是原书内容,而是译者参照教材补充的,请酌情参考和斧正。
\end{remark}

\subsection{曲线的概念}\label{sub:曲线的概念}
\begin{definition}
    给出两个集合$E$和$E'$,如果集合$E$中的每一个点(或称元素)$x$,
    有$E'$中的点$x'$和它对应,
    则我们说给定了$E$到$E'$的一个\keyindex{映射}{mapping}{}$f$.
    $x'$称为点$x$的\keyindex{象}{image}{},
    $x$称为$x'$的\keyindex{原象}{inverse image}{}。
\end{definition}
\begin{definition}
    对于任取集合$E$中的点$x_1$和$x_2$,如果$x_1\neq x_2$时有$f(x_1)\neq f(x_2)$,
    则称映射$f$是\keyindex{一一映射}{one-one mapping}{mapping映射}或\keyindex{单射}{injection}{}。
\end{definition}
\begin{definition}
    在欧氏空间中给出两个集合$E,E'$,
    对于$E$中任一个点$x_0$和任一个数$\varepsilon>0$,
    存在数$\delta>0$,使得对于$E$中与$x_0$的距离小于$\delta$的任意一点$x$来说,
    点$f(x)$与$f(x_0)$间的距离小于$\varepsilon$,
    则称映射$f$是\keyindex{连续的}{continuous}{}。
\end{definition}
\begin{definition}
    如果$f(E)=E'$,则称$f$是从$E$到$E'$的\keyindex{到上映射}{onto mapping}{mapping映射}
    或\keyindex{满射}{surjection}{mapping映射}。
\end{definition}
\begin{definition}
    如果一个开的直线段到三维欧氏空间内建立的对应$f$是一一的、双方连续的到上映射
    (这种映射称为\keyindex{拓扑映射}{topological mapping}{mapping映射}或\keyindex{同态映射}{homeomorphic mapping}{mapping映射}),
    则我们把三维欧氏空间中的映射的象称为\keyindex{简单曲线段}{%simple curve segment
    }{curve曲线}。
\end{definition}

我们以后所讨论的曲线都是简单曲线段,不另做声明。

我们可以确立曲线的方程。在直线段上引入坐标$t(a<t<b)$,
在空间中引入笛卡尔直角坐标$(x,y,z)$,
则上述映射的解析表达式是
\begin{align}\label{eq:03ex01.1}
    \left\{\begin{array}{c}
        x=f(t), \\
        y=g(t), \\
        z=h(t),
    \end{array}
    \right.\quad a<t<b\, .
\end{align}
习惯上常把\refeq{03ex01.1}中的函数关系符号$f,g,h$分别
写作$x,y,z$,于是\refeq{03ex01.1}可写为
\begin{align}\label{eq:03ex01.2}
    \left\{\begin{array}{c}
        x=x(t), \\
        y=y(t), \\
        z=z(t),
    \end{array}
    \right.\quad a<t<b\, .
\end{align}
\refeq{03ex01.2}称为曲线的\keyindex{参数表示}{parametric representation}{}或\keyindex{参数方程}{parametric equation}{},
$t$称为曲线的\keyindex{参数}{parameter}{}。

由于向量函数$\bm r(t)$可表示为$\bm r(t)=x(t)\mathbf{i}+y(t)\mathbf{j}+z(t)\mathbf{k}$,
因而曲线的参数方程\refeq{03ex01.2}也可以写成向量函数的形式:
\begin{align}\label{eq:03ex01.3}
    \bm r=\bm r(t),\quad a<t<b\, .
\end{align}
即在空间中给定一点$O$,以该点作为始点放上对于$t$的所有值的向量$\bm r(t)$.
于是对于$t$的每个值,我们得到确定的向量$\overrightarrow{OM}=\bm r(t)$,
它的始点是点$O$,而终点$M$则与$t$值有关,当$t$在$(a,b)$内变化时,
点$M$在空间中画出一条轨迹,这就是由参数$t$所给定的曲线。
点$M$的向量表达式称为曲线的\keyindex{向量参数表示}{}{parametric representation参数表示}。

\begin{definition}
    如果曲线的参数表示式中的函数
    是\keyindex{$k$阶连续可微}{$k$-times continuously differentiable}{}的函数,
    则把该曲线称为\keyindex{$C^k$阶曲线}{$C^k$-curve}{curve曲线}。
    当$k=1$时,也就是$C^1$阶曲线
    又称为\keyindex{光滑曲线}{smooth curve}{curve曲线}。
\end{definition}

\begin{definition}
    给出$C^1$类的曲线$\bm r=\bm r(t)$,假设对于该曲线上一点($t=t_0$)有
    \begin{align}\label{eq:03ex01.4}
        \bm r'(t_0)\neq \bm 0\, ,
    \end{align}
    则这一点称为曲线的\keyindex{正则点}{regular point}{point点}。
    注意\refeq{03ex01.4}表示$x'(t_0),y'(t_0),z'(t_0)$中至少有一个不等于零。
\end{definition}

以后我们只考虑曲线的正则点。实际上$\bm r'(t_0)=\bm 0$是很特殊的。
如果在一段曲线上$\bm r'(t_0)\equiv\bm 0$,则$\bm r(t)$变成常向量,
这时这段曲线缩成一点,所以一段曲线上$\bm r'(t_0)=\bm 0$的点一般是孤立点。
\begin{definition}
    曲线上所有点都是正则点时,称该曲线为\keyindex{正则曲线}{regular curve}{curve曲线}。
\end{definition}

\begin{definition}
    给出曲线上一点$P$,点$Q$是曲线上$P$的邻近一点,
    使点$Q$沿曲线趋近于点$P$,若\keyindex{割线}{secant line}{}$PQ$趋近于特定位置,
    则把$PQ$的该极限位置称为曲线在$P$点的\keyindex{切线}{tangent line}{},
    定点$P$称为\keyindex{切点}{tangent point}{}。
\end{definition}

\begin{definition}
    若曲线$\bm r=\bm r(t)$上的点$P$对应参数$t_0$且$\bm r(t)$在$t_0$处可微,则
    \begin{align}\label{eq:03ex01.5}
        \bm r'(t_0)=\lim\limits_{\Delta t\rightarrow0}{\frac{\bm r(t_0+\Delta t)-\bm r(t_0)}{\Delta t}}
    \end{align}
    称为曲线在点$P$的\keyindex{切向量}{tangent vector}{vector向量}。
\end{definition}

由于我们已经规定只研究曲线的正则点,所以曲线上一点的切向量是存在的,
它就是切线上的一个非零向量,其正向和曲线参数$t$的增量方向一致。

\begin{corollary}
    曲线在参数$t_0$处的切线方程是
    \begin{align}\label{eq:03ex01.6}
        \frac{X-x(t_0)}{x'(t_0)}=\frac{Y-y(t_0)}{y'(t_0)}=\frac{Z-z(t_0)}{z'(t_0)}\, .
    \end{align}
\end{corollary}

\begin{definition}
    过切点垂直于切线的平面称为
    曲线的\keyindex{法平面}{normal plane}{}。
\end{definition}

\begin{corollary}
    曲线在参数$t_0$处的法面方程是
    \begin{align}\label{eq:03ex01.7}
        x'(t_0)(X-x(t_0))+y'(t_0)(Y-y(t_0))+z'(t_0)(Z-z(t_0))=0\, .
    \end{align}
\end{corollary}

\begin{corollary}
    曲线$\bm r=\bm r(t)$中从$\bm r(a)$到$\bm r(t)$的有向弧长是
    \begin{align}\label{eq:03ex01.8}
        \sigma(t)=\int_a^t{|\bm r'(t)|\mathrm{d}t}\, .
    \end{align}
\end{corollary}

\subsection{曲面的概念}\label{sub:曲面的概念}
\begin{definition}
    平面上不自交的简单\keyindex{闭曲线}{closed curve}{curve曲线}称为\keyindex{Jordan曲线}{Jordan curve}{curve曲线}。
    它将平面分为两个都以该曲线为边界的部分,其中一个是有限的,另一个是无限的;
    有限的区域称为\keyindex{初等区域}{}{curve曲线},即Jordan曲线的内部。
\end{definition}
\begin{example}
    正方形或矩形内部,圆或椭圆内部都是初等区域。
\end{example}

\begin{definition}
    如果平面上初等区域到三维欧氏空间内建立的映射是一一的、双方连续的到上映射,
    则把三维欧氏空间中的象称为\keyindex{简单曲面}{%simple surface
    }{surface曲面}。
\end{definition}
\begin{example}
    矩形纸片(初等区域)卷成的带有裂缝的圆柱面是简单曲面。
\end{example}

我们假定以后所讨论的曲面都是简单曲面,不另作说明。

给出平面上一初等区域$\mathscr{D}$,$\mathscr{D}$中的点的笛卡尔坐标是$(u,v)$,
$\mathscr{D}$经过上述映射$f$后的象是曲面$S$.
对于空间的笛卡尔坐标系来说,$S$上的点的坐标是$(x,y,z)$,
则可以写出$f$的解析表达式:
\begin{align}\label{eq:03ex01.9}
    \begin{array}{l}
        x=f_1(u,v)\, , \\
        y=f_2(u,v)\, , \\
        z=f_3(u,v)\, ,
    \end{array}\quad (u,v)\in \mathscr{D}\, .
\end{align}
称\refeq{03ex01.9}为曲面$S$的\keyindex{参数表示}{parametric representation}{}或\keyindex{参数方程}{parametric equation}{},
$u$和$v$称为曲面的\keyindex{参数}{parameter}{}或\keyindex{曲纹坐标}{curve coordinate}{coordinate坐标}。

习惯上常把\refeq{03ex01.9}中的函数关系符号$f_1,f_2$和$f_3$分别写成$x,y$和$z$,即
\begin{align}\label{eq:03ex01.10}
    \begin{array}{l}
        x=x(u,v)\, , \\
        y=y(u,v)\, , \\
        z=z(u,v)\, ,
    \end{array}\quad (u,v)\in \mathscr{D}\, .
\end{align}
有时也将其简写称向量函数的形式:
\begin{align}\label{eq:03ex01.11}
    \bm r=\bm r(u,v),\quad (u,v)\in \mathscr{D}\, .
\end{align}

\begin{definition}
    初等区域$\mathscr{D}$所在平面上的坐标直线$v=$常数或$u=$常数
    在曲面上的象称为曲面的\keyindex{坐标曲线}{coordinate curve}{}。
    使$v$等于常数$v_0$而$u$变动时的曲线$\bm r=\bm r(u,v_0)$叫$u$-曲线;
    使$u$等于常数$u_0$而$v$变动时的曲线$\bm r=\bm r(u_0,v)$叫$v$-曲线。
    这两族坐标曲线在曲面上构成的坐标网称为曲面上的\keyindex{曲纹坐标网}{curve coordinate net}{}。
\end{definition}

\begin{definition}
    若曲面方程中的函数有直到$k$阶的连续\keyindex{偏微商}{partial derivative}{},
    则该曲面称为\keyindex{$k$阶正则曲面}{}{surface曲面}或\keyindex{$C^k$阶曲面}{$C^k$-surface}{surface曲面}。
    特别地,$C^1$阶曲面又称为\keyindex{光滑曲面}{}{surface曲面}。
\end{definition}

以后我们假定所讨论的曲面都是光滑的。

\begin{definition}
    曲面$\bm r=\bm r(u,v)$上$(u_0,v_0)$点处两条坐标曲线的切向量分别为
    \begin{align}\label{eq:03ex01.12}
        \bm r_u(u_0,v_0) & =\frac{\partial \bm r}{\partial u}(u_0,v_0)\, , \\
        \bm r_v(u_0,v_0) & =\frac{\partial \bm r}{\partial v}(u_0,v_0)\, .
    \end{align}
    若它们不平行,即$\bm r_u\times\bm r_v$在$(u_0,v_0)$点不等于$\bm 0$,
    则称该点为曲面的\keyindex{正则点}{regular point}{point点}。
\end{definition}

以后我们只讨论曲面的正则点。

若曲面上点的曲纹坐标由下列方程确定:
\begin{align}\label{eq:03ex01.13}
    u & =u(t)\, , \\
    v & =v(t)\, ,
\end{align}
其中$t$是自变量,代入曲面的参数方程可得
该点的\keyindex{向径}{radius vector}{vector向量}为
\begin{align}\label{eq:03ex01.14}
    \bm r=\bm r\left(u(t),v(t)\right)=\bm r(t)\, .
\end{align}
当$t$在某区间上变动时,
关于$t$的函数$\bm r$相应的终点在曲面上确定了某一曲线,
该曲线在曲面上$(u_0,v_0)$点处的切方向称为
曲面在该点的\keyindex{切方向}{tangent direction}{direction方向}或\keyindex{方向}{direction}{}。
它平行于
\begin{align}\label{eq:03ex01.15}
    \bm r'(t)=\bm r_u\frac{\mathrm{d}u}{\mathrm{d}t}+\bm r_v\frac{\mathrm{d}v}{\mathrm{d}t}\, ,
\end{align}
其中$\bm r_u$和$\bm r_v$分别是在该点的两条坐标曲线的切向量。
$\bm r'(t),\bm r_u$和$\bm r_v$共面。
\begin{definition}
    曲面上正则点的所有切方向都在
    过该点的坐标曲线的切向量$\bm r_u$和$\bm r_v$所决定的平面上,
    该平面称为曲面在该点的\keyindex{切平面}{tangent plane}{}。
\end{definition}
\begin{corollary}
    曲面$\bm r=\bm r(u,v)$在点$P_0(u_0,v_0)$处的切平面方程为
    \begin{align}\label{eq:03ex01.16}
        \left|
        \begin{array}{ccc}
            X-x(u_0,v_0) & Y-y(u_0,v_0) & Z-z(u_0,v_0) \\
            x_u(u_0,v_0) & y_u(u_0,v_0) & z_u(u_0,v_0) \\
            x_v(u_0,v_0) & y_v(u_0,v_0) & z_v(u_0,v_0)
        \end{array}\right|=0\, .
    \end{align}
\end{corollary}

\begin{definition}
    曲线在正则点处垂直于切平面的方向称为曲面的\keyindex{法方向}{normal direction}{direction方向}。
    过该点平行于法方向的直线称为曲面在该点的\keyindex{法线}{normal}{}。
\end{definition}
\begin{corollary}
    曲面$\bm r=\bm r(u,v)$的法向量为$\displaystyle\bm N=\bm r_u\times\bm r_v$,
    单位法向量$\bm n=\displaystyle\frac{\bm r_u\times\bm r_v}{|\bm r_u\times\bm r_v|}$.
\end{corollary}
\begin{corollary}
    曲面$\bm r=\bm r(u,v)$在点$P_0(u_0,v_0)$处的法线方程为
    \begin{align}\label{eq:03ex01.17}
        \frac{X-x(u_0,v_0)}{\left|
            \begin{array}{cc}
                y_u(u_0,v_0) & z_u(u_0,v_0) \\
                y_v(u_0,v_0) & z_v(u_0,v_0)
            \end{array}
            \right|}=\frac{Y-y(u_0,v_0)}{\left|
            \begin{array}{cc}
                z_u(u_0,v_0) & x_u(u_0,v_0) \\
                z_v(u_0,v_0) & x_v(u_0,v_0)
            \end{array}
            \right|}=\frac{Z-z(u_0,v_0)}{\left|
            \begin{array}{cc}
                x_u(u_0,v_0) & y_u(u_0,v_0) \\
                x_v(u_0,v_0) & y_v(u_0,v_0)
            \end{array}
            \right|}\, .
    \end{align}
\end{corollary}

\begin{definition}
    \keyindex{曲面的第一基本形式}{first fundamental form of a surface}{}为
    \begin{align}\label{eq:03ex01.18}
        \uppercase\expandafter{\romannumeral1}=\mathrm{d}\bm r^2=E\mathrm{d}u^2+2F\mathrm{d}u\mathrm{d}v+G\mathrm{d}v^2\, .
    \end{align}
    其中系数
    \begin{align}\label{eq:03ex01.19}
        E=\bm r_u\cdot\bm r_u,\quad F=\bm r_u\cdot\bm r_v,\quad G=\bm r_v\cdot\bm r_v
    \end{align}
    称为\keyindex{曲面的第一类基本量}{fundamental quantities of first kind for surfaces}{}。
\end{definition}

\begin{corollary}
    曲面的第一基本形式是正定的,即
    \begin{align}\label{eq:03ex01.20}
        E>0,\quad G>0,\quad EG-F^2>0\, .
    \end{align}
\end{corollary}

\begin{corollary}
    曲面的第一基本形式决定曲面上曲线的弧长。
    设曲面上某曲线$C$为$u=u(t),v=v(t)$,
    则其上两点$A(t_0),B(t_1)$沿$C$的有向弧长为
    \begin{align}\label{eq:03ex01.21}
        s=\int_{t_0}^{t_1}{\frac{\mathrm{d}s}{\mathrm{d}t}\mathrm{d}t}=\int_{t_0}^{t_1}
        {\sqrt{E\left(\frac{\mathrm{d}u}{\mathrm{d}t}\right)^2+
            2F\frac{\mathrm{d}u}{\mathrm{d}t}\frac{\mathrm{d}v}{\mathrm{d}t}+
            G\left(\frac{\mathrm{d}v}{\mathrm{d}t}\right)^2}
        \mathrm{d}t}\, .
    \end{align}
\end{corollary}

\begin{corollary}
    曲面的第一基本形式决定曲面的面积。
    曲面$\bm r=\bm r(u,v)$上的曲面域$D$对应的$(u,v)$平面区域为$\mathscr{D}$,
    则$D$是面积为
    \begin{align}\label{eq:03ex01.22}
        \sigma=\iint\limits_{\mathscr{D}}{|\bm r_u\times\bm r_v|\mathrm{d}u\mathrm{d}v}=\iint\limits_{\mathscr{D}}{\sqrt{EG-F^2}\mathrm{d}u\mathrm{d}v}\, .
    \end{align}
\end{corollary}

\begin{definition}
    $C^2$阶曲面中$\bm r(u,v)$有连续的二阶导函数$\bm r_{uu},\bm r_{uv},\bm r_{vv}$,
    单位法向量为$\bm n$,则该\keyindex{曲面的第二基本形式}{second fundamental form of a surface}{}为
    \begin{align}\label{eq:03ex01.23}
        \uppercase\expandafter{\romannumeral2}=\bm n\cdot\mathrm{d}^2\bm r=-\mathrm{d}\bm n\cdot\mathrm{d}\bm r=L\mathrm{d}u^2+2M\mathrm{d}u\mathrm{d}v+N\mathrm{d}v^2\, ,
    \end{align}
    其中系数
    \begin{align}\label{eq:03ex01.24}
        L & =\bm r_{uu}\cdot\bm n=-\bm r_u\cdot\bm n_u,                      \\
        M & =\bm r_{uv}\cdot\bm n=-\bm r_u\cdot\bm n_v=-\bm r_v\cdot\bm n_u, \\
        N & =\bm r_{vv}\cdot\bm n=-\bm r_v\cdot\bm n_v\,
    \end{align}
    称为\keyindex{曲面的第二类基本量}{fundamental quantities of second kind for surfaces}{}。
\end{definition}

曲面的第二基本形式近似等于曲面与切平面的有向距离的两倍,刻画了曲面在空间中的弯曲性。
当曲面在给定点向$\bm n$的正侧弯曲时为正,反之为负。

\subsection{曲面的基本公式}\label{sub:曲面的基本公式}
\begin{notation}
    为了让式子更简洁规律,本小节采用新的符号来表示之前的量:
    \begin{align*}%\label{eq:03ex01.25}
        \bm u^1    & =\bm u,      & \bm u^2    & =v,          & \bm r_1    & =\bm r_u,    & \bm r_2         & =\bm r_v,                                                                                       \\
        \bm r_{11} & =\bm r_{uu}, & \bm r_{22} & =\bm r_{vv}, & \bm r_{12} & =\bm r_{uv}, & \bm r_{21}      & =\bm r_{vu},                                                                                    \\
        g_{11}     & =E,          & g_{22}     & =G,          & g_{12}     & =g_{21}=F,   & g               & =EG-F^2=\left|\begin{array}{cc}g_{11}&g_{12}\\ g_{21}&g_{22}\end{array}\right|,                                                \\
        L_{11}     & =L,          & L_{22}     & =N,          & L_{12}     & =L_{21}=M,   & \sum\limits_{i} & =\sum\limits_{i=1}^{2},\quad \sum\limits_{i,j}=\sum\limits_{i=1}^{2}{\sum\limits_{j=1}^{2}}\, .
    \end{align*}
\end{notation}

对于一个$C^3$阶曲面$S$即$\bm r=\bm r(u^1,u^2)$(这里$u^2$中2是上标,不是平方),它确定了向量
\begin{align}\label{eq:03ex01.26}
    \bm r_1=\frac{\partial \bm r}{\partial u^1}, \quad \bm r_2=\frac{\partial \bm r}{\partial u^2}, \quad \bm n=\frac{\bm r_1\times\bm r_2}{\sqrt{g}}\, .
\end{align}
对这三个向量再求导,并且命
\begin{align}\label{eq:03ex01.27}
    \left\{\begin{array}{l}
        \displaystyle\bm r_{ij}=\lambda_{ij}\bm n+\sum\limits_{k}{\varGamma_{ij}^k\bm r_k}, \\
        \displaystyle\bm n_i=\sum\limits_{j}{\mu_i^j\bm r_j},
    \end{array}\right.\quad i,j=1,2\, .
\end{align}
其中$\varGamma_{ij}^k,\lambda_{ij},\mu_i^j$为待定系数,$i,j,k=1,2$为上下标。

注意到$\bm r_k\cdot\bm n=0$且$\bm r_{ij}\cdot\bm n=L_{ij}$,
于是\refeq{03ex01.27}第一式两边点乘$\bm n$得
\begin{align}\label{eq:03ex01.28}
    \lambda_{ij}=\bm r_{ij}\cdot\bm n=L_{ij}\, .
\end{align}
注意到$g_{ij}=\bm r_i\cdot\bm r_j$,对它们求导得
\begin{align}\label{eq:03ex01.29}
    \frac{\partial g_{ij}}{\partial u^l}=\bm r_{il}\cdot\bm r_{j}+\bm r_{i}\cdot\bm r_{jl}\, .
\end{align}
由于$\bm r_{ij}=\bm r_{ji}$且$g_{ij}=g_{ji}$,所以有
\begin{align}\label{eq:03ex01.30}
    \frac{1}{2}\left(\frac{\partial g_{il}}{\partial u^j}+\frac{\partial g_{jl}}{\partial u^i}-\frac{\partial g_{ij}}{\partial u^l}\right)=\bm r_{ij}\cdot\bm r_l=\sum\limits_{k}{\varGamma_{ij}^k g_{kl}}, \quad i,j,l=1,2\, .
\end{align}
记$[g_{ij}]_{2\times2}$的逆矩阵为$[g^{ij}]_{2\times2}$,即
\begin{align}\label{eq:03ex01.31}
    \sum\limits_{k}{g^{ik}g_{kj}}=\left\{\begin{array}{ll}
        1, & \text{if}\quad i=j,     \\
        0, & \text{if}\quad i\neq j,
    \end{array}\right.\quad i,j=2\, .
\end{align}
于是从\refeq{03ex01.30}可解得系数$\varGamma_{ij}^k$为
\begin{align}\label{eq:03ex01.32}
    \varGamma_{ij}^k=\frac{1}{2}\sum\limits_{l}{g^{lk}\left(\frac{\partial g_{il}}{\partial u^j}+\frac{\partial g_{jl}}{\partial u^i}-\frac{\partial g_{ij}}{\partial u^l}\right)},\quad i,j,k=1,2\, .
\end{align}
对于\refeq{03ex01.27}第二式,两边点乘$\bm r_k$得
\begin{align}\label{eq:03ex01.33}
    \bm n_i\cdot\bm r_k=-\bm r_{ik}\cdot\bm n=-L_{ik}=\sum\limits_{j}{\mu_i^j g_{jk}}, \quad i,k=1,2\, .
\end{align}
因此
\begin{align}\label{eq:03ex01.34}
    \mu_i^j=-\sum\limits_{k}{L_{ik}g^{kj}},\quad i,j=1,2\, .
\end{align}
于是我们得到了$\bm r_1,\bm r_2$和$\bm n$的导向量。
\begin{theorem}
    $C^3$阶曲面$S$有
    \begin{align}\label{eq:03ex01.35}
        \left\{\begin{array}{l}
            \displaystyle\bm r_{ij}=L_{ij}\bm n+\sum\limits_{k}{\varGamma_{ij}^k\bm r_k}, \\
            \displaystyle\bm n_i=-\sum\limits_{j,k}{L_{ik}g^{kj}\bm r_j},
        \end{array}\right.\quad i,j=1,2\, ,
    \end{align}
    上式称为\keyindex{曲面的基本公式}{fundamental formulas for surfaces}{surface曲面},
    其中第一式称为\keyindex{曲面的高斯公式}{Gauss formula of surfaces}{surface曲面},
    第二式称为\keyindex{曲面的外恩加滕公式}{Weingarten formula of surfaces}{surface曲面},
    系数
    \begin{align}\label{eq:03ex01.36}
        \varGamma_{ij}^k=\frac{1}{2}\sum\limits_{l}{g^{lk}\left(\frac{\partial g_{il}}{\partial u^j}+\frac{\partial g_{jl}}{\partial u^i}-\frac{\partial g_{ij}}{\partial u^l}\right)},\quad i,j,k=1,2
    \end{align}
    称为\keyindex{第二类Christoffel符号}{Christoffel symbol of the second kind}{}或\keyindex{联络系数}{coefficient of connection}{}。
\end{theorem}
\begin{notation}
    将曲面基本方程中的新记号用旧记号表示:
    \begin{align*}
        g^{11}                               & =\frac{G}{EG-F^2},                                                       &
        g^{22}                               & =\frac{E}{EG-F^2},                                                         \\
        g^{12}                               & =\frac{-F}{EG-F^2},                                                      &
        g^{21}                               & =\frac{-F}{EG-F^2},                                                        \\
        \frac{\partial g_{11}}{\partial u^1} & =\frac{\partial E}{\partial u}=E_u,                                      &
        \frac{\partial g_{11}}{\partial u^2} & =\frac{\partial E}{\partial v}=E_v,                                        \\
        \frac{\partial g_{22}}{\partial u^1} & =\frac{\partial G}{\partial u}=G_u,                                      &
        \frac{\partial g_{22}}{\partial u^2} & =\frac{\partial G}{\partial v}=G_v,                                        \\
        \frac{\partial g_{12}}{\partial u^1} & =\frac{\partial g_{21}}{\partial u^1}=\frac{\partial F}{\partial u}=F_u, &
        \frac{\partial g_{12}}{\partial u^2} & =\frac{\partial g_{21}}{\partial u^2}=\frac{\partial F}{\partial v}=F_v,   \\
        \varGamma_{11}^1                     & =\frac{GE_u-F(2F_u-E_v)}{2(EG-F^2)},                                     &
        \varGamma_{11}^2                     & =\frac{E(2F_u-E_v)-FE_u}{2(EG-F^2)},                                       \\
        \varGamma_{12}^1                     & =\frac{GE_v-FG_u}{2(EG-F^2)},                                            &
        \varGamma_{12}^2                     & =\frac{EG_u-FE_v}{2(EG-F^2)},                                              \\
        \varGamma_{22}^1                     & =\frac{G(2F_v-G_u)-FG_v}{2(EG-F^2)},                                     &
        \varGamma_{22}^2                     & =\frac{EG_v-F(2F_v-G_u)}{2(EG-F^2)},                                       \\
        \mu_1^1                              & =\frac{-LG+MF}{EG-F^2},                                                  &
        \mu_1^2                              & =\frac{LF-ME}{EG-F^2},                                                     \\
        \mu_2^1                              & =\frac{NF-MG}{EG-F^2},                                                   &
        \mu_2^2                              & =\frac{-NE+MF}{EG-F^2}\, .
    \end{align*}
\end{notation}