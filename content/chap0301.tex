\section{基本形状接口}\label{sec:基本形状接口}

\refvar{Shape}{}的接口定义于
源文件\href{https://github.com/mmp/pbrt-v3/tree/master/src/core/shape.h}{\ttfamily core/shape.h}中,
\href{https://github.com/mmp/pbrt-v3/tree/master/src/core/shape.cpp}{\ttfamily core/shape.cpp}中
可以找到\refvar{Shape}{}公共方法的定义。
基类\refvar{Shape}{}定义了通用形状接口。
它也暴露了一些对所有\refvar{Shape}{}实现有用的公有数据成员。
\begin{lstlisting}
`\initcode{Shape Declarations}{=}`
class `\initvar{Shape}{}` {
public:
`\refcode{Shape Interface}{}`
`\refcode{Shape Public Data}{}`
};
\end{lstlisting}

所有形状都定义在物体的坐标空间中;
例如,所有球体都定义在球心位于原点的坐标系统中。
为了在场景别处放置球体,必须提供描述从物体空间到世界空间映射的变换。
类\refvar{Shape}{}保存了该变换及其逆。

\refvar{Shape}{}也接收一个布尔参数\refvar{reverseOrientation}{}来
表示其曲面法线方向是否与默认的相反。
该能力很有用,因为曲面法线的朝向用于决定形状哪一面是“外面”。
例如,发光形状只在曲面法线所在那侧是发光的。
该参数值通过pbrt输入文件中的{\ttfamily ReverseOrientation}语句控制。

\refvar{Shape}{}还保存了调用\refvar[SwapsHandedness]{Transform::SwapsHandedness}{()}
来进行物体到世界变换的返回值。
每次找到光线交点就调用的\refvar{SurfaceInteraction}{}构造函数需要该值,
所以\refvar{Shape}{}构造函数一次性计算并保存它。
\begin{lstlisting}
`\initcode{Shape Method Definitions}{=}\initnext{ShapeMethodDefinitions}`
`\refvar{Shape}{}`::`\refvar{Shape}{}`(const `\refvar{Transform}{}` *ObjectToWorld,
        const `\refvar{Transform}{}` *WorldToObject, bool reverseOrientation)
    : `\refvar{ObjectToWorld}{}`(ObjectToWorld), `\refvar{WorldToObject}{}`(WorldToObject),
      `\refvar{reverseOrientation}{}`(reverseOrientation),
      `\refvar{transformSwapsHandedness}{}`(ObjectToWorld->`\refvar{SwapsHandedness}{}`()) {
}
\end{lstlisting}

一个重要细节是形状保存了指向其变换的指针而不是直接保存\refvar{Transform}{}
对象。回想\refsec{变换}\refvar{Transform}{}对象由总共32个浮点数表示,需要128字节内存;
因为场景中许多形状经常被施加同样的变换,
pbrt维持了一个\refvar{Transform}{}池使它们可以复用,
并把指向共享\refvar{Transform}{}的指针传给形状。
这样,\refvar{Shape}{}析构函数不会删除其\refvar{Transform}{}指针,
而是让\refvar{Transform}{}控制代码来管理那些内存。
\begin{lstlisting}
`\initcode{Shape Public Data}{=}`
const `\refvar{Transform}{}` *`\initvar{ObjectToWorld}{}`, *`\initvar{WorldToObject}{}`;
const bool `\initvar{reverseOrientation}{}`;
const bool `\initvar{transformSwapsHandedness}{}`;
\end{lstlisting}

\subsection{边界}\label{sub:边界}
pbrt要渲染的场景中经常包含处理计算量极大的物体。
对于许多操作,有一个3D\keyindex{包围盒}{bounding volume}{}来封住物体常常很有用。
例如,如果一条光线没有穿过某个包围盒,对于该光线pbrt可以避免处理里面所有物体。

轴对齐边界框是方便的包围盒,因为它们只需要六个浮点值保存且适合很多形状。
而且测试光线与轴对齐边界框的交点的成本极低。
因此每个\refvar{Shape}{}实现必须能
用\refvar{Bounds3f}{}表示的轴对齐边界框包围自己。
有两种不同的包围方法。
第一个是\refvar{ObjectBound}{()},返回在形状的物体空间里的边界框。
\begin{lstlisting}
`\initcode{Shape Interface}{=}\initnext{ShapeInterface}`
virtual `\refvar{Bounds3f}{}` `\initvar{ObjectBound}{}`() const = 0;
\end{lstlisting}

第二个方法是\refvar[Shape::WorldBound]{WorldBound}{()},返回世界空间的边界框。
pbrt提供该方法的默认实现将物体空间边界变换到世界空间。
然而可以轻松计算更紧致的世界空间边界的形状应该重载该方法。
这种形状的一个例子是三角形(\reffig{3.1})。
\begin{figure}[htbp]
    \centering%LaTeX with PSTricks extensions
%%Creator: Inkscape 1.0.1 (3bc2e813f5, 2020-09-07)
%%Please note this file requires PSTricks extensions
\psset{xunit=.5pt,yunit=.5pt,runit=.5pt}
\begin{pspicture}(629.46002197,511.55999756)
{
\newrgbcolor{curcolor}{1 1 1}
\pscustom[linestyle=none,fillstyle=solid,fillcolor=curcolor]
{
\newpath
\moveto(339.26,140.32999756)
\lineto(567.76,140.32999756)
\lineto(567.76,25.82999756)
\closepath
}
}
{
\newrgbcolor{curcolor}{0 0 0}
\pscustom[linewidth=1,linecolor=curcolor]
{
\newpath
\moveto(339.26,140.32999756)
\lineto(567.76,140.32999756)
\lineto(567.76,25.82999756)
\closepath
}
}
{
\newrgbcolor{curcolor}{1 1 1}
\pscustom[linestyle=none,fillstyle=solid,fillcolor=curcolor]
{
\newpath
\moveto(3.48,364.86999756)
\lineto(172.22,210.78999756)
\lineto(95.02,126.23999756)
\closepath
}
}
{
\newrgbcolor{curcolor}{0 0 0}
\pscustom[linewidth=1,linecolor=curcolor]
{
\newpath
\moveto(3.48,364.86999756)
\lineto(172.22,210.78999756)
\lineto(95.02,126.23999756)
\closepath
}
}
{
\newrgbcolor{curcolor}{0 0 0}
\pscustom[linewidth=1,linecolor=curcolor,linestyle=dashed,dash=4]
{
\newpath
\moveto(0.5,367.80000305)
\lineto(173.5,367.80000305)
\lineto(173.5,124.80000305)
\lineto(0.5,124.80000305)
\closepath
}
}
{
\newrgbcolor{curcolor}{0 0 0}
\pscustom[linewidth=1,linecolor=curcolor,linestyle=dashed,dash=4]
{
\newpath
\moveto(334.76000977,142.32998657)
\lineto(570.76000977,142.32998657)
\lineto(570.76000977,26.32998657)
\lineto(334.76000977,26.32998657)
\closepath
}
}
{
\newrgbcolor{curcolor}{1 1 1}
\pscustom[linestyle=none,fillstyle=solid,fillcolor=curcolor]
{
\newpath
\moveto(339.88,393.97999756)
\lineto(568.38,393.68999756)
\lineto(568.23,279.18999756)
\closepath
}
}
{
\newrgbcolor{curcolor}{0 0 0}
\pscustom[linewidth=1,linecolor=curcolor]
{
\newpath
\moveto(339.88,393.97999756)
\lineto(568.38,393.68999756)
\lineto(568.23,279.18999756)
\closepath
}
}
{
\newrgbcolor{curcolor}{0.60000002 0.60000002 0.60000002}
\pscustom[linewidth=1,linecolor=curcolor,linestyle=dashed,dash=4]
{
\newpath
\moveto(335.68396305,394.16577219)
\lineto(463.57916173,510.66391876)
\lineto(627.21528669,331.01921773)
\lineto(499.32008801,214.52107116)
\closepath
}
}
{
\newrgbcolor{curcolor}{0 0 0}
\pscustom[linewidth=1,linecolor=curcolor,linestyle=dashed,dash=4]
{
\newpath
\moveto(333.95999146,511.05999756)
\lineto(628.95999146,511.05999756)
\lineto(628.95999146,214.05999756)
\lineto(333.95999146,214.05999756)
\closepath
}
}
{
\newrgbcolor{curcolor}{0 0 0}
\pscustom[linewidth=1,linecolor=curcolor]
{
\newpath
\moveto(208.1499939,162.76998901)
\lineto(291.20001221,124.44000244)
}
}
{
\newrgbcolor{curcolor}{0 0 0}
\pscustom[linestyle=none,fillstyle=solid,fillcolor=curcolor]
{
\newpath
\moveto(284.43,121.49999756)
\lineto(290.61,124.70999756)
\lineto(289.05,131.49999756)
\lineto(298.56,121.04999756)
\closepath
}
}
{
\newrgbcolor{curcolor}{0.65098041 0.65098041 0.65098041}
\pscustom[linestyle=none,fillstyle=solid,fillcolor=curcolor]
{
\newpath
\moveto(286.35,121.93999756)
\lineto(297.36,121.59999756)
\lineto(291.18,124.44999756)
\closepath
}
}
{
\newrgbcolor{curcolor}{0.40000001 0.40000001 0.40000001}
\pscustom[linestyle=none,fillstyle=solid,fillcolor=curcolor]
{
\newpath
\moveto(289.96,129.74999756)
\lineto(297.36,121.59999756)
\lineto(291.18,124.44999756)
\closepath
}
}
{
\newrgbcolor{curcolor}{0 0 0}
\pscustom[linewidth=1,linecolor=curcolor]
{
\newpath
\moveto(207.8500061,308.70999146)
\lineto(290.36999512,340.97999573)
}
}
{
\newrgbcolor{curcolor}{0 0 0}
\pscustom[linestyle=none,fillstyle=solid,fillcolor=curcolor]
{
\newpath
\moveto(287.8,334.05999756)
\lineto(289.76,340.73999756)
\lineto(283.79,344.30999756)
\lineto(297.91,343.92999756)
\closepath
}
}
{
\newrgbcolor{curcolor}{0.65098041 0.65098041 0.65098041}
\pscustom[linestyle=none,fillstyle=solid,fillcolor=curcolor]
{
\newpath
\moveto(288.81,335.74999756)
\lineto(296.69,343.44999756)
\lineto(290.35,340.96999756)
\closepath
}
}
{
\newrgbcolor{curcolor}{0.40000001 0.40000001 0.40000001}
\pscustom[linestyle=none,fillstyle=solid,fillcolor=curcolor]
{
\newpath
\moveto(285.68,343.75999756)
\lineto(296.69,343.44999756)
\lineto(290.35,340.96999756)
\closepath
}
}
{
\newrgbcolor{curcolor}{0 0 0}
\pscustom[linestyle=none,fillstyle=solid,fillcolor=curcolor]
{
\newpath
\moveto(337.81564237,191.32581977)
\lineto(336.02657987,191.32581977)
\curveto(335.10470487,192.38311143)(334.38855904,193.53675727)(333.87814237,194.78675727)
\curveto(333.36772571,196.03675727)(333.11251737,197.44561143)(333.11251737,199.01331977)
\curveto(333.11251737,200.5810281)(333.36772571,201.98988227)(333.87814237,203.23988227)
\curveto(334.38855904,204.48988227)(335.10470487,205.6435281)(336.02657987,206.70081977)
\lineto(337.81564237,206.70081977)
\lineto(337.81564237,206.62269477)
\curveto(337.39376737,206.24248643)(336.99012154,205.80238227)(336.60470487,205.30238227)
\curveto(336.22449654,204.8075906)(335.87032987,204.2294656)(335.54220487,203.56800727)
\curveto(335.22970487,202.92738227)(334.97449654,202.2216531)(334.77657987,201.45081977)
\curveto(334.58387154,200.67998643)(334.48751737,199.86748643)(334.48751737,199.01331977)
\curveto(334.48751737,198.12269477)(334.58126737,197.3075906)(334.76876737,196.56800727)
\curveto(334.96147571,195.82842393)(335.21928821,195.12529893)(335.54220487,194.45863227)
\curveto(335.85470487,193.81800727)(336.21147571,193.23988227)(336.61251737,192.72425727)
\curveto(337.01355904,192.20342393)(337.41460071,191.76331977)(337.81564237,191.40394477)
\closepath
}
}
{
\newrgbcolor{curcolor}{0 0 0}
\pscustom[linestyle=none,fillstyle=solid,fillcolor=curcolor]
{
\newpath
\moveto(347.19064237,194.54456977)
\lineto(345.72970487,194.54456977)
\lineto(345.72970487,195.47425727)
\curveto(345.59949654,195.3857156)(345.42241321,195.2607156)(345.19845487,195.09925727)
\curveto(344.97970487,194.94300727)(344.76616321,194.81800727)(344.55782987,194.72425727)
\curveto(344.31303821,194.6044656)(344.03178821,194.50550727)(343.71407987,194.42738227)
\curveto(343.39637154,194.34404893)(343.02397571,194.30238227)(342.59689237,194.30238227)
\curveto(341.81043404,194.30238227)(341.14376737,194.56279893)(340.59689237,195.08363227)
\curveto(340.05001737,195.6044656)(339.77657987,196.2685281)(339.77657987,197.07581977)
\curveto(339.77657987,197.7372781)(339.91720487,198.27113227)(340.19845487,198.67738227)
\curveto(340.48491321,199.0888406)(340.89116321,199.41175727)(341.41720487,199.64613227)
\curveto(341.94845487,199.88050727)(342.58647571,200.03936143)(343.33126737,200.12269477)
\curveto(344.07605904,200.2060281)(344.87553821,200.2685281)(345.72970487,200.31019477)
\lineto(345.72970487,200.53675727)
\curveto(345.72970487,200.8700906)(345.66980904,201.14613227)(345.55001737,201.36488227)
\curveto(345.43543404,201.58363227)(345.26876737,201.75550727)(345.05001737,201.88050727)
\curveto(344.84168404,202.00029893)(344.59168404,202.0810281)(344.30001737,202.12269477)
\curveto(344.00835071,202.16436143)(343.70366321,202.18519477)(343.38595487,202.18519477)
\curveto(343.00053821,202.18519477)(342.57085071,202.13311143)(342.09689237,202.02894477)
\curveto(341.62293404,201.92998643)(341.13335071,201.7841531)(340.62814237,201.59144477)
\lineto(340.55001737,201.59144477)
\lineto(340.55001737,203.08363227)
\curveto(340.83647571,203.16175727)(341.25053821,203.24769477)(341.79220487,203.34144477)
\curveto(342.33387154,203.43519477)(342.86772571,203.48206977)(343.39376737,203.48206977)
\curveto(344.00835071,203.48206977)(344.54220487,203.42998643)(344.99532987,203.32581977)
\curveto(345.45366321,203.22686143)(345.84949654,203.05498643)(346.18282987,202.81019477)
\curveto(346.51095487,202.57061143)(346.76095487,202.2607156)(346.93282987,201.88050727)
\curveto(347.10470487,201.50029893)(347.19064237,201.02894477)(347.19064237,200.46644477)
\closepath
\moveto(345.72970487,196.69300727)
\lineto(345.72970487,199.12269477)
\curveto(345.28178821,199.0966531)(344.75314237,199.0575906)(344.14376737,199.00550727)
\curveto(343.53960071,198.95342393)(343.06043404,198.8779031)(342.70626737,198.77894477)
\curveto(342.28439237,198.6591531)(341.94324654,198.4716531)(341.68282987,198.21644477)
\curveto(341.42241321,197.96644477)(341.29220487,197.6200906)(341.29220487,197.17738227)
\curveto(341.29220487,196.67738227)(341.44324654,196.2997781)(341.74532987,196.04456977)
\curveto(342.04741321,195.79456977)(342.50835071,195.66956977)(343.12814237,195.66956977)
\curveto(343.64376737,195.66956977)(344.11512154,195.7685281)(344.54220487,195.96644477)
\curveto(344.96928821,196.16956977)(345.36512154,196.41175727)(345.72970487,196.69300727)
\closepath
}
}
{
\newrgbcolor{curcolor}{0 0 0}
\pscustom[linestyle=none,fillstyle=solid,fillcolor=curcolor]
{
\newpath
\moveto(354.42501737,199.01331977)
\curveto(354.42501737,197.44561143)(354.16980904,196.03675727)(353.65939237,194.78675727)
\curveto(353.14897571,193.53675727)(352.43282987,192.38311143)(351.51095487,191.32581977)
\lineto(349.72189237,191.32581977)
\lineto(349.72189237,191.40394477)
\curveto(350.12293404,191.76331977)(350.52397571,192.20342393)(350.92501737,192.72425727)
\curveto(351.33126737,193.23988227)(351.68803821,193.81800727)(351.99532987,194.45863227)
\curveto(352.31824654,195.12529893)(352.57345487,195.82842393)(352.76095487,196.56800727)
\curveto(352.95366321,197.3075906)(353.05001737,198.12269477)(353.05001737,199.01331977)
\curveto(353.05001737,199.86748643)(352.95366321,200.67998643)(352.76095487,201.45081977)
\curveto(352.56824654,202.2216531)(352.31303821,202.92738227)(351.99532987,203.56800727)
\curveto(351.66720487,204.2294656)(351.31043404,204.8075906)(350.92501737,205.30238227)
\curveto(350.54480904,205.80238227)(350.14376737,206.24248643)(349.72189237,206.62269477)
\lineto(349.72189237,206.70081977)
\lineto(351.51095487,206.70081977)
\curveto(352.43282987,205.6435281)(353.14897571,204.48988227)(353.65939237,203.23988227)
\curveto(354.16980904,201.98988227)(354.42501737,200.5810281)(354.42501737,199.01331977)
\closepath
}
}
{
\newrgbcolor{curcolor}{0 0 0}
\pscustom[linestyle=none,fillstyle=solid,fillcolor=curcolor]
{
\newpath
\moveto(338.77179538,5.00714437)
\lineto(336.98273288,5.00714437)
\curveto(336.06085788,6.06443603)(335.34471204,7.21808187)(334.83429538,8.46808187)
\curveto(334.32387871,9.71808187)(334.06867038,11.12693603)(334.06867038,12.69464437)
\curveto(334.06867038,14.2623527)(334.32387871,15.67120687)(334.83429538,16.92120687)
\curveto(335.34471204,18.17120687)(336.06085788,19.3248527)(336.98273288,20.38214437)
\lineto(338.77179538,20.38214437)
\lineto(338.77179538,20.30401937)
\curveto(338.34992038,19.92381103)(337.94627454,19.48370687)(337.56085788,18.98370687)
\curveto(337.18064954,18.4889152)(336.82648288,17.9107902)(336.49835788,17.24933187)
\curveto(336.18585788,16.60870687)(335.93064954,15.9029777)(335.73273288,15.13214437)
\curveto(335.54002454,14.36131103)(335.44367038,13.54881103)(335.44367038,12.69464437)
\curveto(335.44367038,11.80401937)(335.53742038,10.9889152)(335.72492038,10.24933187)
\curveto(335.91762871,9.50974853)(336.17544121,8.80662353)(336.49835788,8.13995687)
\curveto(336.81085788,7.49933187)(337.16762871,6.92120687)(337.56867038,6.40558187)
\curveto(337.96971204,5.88474853)(338.37075371,5.44464437)(338.77179538,5.08526937)
\closepath
}
}
{
\newrgbcolor{curcolor}{0 0 0}
\pscustom[linestyle=none,fillstyle=solid,fillcolor=curcolor]
{
\newpath
\moveto(349.04523288,12.65558187)
\curveto(349.04523288,11.9264152)(348.94106621,11.2701652)(348.73273288,10.68683187)
\curveto(348.52960788,10.10349853)(348.25356621,9.6139152)(347.90460788,9.21808187)
\curveto(347.53481621,8.80662353)(347.12856621,8.4967277)(346.68585788,8.28839437)
\curveto(346.24314954,8.08526937)(345.75617038,7.98370687)(345.22492038,7.98370687)
\curveto(344.73012871,7.98370687)(344.29783704,8.0436027)(343.92804538,8.16339437)
\curveto(343.55825371,8.2779777)(343.19367038,8.4342277)(342.83429538,8.63214437)
\lineto(342.74054538,8.22589437)
\lineto(341.36554538,8.22589437)
\lineto(341.36554538,20.38214437)
\lineto(342.83429538,20.38214437)
\lineto(342.83429538,16.03839437)
\curveto(343.24575371,16.37693603)(343.68325371,16.6529777)(344.14679538,16.86651937)
\curveto(344.61033704,17.08526937)(345.13117038,17.19464437)(345.70929538,17.19464437)
\curveto(346.74054538,17.19464437)(347.55304538,16.79881103)(348.14679538,16.00714437)
\curveto(348.74575371,15.2154777)(349.04523288,14.0982902)(349.04523288,12.65558187)
\closepath
\moveto(347.52960788,12.61651937)
\curveto(347.52960788,13.65818603)(347.35773288,14.44724853)(347.01398288,14.98370687)
\curveto(346.67023288,15.52537353)(346.11554538,15.79620687)(345.34992038,15.79620687)
\curveto(344.92283704,15.79620687)(344.49054538,15.70245687)(344.05304538,15.51495687)
\curveto(343.61554538,15.3326652)(343.20929538,15.09568603)(342.83429538,14.80401937)
\lineto(342.83429538,9.80401937)
\curveto(343.25096204,9.61651937)(343.60773288,9.48631103)(343.90460788,9.41339437)
\curveto(344.20669121,9.3404777)(344.54783704,9.30401937)(344.92804538,9.30401937)
\curveto(345.74054538,9.30401937)(346.37596204,9.56964437)(346.83429538,10.10089437)
\curveto(347.29783704,10.6373527)(347.52960788,11.47589437)(347.52960788,12.61651937)
\closepath
}
}
{
\newrgbcolor{curcolor}{0 0 0}
\pscustom[linestyle=none,fillstyle=solid,fillcolor=curcolor]
{
\newpath
\moveto(355.74054538,12.69464437)
\curveto(355.74054538,11.12693603)(355.48533704,9.71808187)(354.97492038,8.46808187)
\curveto(354.46450371,7.21808187)(353.74835788,6.06443603)(352.82648288,5.00714437)
\lineto(351.03742038,5.00714437)
\lineto(351.03742038,5.08526937)
\curveto(351.43846204,5.44464437)(351.83950371,5.88474853)(352.24054538,6.40558187)
\curveto(352.64679538,6.92120687)(353.00356621,7.49933187)(353.31085788,8.13995687)
\curveto(353.63377454,8.80662353)(353.88898288,9.50974853)(354.07648288,10.24933187)
\curveto(354.26919121,10.9889152)(354.36554538,11.80401937)(354.36554538,12.69464437)
\curveto(354.36554538,13.54881103)(354.26919121,14.36131103)(354.07648288,15.13214437)
\curveto(353.88377454,15.9029777)(353.62856621,16.60870687)(353.31085788,17.24933187)
\curveto(352.98273288,17.9107902)(352.62596204,18.4889152)(352.24054538,18.98370687)
\curveto(351.86033704,19.48370687)(351.45929538,19.92381103)(351.03742038,20.30401937)
\lineto(351.03742038,20.38214437)
\lineto(352.82648288,20.38214437)
\curveto(353.74835788,19.3248527)(354.46450371,18.17120687)(354.97492038,16.92120687)
\curveto(355.48533704,15.67120687)(355.74054538,14.2623527)(355.74054538,12.69464437)
\closepath
}
}
\end{pspicture}

    \caption{(a)三角形的世界空间边界框通过将其物体空间边界框
        变换到世界空间再寻找包围结果边界框的边界框算得;可能得到肥大的边界框。
        (b)然而,如果三角形的顶点首先从物体空间变换到世界空间再包围,边界框可能合适得多。}
    \label{fig:3.1}
\end{figure}

\begin{lstlisting}
`\refcode{Shape Method Definitions}{+=}\lastnext{ShapeMethodDefinitions}`
`\refvar{Bounds3f}{}` `\refvar{Shape}{}`::`\initvar[Shape::WorldBound]{WorldBound}{}`() const {
    return (*`\refvar{ObjectToWorld}{}`)(`\refvar{ObjectBound}{}`());
}
\end{lstlisting}

\subsection{光线-边界交点}\label{sub:光线-边界交点}
有了使用\refvar{Bounds3f}{}实例包围形状后,我们将添加一个\refvar{Bounds3}{}方法,即
\refvar{Bounds3::IntersectP}{()},以检查光线-框交点,
并且如果有的话还返回交点的两个参数$t$值。

