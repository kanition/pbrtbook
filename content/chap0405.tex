\section{扩展阅读}\label{sec:扩展阅读04}
引入光线追踪算法之后,涌现了大量尝试寻找高效方法对其加速的研究,
主要是通过开发改进的光线追踪加速结构。
《\citetitle{10.5555/94788}》\citep{10.5555/94788}中Arvo和Kirk的章节
总结了1989年最新进展并为区分不同光线相交加速方法提供了优秀的分类方案。

Kirk和Arvo \parencite*{Kirk88theray}引入了\keyindex{元层次}{meta-hierarchies}{}的统一原则。
它们证明了通过让实现的加速数据结构与场景图元遵照相同的接口,
很容易混合与匹配不同的相交加速框架。
pbrt遵循这一模型,因为\refvar{Aggregate}{}继承自基类\refvar{Primitive}{}。

\subsection{网格}\label{sub:网格}
Fujimoto、Tanaka和Iwata\parencite*{4056861}引入了均匀网格,
即把场景边界分解为等长网格的空间细分方法。
Amanatides和Woo \parencite*{10.2312:egtp.19871000}
以及Cleary和Wyvill \parencite*{Cleary1988}描述了更高效的网格遍历方法。
Snyder和Barr \parencite*{10.1145/37401.37417}\sidenote{译者注:原文参考文献页码标注错误,已修正。}描述了
对该方法的大量改进并证明了网格对于渲染极其复杂场景的用处。
Jevans和Wyvill \parencite*{Jevans1989:23}引入了层次化网格,
即含有许多图元的网格自我细化为小格。
Cazals、Drettakis和Puech \parencite*{cazals1995filtering}以及
Klimaszewski和Sederberg \parencite*{576857}为层次化网格开发了更复杂的技术。

Ize等\parencite*{4061545}为网格的并行创建开发了高效算法。
他们的有趣发现之一是随着所用处理核数量的增长,
网格创建性能很快被有效内存带宽所限制。

选择最优网格分辨率对于从网格中获得优异性能很重要。
Ize等\parencite*{4342587}有该话题的优秀论文,
为完全自动化选择分辨率以及在使用层次化网格时决定何时细化为子网格提供了坚实基础。
他们用大量简化假设推导出理论结果,然后证明了这些结果渲染真实世界场景的适用性。
他们的论文也包括对该领域前人工作很好的筛选引用。

Lagae和Dutré \parencite*{lagae2008compact}基于
哈希法\sidenote{译者注:即hashing,也称散列法。}为均匀网格
描述了一种新颖的表示,它具有的优良性质是不仅每个图元
拥有对网格的单个索引,而且每个网格也只有单个图元索引。
他们证明了该表示有很低的内存使用量且仍然非常高效。

Hunt和Mark \parencite*{4634613}证明了在透视空间中构建网格,
即投影中心是相机或光源时,能让追踪相机或光源发出的光线高效得多。
尽管该方法需要多种加速结构,但从为不同种类光线专门设计的多种结构中获得的性能提升可以很高。
他们的方法也因在某种意义上是栅格化和光线追踪的中间地带而令人瞩目。

\subsection{包围盒层次}\label{sub:包围盒层次}
\citet{10.1145/360349.360354}首先建议为标准可见曲面确定算法使用包围盒来剔除物体集。

