\section{向量}\label{sec:向量}

pbrt提供了2D和3D向量类。
两者都由基本向量元素参数化,
因此易于实例化整数和浮点类型的向量。

\begin{lstlisting}
`\initcode{Vector Declarations}{=}\initnext{VectorDeclarations}`
template <typename T> class `\initvar{Vector2}{}` {
public:
    `\refcode{Vector2 Public Methods}{}`
    `\refcode{Vector2 Public Data}{}`
};

`\refcode{Vector Declarations}{+=}\lastnext{VectorDeclarations}`
template <typename T> class `\initvar{Vector3}{}` {
public:
    `\refcode{Vector3 Public Methods}{}`
    `\refcode{Vector3 Public Data}{}`
};
\end{lstlisting}

接下来,我们一般只介绍\refvar{Vector3}{}方法的实现;
它们都有\refvar{Vector2}{}的相应版本,实现的区别也很简单。

一个向量表示为分量的元组,以所在定义空间$x,y,z$(3D)轴的形式给出其表达。
一个3D向量$\bm v$的各个分量写作$\bm v_x,\bm v_y$和$\bm v_z$。

\begin{lstlisting}
`\initcode{Vector2 Public Data}{=}`
T `\initvar[Vector2::x]{x}{}`, `\initvar[Vector2::y]{y}{}`;

`\initcode{Vector3 Public Data}{=}`
T `\initvar[Vector3::x]{x}{}`, `\initvar[Vector3::y]{y}{}`, `\initvar[Vector3::z]{z}{}`;
\end{lstlisting}

另一实现方式是单个模板类再用一个维度整数参数化,
并以若干{\ttfamily T}值构成的数组表示坐标。
尽管这种方法能减少代码总量,
但向量的单个分量不能以{\ttfamily v.x}等形式访问。
我们认为这种情况下,为了更透明地访问元素,
向量实现中多一点代码是值得的。

然而,一些例程发现能简单地遍历向量的分量是很有用的;
向量类也提供了C++操作符以索引分量,
这样给定向量{\ttfamily v},有{\ttfamily v[0]==v.x}等。
\begin{lstlisting}
`\initcode{Vector3 Public Methods}{=}\initnext{Vector3PublicMethods}`
T operator[](int i) const { 
    Assert(i >= 0 && i <= 2);
    if (i == 0) return x;
    if (i == 1) return y;
    return z;
}
T &operator[](int i) { 
    Assert(i >= 0 && i <= 2);
    if (i == 0) return x;
    if (i == 1) return y;
    return z;
}
\end{lstlisting}

为了方便起见,{\ttfamily typedef}指定了许多向量常用的类型,
这样在别处代码中它们会有更简洁的名字。
\begin{lstlisting}
`\refcode{Vector Declarations}{+=}\lastcode{VectorDeclarations}`
typedef `\refvar{Vector2}{}`<`\refvar{Float}{}`> `\initvar{Vector2f}{}`;
typedef `\refvar{Vector2}{}`<int>   `\initvar{Vector2i}{}`;
typedef `\refvar{Vector3}{}`<`\refvar{Float}{}`> `\initvar{Vector3f}{}`;
typedef `\refvar{Vector3}{}`<int>   `\initvar{Vector3i}{}`;
\end{lstlisting}

\subsection{点积与叉积}\label{sub:点积与叉积}